\documentclass{article}

\usepackage{amsmath, amsthm, amssymb, xypic, setspace}
\usepackage{hyperref}
\usepackage[usenames, dvipsnames]{color}

\hypersetup{
    colorlinks = true,
    linkcolor = RoyalBlue,
    citecolor = Cerulean
}

\numberwithin{equation}{section}

\theoremstyle{definition}
\newtheorem{definition}{Definition}[section]
\newtheorem{exercise}[definition]{Exercise}
\newtheorem{example}[definition]{Example}
\newtheorem{remark}[definition]{Remark}
\newtheorem{note}[definition]{Note}
\newtheorem{notation}[definition]{Notation}

\theoremstyle{theorem}
\newtheorem{proposition}[definition]{Proposition}
\newtheorem{theorem}[definition]{Theorem}
\newtheorem{corollary}[definition]{Corollary}
\newtheorem{lemma}[definition]{Lemma}

\newcommand{\Exterior}{\mathchoice{{\textstyle\bigwedge}}%
    {{\bigwedge}}%
    {{\textstyle\wedge}}%
    {{\scriptstyle\wedge}}}



\newcommand{\norm}[1]{\left\lVert#1\right\rVert}
\newcommand{\ddb}{\partial\bar{\partial}}
\newcommand{\goodwidetilde}[1]{\smash{\widetilde{#1}}}

\begin{document}
These are companion notes to Calabi's paper in which he constructs K\"ahler metrics on holomorphic vector bundles over K\"ahler manifolds. 

\section{K\"ahler metrics on holomorphic vector bundles}



A complex manifold $M$ enjoys a \textbf{complex structure} $J$, that is, an endomorphism of the (real) tangent bundle $TM$ satisfying $J^2 = -\text{id}_{TM}$. The complex structure can be described in local coordinates $(z^k = x^k + \sqrt{-1}y^k)$ by the assignment 
\begin{align*}
J\left(\frac{\partial}{\partial x^k}\right) &= \frac{\partial}{\partial y^k}\\
J\left(\frac{\partial}{\partial y^k}\right) &= - \frac{\partial}{\partial x^k}.
\end{align*}

\begin{definition}
Let $M$ be a complex manifold with complex structure $J$. 
\begin{itemize}
\item By a \textbf{hermitian metric} on $M$ we mean a Riemannian metric $g$ on $M$ such that $J$ is an orthogonal transformation with respect to $g$ in the sense that $g(JU, JV) = g(U,V)$ for each pair of tangent vectors $U,V$. A hermitian metric $g$ on $M$ determines a real $(1,1)$-form $\omega$ on $M$ by the rule $\omega(U,V) = g(JU,V)$. 
\item By a \textbf{K\"ahler metric} on $M$ we mean a hermitian metric $g$ on $M$ whose associated real $(1,1)$-form $\omega$ is closed. By the $\ddb$-lemma, if $g$ is a K\"ahler metric, then locally we may write $\omega = \sqrt{-1} \ddb \Phi$, where $\Phi$ is a local real-valued smooth function on $M$ called a \textbf{K\"ahler potential}. 
\item By a \textbf{K\"ahler form} on $M$ we mean a real $(1,1)$-form $\omega$ on $M$ that is closed. Note that a K\"ahler form determines a symmetric $2$-tensor $g$ by the assignment $g(U,V) = \omega(U, JV)$, which may or may not be positive definite. Note also that this terminology is not consistent with the modern literature (which requires a K\"ahler form to determine a positive definite symmetric tensor), but it is the terminology found in Calabi's original paper. 
\end{itemize}
\end{definition}


\begin{definition}
Let $\pi : L \to M$ be a holomorphic vector bundle over $M$. 
\begin{itemize}
\item By a \textbf{hermitian metric} $h$ on $L$ we mean a collection $\{h_x\}_{x \in M}$ of hermitian metrics on the fibers $L_x = \pi^{-1}(x)$ such that the collection $h_x$ varies smoothly in the parameter $x$. A hermitian metric $h$ on $L$ determines a nonnegative smooth function $t$ defined on the total space of $L$ by the following rule: If $\xi$ is a vector in the fiber $L_x = \pi^{-1}(x)$, then the function $t$ assigns to $\xi$ the value $h_x(\xi, \xi)$. 
\item By a \textbf{section} of $L$ we mean a smooth map $s : M \to L$ such that $\pi \circ s = \text{id}_M$. We let $\Omega^0(L)$ denote the vector space of sections of $L$. We let $\Omega^k(L)$ denote the vector space $\Omega^k(M) \otimes \Omega^0(L) $, where $\Omega^k(M)$ is the space of $k$-forms on $M$. 
\item We say that a section is \textbf{holomorphic} if $s$ is a holomorphic map of complex manifolds. 
\item By a \textbf{connection} $\nabla$ on $L$ we mean a $\mathbb{C}$-linear map $\nabla : \Omega^0(L) \to \Omega^1(L)$ satisfying $\nabla (fs) = df \otimes s + f \cdot \nabla s$ for a smooth function $f$ on $M$ and section $s$ of $L$. A connection may be extended to a map $\nabla : \Omega^k(L) \to \Omega^{k+1}(L)$ by forcing the rule $\nabla(\psi \wedge \theta) = \nabla \psi \wedge \theta + (-1)^{|\psi|} \psi \wedge \nabla \theta$ for a pair of forms $\psi$ and $\theta$. 
\item A connection $\nabla$ determines a map $\nabla^2 : \Omega^0(L) \to \Omega^2(L)$ which is linear over smooth functions on $M$ and hence corresponds to a section of the bundle $\Omega^2(\text{End}(L))$, which is called the \textbf{curvature} $F_\nabla$ of the connection. 
\item A hermitian metric $h$ on $L$ determines a unique connection $\nabla$ called the \textbf{Chern connection} associated with $h$ satisfying the following two properties
\begin{enumerate}
\item[(i)] The connection is compatible with the metric $h$ in the sense that $\nabla h = 0$, meaning that if $s,s'$ are two sections of $L$, then $d(h(s,s')) = h(\nabla s, s') + h(s, \nabla s')$. 
\item[(ii)] If $s$ is a holomorphic section of $L$, then $\nabla s$ has type $(1,0)$. 
\end{enumerate}
\item A connection $\nabla$ determines a decomposition of the tangent bundle of $L$ into \textbf{horizontal} and vertical subbundles in the following manner. If $y$ is a point of $L$, then the vertical subspace $V_y$ is the kernel of the linear map $d\pi_y : T_yL \to T_{\pi(y)}M$. If $s$ is a section of $L$, then say that $s$ is \textbf{horizontal} if $\nabla s = 0$. If $s$ is a horizontal section passing through a point $y = s(x)$ in $L$, define the \textbf{horizontal subspace} $T_yL$ to be the subspace $T_ys(M)$, where $s(M)$ denotes the image of $s$ in $L$. One can show that this subspace is independent of choice of particular horizontal section $s$. 
\end{itemize}
\end{definition}


\begin{lemma}\label{lem:form}
Let $(L,h)$ be a hermitian vector bundle over a K\"ahler manifold $(M,g)$. Then there is a unique K\"ahler form $\omega_L$ on the total space of $L$ satisfying the following three properties.
\begin{enumerate}
\item[(i)] The restriction of the form $\omega_L$ to the zero section of $L$ corresponds to the K\"ahler form $\omega$ on $M$ under the canonical identification of the zero section of $L$ with $M$. 
\item[(ii)] For each point $x \in M$, the restriction of $\omega_L$ to the fiber $\pi^{-1}(x)$ coincides with the K\"ahler form induced by the K\"ahler potential $t|_{\pi^{-1}(x)}$. 
\item[(iii)] For each point $y$ in the total space of $L$ and each tangent vector $X$ in $M$ to $x = \pi(y)$, the horizontal lift $\tilde{X}_y$ of $X$ to $y$ determined by the Chern connection of $L$ is orthogonal to the fiber $\pi^{-1}(x)$ passing through $y$. 
\end{enumerate}
Moreover, the form $\omega_L$ is given by 
\[
\omega_L = \pi^* \omega + \sqrt{-1} \ddb t. 
\]
In other words, a local potential for $\omega_L$ is given by 
\[
\Psi = \Phi \circ \pi + t
\]
where $\Phi$ is a local potential for the K\"ahler metric $g$ on $M$. 
\end{lemma}


\begin{definition}
Let $(L,h)$ be a hermitian vector bundle of rank $m$ over a K\"ahler manifold $(M,g)$ of complex dimension $n$, and let $\nabla$ denote the corresponding Chern connection. Let $\{s_\lambda\}$ be a local holomorphic frame for $L$. 
\begin{itemize}
\item Define local smooth functions $L^\lambda{}_{\mu j}$ on $M$ by the rule 
\[
\nabla s_\mu = L^\lambda{}_{\mu j} dz^j \otimes s_\lambda. 
\]
It can be shown that 
\[
L^\lambda{}_{\mu j} = (\partial_j h_{\mu \nu}) h^{\nu \lambda}.
\]
\item Define local smooth functions $F^\lambda{}_{\mu j \bar{k}}$ on $M$ by the rule 
\[
\nabla^2 s_\mu = (F^\lambda{}_{\mu j \bar{k}} dz^j \wedge d\bar{z}^k) \otimes s_\lambda.
\]
It can be shown that 
\[
F^\lambda{}_{\mu j\bar{k}} = \partial_{\bar{k}} L^\lambda{}_{\mu j}. 
\]
\item Use coordinates $(z,\zeta) = (z^1, \ldots, z^n, \zeta^1, \ldots, \zeta^m)$ for the total space of $L$ by the assignment 
\[
(z, \zeta) \mapsto \zeta^\lambda s_\lambda(z).
\]
\item With such coordinates, introduce local vector fields $\nabla_j$ on the total space of $L$ by the rule 
\[
\nabla_j = \frac{\partial}{\partial z^j} - L^\lambda{}_{\mu j} \zeta^\mu \frac{\partial}{\partial \zeta^\lambda}.
\]
Then the list $(\nabla_1, \ldots, \nabla_n, \partial/\partial \zeta^1, \ldots, \partial/\partial \zeta^m)$ is a frame for the tangent bundle $TL$ which is compatible with the decomposition of the tangent bundle into horizontal and vertical subbundles respectively. 
\item Denote by $(dz^1, \ldots, dz^n, \nabla \zeta^1, \ldots, \nabla \zeta^m)$ the dual $1$-forms so that 
\[
\nabla \zeta^\lambda = d\zeta^\lambda + L^{\lambda}{}_{\mu j} \zeta^\mu dz^j
\]
and the list is a frame for the cotangent bundle compatible with the decomposition into horizontal and vertical subbundles. 
\end{itemize} 
\end{definition}

\begin{lemma}
The nonnegative function $t$ admits a description in local coordinates $(z, \zeta)$ given by 
\[
t = h_{\mu \nu} \zeta^\mu \bar{\zeta}^\nu.
\]
Moreover, the $(1,1)$-form $\ddb t$ is given by 
\[
\ddb t = h_{\mu \nu}F^\mu{}_{\lambda j\bar{k}} \zeta^\lambda \bar{\zeta}^\nu dz^j \wedge d\bar{z}^k + h_{\mu \nu} \nabla \zeta^\mu \wedge \overline{\nabla \zeta^\nu}.
\]
It follows that if $\Psi$ is the local potential for $\omega_L$ given by 
\[
\Psi = \Phi \circ \pi + t,
\]
then a local expression for the $(1,1)$-form $\ddb \Psi$ is given by 
\[
\ddb \Psi = (g_{j \bar{k}} + h_{\mu \nu}F^\mu{}_{\lambda j\bar{k}} \zeta^\lambda \bar{\zeta}^\nu) dz^j \wedge d\bar{z}^k + h_{\mu \nu} \nabla \zeta^\mu \wedge \overline{\nabla \zeta^\nu}.
\]
\end{lemma}



\begin{definition}\label{def:poscurv}
We say that $h$ is \textbf{nonnegative} if the inequality 
\[
h_{\mu \nu} F^{\mu}{}_{\lambda j \bar{k}} Z^j \bar{Z}^k\zeta^\lambda \bar{\zeta}^\nu \geqslant 0
\]
holds for all nonzero vectors $Z$ in $\mathbb{C}^n$ and all nonzero vectors $\zeta$ in $\mathbb{C}^m$.

\end{definition}


\begin{proposition}\label{prop:exis}
Let $(L,h)$ be a hermitian vector bundle over a K\"ahler manifold $(M,g)$, and let $\omega_L$ be the K\"ahler form from Lemma \ref{lem:form}. 
\begin{enumerate}
\item[(i)] The form $\omega_L$ is positive definite in a neighborhood of the zero section. 
\item[(ii)] If the curvature of $h$ is nonnegative in the sense of Definition \ref{def:poscurv}, then the form $\omega_L$ is positive definite globally on $L$ and hence defines a K\"ahler form on $L$. 
\end{enumerate}
\end{proposition}


\begin{proposition}\label{prop:u}
Let $u(x)$ be a smooth real-valued function of a single variable and let $E$ denote the subset of the total space of $L$ such that $u \circ t$ is defined. Define a form $\omega_u$ on the subset $E$ by the rule
\[
\omega_u = \pi^* \omega + \sqrt{-1}\ddb (u \circ t).
\]
Then $\omega_u$ is a K\"ahler form on $E$. Moreover, the associated symmetric $2$-tensor is positive definite along the vertical directions if and only if  $u$ satisfies the conditions 
\[
\begin{cases}
u'(x) > 0 \\
u'(x) + x u''(x) > 0
\end{cases}.
\]
The condition $u'(x) > 0$ ceases to be necessary if $L$ is a line bundle and the zero section does not belong to $E$. 
\end{proposition}

\begin{proof}
One computes analogously to the above for $\omega_L$ that a local expression for $\omega_u$ is given by 
\begin{align*}
\omega_u &= \sqrt{-1}(g_{j \bar{k}} + (u' \circ t) h_{\mu \nu}F^\mu{}_{\lambda j\bar{k}} \zeta^\lambda \bar{\zeta}^\nu) dz^j \wedge d\bar{z}^k  \\
&\;\;\; + \sqrt{-1}((u' \circ t)h_{\mu \nu} + (u'' \circ t) h_{\mu \beta} h_{\alpha \nu} \zeta^\alpha \bar{\zeta}^\beta) \nabla \zeta^\mu \wedge \overline{\nabla \zeta^\nu}.
\end{align*}

We first show that the conditions on $u$ are sufficient for the tensor associated to $\omega_u$ to be positive definite along the vertical directions. It suffices to show that for each $\zeta, \eta \in \mathbb{C}^m$ we have 
\[
A(\zeta, \eta) := ((u' \circ t)h_{\mu \nu} + (u'' \circ t) h_{\mu \beta} h_{\alpha \nu} \zeta^\alpha \bar{\zeta}^\beta) \eta^\mu \bar{\eta}^\nu \geqslant 0
\]
with equality if and only if $\eta = 0$. Indeed, we find that
\begin{align*}
A(\zeta, \eta) &= (u' \circ t)  \norm{\eta}^2 + (u'' \circ t) \langle \eta, \zeta \rangle \langle \zeta, \eta \rangle  \\
&= (u' \circ t) \norm{\eta}^2 + (u'' \circ t) |\langle \eta, \zeta \rangle |^2.
\end{align*}
When $\zeta = 0$, the inequality $u'(x) > 0$ ensures that $A(0, \eta) \geqslant 0$ with equality if and only if $\eta = 0$. Otherwise, the condition $u'(x) + x u''(x) > 0$ and the relation $t = \norm{\zeta}^2$ ensures that 
\[
A(\zeta, \eta) \geqslant (u' \circ t) (\norm{\eta}^2 - |\langle \eta, \zeta \rangle|^2 \norm{\zeta}^{-2}).
\]
The term $(u' \circ t)$ is positive from our assumption on $u$, and the second term is nonnegative by the Cauchy-Schwarz inequality. 

Now suppose that $A(\zeta, \eta) \geqslant 0$ with equality if and only if $\eta = 0$. We show that $u$ satisfies the two required conditions.  If we take $\eta = \zeta$, then we find that 
\[
A(\zeta, \zeta) = (u' \circ t) t + (u'' \circ t) t^2 \geqslant 0,
\]
and since $t \geqslant 0$, the second condition on $u$ follows.  For the first condition, we may choose $\eta \ne 0$ such that $\langle \eta, \zeta \rangle = 0$. 

Finally suppose that $L$ is a line bundle and the zero section does not belong to $E$. Then in this case, the condition $u'(x) > 0$ ceases to be necessary because 
\[
A(\zeta, \eta) = |\eta|^2 \{ (u' \circ t) +t (u'' \circ t) \}
\]
so that the second condition is sufficient. 
\end{proof}

\begin{definition}
A lower bound on the curvature of $(L,h)$ is a real number $c$ such that 
\[
h_{\mu \nu} F^\mu{}_{\lambda j \bar{k}} Z^j \bar{Z}^k \zeta^\lambda \bar{\zeta}^\nu \geqslant c h_{\mu \nu} g_{j\bar{k}} Z^j \bar{Z}^k \zeta^\mu \bar{\zeta}^\nu.
\]
\end{definition}

\begin{theorem}
Let $(L,h)$ be a hermitian vector bundle over a K\"ahler manifold $(M,g)$. If the curvature of $(L,h)$ is bounded from below, then there is a K\"ahler metric $g_L$ on the total space of $L$. Moreover, if the metric $g$ on $M$ is complete, then so is the metric $g_L$ on $L$. 
\end{theorem}

\begin{proof}
If the bound $c$ on the curvature satisfies $c \geqslant 0$, then we may appeal to Proposition \ref{prop:exis}. So we may suppose that $c$ is strictly negative and equal to $c = -b$ for some $b > 0$. The idea is to choose a function $u$ such that the form $\omega_u$ from Proposition \ref{prop:u} defines a K\"ahler form whose associated symmetric $2$-tensor is positive definite. In particular, we may choose the function  
\[
u(x) = \frac{2}{3b} \log(a + x) - \frac{1}{3}\log(\log(a + x))
\]
where $a = e^b$. 

One checks by direct computation that 
\begin{align*}
u'(x) = \frac{1}{3(a + x)} \left(\frac{2}{b} - \frac{1}{\log(a + x)}\right) \geqslant \frac{1}{3b(a + x)}
\end{align*}
and also that 
\begin{align*}
xu''(x) + u'(x) &= \frac{1}{3} (a + x)^{-2} \left(\frac{2a}{b} - \frac{a}{\log(a + x)} + \frac{x}{(\log(a + x))^2}\right) \\
&\geqslant \frac{a}{2b(a + x)^2} + \frac{x}{3(a + x)^2 (\log(a + x))^2}.
\end{align*}
Hence the associated two-tensor is positive definite along the vertical directions by the previous lemma. 

To check for positivity along the horizontal directions, we use the fact that the curvature is bounded from below by $-b$ to find that 
\[
(g_{j\bar{k}} + (u' \circ t) h_{\mu \nu} F^{\nu}{}_{\lambda j \bar{k}} \zeta^\lambda \bar{\zeta}^\mu) dz^j \otimes d\bar{z}^k \geqslant (1 - b t (u' \circ t)) g_{j \bar{k}}dz^j \otimes d\bar{z}^k 
\]
and the direct computation that
\[
1 - b x u'(x) = \frac{1}{3} + \frac{1}{3(a + x)} \left(2a + \frac{bx}{\log(a + x)}\right) \geqslant \frac{1}{3}
\]
to conclude that the metric $g_u$ associated to $\omega_u$ is bounded from below by the metric with local expression  
\[
g_u \geqslant \frac{1}{3} g_{j \bar{k}} dz^j \otimes d\bar{z}^k +((u' \circ t)h_{\mu \nu} + (u'' \circ t) h_{\mu \beta} h_{\alpha \nu} \zeta^\alpha \bar{\zeta}^\beta) \nabla \zeta^\mu \otimes \overline{\nabla \zeta^\nu},
\]
which is positive definite along the horizontal directions. 

Along each fiber the metric associated to $\ddb u \circ t$ is complete because the integral 
\[
\int_0^\infty (x u''(x) + u'(x))^{1/2} \: dx
\] 
diverges. 
\end{proof}

\begin{corollary}
There is a complete K\"ahler metric on the complement of a point in projective space $\mathbb{CP}^n$. 
\end{corollary}

\begin{proof}
The complement of a point in projective space $\mathbb{CP}^n$ is biholomorphic to the total space of a line bundle over $\mathbb{CP}^{n-1}$. 
\end{proof}


\section{K\"ahler-Einstein metrics on holomorphic vector bundles}

\begin{definition}
Let $(M,g)$ be a K\"ahler manifold of complex dimension $n$. 
\begin{itemize}
\item The \textbf{Ricci form} of $g$ is the $2$-form with local expression 
\[
\text{Ric}(g) = - \sqrt{-1} \ddb \log \det (g_{j\bar{k}}).
\]
\item We say that $g$ is \textbf{K\"ahler-Einstein} if there is a constant $k_0$ such that 
\[
\text{Ric}(g) = k_0 \omega.
\]
If $g$ is K\"ahler-Einstein, then we have 
\[
\det g_{j \bar{k}} = |\text{hol}|^2 e^{-k_0 \Phi}
\] 
for some unspecified holomorphic function, where $\Phi$ is a local K\"ahler potential for $g$. 
\end{itemize}
\end{definition}

\begin{definition}
Let $(L,h)$ be a hermitian vector bundle over a K\"ahler manifold $(M,g)$. We say that $(L,h)$ is \textbf{Hermitian-Einstein} if there is a constant $\ell$ such that 
\[
F_\nabla = -\sqrt{-1} \ell \omega \otimes \text{Id}_L. 
\]
In local coordinates, this condition is stated as 
\[
F^{\mu}{}_{\nu j\bar{k}} = \ell g_{j \bar{k}}\cdot \delta^\mu{}_{\nu}.
\]
\end{definition}

\begin{lemma}
If $(L,h)$ is a Hermitian-Einstein line bundle with constant $\ell$, then there is a nonzero holomorphic function $\textnormal{hol}$ on the base $M$ such that 
\[
t = |\textnormal{hol}|^2 e^{\ell \Phi} |\zeta|^2
\]
where $\Phi$ is a K\"ahler potential for $g$. 
\end{lemma}

\begin{proof}
Choose a local holomorphic nonvanishing section $s$ of $L$. Then in this case, the curvature reduces to 
\[
F = \ddb \log h
\] 
where $h$ is the function $\norm{s}_{h}^2$. 
The Hermitian-Einstein equation asserts that 
\[
\ddb (\log h - \ell \Phi ) = 0.
\]
It follows that there is a holomorphic function $\text{hol}$ on the base such that 
\[
h = |\text{hol}|^2 e^{\ell \Phi} \delta_{\mu \nu}. 
\]
The claim follows. 
\end{proof}

\begin{lemma}
Let $(M,g)$ be an $(n - 1)$-dimensional K\"ahler-Einstein manifold with Ricci constant equal to $k_0$, and let $(L,h)$ be a rank $m$ Hermitian-Einstein vector bundle with constant equal to $\ell$. For a nonnegative number $x_0$, let $H_{x_0}$ denote the hypersurface in $L$ given by $H_{x_0} = \{ q \in L : t(q) = x_0\}$. Let $u(x)$ be a function defined on an interval $I \subset \mathbb{R}_{\geqslant 0}$ containing $x_0$. Then the form $\omega_u$ determined by $u$ on the neighborhood $t^{-1}(I)$ of $H_{x_0}$ in $L$ is positive definite if and only if $u$ satisfies the conditions 
\begin{align*}
\begin{cases}
1 + \ell x u'(x) > 0 \\
u'(x) + x u''(x) > 0.
\end{cases}
\end{align*}
If these conditions are satisfied in $I$ and if $\ell \ne 0$, then the metric defined by the form $\omega_u$ is K\"ahler-Einstein with Ricci constant equal to $k$ if and only if $u$ satisfies the differential equation 
\[
(1 + \ell x u'(x))^{n-1} (x u''(x) + u'(x)) = c x^{\ell^{-1}(k_0 - k - \ell)}e^{-k u(x)}
\]
where $c$ is a fixed positive constant; if $\ell = 0$, then this forces $k = k_0$ and the differential equation becomes 
\[
\begin{cases}
u(x) = 2k^{-1} \log(1 + c_0^2 k x^c) + c_1 \log x + c_2 & \ell = 0, k = k_0 \ne 0 \\
u(x) = c_0^2 x^c + c_1 \log x + c_2 & k = k_0 = \ell = 0
\end{cases}
\]
where $c, c_0, c_1, c_2$ are real constants with $c_0 \ne 0$. 
\end{lemma}

\begin{proof}
With the given hypotheses, the form $\omega_u$ is given locally by 
\[
\ddb \Psi = (1 + \ell xu'(x)) g_{j \bar{k}} dz^j \wedge d\bar{z}^k + c_0 e^{\ell \Phi}(x u''(x) + u'(x))  \nabla \zeta \wedge \overline{\nabla \zeta}.
\]
Taking the determinant, we find that 
\[
\det g_L = c_0^m (1 + \ell x u'(x))^{n-1} e^{-k_0 \Phi} e^{ \ell \Phi} (x u''(x) + u'(x))^.
\]
The K\"ahler-Einstein condition $\det g_L = |\text{hol}|^2 e^{-k \Psi}$ becomes 
\[
|\text{hol}|^2 e^{-k(\Phi + u(x))} = c_0 (1 + \ell x u'(x))^{n-1} e^{( \ell - k_0) \Phi} (x u''(x) + u'(x))^,
\]
or equivalently 
\[
(1 + \ell x u'(x))^{n-1}(x u''(x) + u'(x)) = |\text{hol}|^2  e^{(k_o - k -  \ell )\Phi -k u(x)}.
\]
If $\ell \ne 0$, the previous lemma implies that this equation is equivalent to 
\[
(1 + \ell x u'(x))^{n-1} (x u''(x) + u'(x)) = |\text{hol}|^2 x^{\ell^{-1}(k_0 - k - \ell)}e^{-k u(x)}.
\]
By an appropriate choice of coordinates, Calabi shows that $|\text{hol}|^2$ may be reduced to a constant. 

Indeed, suppose that $\varphi$ represents the holomorphic function in question. 
\end{proof}

\begin{lemma}
Suppose we are given a hermitian vector bundle $(L,h)$ of rank $m$ over a K\"ahler-Einstein manifold $(M,g)$ with the same hypotheses of the previous lemma. A K\"ahler-Einstein metric on $L$ with Ricci constant equal to $k$ determined by a function $u(x)$ (defined for $0 \leqslant x < x_0$) extends smoothly across the zero section of $L$ if and only if $k = k_0 - m \ell$ and $u'(0) > 0$ for the case $\ell \ne 0$; for the case $\ell = 0$, the function $u(x)$ is determined to be 
\[
\begin{cases}
u(x) = 2 k^{-1} \log(1 + c_0^2 k x) + c_2 & k = k_0 \ne 0 , c_0 \ne 0, \ell = 0 \\
u(x) = c_0^2x + c_2 & k = k_0 = \ell = 0, c_0 \ne 0.
\end{cases}
\]
\end{lemma}


\begin{theorem}
Suppose we are given a hermitian vector bundle $(L,h)$ of rank $m$ over a K\"ahler-Einstein manifold $(M,g)$ and a function $u$ defined in a neighborhood of $0$ determining a K\"ahler-Einstein metric $\omega_u$ on a maximal neighborhood $E \subset L$ of the zero section of $L$. Then the metric determined by $\omega_u$ is a complete K\"ahler-Einstein metric if and only if 
\begin{enumerate}
\item[(i)] $M$ is complete with respect to $g$
\item[(ii)] $k = k_0 - m \ell$
\item[(iii)] $\ell \geqslant 0$ and $k \leqslant 0$. 
\end{enumerate}

\end{theorem}


\section{Extra}


Let $M$ be a complex manifold of dimension $n$ with complex structure $J$. By a hermitian metric on $M$ we mean a Riemannian metric $g$ on $M$ such that $J$ is an orthogonal transformation with respect to $g$ in the sense that $g(JU, JV) = g(U,V)$ for vector fields $U,V$ on $M$. The hermitian metric $g$ determines a $(1,1)$-form $\omega$ by the rule $\omega(U,V) = g(JU,V)$. We say that a hermitian metric $g$ is K\"ahler if the $(1,1)$-form $\omega$ is closed.  

Let $\pi : L \to M$ be a holomorphic vector bundle of rank $m$ over $M$. By a hermitian metric $h$ on $L$ we mean a collection $h_x$ of hermitian metrics on the fibers $L_x = \pi^{-1}(x)$ of $L$ in such a way that the collection $h_x$ varies smoothly in the parameter $x \in M$.

A smooth section of $L$ is a map $s : M \to L$ such that $\pi \circ s = \text{id}_M$. We let $\Omega^0(L)$ denote the vector space of smooth sections. We let $\Omega^k(L)$ denote the vector space $\Omega^k(L) = \Omega^0(L) \otimes \Omega^k(M)$, where $\Omega^k(M)$ is the space of $k$-forms on $M$. 

A smooth section $s : M \to L$ is called holomorphic if $s$ is a holomorphic map of complex manifolds.  

A hermitian metric $h$ on $L$ (together with the holomorphic structure) determines a unique connection $\nabla : \Omega^0(L) \to \Omega^1(L)$ called the Chern connection which satisfies the following two properties
\begin{enumerate}
\item[(i)] $\nabla h = 0$
\item[(ii)] If $s$ is a holomorphic section of $L$, then $\nabla s$ has type $(1,0)$. 
\end{enumerate}
For a smooth section $s$ of $L$, we use the notation $\nabla_j s$ for the local section given by the contraction of $\nabla s$ with the local vector field $\partial/\partial z^j$. 

\begin{notation}
For a smooth frame $\{s_\lambda\}$ of $L$, define local functions $L^{\lambda}_{\mu j}$ by the rule 
\[
\nabla_j s_\mu = L^{\lambda}_{\mu j} s_\lambda .
\]
\end{notation}

\begin{lemma}
If $\{s_\lambda\}$ is a holomorphic frame for $L$, then the functions $L^{\lambda}_{\mu j}$ are given by 
\[
L^{\lambda}_{\mu j} =  h^{\nu \lambda}\partial_j h_{\mu \nu}. 
\]
\end{lemma}

\begin{proof}
Because $\{s_\lambda\}$ is a holomorphic frame, each $\nabla s_\lambda$ has type $(1,0)$ and is given by 
\[
\nabla s_\lambda = (\nabla_j s_\lambda)dz^j.
\]  The fact that the connection $\nabla$ is compatible with $h$ means that 
\begin{align*}
d h_{\mu \nu} &= d(h(s_\mu, s_\nu)) \\
&= h(\nabla s_\mu, s_\nu) + h( s_\mu, \nabla s_\nu) \\
&= L^\lambda_{\mu j} h_{\lambda \nu} dz^j + \overline{L}^{\lambda}_{\nu j} h_{\mu \lambda} d\bar{z}^j.
\end{align*}
Taking the $(1,0)$-part of the above equation gives that 
\[
\partial_j h_{\mu \nu} =  L^\lambda_{\mu j} h_{\lambda \nu}. 
\]
Solving for $L^{\lambda}_{\mu j}$, we find that 
\[
L^{\lambda}_{\mu j} = h^{\nu \lambda}\partial_j h_{\mu \nu},
\]
as desired. 
\end{proof}


\begin{definition}
A section $s$ of $L$ is called horizontal if $\nabla s$ vanishes. 
\end{definition}

\begin{lemma}
Locally we may always find a horizontal section $s$. 
\end{lemma}

\begin{proof}
Let $\{s_\lambda\}$ be a smooth frame for $L$. Any section can be written as 
\[
s = a^\lambda s_\lambda
\]
for some local smooth functions $a^\lambda$ on $M$. The section $\nabla_j s$ is given by 
\[
\nabla_j s = (\partial_j a^\lambda  + a^\mu L^\lambda_{\mu j})s_\lambda .
\]
It suffices therefore to solve the system of differential equations
\[
\partial_j a^\lambda  + a^\mu L^\lambda_{\mu j} = 0
\]
for each $j,\lambda$. 
\end{proof}



\begin{lemma}
If $s$ and $t$ are two horizontal sections of $L$ passing through a point $y = s(x) = t(x)$ in the total space of $L$, then $T_ys(M) = T_y t(M)$. 
\end{lemma}

\begin{proof}
Let $\{s_\lambda\}$ be a smooth frame for $L$ near $y$. We may write 
\begin{align*}
s &= a^\lambda s_\lambda \\
t &= b^\lambda s_\lambda
\end{align*} 
for some local smooth functions $a^\lambda, b^\lambda$ near $x$ on $M$. 
The tangent space $T_ys(M)$ is spanned by the collection of tangent vectors 
\[
\frac{\partial}{\partial z^j} + (\partial_j a^\lambda) \frac{\partial}{\partial \zeta^\lambda} \hspace{5mm} 1 \leqslant j \leqslant n.
\]
Since $s$ is horizontal, we have $\partial_j a^\lambda = -a^\mu L^\lambda_{\mu j}$, and it follows that $T_ys(M)$ is spanned by 
\[
\frac{\partial}{\partial z^j} - a^\mu L^\lambda_{\mu j} \frac{\partial}{\partial \zeta^\lambda} \hspace{5mm} 1 \leqslant j \leqslant n.
\]
Similarly the space $T_yt(M)$ is spanned by 
\[
\frac{\partial}{\partial z^j} - b^\mu L^\lambda_{\mu j} \frac{\partial}{\partial \zeta^\lambda} \hspace{5mm} 1 \leqslant j \leqslant n.
\]
At the point $x$, the quantities $a^\mu L^\lambda_{\mu j}$ and $b^\mu L^\lambda_{\mu j}$ are equal, so the claim is proved. 
\end{proof}

For a point $y$ in the total space of $L$, define the horizontal subspace at $y$ to be the subspace of $T_yL$ given by the tangent space at $y$ to a horizontal section $s$ of $L$ passing through $y$. This is well-defined by the previous lemma. In this way, we obtain a collection of horizontal subspaces $H_y \subset T_yL$. Moreover, if we define the vertical subspace $V_y = \ker d\pi_y$, then we find that the tangent space $T_yL$ enjoys the decomposition $T_yL = H_y \oplus V_y$. Moreover, this decomposition varies smoothly in the parameter $y$ so that the tangent bundle $TL$ decomposes into the sum of horizontal and vertical bundles.  

\begin{notation}
Let $\{s_\lambda\}$ be a local holomorphic frame for $L$. Let $(z, \zeta) = (z^1, \ldots, z^n, \zeta^1, \ldots, \zeta^m)$ be local coordinates for the total space of $L$ with respect to this frame in the sense that the assignment 
\[
(z, \zeta) \mapsto \zeta^\lambda s_\lambda(z)
\]
describes a coordinate chart on $L$. 
Define local vector fields $\nabla_j$ on the total space of $L$ by the rule 
\[
\nabla_j = \frac{\partial}{\partial z^j} - L^{\lambda}_{\mu j} \zeta^\mu \frac{\partial}{\partial \zeta^\lambda}
\]
so that $(\nabla_1, \ldots, \nabla_n, \frac{\partial}{\partial \zeta^1}, \ldots, \frac{\partial}{\partial \zeta^m})$ is a local frame for the tangent bundle to $L$. Moreover, this local frame is compatible with the decomposition of the bundle $TL$ into horizontal and vertical subbundles by the above discussion. 
Denote $1$-forms by the notation 
\[
\nabla \zeta^\lambda = d\zeta^\lambda + L^{\lambda}_{\mu j} \zeta^\mu dz^j 
\]
so that $(dz^1, \ldots, dz^n, \nabla \zeta^1, \ldots, \nabla \zeta^m)$ is a local frame of the cotangent bundle of $L$, which is in fact dual to $(\nabla_1, \ldots, \nabla_n, \frac{\partial}{\partial \zeta^1}, \ldots, \frac{\partial}{\partial \zeta^m})$. 
\end{notation}



For the Chern connection, its curvature $F_\nabla = \nabla^2$ has type $(1,1)$. It follows that for a smooth section $s$ we have  
\begin{align*}
\nabla^2 s &= \nabla (\nabla_j s dz^j + \nabla_{\bar{k}}s d\bar{z}^k) \\
&= (\nabla_{\bar{k}} \nabla_j s )dz^j \wedge d\bar{z}^k + (\nabla_j \nabla_{\bar{k}} s) d\bar{z}^k \wedge dz^j \\
&= (\nabla_{\bar{k}}\nabla_j - \nabla_{j}\nabla_{\bar{k}})s \: dz^j \wedge d\bar{z}^k. 
\end{align*}

\begin{notation}
For a smooth frame $\{s_\lambda\}$ for $L$, define local functions $F^{\lambda}_{\mu j \bar{k}}$ by the rule 
\[
F_{\nabla} s_{\mu} = F^{\lambda}_{\mu j \bar{k}} s_\lambda dz^j \wedge d\bar{z}^k.
\]
It follows from the above expression that the functions $F^{\lambda}_{\mu j\bar{k}}$ satisfy 
\[
 (\nabla_{\bar{k}}\nabla_j - \nabla_{j}\nabla_{\bar{k}})s_\mu = F^{\lambda}_{\mu j\bar{k}} s_{\lambda} 
\]
\end{notation}

\begin{lemma}
For a holomorphic frame $\{s_\lambda\}$ for $L$, the functions $F^{\lambda}_{\mu j \bar{k}}$ are given by 
\[
F^{\lambda}_{\mu j \bar{k}} = \partial_{\bar{k}} L^{\lambda}_{\mu j}.
\]
\end{lemma}

\begin{proof}
For a holomorphic frame $\{s_\lambda\}$, we have $\nabla_{\bar{k}} s_\lambda =0$ so that   
\begin{align*}
 (\nabla_{\bar{k}}\nabla_j - \nabla_{j}\nabla_{\bar{k}}) s_\mu &= \nabla_{\bar{k}} \nabla_j s_\mu \\
 &= \nabla_{\bar{k}} (L^{\lambda}_{\mu j} s_\lambda) \\
&= \partial_{\bar{k}} L^{\lambda}_{\mu j} s_\lambda.
\end{align*}
This completes the proof. 
\end{proof}



\end{document}
