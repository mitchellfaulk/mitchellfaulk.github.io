\documentclass{article}

\usepackage{amsmath, amssymb, xypic, color}
\usepackage[margin=1in]{geometry}

\begin{document}
\begin{center}
MATH S1202: Calculus IV \\ 
Quiz 3 \\
June 7, 2018
\end{center}


\noindent \textbf{1.} Compute the gradient $\nabla f(3,4)$ where 
\[
f(x,y) = \sqrt{x^2 + y^2}.
\]

\medskip

{\color{blue}
\noindent \emph{Solution.} We compute that 
\[
\nabla f(x,y) = \frac{1}{f} \langle x,y\rangle.
\]
We have $f(3,4) = 5$ and hence 
\[
\nabla f(3,4) = \langle \tfrac{3}{5}, \tfrac{4}{5} \rangle.
\]
}

\vspace{3mm}

\noindent \textbf{2.} Compute the area enclosed by the ellipse 
\[
\frac{x^2}{4} + \frac{y^2}{25} = 1
\]
using ideas from Chapter 15. (Hint: There are several ways of doing this problem, but perhaps the easiest is to use the change of variables $x = 2 r \cos(\theta)$ and $y = 5 r \sin(\theta)$.)


\medskip

{\color{blue}
\noindent \emph{Solution.} Using the coordinates 
\begin{align*}
x &= 2r \cos\theta \\
y &= 5r \sin\theta,
\end{align*}
the region $R$ in the $xy$-plane enclosed by the ellipse is mapped to by the region $S$ in the $(r,\theta)$-plane given by 
\[
S = \begin{cases}
0 \leqslant \theta \leqslant 2\pi \\
0 \leqslant r \leqslant 1.
\end{cases}
\]
The Jacobian is given by 
\[
\frac{\partial(x,y)}{\partial(r, \theta)} = 10r.
\]
It follows that the area of $R$ is given by 
\begin{align*}
\text{Area}(R) &= \iint_R dA \\
&= \int_0^{2\pi}\int_0^1 10r \: dr \: d\theta \\
&= 10 \pi.
\end{align*}
}

\vspace{3mm}

\noindent \textbf{3.} Let $C$ be the line segment from $(0,0)$ to $(1,1)$ in the $xy$-plane. Compute the line integral $\int_C xy \: ds$. 


\medskip

{\color{blue}
\noindent \emph{Solution.} A parametrization of $C$ is given by $\mathbf{r}(t) = (t,t)$ for $0 \leqslant t \leqslant 1$. We compute that 
\[
|\mathbf{r}'(t)| = \sqrt{2}.
\]
It follows that 
\begin{align*}
\int_C xy \: ds &= \int_0^1 t \cdot t \cdot \sqrt{2} \: dt \\
&= \sqrt{2}\int_0^1 t^2 \: dt \\
&= \frac{\sqrt{2}}{3}.
\end{align*}
}

\end{document}