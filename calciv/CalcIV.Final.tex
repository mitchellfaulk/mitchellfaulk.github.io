\documentclass{article}

\usepackage[margin=1in]{geometry}
\usepackage{amsmath, amssymb, fancyhdr, setspace, mathtools}

%Commands
\newcommand{\class}{MATH S1202: Calculus IV}
\newcommand{\term}{Summer 2018}
\newcommand{\examnum}{Final Exam}
\newcommand{\examdate}{June 28, 2018}
\newcommand{\timelimit}{95 minutes}

%Formatting
\parindent 0ex
 
\pagestyle{fancy}
\fancyhf{}
\rhead{\examdate}
\lhead{\class}
\chead{\examnum}
\rfoot{Page \thepage}


\begin{document}
\begin{spacing}{2}
\thispagestyle{empty}
\begin{flushleft}
\begin{tabular}{p{2.8in} r l}
\textbf{\class} & \textbf{Name:} & \makebox[2.8in]{\hrulefill}\\
\textbf{\term} &&\\
\textbf{\examnum} &&\\
\textbf{\examdate} &&\\
\textbf{Time Limit: \timelimit} & 
\end{tabular}\\
\end{flushleft}
\rule[1ex]{\textwidth}{.1pt}

\vspace{5mm}

\begin{center}
\begin{tabular}{|l|l|l|} \hline
Question & Points & Score \\  \hline
1 & 10  & \\ \hline
2 & 20  & \\ \hline 
3 & 30 & \\ \hline 
4 & 40  & \\ \hline 
Total & 100 &  \\ \hline
\end{tabular}
\end{center}

\newpage

\noindent \textbf{1. [10 points]} Let $C$ be a circle of radius $R$ centered about the point $(a,b)$ in the $xy$-plane. If $f$ is the function $f(x,y) = x+y+1$, compute the line integral $\int_C f \: ds$. 
 
\newpage 

\noindent \textbf{2. [20 points]} Let $\mathbf{G}$ denote the vector field $\mathbf{G}(x,y,z) = (2(x+y), 2(x + y) + e^z, y e^z)$ defined on $\mathbb{R}^3$. 

\noindent \textbf{(a) [10 points]} If $\mathbf{G}$ is conservative, find a function $g$ such that $\nabla g = \mathbf{G}$. If $\mathbf{G}$ is not conservative, explain why not.  

\newpage 


\noindent \textbf{(b) [10 points]} Compute the work done by $\mathbf{G}$ on a particle moving along $\mathbf{r}(t) = (\sin(t), \cos(t), t)$ for $0 \leqslant t \leqslant 2\pi$. 

\newpage 




\noindent \textbf{3. [30 points]} Let $E$ be the solid region under the paraboloid $z = 1 - x^2 - y^2$ lying in the first octant.  Let $S$ denote the boundary of $E$ with the induced orientation. 

\medskip

\noindent \textbf{(a) [10 points]} Compute the flux of the vector field $\mathbf{H}(x,y,z) = (x,y,z -x^2 - y^2)$ over $S$. 

\newpage 



\noindent \textbf{(b) [10 points]} Let $S_1$ denote the portion of the boundary of $E$ lying along the paraboloid $z = 1 - x^2 - y^2$ with induced orientation. Compute the flux of $\mathbf{H}(x,y,z) = (x,y,z - x^2 - y^2)$ over $S_1$. 

\newpage 

\noindent \textbf{(c) [10 points]} Let $C_1$ denote the boundary of $S_1$ with the corresponding induced orientation. (Note that $C_1$ consists of 3 arcs.) If $\bf{H}$ is the vector field $\mathbf{H}(x,y,z) = (x,y,z - x^2 - y^2)$, compute the work done by $\mathbf{H}$ on a particle moving along $C_1$. 


\newpage

\noindent \textbf{4. [40 points]} Let $S_2$ denote the portion of the cylinder $x^2 + y^2 - 4y = 0$ lying between the planes $z = 0$ and $z = y$. 

\noindent \textbf{(a) [10 points]} Compute the area of $S_2$. 

\newpage 

\noindent \textbf{(b) [10 points]} If a thin sheet occupies the surface $S_2$ with constant density $\rho(x,y,z) = c$, then compute the $x$-coordinate $\bar{x}$ of the center of mass. 

\newpage 

\noindent \textbf{(c) [10 points]} Find an equation of the tangent plane to $S_2$ at the point $(0,4,1)$. 

\newpage

\noindent \textbf{(d) [10 points]} If $\mathbf{F}$ is the vector field $\mathbf{F}(x,y,z) = (2yz - 2z, 0 , xy)$, then compute the flux  $\iint_{S_2} \text{curl} \: \mathbf{F} \cdot d\mathbf{S}$, where $S_2$ is oriented so that the unit normal vector points \emph{away} from the inside of the cylinder $x^2 + y^2 - 4y = 0$. 
\end{spacing}
\end{document}