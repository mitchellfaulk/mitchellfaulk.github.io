\documentclass{article}

\usepackage{amsmath, amssymb, xypic}
\usepackage[margin=1in]{geometry}

\begin{document}
\begin{center}
MATH S1202: Calculus IV \\ 
Quiz 5 \\
June 21, 2018
\end{center}

\noindent 

\vspace{3mm}


\noindent \textbf{1.} Let $E$ be the solid region above the $xy$-plane, inside the cylinder $x^2 + z^2 = 9$, and bounded by the planes $y = -4$ and $y = 4$.  Let $S_1$ be the portion of the boundary of $E$ lying along the cylinder $x^2 + z^2 = 9$.

\begin{enumerate}
\item[(a)] Use cylindrical coordinates about the $y$-axis with fixed radius $r = 3$ to give a parametrization $\mathbf{r}(\theta, y)$ of $S_1$. In particular, specify the domain $D$ of your parametrization $\mathbf{r}(\theta, y)$. 
\item[(b)] Find the area of $S_1$ using (a). 
\item[(c)]  If a thin sheet occupies $S_1$ with density $\rho(x,y,z) = 1$, find the $z$-coordinate $\bar{z}$ of the center of mass $(\bar{x}, \bar{y}, \bar{z})$. 
\end{enumerate} .


\vspace{3mm}

\noindent \textbf{2.} Let $S_1$ be as above and equip $S_1$ with the induced orientation as a portion of the boundary of the solid region $E$. Let $\mathbf{F}$ be the vector field $\mathbf{F}(x,y,z) = (x, x+y, z)$. 
\begin{enumerate}
\item[(a)] Compute the flux of $\mathbf{F}$ across $S_1$. (Hint: Pay attention to orientation. In particular, is the parametrization $\mathbf{r}(\theta, y)$ of $S_1$ you gave in Problem 1 compatible with the orientation specified on $S_1$? If not, you may need a minus sign somewhere.) 
\item[(b)] Let $C_1$ be the boundary curve of $S_1$ with induced orientation. Compute the work done by $\mathbf{F}$ on a particle traveling along $C_1$. (Hint: Again, pay attention to the orientation of $C_1$.)
\end{enumerate}


\end{document}

