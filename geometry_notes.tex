\documentclass{amsart}

\usepackage{amsmath, amsthm, amssymb, xypic, hyperref}

\numberwithin{equation}{section}

% mathbb

\newcommand{\bC}{\mathbb{C}}
\newcommand{\bR}{\mathbb{R}}
\newcommand{\bZ}{\mathbb{Z}}

\newcommand{\ep}{\epsilon}


% mathrm

\newcommand{\Ad}{\mathrm{Ad}}
\newcommand{\ad}{\mathrm{ad}}
\newcommand{\End}{\mathrm{End}}
\newcommand{\can}{\mathrm{can}}

% mathfrak

\newcommand{\fg}{\mathfrak{g}}
\newcommand{\fgl}{\mathfrak{gl}}
\newcommand{\fX}{\mathfrak{X}}

\theoremstyle{definition}
\newtheorem{definition}{Definition} [section]
\newtheorem{exercise}[definition]{Exercise}
\newtheorem{example}[definition]{Example}
\newtheorem{remark}[definition]{Remark}
\newtheorem{note}[definition]{Note}
\newtheorem{notation}[definition]{Notation}

\theoremstyle{theorem}
\newtheorem{proposition}[definition]{Proposition}
\newtheorem{theorem}[definition]{Theorem}
\newtheorem{corollary}[definition]{Corollary}
\newtheorem{lemma}[definition]{Lemma}





\begin{document}

\input{../G4402F15_190905.synctex.gz}

\begin{center}
{\large \bf Mathematics G4402. Modern Geometry I, Fall 2015\\ Lecture Notes}
\end{center}

\bigskip

\noindent
These notes are prepared by the instructor Melissa Liu and the teaching assistant Mitchell Faulk.\\
 {\small (Last updated by Melissa Liu on \today. Many thanks 
to Myeonhu Kim for correcting numerous errors in an earlier version.)}

 
\tableofcontents
 
\section{Wednesday, September 9, 2015}


\noindent
{\bf \large Abstract manifolds}

\begin{definition}[topological manifolds] 
A {\em topological $n$-manifold} (or a {\em topological manifold of dimension $n$}) is a topological space $M$ which is 
\underline{locally homeomorphic to $\mathbb{R}^n$}, that is, for each $p \in M$, there is an open neighborhood $U$ of $p$ in $M$ 
and a homeomorphism $\phi$ from $U$ to an open set $\Omega$ in $\bR^n$. 
We call such a pair $(U,\phi)$ a {\em chart} (or {\em coordinate system}) for $M$ around $p$, and $U$ is called a \input{../Slides.snm}

{\em coordinate neighborhood} at $p$. 
\end{definition}

\begin{remark}[{cf. \cite[page 6]{Bo}, \cite[page 29-30]{dC}}] 
Some textbooks  require that the topology of $M$ satisfy the following \emph{additional} two properties. 
\begin{enumerate}
\item[(i)] The topology of $M$ is Hausdorff.
Recall that, a topologial space $M$ is Hausdorff if for any two distinct points
$p$ and $q$ in $M$, there exist open sets $U$ and $V$ in $M$ such that $p\in U$, $q\in V$, and $U\cap V$ is empty. 
\item[(ii)] The topology of $M$ has a countable basis of open sets.\\ 
Recall that a collection $\mathcal{B}$ of open subsets in a topological space $M$ is a basis of open sets of $M$
if every open subset of $M$ can be written as a union of elements of $\mathcal{B}$. 
\end{enumerate}
\end{remark}

\begin{example}[a non-Hausdorff manifold]
Let $M = \mathbb{R} \sqcup \{p\}$ be the disjoint union of the real line $\bR$ and a point $p$. 
Define a topology on $M$ by the topology generated by open subsets of $\bR$ and sets
of the form $(U\setminus\{0\})\cup \{p\}$, where $U$ is an open neighborhood of 
$0$ in $\bR$.   Note that any neighborhoods of $p$ and   $0$ intersect, so $M$ is a non-Hausdorff topological space.

For any $q\in \bR = M\setminus\{p\}$, $\bR\subset M$ is an open neighborhood of $q$ in $M$, and 
the identity map $\bR\to \bR$ is a homemorphism from $\bR$ to $\bR$. The
set $U= (\bR\setminus \{0\})\cup \{ p\}$ is an open neighbhorhood of $p$ in $M$, and\includegraphics[]{../Lecture01.pdf}

the map $\phi: U\to \bR$ given by $\phi(x)=x$ for $x\in \bR\setminus \{0\}$ and 
$\phi(p)=0$ is a homeomorphism. Therefore, $M$ is a topological 1-manifold.
\end{example}

\begin{example}
An example of a topological manifold which does not have a countable basis is the \emph{long line}. A proper discussion of this manifold would be quite 
lengthy and would require a digression on set theory, so we choose not to discuss this example further here. 
\end{example}

\begin{definition}[atlas]
An {\em atlas} of a topological $n$-manifold $M$ is a collection $\{(U_\alpha,\phi_\alpha) : \alpha \in I\}$ of charts such that the collection 
$\{U_\alpha : \alpha \in I\}$ is an open cover of $M$. The maps $\phi_\beta \circ \phi_\alpha^{-1} : \phi_\alpha(U_\alpha \cap U_\beta) \to \phi_\beta(U_\alpha \cap U_\beta)$ are called {\em transition functions}. 
\end{definition}

\begin{remark}
\begin{itemize}
\item $I$ is some index set, which can be finite, countably infinite, or uncountably infinite. 
\item If follows from the definitions that the transition functions are homeomorphisms. 
\item If $M$ has a countable atlas, then $M$ has a countable basis of open sets. 
\end{itemize}
\end{remark}

\begin{definition}[$C^k$ atlas]
Let $k$ be a positive integer or $\infty$. 
A {\em $C^k$-atlas} for an $n$-manifold $M$ is an atlas $\Phi = \{(U_\alpha, \phi_\alpha):\alpha\in I\}$ such that all transition functions are 
$C^k$ diffeomorphisms of open subsets of $\mathbb{R}^n$. 
\end{definition}

\begin{definition}
We say that two $C^k$-atlases $\Phi = \{(U_\alpha,\phi_\alpha):\alpha\in I \}$ and $\Psi = \{(\psi_\beta, V_\beta): \beta\in J\}$ for a 
topological manifold $M$ are {\em equivalent} if their union is a $C^k$-atlas. 
A {\em $C^k$ differentiable structure} on a topological manifold $M$ is a choice of an equivalence class of $C^k$-atlases. 
A {\em $C^k$ manifold} is a topological manifold equipped with a $C^k$-structure.

A $C^\infty$ differentiable structure is also called a {\em smooth structure}, and a $C^\infty$ manifold
is also called a {\em smooth manifold}. 
\end{definition}

\begin{example} Let $k$ be a positive integer.
We endow $M = \bR$ with two non-equivalent $C^k$atlases. 
For the first atlas, take $\Phi = \{ (\bR, \phi)\}$ where $\phi(x) = x$. 
For the second atlas, take $\Psi = \{ (\bR, \psi) \}$ where $\psi(x) = x^3$. 
Let $k$ be any postive integer, or $\infty$. Both $\Phi$ and $\Psi$ are $C^k$-atlases
since all of their transition functions (which consist of simply the identity map) are $C^k$-differentiable. 
However, their union $\Phi \cup \Psi$ is not a $C^k$-atlas, since the transition function 
$\phi \circ \psi^{-1}(x) = x^{1/3}$ is not $C^k$-differentiable. 
\end{example}

\begin{example}[{The real projective space $P_n(\bR)$}] \label{PR}$ $\\
1. As a {\em set},  
 $P_n(\bR)$ is the set of one-dimensional $\bR$-linear subspace of $\bR^{n+1}$.
 
\smallskip

\noindent
2. {\em Topology}.\\
Define a surjective map $\pi : \bR^{n+1}\setminus\{0\} \to P_n(\bR)$ by sending a nonzero vector 
in $\bR^{n+1}$ to the one-dimensional $\bR$-linear
subspace of $\bR^{n+1}$ spanned by that vector. 
For any nonzero vector $x=(x_1,\ldots,x_{n+1})$ in $\bR^{n+1}$ we let $[x_1,\ldots, x_{n+1}]$ denote its image in $P_n(\bR)$. 
Note that $[x_1,\ldots, x_{n+1}] = [y_1,\ldots, y_{n+1}]$ if and only if 
$(y_1,\ldots, y_{n+1})=\lambda (x_1,\ldots, x_{n+1})$ for some nonzero $\lambda\in \bR$.  
Equip the set $P_n(\bR)$ with the quotient topology determined by the map $\pi$.  This means that a subset $U$ 
of $P_n(\bR)$ is open if and only if $\pi^{-1}(U)$ is open in $\bR^{n+1} \setminus \{0\}$. 

Let $S^n=\{ (x_1,\ldots, x_{n+1})\in \bR^{n+1}: \sum_{i=1}^{n+1}x_i^2=1\}\subset 
\bR^{n+1}$
be the unit sphere with center at the origin, equipped with the subset topology. Then
$\pi|_{S^n}:S^n\to P_n(\bR)$ is a covering map of degree 2. The quotient
topology determined by $\pi|_{S^n}: S^n\to P_n(\bR)$  
agrees with the quotient  topology determined by $\pi:\bR^{n+1}\setminus\{0\}\to P_n(\bR)$. 
It is easy to see that the quotient topology determined by $\pi|_{S^n}$ is compact
and Hausdorff. 

\smallskip

\noindent
3. {\em Atlas}.\\
For each positive integer $i$ satisfying $1 \le i \le n+1$, let $U_i$ denote the subset of $P_n(\bR)$ given by 
\[
U_i = \{[x_1, \ldots, x_{n+1}]\in P_n(\bR) : x_i \ne 0 \}.
\]
Note that $U_i$ is an open subset of $P_n(\bR)$ since the set $\pi^{-1}(U_i)$ is open in $\mathbb{R}^{n+1}\setminus\{0\}$. Also note that the collection $\{U_i : 1 \le i \le n+1\}$ forms an open cover of $P_n(\bR)$. \\
Let $\widetilde{\phi}_i : \pi^{-1}(U_i) \to \bR^n$ denote the map given by 
\[
\widetilde{\phi}_i(x_1, \ldots, x_{n+1}) = \left(\frac{x_1}{x_i}, \ldots, \frac{x_{i-1}}{x_i}, \frac{x_{i+1}}{x_i}, \ldots, \frac{x_{n+1}}{x_i}\right).
\] 
Note that $\widetilde{\phi}_i$ satisfies $\widetilde{\phi}_i(\lambda x) = \widetilde{\phi}_i(x)$ for each $x \in \pi^{-1}(U_i)$ and each scalar $\lambda \in \mathbb{R}$. It follows that $\widetilde{\phi}_i$ induces a well-defined map $\phi_i : U_i \to \mathbb{R}^n$ described by $\widetilde{\phi}_i = \phi_i \circ \pi$. Since $\widetilde{\phi}_i$ is continuous, we see that $\phi_i$ is continuous as well. The map
$\phi_i^{-1}: \bR^n\to U_i$  given by 
$$
\phi_i^{-1}(x_1, \ldots, x_n) = [x_1, \ldots, x_{i-1}, 1, x_{i}, \ldots, x_n]
$$
is the inverse of $\phi_i: U_i\to \bR^n$. 
The map $\phi_i^{-1}$ is also continuous since it can be written as the composition $\phi_i^{-1} = \pi \circ s_i$ where $s_i : \bR^n \to \mathbb{R}^{n+1}\setminus  \{0\}$ is the continuous map given by 
\[
s_i(x_1, \ldots, x_n) = (x_1, \ldots, x_{i-1}, 1, x_i, \ldots, x_n).
\]
It follows that $\phi_i : U_i \to \mathbb{R}^n$ is a homeomorphism. \\
Therefore the topogical space $P_n(\bR)$ is a topological $n$-manifold,
and $\Phi=\{ (U_i,\phi_i): i=1,\ldots,n+1\}$ is an atalas on $P_n(\bR)$.

\smallskip

\noindent
4. {\em Transition functions}.
$$
\phi_2\circ \phi_1^{-1}(y_1,\ldots, y_n)
=\phi_1([1,y_1,\ldots, y_n])
=(\frac{1}{y_1},\frac{y_2}{y_1},\ldots, \frac{y_n}{y_1})
$$
$\phi_2\circ \phi_1^{-1}:
\phi_1(U_1\cap U_2)= (\bR\setminus\{0\})\times \bR^{n-1}
\to \phi_2(U_1\cap U_2)= (\bR\setminus\{0\})\times \bR^{n-1}$
is a $C^\infty$ diffeomorphism. \\
The general case $\phi_j\circ \phi_i^{-1}$ ($i\neq j$) is similar. \\
Therefore $\Phi=\{(U_i,\phi_i):i=1,\ldots,n+1\}$ is 
a $C^\infty$ atlas on $P_n(\bR^n)$, an defines a $C^\infty$ differentible
structure on $P_n(\bR^n)$. $(P^n(\bR),\Phi)$ is a $C^\infty$ $n$-manifold.
\end{example}


\begin{remark}
Note that the transition functions $\phi_j\circ \phi_i^{-1}$ are real
analytic ($C^\omega$), so $\Phi$ is indeed a real analytic atlas, and
$(P^n(\bR),\Phi)$ is a real analytic manifold of dimension $n$.
\end{remark}

\begin{remark}
Replacing $\bR$ by $\bC$ in Example \ref{PR}, we obtain the
definition of the $n$-dimensional complex projective space $P_n(\bC)$, equipped
with the quotient topology determined by $\pi:\bC^{n+1}-\{0\}\to P_n(\bC)$.
$P_n(\bC)$ is locally homeomorphic to $\bC^n=\bR^{2n}$, so it is 
a topological $2n$-manifold. $\Phi=\{(U_i,\phi_i): i=1,\ldots, n+1\}$,
where $\phi_i:U_i\to \bC^n=\bR^{2n}$, is a $C^\infty$ atlas on $P_n(\bC)$,
and $(P_n(\bC), \Phi)$ is a $C^\infty$ $2n$-manifold. 

The transition functions $\phi_j\circ \phi_i^{-1}$ are indeed complex
analytic, so $\Phi$ defines a complex struture on $P_n(\bC)$,
and $(P_n(\bC),\Phi)$ is a complex manifold of dimensiona $n$. (cf. 
Phong's class "Complex Analysis and Riemann Surfaces")
\end{remark}




\section{Monday, September 14, 2015}

\noindent
{\bf \large $C^k$-differentiable maps}


\begin{definition}
Let $M$ and $N$ be $C^l$-manifolds of dimension $m$ and $n$ respectively. A continuous map $f : M \to N$ is called 
{\em $C^k$-differentiable} for some $k \le l$ if for any $p \in M$, there is a coordinate chart $(U,\phi)$ around $p$ in some atlas representing the $C^l$-structure on $M$ and a coordinate chart $(V,\psi)$ around $f(p)$ in some atlas representing the $C^l$-structure on $N$ such that 
\begin{itemize}
\item $f(U) \subset V$
\item the composition $g = \psi \circ f \circ \phi^{-1} : \phi(U) \to \psi(V)$ is $C^k$-differentiable. 
\end{itemize}
\end{definition}

\begin{remark}
There are two subtleties to this definition. 
\begin{itemize}
\item The definition seems to depend on choices of coordinate charts in fixed atlases for $M$ and $N$ respectively. Indeed, one might worry that while the $g = \psi \circ f \circ \phi^{-1}$ is $C^k$-differentiable, there is another such composition $\widetilde{g} = \widetilde{\psi} \circ f \circ \widetilde{\phi}^{-1}$ that is not. However, because the transition maps in a $C^l$ atlas are $C^l$-differentiable and $k \le l$, the chain rule forbids this from happening. It follows that the definition does not depend on the choices of coordinate charts in fixed atlas for $M$ and $N$. 
\item One might worry, nevertheless, that the definition depends on the choice of atlases representing the given $C^l$-structures. But again, because of the equivalence condition we placed on $C^l$-atlases, we see that the chain rule guarantees that the definition does not depend on the choice of atlases representing the given $C^l$-structures. 
\end{itemize}
These subtleties will appear in forthcoming definitions as well, but we will neglect to remark on them and leave the details to the interested reader. 
\end{remark}

\begin{definition}
A $C^\infty$-differentiable map $f : M \to N$ is also called a {\em smooth map}. 
\end{definition}

\begin{example}
As an example, let us view $\bR^{n+1} \setminus \{0\}$ as a smooth manifold where the $C^\infty$-structure is the one determined by the atlas consisting only of the identity map, and let us equip $P_n(\bR)$ with the $C^\infty$-structure described in Example \ref{PR}. Then the natural map 
$\pi : \mathbb{R}^{n+1}\setminus\{0\} \to P_n(\mathbb{R})$ is a smooth map. This can be seen because the compositions 
$$
g_i := \phi_i \circ \pi \circ \text{id}^{-1}: \pi^{-1}(U_i) \to \mathbb{R}^n,
\quad (x_1,\ldots,x_{n+1})\mapsto (\frac{x_1}{x_i},\ldots, \frac{x_{i-1}}{x_i},\frac{x_{i+1}}{x_i},\ldots, 
\frac{x_{n+1}}{x_i}) 
$$
are smooth at each point of their domains. 
\end{example}

\begin{remark}
If $M$ is a $C^l$ manifold and $U$ is an open subset of $M$, then the $C^l$-differentiable structure on $M$ 
restricts to a $C^l$-differentiable structure on $U$. 
\end{remark}

\begin{definition}
Let $M, N$ be smooth manifolds. We say that $f : M \to N$ is a {\em diffeomorphism} if 
\begin{itemize}
\item $f$ is a homeomorphism, and 
\item $f$ and $f^{-1}$ are smooth.
\end{itemize}
We say that $f$ is a {\em local diffeomorphism} at $p \in M$ if there is an open neighborhood $U$ 
of $p$ in $M$ and an open neighborhood $V$ of $f(p)$ in $N$ such that 
$f(U) = V$ and  $f|_{U} : U \to V$ is a diffeomorphism. 
\end{definition}

\begin{example}
Let $\phi : \bR \to \bR$ be the map $\phi(x) = x$ and let $\psi : \bR \to \bR$ be the map $\psi(x) = x^3$. We have seen that $\Phi=\{ (\bR,\phi)\}$ and $\Psi=\{(\bR,\psi)\}$ are two $C^\infty$ atlases on $\bR$ which are not equivalent.
Let $f : (\bR, \Phi) \to (\bR, \Psi)$ denote the map $f(x) = x^{1/3}$. 
Then $f$ is a diffeomorphism since $\psi\circ f\circ \phi^{-1}: \phi(\bR)=\bR \to \Psi(\bR)=\bR$ is the identity map.
\end{example}

\begin{definition}
Given an open subset $U$ of $\bR^m$ and a smooth map $f : U \to \bR^n$, 
we say that $f$ is a {\em submersion} (resp. {\em immersion}) at $x \in U$ if the differential 
$df_x : \bR^m \to \bR^n$ is a surjective (resp. injective) linear map.
 \end{definition}

\begin{example}[Canonical submersion]
Let $m$ and $n$ be positive integers satisfying $m \ge n$. Consider the map $\pi : \mathbb{R}^m \to \mathbb{R}^n$ given by 
\[
\pi(x_1, \ldots, x_m) = (x_1, \ldots, x_n).
\]
Since $\pi$ is a linear map, we see that $d \pi_x = \pi$ for each $x \in \bR^m$. 
It follows that $\pi$ is a submersion at any $x\in \bR^m$; $\pi$ is called the {\em canonical submersion}.
\end{example}

\begin{example}[Canonical immersion]
Let $m$ and $n$ be positive integers satisfying $m \leq n$. Consider the map 
$i : \bR^m \to \bR^n$ given by 
\[
i(x_1, \ldots, x_m) = (x_1, \ldots, x_m, 0, \ldots, 0).
\]
Since $i$ is a linear map, we have $d i_x = i$ for each $x \in \bR^m$. It follows that $i$ is an immersion
at any $x\in \bR^m$; $i$ is called the {\em canonical immersion}. 
\end{example}

\begin{definition}\label{def:submersion}
Let $f : M \to N$ be a smooth map between smooth manifolds and let $p$ be a point of $M$. 
We say that $f$ is a {\em submersion} (resp. {\em immersion}) at $p$ if there is a chart 
$(U,\phi)$ for $M$ around $p$ and a chart $(V,\psi)$ for $N$ around $f(p)$ such that 
\begin{itemize}
\item $f(U) \subset V$, and 
\item the composition $g = \psi \circ f \circ \phi^{-1} : \phi(U) \to \psi(V)$ is a submersion 
(resp. immersion) at $\phi(p)$. 
\end{itemize}
\end{definition}

\begin{proposition} \label{prop:canonical-form}
Let $f : M \to N$ be a smooth map between smooth manifolds of dimension $m$ and $n$ respectively. 
\begin{enumerate}
\item (Canonical form for submersions and immersions) 
If $f$ is a submersion (resp. immersion) at $p \in M$, so that $m\geq n$ (resp. $m\leq n$),  
there is a chart $(U,\phi)$ for $M$ around $p$ and a chart $(V,\psi)$ for $N$ around $f(p)$ such that 
\begin{itemize}
\item $\phi(p) = 0\in \bR^m$,
\item $\psi(f(p)) = 0 \in \bR^n$, and 
\item the composition $\psi \circ f \circ \phi^{-1}$ is the restriction of 
the canonical submersion (resp. immersion) to $\phi(U) \subset \mathbb{R}^m$.  
\end{itemize}
\item If $f$ is a submersion and an immersion at $p \in M$, then $f$ is a local diffeomorphism at $p$. 
\end{enumerate}
\end{proposition}

\begin{proof} Roundtable on September 18. Reference: \cite[II.7, III.4]{Bo}.
\end{proof}

\begin{definition}
Let $f : M \to N$ be a smooth map between smooth manifolds. We say that $f$ is a 
{\em submersion} (resp. {\em immersion}) if $f$ is a submersion (resp. immersion) 
at each point $p \in M$. 
\end{definition}

\begin{definition}
Let $f : M \to N$ be a smooth map between smooth manifolds. We say that $f$ is an 
{\em embedding} if 
\begin{itemize}
\item $f$ is an immersion 
\item $f : M \to f(M)$ is a homeomorphism onto $f(M)$, where $f(M)$ is equipped 
with the subspace topology. 
\end{itemize}
In this case, we say that $f(M)$ is a {\em submanifold} of $N$.
\end{definition}

From Proposition \ref{prop:canonical-form} (1), 
We also have the following alternative definition of a submanifold. 

\begin{definition}
Let $N$ be a smooth $n$-dimensional manifold, and let $M$ be a subset of $N$. 
We say that $M$ is a {\em submanifold} of $N$ of dimension $m$ (which is not greater than $n$) 
if for each $p$ in $M$, there is a chart $(U,\phi)$ for $N$ around $p$ such that 
$\phi(p)=0$ and $\phi(U \cap M) = \phi(U) \cap (\bR^m \times \{0\})$. 
\end{definition}

\begin{example}
These examples are to illuminate the definition of an embedding. 
Given a smooth map $f:\bR\to \bR^2$, $df_t:\bR\to \bR^2$ is given by 
$df_t(u)=f'(t)u$. So $f$ is an immersion at $t\in \bR$ iff $f'(t)$ is nonzero.
\begin{enumerate}
\item Let $f : \bR \to \bR^2$ denote the parabola given by  $f(t) = (t,t^2)$. Then $f'(t)=(1,2t)$ is nonzero for any $t\in \bR$, 
and hence $f$ is an immersion. We see also that $f$ is a homeomorphism from $\bR$ onto the 
image $f(\bR)$, so $f$ defines an embedding. 

\item  Let $f : \bR \to \bR^2$ denote the covering of the unit circle given by $f(t) = (\cos(t), \sin(t))$. 
Then $f'(t)=(-\sin t,\cos t)$ is nonzero for any $t\in \bR$, so $f$ is an immersion, but $f$ is not an embedding because 
it is not injective. 

\item  Let $f : \bR \to \bR^2$ be the nodal cubic defined by $f(t) = (t^3 - 4t, t^2 - 4)$. 
Then $f'(t)=(3t^2-4, 2t)$ is always nonzero, so $f$ is an immersion. However, $f$ is not an embedding since it is not injective: $f(2)=f(-2)=(0,0)$.

\item  Let $f : \bR \to \bR^2$ be the cuspidal cubic defined by $f(t) = (t^3, t^2)$. Then we see that $f$ is injective and a homeomorphism onto its image, but $f$ is not an immersion at $t = 0$, because the derivative vanishes there.
\end{enumerate}
\end{example}

\begin{definition}
Let $f : M \to N$ be a smooth map between smooth manifolds and assume that the dimension of $M$ is greater than or equal to the dimension of $N$. A point $p \in M$ is a {\em critical point} of $f$ if $f$ is not a submersion at $p$. 
In this case, $f(p)$ is called a {\em critical value} of $f$, that is, a point $q \in N$ is a critical value 
if there is a point $p \in f^{-1}(q)$ such that $p$ is a critical point. We say that $q \in N$ is a 
{\em regular value} if $q$ is not a critical value. 
\end{definition}

\begin{theorem} \label{preimage}
Let $f : M \to N$ be a smooth map between smooth manifolds of dimensions $m$ and $n$ respectively, with $m \ge n$. 
If $q \in N$ is a regular value of $f$ then the preimage $f^{-1}(q)$ is a closed submanifold of $M$ of 
dimension $m - n$. ($f^{-1}(q)$ can be empty.) 
\end{theorem}

\begin{proof}
Roundtable on September 18. Reference: \cite[III.5]{Bo}.  Idea: use canonical form of submersion.
\end{proof}


\begin{example}
Let $f : \bR^{n+1} \to \bR$ be the smooth map given by 
\[
f(x_1, \ldots, x_{n+1}) = x_1^2 + \cdots + x_{n+1}^2.
\]
Then $df_x:\bR^{n+1}\to \bR$ is given by $df_x=[2x_1 \cdots 2x_{n+1}]$, which is
surjective iff $x\ne 0$. So the only critical point of $f$ is $0\in \bR^{n+1}$
and the only critical value of $f$ is $0\in \bR$. It follows that every nonzero real number
is a regular value of $f$. 
If $a > 0$, then we see that $f^{-1}(a)$ is a $n$-dimensional smooth submanifold of $\bR^{n+1}$. 
Note that $f^{-1}(a)$ is the $n$-dimensional sphere of radius $\sqrt{a}$. 
We have $f^{-1}(0)=\{0\}$, and $f^{-1}(a)$ is empty when $a<0$. 
\end{example}

\begin{example}
Let $p$ denote the composition $S^n \hookrightarrow \mathbb{R}^{n+1}\setminus \{0\} \to P_n(\mathbb{R})$. Then $p$ is a covering map of degree $2$. Moreover, $p$ is a local diffeomorphism. 
\end{example}





\section{Wednesday, September 16, 2015}

\begin{example}\label{ex:orthogonal}
Let $O(n)$ denote the set of all $n\times n$ orthognal matrices: 
$$
O(n)=\{ A\in M_n(\bR): AA^T = I_n\}
$$
where $M_n(\bR)$ is the set of real $n\times n$ matrics, $A^T$ is the transpose of $A$, 
and $I_n$ denotes the $n\times n$ identity matrix. We may identify $M_n(\bR)$ with $\bR^{n^2}$ as an
$n^2$-dimensional real vector space. We claim that $O(n)$ is a submanifold of $M_n(\bR)\cong \bR^{n^2}$ of dimension $\frac{n(n-1)}{2}$. 
To prove this, we will use the preimage theorem. 

Let $S_n(\bR)$ denote the set of all real symmetric $n\times n$ matrices:
$$
S_n(\bR)=\{ A\in M_n(\bR):A=A^T\}.
$$ 
Then $S_n(\bR)$ is an $\frac{n(n+1)}{2}$-dimensional
subspace of $M_n(\bR)$. Define a map 
$$
f : M_n(\bR)\cong \bR^{n^2} \longrightarrow S_n(\bR)\cong \bR^{\frac{n(n+1)}{2}},\quad  A \mapsto AA^T.
$$ 
Then $f$ is a smooth map, since it is a polynomial map in the entries of $A$:
if $A=(a_{ij})$ then $(AA^T)_{kl}= \sum_{m=1}^n a_{km}a_{lm}$. 

By the preimage theorem, it remains to show that $I_n$ is a regular value of $f$. 
For $A \in M_n(\mathbb{R})$, the differential $df_A: M_n(\bR)\to S_n(\bR)$
at $A$ is given by 
\[
df_A(B) = \lim_{h \to 0} \frac{f(A + hB) - f(A)}{h} = 
\lim_{h\to 0} \frac{(A+hB)(A^T+hB^T)-AA^T}{h}=  AB^T + BA^T.
\]
If $A \in f^{-1}(I_n)= O(n)$ and $C \in S_n(\mathbb{R})$ are arbitrary, then $B = \tfrac{1}{2}CA =\tfrac{1}{2}C^TA$ satisfies 
\[
df_A(B) = C,
\]
showing that $df_A$ is surjective for all $A\in f^{-1}(I_n)$. It follows that $I_n$ is a regular value 
of $f$ as desired. 
\end{example}

\noindent
{\large \bf Orientation}

\begin{definition}
Let $M$ be a $C^k$ manifold, where $k\geq 1$.  
We say that $M$ is {\em orientable} if there is a $C^k$-atlas $\Phi = \{(U_\alpha, \phi_\alpha) : \alpha \in I\}$ representing the $C^k$-structure on $M$ such that 
\begin{itemize}
\item[($\star$)] For each $\alpha, \beta \in I$ such that $U_\alpha \cap U_\beta \ne \varnothing$, the transition function $\phi_\beta \circ \phi_\alpha^{-1}$ satisfies $\det(d(\phi_\beta \circ \phi_\alpha^{-1})_x) >0$ for each $x \in \phi_\alpha(U_\alpha \cap U_\beta)$.  
\end{itemize}
If $M$ is orientable, an {\em orientation} of $M$ is a choice of a $C^k$-atlas satisfying ($\star$). 
If $\Phi$ and $\Psi$ are two $C^k$-atlases satisfying ($\star$), then they determine the same orientation if their union $\Phi\cup \Psi$ 
satisfies ($\star$)
\end{definition}

\begin{example}
Suppose that $\Phi = \{(U_1,\phi_1), (U_2, \phi_2)\}$ is a $C^k$-atlas of a $C^k$-manifold $M$ such that the intersection $U_1 \cap U_2$ is connected. We claim 
that $M$ is orientable. Indeed, since the determinant of $\det(d(\phi_2\circ\phi_1^{-1})_x)$ is a continuous map from the connected set $\phi_1(U_1\cup U_2)$ 
to $\bR\setminus\{0\}$, it is either always positive or always negative on $\phi_1(U_1\cup U_2)$. If it is always positive then $\Phi$ determines an orientation;
if it is always negative, then we can change the sign of one of the coordinates of $\phi_2$ to make it always positive.

By Assignment 1 (1) and the above observation, $S^n$ is orientable for any $n\geq 2$. It is easy to see that $S^1$ is also orientable. 
\end{example}

\begin{lemma}
Let $L : \mathbb{C}^n \to \mathbb{C}^n$ be a $\mathbb{C}$-linear isomorphism given by $v \mapsto C v$ for some complex $n\times n$ matrix 
$C\in M_n(\bR)$. Write $C = A + iB$ for some $A, B\in M_n(\bR)$.  Let $i : \bR^{2n} \to \bC^n$ be the $\bR$-linear map given by $(x,y) \mapsto x + iy$.
Let $L' : \bR^{2n} \to \bR^{2n}$ denote the $\mathbb{R}$-linear map such that $L \circ i = i \circ L'$. Then we see that $L'$ is given by 
\[
\begin{bmatrix}
x \\ y
\end{bmatrix} \mapsto 
\begin{bmatrix}
A & - B \\
B & A
\end{bmatrix}\begin{bmatrix}
x \\ y
\end{bmatrix}
\]
and 
$$
\det \begin{bmatrix} A & -B\\ B & A \end{bmatrix} = |\det C|^2.
$$ 
\end{lemma}

\begin{example}
We may form complex projective space $P_n(\mathbb{C})$ in a similar fashion to real projective space. We claim that this $2n$-dimensional manifold is orientable. Indeed, for each $x \in \phi_i(U_i)$, the differential $d(\phi_j \circ \phi_i^{-1})_x : \bC^n  \to \bC^n$ is a $\bC$-linear isomorphism. By the Lemma, it follows that if we view the differential as an $\bR$-linear map from $\bR^{2n}$ to $\bR^{2n}$, then it has positive determinant.  

This argument shows that a complex $n$-manifold is an orientable $C^\infty$ $2n$-manifold;
indeed, the orientation is determined by the complex structure, so it is an {\em oriented}
$C^\infty$ $2n$-manifold.  
\end{example}

\begin{example}
We will see later the real projective space $P_n(\bR)$ is orientable iff $n$ is odd.
In particular, the real projective line $P_1(\bR)\cong S^1$ is orientable, and
the real projective plane $P_2(\bR)$ is nonorientable. 

\end{example}

\bigskip

\noindent
{\large \bf Tangent spaces and tangent bundles}

\medskip

Let $M$ be a $C^k$ manifold of dimension $n$, where $k\geq 1$.
\begin{definition}[tangent space, tangent vector]
Let $(U,\phi)$ and $(V,\psi)$ be two charts for $M$ around $p \in M$. 
For vectors $\vec{u}, \vec{v} \in \mathbb{R}^n$, we write $(U,\phi,\vec{u}) \sim_p (V,\psi, \vec{v})$ if 
\[
d(\psi \circ \phi^{-1})_{\phi(p)}(\vec{u}) = \vec{v}. 
\]
This defines an equivalence relation on such triples, and we let $[(U,\phi,\vec{u})]$ denote the equivalence class of such a triple under this relation. We define the {\em tangent space to $M$ at $p$} to be the set 
\[
T_pM = \{[(U,\phi,\vec{u})] : \text{$(U,\phi)$ is a chart around $p$},\; \vec{u} \in \mathbb{R}^n\}. 
\]

For a fixed chart $(U,\phi)$ around $p$, the map $\theta_{U,\phi,p} : \bR^n \to T_pM$ described by 
\[
\theta_{U,\phi,p}(\vec{u}) = [(U,\phi, \vec{u})]
\]
is a bijection (Assignment 3 (1)).  This implies that we may endow the space $T_pM$ with an $\bR$-linear structure. 
Moreover, this structure does not depend on the choice of chart: Indeed if $(V,\psi)$ is another chart around $p$, then the following diagram commutes
\[
\xymatrix{
\mathbb{R}^n \ar[rr]^{\theta_{U,\phi,p}}  \ar[d]_{d(\psi \circ \phi^{-1})_{\phi(p)}}& & T_pM \\
\mathbb{R}^n \ar[rru]_{\theta_{V,\psi,p}}
}
\]
and the map $d(\psi \circ \phi^{-1})_{\phi(p)}$ is an $\bR$-linear isomorphism. 

A {\em tangent vector at $p$} is a vector in the $n$-dimensional real vector space $T_pM$.

\end{definition}


We construct now a $2n$-dimensional manifold called the {\rm tangent bundle} of $M$, denoted $TM$. 

\noindent
1. As a {\em set}, the tangent bundle of $M$ is given by
\[
TM = \{(p,v) : p \in M, v \in T_pM\}.
\]
There is a surjective map $\pi : TM \to M$ sending $(p,v)$ to $p$. 

\noindent
2. {\em Topology}:
For a chart $(U,\phi)$ for $M$, let $\tilde{\phi} : \pi^{-1}(U) \to \phi(U) \times \mathbb{R}^n$ be the map described by 
\[
\tilde{\phi}(p,v) = (\phi(p), \theta^{-1}_{U,\phi,p}(v)).
\]
Equip the set $TM$ with the topology such that $\tilde{\phi}$ is a homeomorphism for each chart $(U,\phi)$. This means that a subset $A$ of $TM$ is open if and only if for each chart $(U,\phi)$ for $M$, the set $\tilde{\phi}(\pi^{-1}(U) \cap A)$ is open in $\phi(U) \times \mathbb{R}^n$. 
With this topology, $TM$ is a topological manifold of dimension $2n$.

It can be shown that that if $M$ is Hausdorff (resp. has a countable basis), then $TM$ is Hausdorff (resp. has a countable basis) as well. 

\noindent
3. {\em Transition functions}: Note that if $U$ is an open subset of $M$ then 
$\pi^{-1}(U)$ can be identified with $TU$. We have $\pi^{-1}(U)\cap \pi^{-1}(V)= TU\cap TV = T(U\cap V) =\pi^{-1}(U\cap V)$.
Given two charts $(U,\phi)$ and $(V,\psi)$ for $M$, $(TU, \widetilde{\phi})$ and 
$(TV,\widetilde{\psi})$ are charts for $TM$, and the transition function
$$
\widetilde{\psi}\circ \widetilde{\phi}^{-1}: \tilde{\phi}(TU\cap TV) =\tilde{\phi}(T(U\cap V))
\to \tilde{\psi}(TU\cap TV) = \tilde{\psi}(T(U\cap V))  
$$
is given by 
$$
\widetilde{\psi} \circ \widetilde{\phi}^{-1}(\vec{x},\vec{u}) = (\psi \circ \phi^{-1}(\vec{x}), d(\psi \circ \phi^{-1})_{\vec{x}}(\vec{u}))
$$
where $\psi\circ\phi^{-1}(\vec{x})$ is $C^k$ in $\vec{x}$ and
the map $\vec{x} \mapsto d(\psi \circ \phi^{-1})_{\vec{x}}$ is $C^{k-1}$ in $\vec{x}$. So 
$\widetilde{\psi} \circ \widetilde{\phi}^{-1}$ is a $C^{k-1}$ diffeomorphism. It follows that $TM$ is a $C^{k-1}$-manifold.
In particular, if $M$ is a $C^\infty$ manifold then $TM$ is a $C^\infty$ manifold. 

\begin{lemma}
The projection map $\pi : TM \to M$ is a $C^{k-1}$ map. 
In particular, when $k=\infty$, $\pi:TM\to M$ is a smooth map and a submersion.
\end{lemma}
\begin{proof}
Given a point $(p,v)$ in $TM$, where $p\in M$ and $v\in T_pM$, let $(U,\phi)$ be a $C^k$ chart for $M$ around $p=\pi(p,v)$. Then
$(\pi^{-1}(U)=TU,\widetilde{\phi})$ is a $C^{k-1}$ chart around $(p,v)$, and we have the following commutative diagram 
\[
\xymatrix{
\pi^{-1}(U) \ar[r]^{\pi} \ar[d]_{\widetilde{\phi}} & U \ar[d]_{\phi} \\
\phi(U) \ar[r]^g \times \mathbb{R}^n & \phi(U)
}
\]
where $g(\vec{x},\vec{u})= \vec{x}$ is the restriction of the canonical submersion $\bR^{2n} \to \bR^n$. 
\end{proof}


Assignment 2 (2): $TM$ is orientable, even though $M$ may not be.

\section{Monday, September 21, 2015}


\noindent
{\bf \large The differential of a $C^k$ map}

\begin{definition}
Let $f : M \to N$ be a $C^k$ map between $C^k$ manifolds of dimension $m$ and $n$ respectively,
where $k\geq 1$.  The {\em differential of $f$ at $p$} is the linear map 
\[
df_p : T_pM \to T_{f(p)}N
\]
defined as follows: Given a chart $(U,\phi)$ for $M$ around $p$ and a chart $(V,\psi)$ for $N$ around $f(p)$ such that 
$f(U) \subset V$, let $g:= \psi\circ f\circ \phi^{-1}:\phi(U)\to \psi(V)$, and let $df_p$ denote the composition 
\[
df_p = \theta_{V,\psi,f(p)} \circ dg_{\phi(p)} \circ \theta^{-1}_{U,\phi,p}.
\]
In terms of diagrams, this is the map given below
\[
\xymatrix{
T_pM \ar[r]^{df_p} \ar[d]_{\theta_{U,\phi,p}^{-1}} & T_{f(p)}N  \\
\mathbb{R}^m \ar[r]^{dg_{\phi(p)}} & \mathbb{R}^n \ar[u]_{\theta_{V,\psi, f(p)}}
}
\]
\end{definition}

\begin{remark}
At first glance, it seems that the differential $df_p$ may be ill-defined: a different choice of charts seems to lead to a different definition of $df_p$. However, the chain rule again comes to our rescue, and one can indeed show that $df_p$ is a well-defined map that is independent of the choice of charts. 

Note that $df_p$ is indeed a linear map since the $\theta$ and $dg_{\phi(p)}$ are. 

Finally, note that this definition is consistent with the case when $M$ is an open subset of $\bR^m$ and 
$N$ is an open subset of $\bR^n$. 
\end{remark}

\begin{theorem}[Chain Rule]
Let $f : M_1 \to M_2$ and $g : M_2 \to M_3$ be $C^k$ maps between $C^k$ manifolds, where $k\geq 1$. Then 
\begin{enumerate}
\item The composition $g \circ f : M_1 \to M_3$ is a $C^k$ map.
\item For each point $p$ in $M_1$, the differential of the composition is given by 
\[
d(g \circ f)_p = dg_{f(p)} \circ df_p.
\]
\end{enumerate}
\end{theorem}

The following definition is equivalent to Definition \ref{def:submersion} when $k=\infty$.
\begin{definition}
Let $f : M \to N$ be a $C^k$ map between $C^k$manifolds, where $k\geq 1$. 
We say $f$ is a {\em submersion at $p$} (resp. {\em  immersion at $p$}) if $df_p$ is surjective (resp. injective).
\end{definition}

\begin{remark}
Suppose that $M$ is a submanifold of $N$. Then for each $p$ in $M$, the tangent space $T_pM$ can be viewed as a subspace of $T_pN$. Indeed, if $i : M \to N$ denotes the inclusion, then $di_p : T_pM \to T_pN$ is injective. 
\end{remark}


\begin{remark}
Suppose that $f : M \to N$ is a smooth map. Let $q \in N$ be a regular value. By Theorem \ref{preimage} (the preimage
theorem),  $S=f^{-1}(q)$ is a submanifold of $M$ of dimension $m - n$, where $m=\dim M$ and 
$n=\dim N$. For each $p \in S$, the tangent space $T_pS$ is given by $T_pS = \ker (df_p : T_pM \to T_{f(p)}N)$. 
That is, we have the following short exact sequence of real vector spaces 
\[
0 \longrightarrow T_pS \longrightarrow T_pM \longrightarrow T_qN \longrightarrow 0.
\]
\end{remark}

\begin{remark} For every point $p\in \bR^n$, we have an isomorphism 
$T_p\bR^n\cong \bR^n$ given by $v\mapsto \theta_{\bR^n,\mathrm{id},p}^{-1}(v)$.
We also have $\tilde{\mathrm{id}}: T \bR^n \to \bR^n\times \bR^n$.
\end{remark}

\begin{example}
Let $f : \bR^{n+1} \to \bR$ be the map $f(x_1, \ldots, x_{n+1}) = x_1^2 + \cdots + x_{n+1}^2$. We have already seen that $1$ is a regular value of $f$, and thus the unit sphere $S^n=f^{-1}(1)$ is a submanifold of $\mathbb{R}^{n+1}$. 
For each $p \in S^n$, we compute 
\[
T_pS^n = \{v \in \bR^{n+1} : df_p(v) = 0\} = \{v \in \bR^{n+1} : p \cdot v = 0\}
\]
\end{example}

\begin{example}
Let $f : M_n(\mathbb{R}) \to S_n(\mathbb{R})$ be the map of Example \ref{ex:orthogonal}, that is, $f(A) = AA^T$. 
Recall that the orthogonal group $O(n)$ is the preimage of the regular value $I_n$. For $A \in O(n)$, we compute 
\[
T_AO(n) = \{B \in M_n(\bR) : df_A(B) = 0\} = \{B \in M_n(\bR) : BA^T + AB^T = 0\}.
\]
In particular, $T_{I_n}O(n)=\{ B\in M_n(\bR): B+B^T=0\} \cong \bR^{\frac{n(n-1)}{2}}$ is the 
set of real $n\times n$ skew-symmetric matrices.  
\end{example}

\begin{definition}
Let $f : M \to N$ be a $C^k$ map between $C^k$ manifolds. Define $df : TM \to TN$ by the rule 
\[
df(p,v) = (f(p), df_p(v)).
\]
\end{definition}

\begin{proposition}
Let $f : M \to N$ be a $C^k$ map between $C^k$ manifolds. Then $df : TM \to TN$ is a $C^{k-1}$ map
between $C^{k-1}$ manifolds.
\end{proposition}

\begin{proposition}
If $M$ is a smooth submanifold of $N$ of dimension $m$, then $TM$ is a smooth submanifold of $TN$ of dimension $2m$. 
\end{proposition}

\begin{example}
The tangent bundle of the sphere $S^n$ is given by 
\[
TS^n = \{(x,v) \in \bR^{n+1} \times \bR^{n+1} : |x|^2 = 1, x \cdot v = 0\} \subset \bR^{n+1}\times \bR^{n+1}.
\]
\end{example}

\begin{example}
The tangent bundle of the orthogonal group $O(n)$ is given by 
\[
TO(n) = \{(A,B) \in M_n(\bR) \times M_n(\bR) : AA^T = I_n, BA^T + AB^T = 0 \} \subset M_n(\bR)\times M_n(\bR).
\]
\end{example}

\noindent
{\bf \large Vector bundles}

Roughly speaking, a real vector bundle of rank $r$ over a manifold $M$ consists of a family of $r$-dimensional 
real vector spaces parametrized by $M$. 

\begin{definition} \label{vector-bundle}
Let $M$ be a $C^k$ manifold. A {\em real $C^k$ vector bundle of rank $r$ over $M$} consists of
\begin{itemize}
\item a $C^k$ manifold $E$ called the {\em total space} and 
\item a $C^k$ surjective map $\pi : E \to M$
\end{itemize}
such that 
\begin{enumerate}
\item[(i)] (local trivialization) There is an open cover $\{ U_\alpha:\alpha\in I\}$ of $M$ 
(where $U_\alpha$ is not necessarily a coordinate neighborhood) and $C^k$ diffeomorphisms 
$h_\alpha : \pi^{-1}(U_\alpha) \to U_\alpha \times \mathbb{R}^r$ (called {\em local trivializations}) 
such that the following diagram commutes 
\[
\xymatrix{
\pi^{-1}(U_\alpha) \ar[r]^{h_\alpha} \ar[d]_{\pi_\alpha} & \ar[dl]^{\mathrm{pr}_1} U_\alpha \times \mathbb{R}^r \\
U_\alpha
}
\]
where $\pi_\alpha$ is the restriction of $\pi$ to $\pi^{-1}(U_\alpha)$, and $\mathrm{pr}_1$ is the projection
to the first factor. 
\item[(ii)] (transition functions) If the intersection $U_\alpha \cap U_\beta$ is nonempty, then the map 
\[
h_\beta \circ h_\alpha^{-1} : (U_\alpha \cap U_\beta) \times \mathbb{R}^r \to (U_\alpha \cap U_\beta) \times \mathbb{R}^r
\]
is a $C^k$ diffeomorphism of the form $h_\beta \circ h_\alpha^{-1}(x,v) = (x, g_{\beta\alpha}(x)v)$ where $g_{\beta\alpha} : U_\alpha \cap U_\beta \to GL(r,\bR)$ is a $C^k$ map. (Note that $GL(r,\bR)=\{ A\in M_r(\bR):\det(A)\neq 0\}$ is an open subset
of $M_r(\bR)\cong \bR^{r^2}$.)
\end{enumerate}
\end{definition}

\begin{remark}
From condition (i), we know that $h_\beta \circ h_\alpha^{-1}$ is a $C^k$ diffeomorphism of the form 
$(x,v) \mapsto (x, g_{\beta\alpha}(x)v)$ where $g_{\beta\alpha}(x) : \bR^r \to \bR^r$ is a $C^k$ diffeomorphism 
(depending on $x\in U_\alpha\cap U_\beta$). 
However, in condition (ii), we require something stronger: namely that $g_{\beta\alpha}(x)$ is a linear isomorphism. 
If we only had the weaker condition, then we would say that $\pi : E \to M$ is a {\em fiber bundle} with 
total space $E$ and fiber $\bR^r$. 
\end{remark}

\begin{example}[product vector bundle]
The product vector bundle of rank $r$ consists of $\pi= \mathrm{pr}_1 : E= M \times \bR^r \to M$ where $\rm{pr}_1$ 
denotes the projection onto the first factor. 
\end{example}

\begin{definition}[trivial vector bundle]\label{trivial}
We say that $\pi : E \to M$ is a {\em trivial vector bundle of rank $r$} if there is a $C^k$ diffeomorphism (when $k\geq 1$)
or a homeomorphism (when $k=0$) $h : E \to M \times \bR^r$ such that 
\begin{itemize}
\item $h$ commutes with the projection maps in the sense that $\pi = \mathrm{pr}_1 \circ h$
\item the restriction of $h$ to each fiber $h_x : E_x \to \{x\} \times \bR^r$ is a linear isomorphism. 
\end{itemize}
In other words, $\pi:E\to M$ is a trivial vector bundle of rank $r$ if there exists
a {\em global} trivialization $h:E\to M\times\bR^r$.
\end{definition}



\section{Wednesday, September 23, 2015}

\noindent
{\large {\bf Vector bundles} (continued)}

\begin{example}[tangent bundle]
Suppose that $M$ is a $C^k$ manifold with dimension $n$. Then $\pi : TM \to M$ is a $C^{k-1}$ vector bundle of rank $n$ over $M$. 

To see this, let $\Phi=\{(U_\alpha,\phi_\alpha):\alpha\in I\}$ be a $C^k$-atlas of the $C^k$ manifold $M$, define
local trivializations $h_\alpha: \pi^{-1}(U_\alpha)\to U_\alpha\times \bR^n$ by 
$$
h_\alpha(p,v)= \big( p, \theta_{U_\alpha,\phi_\alpha,p}^{-1}(v) \big)
$$
where $p\in U_\alpha$ and $v\in T_pM$. Then each $h_\alpha$ is $C^{k-1}$ diffeomorphism which satisfies (i) in Definition \ref{vector-bundle}.
If $U_\alpha\cap U_\beta\neq \emptyset$, the transition function 
$$
h_\beta\circ h_\alpha^{-1}: (U_\alpha\cap U_{\beta})\times \bR^n   \to (U_\alpha\cap U_\beta)\times \bR^n
$$
is given by 
$$
h_\beta\circ h_\alpha^{-1}(p,\vec{u})= \big(p,d(\phi_\beta\circ\phi^{-1}_\alpha)_{\phi_\alpha(p)}(\vec{u})\big).
$$
Note that $p\mapsto d(\phi_\beta\circ \phi_\alpha^{-1})_{\phi_\alpha(p)}$ defines a $C^{k-1}$ map from $U_\alpha\cap U_\beta$ to $GL(n,\bR)$.
So the transition functions satisfy (ii) in Definition \ref{vector-bundle}.
\end{example}


\begin{example}[universal line bundle over $P_n(\bR)$] 
See Assignment 3 (2).
\end{example}

\begin{definition}
Let $\pi : E \to M$ be a $C^k$ vector bundle over a $C^k$ manifold $M$. A {\em $C^k$ section} 
of $\pi$ is a $C^k$ map $s : M \to E$ such that $\pi \circ s = \text{id}_M$. 
\end{definition}

\begin{lemma}
Let $\pi : E \to M$ be a $C^k$ vector bundle of rank $r$ over a $C^k$ manifold $M$. Then $\pi : E \to M$ is trivial 
if and only if there are $C^k$ sections $s_1, \ldots, s_r$ of $\pi:E\to M$ such that for each point $x \in M$, 
the collection $\{s_1(x), \ldots, s_r(x)\}$ forms a basis of $E_x$. 
\end{lemma}

\begin{proof}
$(\Rightarrow)$ Suppose that $\pi :E \to M$ is trivial and let $h : E \to M \times \bR^r$ be a trivialization as in Definition \ref{trivial}.
Let $e_1=(1,0,\ldots,0), e_2=(0,1,0,\ldots,0), \ldots, e_r =(0,\ldots, 0,1)$ be the standard basis of $\bR^r$.
Define $s_i: M\to E$ by $s_i(x)=h^{-1}(x, e_i)$, $i=1,\ldots,r$. Then $s_i$ are $C^k$ sections of $\pi:E\to M$, and
for each $x\in M$ the collection $\{s_1(x), \ldots, s_r(x)\}$ forms a basis of $E_x\cong \bR^r$.

\noindent
$(\Leftarrow)$ Conversely, if we are given $C^k$ sections $s_1,\ldots,s_r$ of $\pi:E\to M$ such that
the collection $\{ s_1(x),\ldots, s_r(x)\}$ forms a basis of $E_x\cong \bR^r$ for all $x\in M$, 
we define $\psi: M \times \mathbb{R}^r \to E$ by  
\[
(x,(v_1,\ldots, v_r)) \mapsto (x, \sum_{i=1}^r v_i s_i(x)).
\]
where $x\in M$, $(v_1,\ldots, v_r)\in \bR^r$, and $\sum_{i=1}^r v_i s_i(x)\in E_x$. Then 
$\psi$ is a $C^k$-diffeomorphism (when $k\geq 1$) or a homeomorphism (when $k=0$), and
$h:=\psi^{-1}:E\to M\times \bR^r$ is a global trivialization as in Definition \ref{trivial}. 
\end{proof}

\begin{definition}
Let $M$ be a smooth manifold. A {\em smooth vector field} on $M$ is a smooth section of $TM$. 
\end{definition}


\bigskip

\noindent
{\large \bf Derivations}

\begin{definition}\label{germ}
Let $M$ be a $C^k$ manifold and let $p$ be a point of $M$. Let $U$ and $V$ be open neighborhoods 
of $p$ in $M$ and let $f : U \to \bR$ and $g : V \to \mathbb{R}$ be $C^k$ functions. We define an equivalence relation $\sim_p$ by the rule 
$(f:U\to \bR) \sim_p (g:V\to \bR)$ if and only if there is an open neighborhood $W$ of $p$ such that $W\subset U\cap V$ and 
$f|_W \equiv g|_W$. 

A {\em germ of $C^k$ functions at $p$} is an equivalence class under this 
equivalence relation. Let $[f:U\to \bR]$ denote the equivalence class represented by $f:U\to \bR$.
We let $C_p^k(M)$ denote the collection of all such equivalence classes:
$$
C^k_p(M) := \{ (f:U\to \bR): U \textup{ is an open neighborhood of $p$ in $M$, $f$ is a $C^k$ function on $U$} \}/\sim_p.
$$
\end{definition}

\begin{lemma}
The set $C_p^k(M)$ of germs of $C^k$-functions at $p$ has the natural structure of a ring:
\begin{eqnarray*}
{[f:U\to \bR]+ [g:V\to \bR]} &=& [f+g:U\cap V\to \bR],\\
{[f:U\to \bR]\cdot [g:V\to \bR]} &=& [f\cdot g: U\cap V\to \bR], 
\end{eqnarray*}
where $(f+g)(q)=f(q)+g(q)$ and $(f\cdot g)(q)=f(q)g(q)$ for $q\in U\cap V$.
\end{lemma}

\begin{remark}
In the definition of $C^k_p(M)$ in Definition \ref{germ},  we may assume that $U$ is contained in some fixed coordinate chart 
$(U_0, \phi_0)$ for $M$ around $p$, and hence we get a map 
\begin{align*}
C_p^k(M) &\to C_0^k(\bR^n) \\
[f : U \to \bR] &\mapsto [f \circ \phi_0^{-1} : \phi_0 (U) \to \bR].
\end{align*}
which is a ring isomorphism. Therefore, it is sufficient to study germs of $C^k$ functions at $0$ in $\bR^n$. 
\end{remark}

\begin{lemma}
Let $C^k(M)$ be the set of all $C^k$-functions on $M$.
The natural map $C^k(M) \to C_p^k(M)$ given by $f\mapsto [f:M\to \bR]$ is surjective. 
\end{lemma}

\begin{proof}
Suppose we have a $C^k$ function $f : U \to \bR$ defined on a open neighborhood $U$ of $p$. 
We claim that there is a neighborhood $U'$ containing $p$ and a $C^k$-map $\beta : U' \to \mathbb{R}$ such that 
\begin{itemize}
\item $U' \subset U$
\item $\overline{U'}$ is compact
\item $\beta(x) = 1$ for each $x \in U'$
\item $\text{supp}(\beta)$ is relatively compact in $U$ 
\item $\beta(x) = 0$ for all $x \notin U$. 
\end{itemize}
Then the multiplication $(\beta f : U \to \mathbb{R})\sim_p (f : U \to \mathbb{R})$. But $\beta f$ extends to a $C^k$ 
function defined on all of $M$. The result now follows. 
\end{proof}

\begin{definition}
A {\em derivation} on $C_p^k(M)$ is an $\bR$-linear map $\delta : C_p^k(M) \to \mathbb{R}$ such that 
\[
\delta(fg) = \delta(f)g(p) + f(p)\delta(g) \quad\quad \textup{(Leibniz rule)}
\]
for each $f,g \in C_p^k(M)$. 
\end{definition}

\begin{remark}
The set of derivations on $C_p^k(M)$ is an $\bR$-linear space. 
\end{remark}

\begin{example}  Suppose that $k\geq 1$. For $i=1,\ldots,n$,
$$
\frac{\partial}{\partial x_i}(0): C_0^k(\mathbb{R}^n)\to \bR,\quad
f \mapsto \frac{\partial f}{\partial x_i}(0).
$$
is a derivation on $C_0^k(\bR^n)$. For any $a_1,\ldots, a_n\in \bR$,
$$
\sum_{i=1}^n a_i \frac{\partial}{\partial x_i}(0): C_0^k(\mathbb{R}^n)\to \bR,\quad
f \mapsto \sum_{i=1}^n a_i \frac{\partial f}{\partial x_i}(0)
$$
is a derivation on $C_0^k(\bR)$.

\end{example}


\begin{lemma}
This lemma has three parts. 
\begin{enumerate}
\item[(a)] If $\delta$ is a derivation on $C_0^k(\mathbb{R}^n)$ and $f$ is constant near 0, then $\delta(f) = 0$. 
\item[(b)] If $\delta$ is a derivation on $C_0^0(\mathbb{R}^n)$, then $\delta \equiv 0$. 
\item[(c)] If $\delta$ is a derivation on $C_0^\infty(\mathbb{R}^n)$, then we may write 
\[
\delta = \sum_{i=1}^n a_i \frac{\partial}{\partial x_i}(0)
\]
where $a_i = \delta(x_i)$.
\end{enumerate}
\end{lemma}

\begin{proof}
(a) Since $\delta$ is linear, it suffices to show that $\delta(1) = 0$, but this is indeed the case as 
\[
\delta(1) = \delta(1 \cdot 1) = \delta(1)1 + 1 \delta(1) = 2 \delta(1). 
\]

\noindent
(b) Assignment 3 (3).

\noindent
(c) Let $f$ be a smooth function on $\mathbb{R}^n$ defined on a neighborhood of $0$. Take $x$ small enough such that the map 
$g : (-2,2) \to \mathbb{R}$ defined by $g(t) = f(tx)$ is defined. Then $g(t)$ is a smooth function on $(-2,2)$.  
$$
f(x) - f(0) = g(1) - g(0) 
= \int_{0}^1 g'(t)dt 
= \int_0^1 \Big( \sum_{i=1}^n x_i \frac{\partial f}{\partial x_i}(tx)\Big) dt 
= \sum_{i=1}^n x_i \int_0^1 \frac{\partial f}{\partial x_i}(tx)dt
$$ 
Let $h_i(x) = \int_0^1 \frac{\partial f}{\partial x_i}(tx)dt$. Then $h_i \in C_0^\infty(\mathbb{R}^n)$ and 
$$ 
h_i(0) = \int_0^1\frac{\partial f}{\partial x_i}(0)dt =\frac{\partial f}{\partial x_i}(0).
$$
It then follows that 
$$
\delta(f) = \delta(f - f(0)) = \delta\Big(\sum_{i=1}^n x_i h_i(x)\Big) 
= \sum_{i=1}^n (\delta(x_i) h_i(0) + x_i(0) \delta(h_i)) 
= \sum_{i=1}^n \delta(x_i) \frac{\partial f}{\partial x_i}(0)
$$
as desired.
\end{proof}


Let $D_pM$ denote the space of derivations on $C_p^\infty(M)$. We claim that there is a linear isomorphism 
\begin{align*}
T_pM &\longrightarrow D_pM \\
[(U, \phi, \vec{u})] &\mapsto \sum_{i=1}^n u_i \frac{\partial}{\partial x_i}(p).
\end{align*}
where the derivation $\frac{\partial}{\partial x_i}(p):C^\infty_p(M)\to \bR$ is defined by 
$f\mapsto \frac{\partial(f \circ \phi^{-1})}{\partial x_i}(\phi(p))$. 
Indeed, if this is well-defined, it is clearly a linear isomorphism, so it suffices to show that it is well-defined. 

Let $(V,\psi)$ be another chart for $M$ around $p$. Let $v \in \bR^n$ be such that $[(U,\phi, \vec{u})] = [(V, \psi, \vec{v})]$. 
Then this means that $\vec{v} = d(\psi \circ \phi^{-1})_{\phi(p)}(\vec{u})$. Write $\phi = (x_1, \ldots, x_n)$ and 
$\psi = (y_1, \ldots, y_n)$. Then the fact that $\vec{v} = d(\psi \circ \phi^{-1})_{\phi(p)}\vec{u}$ implies that 
\[
v_j = \sum_{i=1}^n \frac{\partial y_j}{\partial x_i} (\phi(p)) u_i,
\]


We then apply the chain rule to see that    
$$
\sum_{i=1}^n u_i \frac{\partial }{\partial x_i}(p) = \sum_{i,j=1}^n u_i \frac{\partial y_j}{\partial x_i}(\phi(p)) \frac{\partial }{\partial y_j}(p) 
= \sum_{j=1}^n v_j \frac{\partial}{\partial y_j}(p).
$$

$\displaystyle{ \sum_{i=1}^n u_i \frac{\partial}{\partial x_i}(p) }$ is the notation of a tangent vector at $p\in M$ in do Carmo's book.


\bigskip



Let $(U, \phi)$ be a coordinate chart for $M$ and write $\phi = (x_1, \ldots, x_n)$.  Recall that
$\widetilde{\phi}:TU\to \phi(U)\times \bR^n$ is defined by 
$\widetilde{\phi}(p,v)= (\phi(p), \theta_{U,\phi,p}^{-1}(v))$, 
and the linear isomorphism $T_p M\stackrel{\cong}{\longrightarrow} D_p M$ is given by $\theta_{U,\phi,p}(\vec{u}) \mapsto \sum_{i=1}^n u_i\frac{\partial}{\partial x_i}(p)$.
So $\widetilde{\phi}^{-1}: \phi(U)\times\bR^n \to TU$ is given by 
$$
\widetilde{\phi}^{-1}(x,\vec{u}) = (\phi^{-1}(x), \sum_{i=1}^n u_i \frac{\partial}{\partial x_i}(p) )
$$
where $x\in \phi(U)\subset \bR^n$ and $\vec{u}=(u_1,\ldots, u_n)\in \bR^n$. For $i=1,\ldots, n$
\[
\frac{\partial}{\partial x_i} : U \to TU,\quad p\mapsto (p, \frac{\partial}{\partial x_i}(p))
\]
are smooth sections of $TU\to U$. 
Moreover, for each point $p \in U$, the collection $\{ \frac{\partial}{\partial x_i}(p): i=1\ldots, n\}$ 
forms a basis for $T_pU$, and hence the collection $\{\frac{\partial}{\partial x_i}: i=1,\ldots,n\}$ 
forms a $C^\infty$ frame for $TU\to U$. We let $C^\infty(U,TU)$ denote the space of $C^\infty$ sections of $TU\to U$. 
We have an isomorphism
$$
T_p M =\bigoplus_{i=1}^n \bR\frac{\partial}{\partial x_i}(p)
$$
as real vector spaces, and an isomorphism
$$
C^\infty(U,TU) =\bigoplus_{i=1}^n C^\infty(U)\frac{\partial}{\partial x_i}
$$
as $C^\infty(U)$-modules. Therefore, any $C^\infty$ vector field on $U$ is of the form 
$$
\sum_i a_i \frac{\partial}{\partial x_i},\quad a_i \in C^\infty(U).
$$ 






\section{Monday, September 28, 2015}

\noindent
{\bf \large Lie derivative and Lie bracket}

Last time we defined derivations on the germs of smooth functions of $M$ at $p$. We also identified the set of derivations $D_pM$ with the tangent space $T_pM$. 

\begin{definition}
Let $M$ be a smooth manifold. A {\em derivation} on $C^\infty(M)$ is an $\mathbb{R}$-linear map $\delta : C^\infty(M) \to C^\infty(M)$ satisfying the Leibniz rule 
\[
\delta(fg) = \delta (f)g + f \delta(g).
\]
Let $D(M)$ denote the set of derivations on $C^\infty(M)$. 
\end{definition}

\begin{remark}
This is a sort of global extension of the previous definition. 
\end{remark}

\begin{remark}
Note that $D(M)$ is a $C^\infty(M)$-module: Indeed if $\delta \in D(M)$ and $h \in C^\infty(M)$, then we can define $h \delta \in D(M)$ by the rule 
\[
(h\delta)(f) = h \delta(f).
\]
\end{remark}

Now we relate this notion to vector fields, via Lie derivatives. 

\begin{definition}
Let $X$ be a smooth vector field on a smooth manifold $M$. Define a map $L_X : C^\infty(M) \to C^\infty(M)$ called the {\em Lie derivative} by the rule 
\[
(L_Xf)(p) = X(p)f
\]
for any $p \in M$. Recall that a smooth vector field is a smooth section $M \to TM$, so that means that $X(p) \in T_pM = D_pM$, so we may apply $X(p)$ to the germ determined by $f$ at $p$. We sometimes denote $L_Xf$ by $Xf$. 
\end{definition}

To see $Xf$ is a smooth function on a coordinate neighborhood $U$ of $p$, recall that $X$ restricted to $U$ is given by $X = \sum_{i=1}^n a_i \frac{\partial}{\partial x_i}$ where $a_i \in C^\infty(U)$. Then we see that 
\[
(Xf)(p) = \sum_{i=1}^n a_i(p) \frac{\partial }{\partial x_i}(f \circ \phi^{-1})(\phi(p)).
\]
In do Carmo's notation, we write 
\[
(Xf)(p) = \sum_{i=1}^n a_i(p) \frac{\partial f}{\partial x_i}(p).
\]



\begin{theorem}
The assignment 
\begin{align*}
C^\infty(M,TM) &\to D(M) \\
X &\mapsto L_X
\end{align*}
is an isomorphism of $C^\infty(M)$-modules. 
\end{theorem}

\begin{proof}
We provide an outline of the proof. First it is clear that the map is $C^\infty(M)$-linear. 

To see that the map is surjective, suppose we are given $\delta \in D(M)$, we will define $X \in C^\infty(M,TM)$ such that $L_X = \delta$. For any $p \in M$, we let $(U,\phi)$ be a coordinate chart for $M$ around $p$ and we let $X(p) = \sum_{i=1}^n \delta_p(x_i) \frac{\partial}{\partial x_i}(p)$. Here the notation $\delta_p$ means that we restrict the derivation $\delta$ to the germs of functions at $p$. 

To see that the map is injective, we want to show that if $X \in C^\infty(M,TM)$ is not identically zero, then $L_X$ is not identically zero.  If $X \ne 0$, then there is a point $p \in M$ such that $X(p) \ne 0 \in T_pM = D_pM$. So there is an $f \in C^\infty_p(M)$ such that $X(p)f \ne 0$. We may assume that $f \in C^\infty(M)$. Then $(L_Xf)(p) = X(p)f \ne 0$. 
\end{proof}

\begin{definition}[Lie bracket]
Let $X,Y$ be smooth vector fields on $M$. We define $[X,Y] : C^\infty(M) \to C^\infty(M)$ by the rule 
\[
[X,Y](f) = XYf - YXf = L_XL_Yf - L_YL_Xf.
\]
\end{definition}

\begin{lemma}
The map $[X,Y]$ is a derivation. 
\end{lemma}

\begin{proof}
It is clear that $[X,Y]$ is $\mathbb{R}$-linear. We need to check the Leibniz rule. But this is straightforward and left as an exercise.  
\end{proof}

By the Lemma and the Theorem, we may view $[X,Y]$ as a smooth vector field. In local coordinates $(U,\phi)$, suppose that $X = \sum_{i}a_i \frac{\partial}{\partial x_i}$ and $Y = \sum_{i} b_i \frac{\partial}{\partial x_i}$. Then in terms of local coordinates we find that 
\[
[X,Y] = \sum_{i,j}\left(a_i \frac{\partial b_j}{\partial x_i} - b_i \frac{\partial a_j}{\partial x_i}\right) \frac{\partial}{\partial x_j}
\]

\begin{proposition}\label{Lie-bracket}
The map $[-,-] : C^\infty(M,TM) \times C^\infty(M,TM) \to C^\infty(M,TM)$ defines a map which satisfies the following properties:
\begin{enumerate}
\item[(i)] $[-,-]$ is $\mathbb{R}$-bilinear. 
\item[(ii)] $[-,-]$ is anti-commutative in the sense that $[X,Y] = -[Y,X]$. 
\item[(iii)] $[-,-]$ satisfies the Jacobi identity in the sense that 
\[
[X,[Y,Z]] + [Y,[Z,X]] + [Z,[X,Y]] = 0.
\]
\item[(iv)]  If $f,g \in C^\infty(M),$ then 
\[
[fX, gY] = fg[X,Y] + f X(g)Y - gY(f)X.
\]
\end{enumerate}
\end{proposition}

\begin{remark}
The first three properties show that $(C^\infty(M,TM), [-,-])$ is a Lie algebra over $\mathbb{R}$. 
\end{remark}

\begin{proof}[Proof of Proposition \ref{Lie-bracket}]
(i) and (ii) are clear from definition. It is straightforward to check (iii) and (iv); you will be asked
to verify (iii) in Assignment 4 (1). 
\end{proof}

We now discuss the differential in terms of derivations. 

\begin{definition}
Let $F: M \to N$ be a $C^k$ map between $C^k$ manifolds. Let $l$ be a positive integer satisfying $l \le k$. 
Then $F$ induces a map $F^* : C^l(N) \to C^l(M)$ called the {\em pullback}  defined by the rule $f \mapsto  f \circ F$. 
If $p$ is a point in $M$, we get a map $F_p^* : C_{F(p)}^l(N) \to C_p^l(M)$ defined by $[(V,f)] \mapsto [(F^{-1}(V),  f \circ F)]$. 
\end{definition}

\begin{remark}
If $M$ and $N$ are $C^k$ manifolds and $F:M \to N$ is a continuous map, then for each $p \in M$, we get a map 
$F^*_p : C^0_{F(p)}(N) \to C^0_p(M)$. Then $F$ is a $C^k$ map if and only if for each $p$ in $M$, 
the image $F_p^*(C^k_{F(p)}(N))$ is a subring of $C_p^k(M)$. We may use this to define $C^k$ maps. 
(cf. Roundtable on September 25, and Well's {\em Differential Analysis on Complex Manifolds}, Chapter I)
\end{remark}

\begin{lemma} \label{dF-derivation}
Let $F : M \to N$ be a smooth map between smooth manifolds. For each point $p$ in $M$, the map 
$dF_p : T_pM= D_pM \to T_{F(p)}N= D_{F(p)}N$ is given by 
\begin{equation}\label{eqn:dF-derivation}
dF_p(X)f = X(F^*f)
\end{equation}
for any $X \in T_pM= D_pM$ and $f \in C^\infty_{F(p)}(N)$. 
\end{lemma}

\begin{proof}
This follows from the chain rule. Passing to local coordinates, we may assume that $M$ is an open subset of $\mathbb{R}^m$,  
$N$ is an open subset of $\mathbb{R}^n$, $p = 0$, and $F(p) = 0$. We write $F(x) = (y_1(x), \ldots, y_n(x))$. 
Then any derivation $X\in D_0\bR^m$ is given by $X = \sum_{i=1}^ma_i \frac{\partial}{\partial x_i}(0)$. Note that 
\[
dF_p(X) = \sum_{j=1}^n \left(\sum_{i=1}^m \frac{\partial y_j}{\partial x_i}(0) a_i\right) \frac{\partial}{\partial y_j}(0).
\]
The LHS and RHS of \eqref{eqn:dF-derivation} are
$$
\mathrm{LHS}=dF_p(X)f =\sum_{i=1}^m\sum_{j=1}^n a_i\frac{\partial y_j}{\partial x_i}(0)\frac{\partial f}{\partial y_j}(0),\quad\ 
\mathrm{RHS}= \sum_{i=1}^m a_i \frac{\partial}{\partial x_i}(f\circ F)(0).
$$
which are equal by the chain rule. 
\end{proof}

\begin{remark}
We may use \eqref{eqn:dF-derivation} to {\em define} $dF_p$. 
\end{remark}


\begin{definition}
Let $M$ be a smooth manifold. A {\em smooth curve in $M$} is a smooth map $\gamma : (a,b) \to M$ where $-\infty \le a < b \le +\infty$. 
\end{definition}

\begin{notation}
For any $t \in (a,b)$, let $\gamma'(t)$ (or $\frac{d\gamma}{dt}(t)$) denote the tangent vector 
$d\gamma_t\left(\frac{\partial}{\partial t}\right) \in T_{\gamma(t)}M$. 
\end{notation}

\begin{example}
If $M = \mathbb{R}^n$, then a smooth map $\gamma : (a,b) \to M$ is given by 
$$
\gamma(t)=(x_1(t),\ldots x_n(t))
$$
where $x_i:(a,b)\to \bR$ are $C^\infty$ function on $(a,b)$.
$$
\gamma'(t)=(x_1'(t),\ldots, x_n'(t))=\sum_{i=1}^n x_i'(t)\frac{\partial}{\partial x_i}(\gamma(t)).
$$
\end{example}

\begin{lemma}
Let $M$ be a smooth manifold and let $\gamma : (-\epsilon, \epsilon) \to M$  be a smooth curve. Let $\gamma(0) = p$. Then $\gamma'(0)$ is a derivation at $p$ given by 
\begin{equation}\label{eqn:gamma-derivation}
\gamma'(0)f = \frac{d}{dt}(f \circ \gamma)|_{t=0} 
\end{equation}
\end{lemma}
\begin{proof}
This is a special case of Lemma \ref{dF-derivation}.
\end{proof}

\begin{remark} 
do Carmo uses \eqref{eqn:gamma-derivation} to {\em define} a derivation $\gamma'(0):C^\infty_p(M) \to \mathbb{R}$ for
each smooth curve passing through $p\in M$ at $t=0$. The tangent space $T_pM$ is {\em defined} to be the collection of such $\gamma'(0)$. 
Under this definition, the differential $dF_p:T_p M\to T_{F(p)}N$ of a smooth map $F: M \to N$ at $p\in M$ is defined by 
\[
\gamma'(0) \mapsto (F \circ \gamma)'(0).
\]
\end{remark}


\section{Wednesday, September 30, 2015}

\noindent
{\bf\large Integral Curves}

\begin{definition}
Let $X$ be a smooth vector field on a smooth manifold $M$ and let $\gamma : I \to M$ be a smooth curve. We say that $\gamma$ is an 
{\em integral curve} of $X$ if $\gamma'(t) = X(\gamma(t))$ for all $t \in I$. 
\end{definition}

\begin{example} $M = \mathbb{R}^n$ 
$$
\gamma(t) = (x_1(t), \ldots, x_n(t))
$$
where $x_i:I\to \bR$ are smooth functions on $I$. A smooth vector field on $\bR^n$ is
of the form
$$
X(x) = (a_1(x), \ldots, a_n(x)) =\sum_i a_i(x) \frac{\partial}{\partial x_i}
$$
where $a_i$ are smooth functions on $\bR^n$, so $X$ can be viewed as a smooth map from 
$\bR^n$ to $\bR^n$. The statement that $\gamma$ is an integral curve of $X$ is equivalent to a system of ODE's given by 
\[
\frac{d x_i}{dt}(t) = a_i(x_1(t), \ldots, x_n(t)),\quad i=1,\ldots,n.
\]
\end{example}

\begin{theorem}\label{flow-existence}
Let $M$ be a smooth manifold and let $X$ be a smooth vector field on $M$.
\begin{enumerate}
\item[(i)] For any point $p \in M$, there is an open interval $I_p \subset \mathbb{R}$ containing $0$ and an integral curve $\phi_p : I_p \to M$ of $X$ such that $\phi_p(0) = p$ and  $I_p$ is a maximal interval for such a $\phi_p$. Moreover, this integral curve is unique in the following sense. If $\gamma : I' \to M$ is an integral curve of $X$ on an open interval $I'$ containing $0$ such that $\gamma(0) = p$, then $I' \subset I_p$ and $\gamma = \phi_p|_{I'}$. 
\item[(ii)] For any $p \in M$ there is 
\begin{itemize}
\item an open neighborhood $U$ of $p$ in $M$
\item an open interval $I$ of $0$ in $\mathbb{R}$
\item a smooth map $\phi : I \times U \to M$
\end{itemize}
such that 
$$
\begin{cases}
\frac{\partial \phi}{\partial t}(t,q) =& X(\phi(t,q))\\
\phi(0,q)=&q
\end{cases}
$$
\end{enumerate}
\end{theorem}

\begin{proof}
We may assume $M = \mathbb{R}^n$ and $p = 0$. Then the proof becomes one in ODE's. Reference: Boothby Chapter IV. 
\end{proof}

\begin{example}
If $M = \mathbb{R}^n$ and $ p =(a_1, \ldots, a_n)$. Suppose that $X$ is the identity vector field,
i.e. $X(\vec{x})=\vec{x}$ for all $\vec{x}=(x_1,\ldots,x_n) \in \bR^n$.  
Then the integral curves are straight lines emanating from the origin. In terms of local coordinates, 
\[
\begin{cases}
\frac{d x_i}{dt} & = x_i \\
x_i(0) & = a_i
\end{cases}
\quad i=1,\ldots,n,
\]
which implies $x_i(t) = a_ie^t$. And $\phi : \bR \times \bR^n \to \bR^n$ 
is given by $\phi(t, x_1, \ldots, x_n) = (x_1 e^{t}, \ldots, x_n e^t)$, or equivalently,
$\phi(t,\vec{x})= e^t\vec{x}$.
\end{example}

\begin{example}
Let $M = \{\vec{x} \in \bR^n : |\vec{x}| < 1\}$ and again $X$ is the identity vector field. 
If $p=\vec{a}=(a_1,\ldots,a_n)$ then $\phi_p: I_p \to \bR^n$ is given by 
$\phi_p(t)= e^t \vec{a}$, where $I_p=(-\infty,-\log|\vec{a}|)$. 
\end{example}

\begin{remark}\label{tzero}
If $q = \phi_p(t_0)$, then $\phi_q(t) = \phi_p(t + t_0)$. 
\end{remark}

Now we change our point of view. Instead of fixing $p$, we fix time $t$ in the function $\phi(t,p)$. Define $\phi_t : U \to M$ by the rule $\phi_t(q) = \phi(t,q)$. We should think of this as telling us where points in $M$ get mapped after flowing a certain time $t$. Because of this, we call $\phi_t$ the {\em local flow} of $X$. 

\begin{remark}
By the previous remark (Remark \ref{tzero}), we find that $\phi_{t_1} \circ \phi_{t_2} = \phi_{t_1 + t_2}$ when both hand sides of the equality are
defined.  
\end{remark}

\begin{lemma}\label{flow-compact}
Let $X$ be a smooth vector field on a smooth manifold $M$ such that the support of $X$ is compact. Recall that the support of $X$ is 
\[
\textup{Supp}(X) = \overline{\{p \in M : X(p) \ne 0\}}.
\]
Then there exists a unique smooth map $\phi : \mathbb{R} \times M \to M$ such that 
\begin{equation}\label{phi-t-q}
\frac{\partial \phi}{\partial t}(t,q) = X(\phi(t,q)), \quad
\phi(0,q) = q.
\end{equation}
(In other words, we have a {\em global} flow $\phi_t: M\to M$ which exists for all time $t\in \bR$.)
\end{lemma}

\begin{proof} It suffices to prove the existence; the uniqueness follows from part (i) of Theorem \ref{flow-existence}. 
Let $K=\mathrm{Supp}(X)$.

\noindent
1. The set $V:= M\setminus K$ is open, and $X(q)=0$ for $q\in V$. Define
$\phi: \bR\times V \to M$ by $\phi(t,q)=q$. Then $\phi$ is smooth, and it satisfies
$$
\frac{\partial \phi}{\partial t}(t,q)=0 = X(q)= X(\phi(t,q)),\quad \phi(0,q)=q.
$$

\noindent
2. Given any $p\in K$, by Theorem \ref{flow-existence} (ii), there exists an open neighbhood $U_p$ of $p$ in $M$ and a positive number $\ep_p>0$ such that
there is a $C^\infty$ map $\psi_p:(-\ep_p,\ep_p)\times U_p\to M$ satisfying
$$
\frac{\partial \psi_p}{\partial t}(t,q) = X(\psi_p(t,q)),\quad \psi_p(0,q) = q.
$$ 
Moreover, if $p_1,p_2\in K$ and $U_{p_1}\cap U_{p_2}\neq \emptyset$ then part (i) of Theorem \ref{flow-existence} implies
$$
\psi_{p_1}|_{(-\ep,\ep)\times (U_{p_1}\cap U_{p_2})}= \psi_{p_2}|_{(-\ep,\ep)\times (U_{p_1}\cap U_{p_2}) }
$$
where $\ep=\min\{\ep_{p_1},\ep_{p_2}\}>0$. So we obtain a smooth map 
$\psi(t,q)$ defined on $(-\ep,\ep)\times (U_{p_1}\cup U_{p_2})$. 

$K$ is compact and $K\subset \cup_{p\in K}U_p$, so there are finitely many $p_1,\ldots, p_N \in K$ such that
$K\subset \cup_{i=1}^N U_{p_i}$. Let $\ep:= \min\{\ep_{p_1},\ldots,\ep_{p_N}\} >0$, and define
$U:= \cup_{i=1}^N U_{p_i}$. Then we obtain a smooth map $\psi: (-\ep,\ep)\times U \to M$ satisfying 
$$
\frac{\partial \psi}{\partial t}(t,q) = X(\psi(t,q)),\quad \psi(0,q) = q.
$$


\noindent
3. By part (i) of Theorem \ref{flow-existence}, $\phi|_{(-\ep,\ep)\times (U\cap V)} = \psi|_{(-\ep,\ep)\times (U\cap V)}$, where
$\phi:\bR\times V\to M$ is defined in Step 1 above  and $\psi:(-\ep,\ep)\times U\to M$ is defined in Step 2 above.
We also have $U\cup V=M$, so we obtain a smooth map $\phi: (-\ep,\ep)\times M\to M$
satisfying \eqref{phi-t-q}.

\noindent
4. For any $t\in \bR$, there exists a positive integer $n$ such that $|t|< n \ep$. Define
$$
\phi(t,q) := \underbrace{\phi(\frac{t}{n}, \phi(\frac{t}{n}, \cdots \phi(\frac{t}{n},}_{n\textup{ times}}\ q\  
\underbrace{))\cdots)}_{n\textup{ times}}
$$
where $q\in M$; the definition is independent of choice of $n>|t|$.  Then
$\phi:\bR\times M\to M$ is a smooth map satisfying \eqref{phi-t-q}.  
\end{proof}

If $\phi_t$ is defined on all of $M$ and for all $t \in \bR$, then we have a group homomorphism 
$(\bR, +) \to (\text{Diff}(M), \circ)$ defined by $t \mapsto \phi_t$. In particular, $\phi_0$ is the identity map. 
The inverse of $\phi_t$ is the map $\phi_{-t}$. The image of this group homomorphism lies in the connected component of the 
identity diffeomorphism, since $\bR$ is connected. 


\bigskip

\noindent{\bf \large Flow and Lie derivative}

Let $M$ be a smooth manifold and let $X$ be a smooth vector field. We have defined the Lie derivative of $X$ by the rule $L_X(f)(p) = X(p)(f)$. Recall that $L_X : C^\infty(M) \to C^{\infty}(M)$ is $\mathbb{R}$-linear and satisfies the Leibniz rule. Now we want to extend $L_X$ to a map $L_X : C^\infty(M,TM) \to C^\infty(M,TM)$. 

\begin{definition}
We define $L_X : C^\infty(M, TM) \to C^\infty(M,TM)$ by the rule 
\[
L_X(Y) = [X,Y].
\]
\end{definition}

Then $L_X$ is an $\bR$-linear map. Moreover, it satisfies the following Leibniz rule:  
\[
L_X(fY) = L_X(f)Y + f(L_X(Y))
\]
for any smooth function $f$ and any vector fields $Y$ on $M$. 

\begin{remark}We have a few remarks. 
\begin{itemize}
\item If we consider $L_X : C^\infty(M) \to C^\infty(M)$, then we can see that $L_{fX} = fL_X$
if $f\in C^\infty(M)$ and $X\in C^\infty(M,TM)$.  
So the operator $L_X$ on $C^\infty(M)$ is $C^\infty(M)$-linear in $X$.  
\item If we consider $L_X : C^\infty(M,TM) \to C^\infty(M,TM)$, then we can see that 
\[
L_{fX}(Y) = [fX,Y] = f[X,Y] - Y(f)X = fL_X(Y) - Y(f)X.
\]
So the operator $L_X$ on $C^\infty(M,TM)$ is $\bR$-linear but not $C^\infty(M)$-linear in $X$.
\end{itemize}
\end{remark}

We now discuss the pushforward and pullback of a vector field under a diffeomorphism. 

\begin{definition}
Let $F : M \to N$ be a smooth diffeomorphism and let $X$ be a smooth vector field on $M$. 
Then we define the {\em pushforward} $F_*X$ to be the smooth vector field on $N$ defined by 
\[
F_*X(p) =  (dF)_{F^{-1}(p)}(X(F^{-1}(p))). 
\]
Given a smooth vector field $Y$ on $N$, we define the {\em pullback} of $Y$ to be $F^*Y = (F^{-1})_*(Y)$,
which is a smooth vector field on $M$. 
\end{definition}

\begin{proposition}
Let $X$ be a smooth vector field on a smooth manifold $M$. Let $p$ be a point of $M$. 
By Theorem \ref{flow-existence} (ii), there is an open neighborhood $U$ of $p$ in $M$ and a local 
flow $\phi_t : U \to M$ of $X$ for $t$ in some small neighborhood $(-\epsilon, \epsilon)$ of $0$. Then 
\begin{enumerate}
\item[(a)] For each $f \in C_p^\infty(M)$, we compute that 
\[
(L_Xf)(p) = \frac{d}{dt}\Big|_{t=0} (\phi_t^*f)(p) = \frac{d}{dt}\Big|_{t=0}  (f \circ \phi_t)(p).
\]
\item[(b)] For a smooth vector field $Y$ defined an on open neighborhood $V$ of $p$ in $U$, we compute that 
\[
(L_XY)(p) = - \frac{d}{dt}\Big|_{t=0}((\phi_t)_*Y)(p) = \lim_{t\to 0}\frac{Y(p) - ((\phi_t)_*Y)(p)}{t}.
\]
\end{enumerate}
\end{proposition}

\begin{proof}
(a) We compute 
\begin{align*}
\frac{d}{dt}\Big|_{t=0} (f \circ \phi_t)(p) =& \frac{d}{dt}\Big|_{t=0} f(\phi_t(p)) = \frac{d}{dt}\Big|_{t=0} f(\phi_p(t)) \\
= & \frac{d}{dt}\Big|_{t=0} (f \circ \phi_p)(t) = \phi_p'(0)f = X(p)f.
\end{align*}
(b) It suffices to show that for any $f \in C_p^\infty(M)$, we have 
\[
[X,Y](p)f = - \frac{d}{dt}\Big|_{t=0} ((\phi_t)_* Y)(p)f.
\]

We then compute 
$$
((\phi_t)_* Y)(p)f  = (d\phi_t)_{\phi_{-t}(p)} (Y(\phi_{-t}(p)))f 
= Y(\phi_{-t}(p))(f \circ \phi_t),
$$
where the second equality follows from Lemma \ref{dF-derivation}.
Let $h(t,q) = f \circ \phi_t(q) - f(q)$. Then note that $h$ is a smooth map from $(-\delta, \delta) \times V \to \mathbb{R}$ for some small $\delta$ and some open neighborhood $V$ of $p$ in $M$. Then $h(0,q) = 0$ for all $q \in V$. By Lemma \ref{h-hzero} below, we may write 
\[
h(t,q) = tg(t,q)
\]
where $g : (-\delta, \delta) \times V \to \mathbb{R}$ is some smooth function. 
Define $g_t : V \to \bR$ by the rule $g_t(q) = g(t,q)$. Then $g_t\in C^\infty(V)$. 
By part (a), 
\[
(L_Xf)(q) = \frac{d}{dt}\Big|_{t=0} (f \circ \phi_t)(q) =
\lim_{t\to 0}\frac{f\circ\phi_t(q)-f(q)}{t}= \lim_{t\to 0} g(t,q)= g(0,q)=g_0(q).
\]
It follows that $g_0 = Xf \in C_p^\infty(M)$. Then we find that 
$$
Y(\phi_{-t}(p))(f \circ \phi_t) = Y(\phi_{-t}(p))(f + tg_t) 
= Y(\phi_{-t}(p))(f) + tY(\phi_{-t}(p))(g_t).
$$
Let $r(t) = Y(\phi_{-t}(p))(g_t)$, which is a smooth function in one variable $t$. Then we find 
$$
Y(\phi_{-t}(p))(f \circ \phi_t) = (Yf)(\phi_{-t}(p)) + t\cdot r(t). 
$$
We now differentiate to find 
\begin{align*}
\frac{d}{dt}\Big|_{t=0} ((\phi_t)_*Y)(p)f =& \frac{d}{dt}\Big|_{t=0} (Yf)\circ \phi_{-t}(p) + r(0) =  -X(p)(Yf) + Y(p)g_0 \\
=& -X(p)(Yf) + Y(p)(Xf) = -[X,Y](p)f
\end{align*}
as desired. 
\end{proof}

\begin{lemma}\label{h-hzero}
Let $\delta$ be a small positive number, let $U$ be an open subset of $M$, and let $h : (-\delta, \delta) \times U \to \mathbb{R}$ be smooth. Suppose that $h(0,q) = 0$ for any $q \in U$. Then $h(t,q) = t g(t,q)$ for some smooth function $g : (-\delta, \delta) \times U \to \mathbb{R}$.
\end{lemma}

\begin{proof}
Fix $t,q$. Let $u(s) = h(st, q)$. Then $u(s)$ is $C^\infty$ function of one variable $s$. 
\begin{align*}
h(t,q) &= h(t,q) - h(0,q) = u(1) - u(0) = \int_{0}^1 u'(s)ds = \int_0^1 t \frac{\partial h}{\partial t}(st,q)ds \\
&= t \int_0^1 \frac{\partial h}{\partial t}(st,q)ds = tg(t,q).
\end{align*}
where 
$$
g(t,q):= \int_0^1 \frac{\partial h}{\partial t}(st,q)ds
$$
is a $C^\infty$ function in $(t,q)$ since $h$ is.  
\end{proof}



\section{Monday, October 5, 2015}


\begin{definition}[Subbundle]
Let $\pi : E \to M$ be a smooth vector bundle of rank $r$ over $M$. A subset $F$ of $E$ is called a 
{\em smooth subbundle of rank $k$} if for any $p \in M$, there is an open neighborhood $U$ of $p$ in $M$ and a local trivialization  $h : \pi^{-1}(U) \to U \times \bR^r$ such that $h\left(F \cap \pi^{-1}(U)\right) = U \cap (\bR^k \times \{0\})$. 
\end{definition}

\begin{remark}
We have some remarks. 
\begin{enumerate}
\item[(i)] For any $p \in M$, the fiber $F_p = F \cap E_p$ is a $k$-dimensional subspace of $E_p$. Moreover, $F_p$ depends smoothly on the choice of $p$. 
\item[(ii)] The map $\pi|_F : F \to M$ is a smooth vector bundle of rank $k$ over $M$. Moreover, the transition functions
$g^F_{\beta\alpha}$  for this vector bundle are found by restricting the transition functions $g^E_{\beta\alpha}$ for $E$:
for $x\in U_\alpha\cap U_\beta$, 
$$
g_{\beta\alpha}^E(x) = \left [\begin{array}{cc} g_{\beta\alpha}^F(x) & \star \\ 0 & \star \end{array}\right]
\in GL(r,\bR)
$$ 
where $g_{\beta\alpha}^F(x)\in GL(k,\bR)$.
\end{enumerate}
\end{remark}

\begin{proposition}\label{subbundle}
Let $\pi : E \to M$ be a smooth vector bundle of rank $r$ over a smooth manifold $M$. Let $\{ F_p: p\in M\}$ be 
a collection of $k$-dimensional linear subspaces $F_p$ of $E_p$ and set $F = \cup_p F_p \subset E$. 
Then $F$ is a smooth subbundle of $E$ of rank $k$ if and only if for each $p \in M$, there is an open neighborhood 
$U$ of $p$ in $M$ and smooth sections $s_1, \ldots, s_k$ of $\pi : \pi^{-1}(U)=E|_U \to U$ 
such that for each $q \in U$, the collection $\{s_i(q)\}_{i=1}^k$ form a basis of $F_q$. 
\end{proposition}




\begin{example}
The universal line bundle 
\[
E = \{(l,v) : l \in P_n(\mathbb{R}) , v \in l\} \subset P_n(\bR) \times \bR^{n+1}
\]
is a smooth subbundle of the product bundle. For any $l\in P_n(\bR)$, 
$l\in U_i$ for some $i\in \{1,\ldots,n+1\}$, where $U_i=\{ [x_1,\ldots, x_{n+1}]\in P_n(\bR): x_i\neq 0\}$.
On $U_i$, we define $s_i:U_i\to E|_{U_i}$ by 
$$
s_i([y_1,\ldots, y_{i-1},1, y_i,\ldots, y_n]) = ([y_1,\ldots, y_{i-1},1, y_i,\ldots, y_n], 
(y_1,\ldots,y_{i-1},1,y_i,\ldots, y_n).
$$
Then $s_i$ is a smooth section of $U_i\times \bR^{n+1}\to U_i$, and 
$E_l = \bR s_i(l)$ for any $l\in U_i$.  By Proposition \ref{subbundle},
$E$ is a rank 1 smooth subbundle of $P_n(\bR)\times \bR^{n+1}$.

\end{example}


\begin{definition}[Distribution]
Let $M$ be a smooth manifold. A {\em smooth distribution of dimension $k$ on $M$} 
is a collection $\{F_p \subset T_pM : p \in M\}$ of $k$-dimensional subspaces $F_p$ of 
$T_pM$ such that $F = \cup_pF_p$ is a smooth subbundle of rank $k$ of $TM$. 
\end{definition}

\begin{remark}
By Proposition \ref{subbundle}, a collection $\{F_p \subset T_pM : p \in M\}$ of $k$-dimensional subspaces $F_p$ of $T_pM$ 
is a smooth distribution if and only if for each $p \in M$, there is an open neighborhood $U$ of $p$ and smooth vector fields $X_1, \ldots, X_k$ on $U$ such that for each $q \in U$, the list $\{X_1(q), \ldots, X_k(q)\}$ forms a basis for $F_q$. 
\end{remark}

\begin{remark}
Let $C^\infty(M,F)$ denote the space of smooth sections of the subbundle $F \to M$. 
Note that $C^\infty(M, F)$ is a $C^\infty(M)$-submodule of the space $C^\infty(M,TM)$ of smooth sections of $TM$, 
that is, the space of smooth vector fields on $M$. 
\end{remark}

\begin{definition}
Let $F$ be a smooth distribution of dimension $k$ on a smooth manifold $M$ of dimension $n$. 
\begin{enumerate}
\item[(i)] We say that $F$ is {\em involutive} if $C^\infty(M,F)$ is a Lie subalgebra of 
           $(C^\infty(M,TM), [-,-])$. 
\item[(ii)] We say that $F$ is {\em completely integrable} if for each $p$ in $M$, there is a chart 
$(U,\phi)$ for $M$ around $p$ such that for each $q \in U$, the subspace $F_q$ is spanned by the list 
$\{\frac{\partial}{\partial x_1}(q), \ldots, \frac{\partial}{\partial x_k}(q)\}$, where
$(x_1, \ldots, x_n)$ are local coordinates on $U$. 
\end{enumerate}
\end{definition}

\begin{remark}
Note that $F$ is completely integrable if and only if for each $p \in M$, there is a $k$-dimensional 
submanifold $S \subset M$ such that $p\in S$ and for any $q \in S$, the subspace $T_qS = F_q$. 
\end{remark}

\begin{example}
We see that a smooth distribution $F$ has the same dimension as $M$ if and only if $F = TM$. 
And of course  $F$ is involutive and completely integrable. 
\end{example}

\begin{example}
If the dimension of $F$ is $1$, then $F$ is both involutive and completely integrable. 
For each point $p \in M$, there is an open neighborhood $U$ of $p$ in $M$ and a smooth vector field $X$ on $U$ such that $F_q = \mathbb{R}X(q)$ for each $q \in U$. There is an integral curve of $X$ on this neighborhood showing that $F$ is completely integrable. Moreover, to see that $F$ is involutive, we note that any smooth section of $F$ is locally a multiple of $X$ and hence 
\[
[fX, gX] = (fX(g) - gX(f))X.
\]  
\end{example}

\begin{lemma}
If $F$ is completely integrable then $F$ is involutive. 
\end{lemma}

\begin{proof}
Suppose that $X$ and $Y$ are smooth sections of $F$. On a coordinate chart $(U,\phi)$, we may write 
\begin{align*}
X &= \sum_{i=1}^k a_i \frac{\partial}{\partial x_i} \\
Y &= \sum_{i=1}^k b_i \frac{\partial}{\partial x_i}
\end{align*}
for some smooth functions $a_i, b_i \in C^\infty(U)$. Then we compute that
\[
[X,Y] = \sum_{i,j=1}^k \left(a_i \frac{\partial b_j}{\partial x_i} - b_i \frac{\partial a_j}{\partial x_i}\right) \frac{\partial}{\partial x_j}
\]
belongs to the span of $\{\frac{\partial}{\partial x_1}(q), \ldots, \frac{\partial}{\partial x_k}(q)\}$. 
\end{proof}



The converse is also true:
\begin{theorem}[Frobenius]
A smooth distribution $F$ on a smooth manifold is completely integrable if and only if $F$ is involutive. 
\end{theorem}

\begin{proof}
A reference is \cite[Chapter IV, Section 8]{Bo}. 
\end{proof}


\noindent
{\bf \large Operations on vector bundles}

Let $\pi : E \to M$ be a smooth vector bundle of rank $r$ over a smooth manifold $M$. We will construct a smooth vector bundle $\pi^* : E^* \to M$ called the dual bundle, whose fibers are given by $E^*_p = (E_p)^*$. 


Let $\pi : E \to M$ be a smooth vector bundle of rank $r$ over a smooth manifold $M$. Let $E^*$ denote the set 
\[
E^* = \bigcup_{p \in M}E_p^*.
\]
Define $\pi^*:E^*\to M$ such that $\pi^*(E^*_p)=\{p\}$. 
We wish to equip $E^*$ with the structure of a smooth manifold. 

\noindent
1. Suppose that $\{ U_\alpha:\alpha\in I\}$ is an open cover of $M$ and
$h_\alpha^E :\pi^{-1}(U_\alpha)= E|_{U_\alpha}\to U_\alpha\times \bR^r$ are local trivializations of $E$.
Let $\{e_1,\ldots,e_r\}$ be the standard basis of $\bR^r$, and define
$s_{\alpha i}: U_\alpha\to \pi^{-1}(U_\alpha)$ by $h_\alpha^{-1}(x,e_i)$. Then
$\{ s_{\alpha 1},\ldots, s_{\alpha r}\}$ is a $C^\infty$ frame of $E|_{U_\alpha}\to U_\alpha$. 
Suppose that $U_\alpha\cap U_\beta\neq \emptyset$. Then there exists a $C^\infty$ map
$g_{\beta\alpha}^E:U_\alpha\cap U_\beta\to GL(r,\bR)$ such that 
$$
s_{\alpha j}(x)=\sum_{i=1}^r s_{\beta i}(x) g_{\beta\alpha}^E(x)_{ij}.
$$
The transition function $h^E_\beta\circ (h^E_\alpha)^{-1}: (U_\alpha\cap U_\beta)\times \bR^r \to (U_\alpha\cap U_\beta)\times \bR^r$ is
given by 
\begin{eqnarray*}
h_\beta^E \circ (h_\alpha^E)^{-1}(x,v) &=& (h_\beta^E)(x,\sum_{j=1}^r v_j s_{\alpha j}(x))
=h_\beta^E(x, \sum_{i,j=1}^r v_j s_{\beta i}(x) g_{\beta\alpha}^E(x)_{ij}) \\
&=& h_\beta^E (x, \sum_{i=1}^r u_i s_{\beta i}(x))= (x, u_i)
\end{eqnarray*}
where $u_i =\sum_{j=1}^r g^E_{\beta\alpha}(x)_{ij} v_j$. So the transition function is given by 
$$
h^E_\beta\circ (h_\alpha^E)^{-1}(x,v)= (x,g^E_{\beta\alpha}(x) v).
$$

\noindent
2. Let $\Gamma(U_\alpha, E^*|_{U_\alpha})$ denote the
set of maps $s: U_\alpha \to (\pi^*)^{-1}(U_\alpha)= \cup_{x \in U_\alpha} E^*_x$ such that
$s(x)\in E^*_x$. For any $x\in U_\alpha$, let
$\{s_{\alpha 1}^*(x),\ldots, s_{\alpha r}^*(x)\}$  be the basis of $E^*_x$ dual to
the basis $\{s_{\alpha 1}(x),\ldots, s_{\alpha r}(x)\}$ of $E_x$:
$$
\langle s_{\alpha i}^*(x),s_{\alpha j}(x)\rangle = \delta_{ij}.
$$ 
Then $s_{\alpha 1}^*,\ldots, s_{\beta r}^* \in \Gamma(U_\alpha, E^*|_{U_\alpha})$, and there is a bijection 
$$
\Phi_\alpha: U_\alpha\times \bR^r \to (\pi^*)^{-1}(U_\alpha),\quad
(x,v)\mapsto (x, \sum_{i=1}^r v_i s_{\alpha i}^*(x)). 
$$
We equip $(\pi^*)^{-1}(U_\alpha)$ with the topological structure and $C^\infty$ structure
such that the bijection $\Phi_\alpha$ is a $C^\infty$ diffeomorphism. Define
$h^{E^*}_\alpha:=\Phi_\alpha^{-1}: (\pi^*)^{-1}(U_\alpha)\to U_\alpha \times \bR^r$.
Then $\pi^*|_{(\pi^*)^{-1}(U_\alpha)} = \mathrm{pr}_1\circ h^{E^*}_\alpha$ and
$h^{E^*}_\alpha|_{E^*_x}$ is a linear isomorphism from $E_x$ to $\{x\}\times \bR^r \cong \bR^r$ 
for all $x\in U_\alpha$

\noindent
3. Suppose that  $U_\alpha \cap U_\beta \neq \emptyset$.
$$
s_{\beta i}^*(x) =\sum_{i=1}^r \langle s_{\beta i}^*(x), s_{\alpha j}(x)\rangle s_{\alpha j}^*(x)
=\sum_{j=1}^r g_{\beta\alpha}^{E}(x)_{ij} s_{\alpha j}^*(x) = \sum_{j=1}^r s_{\alpha j}^*(x) \left(g_{\beta\alpha}^{E}(x)^T \right)_{ji}. 
$$
where $A^T$ denote the transpose of $A$. Therefore,
$$
h^{E^*}_\alpha\circ (h^{E^*}_\beta)^{-1}: (U_\alpha\cap U_\beta)\times\bR^r
\to (U_\alpha\cap U_\beta)\times \bR^r
$$
is given by $h^{E^*}_\alpha\circ (h^{E^*}_\beta)^{-1}(x,v)= (x, g^E_{\beta\alpha}(x)^T v)$. Its inverse
map 
$$
h^{E^*}_\beta \circ (h^{E^*}_\alpha)^{-1}: (U_\alpha\cap U_\beta)\times\bR^r
\to (U_\alpha\cap U_\beta)\times \bR^r
$$
is given by 
\begin{equation}\label{eqn:Edual}
h^{E^*}_\beta \circ (h^{E^*}_\alpha)^{-1}(x,v)= (x, (g^E_{\beta\alpha}(x)^T)^{-1} v)
\end{equation}
which is a $C^\infty$ diffeomorphism. This shows that the topological structures and $C^\infty$ structures on $(\pi^*)^{-1}(U_\alpha)$ and $(\pi^*)^{-1}(U_\beta)$ defined in Step 2 
coincide on their intersection $(\pi^*)^{-1}(U_\alpha\cap U_\beta)$, so we obtain 
the structure of a $C^\infty$ manifold on $E^*$.
Indeed, by shrinking $U_\alpha$ we may assume that there is a $C^\infty$  atlas on $M$
of the form $\{ (U_\alpha, \phi_\alpha): \alpha\in I\}$.
Define 
$$
\tilde{\phi}_\alpha:(\pi^*)^{-1}(U_\alpha)\to \phi_\alpha(U_\alpha) \times \bR^r,
\quad \tilde{\phi}_\alpha(x,\sum_{i=1}^r v_i s^*_{\alpha i}(x)) = (\phi_\alpha(x), (v_1,\ldots, v_r)).
$$ 
Then  $\{ \Big((\pi^*)^{-1}(U_\alpha), \tilde{\phi}_\alpha\Big):\alpha\in I\}$ is
a $C^\infty$ atlas for $E^*$. Moreover, $h_\alpha^{E^*}$ and 
$h_\beta^{E^*}\circ h_\alpha^{E^*}$ satisfy 
(i) and (ii) in Definition \ref{vector-bundle}, respectively.
Finally, \eqref{eqn:Edual} tells us
$g^{E^*}_{\beta \alpha}(x)= (g^E_{\beta \alpha}(x)^T)^{-1}$ for $x\in U_\alpha\cap U_\beta$.

\begin{remark}
The $C^\infty$ structure on $E^*$ is characterized as follows. 
Let $\Gamma(M,E^*)$ denote the set of maps $\phi: M\to E^*=\cup_{x\in M} E^*_x$ such that $\phi(x)\in E_x^*$.
We say $\phi\in \Gamma(M,E^*)$ is a smooth section of $E^*\to M$ if, for every smooth section 
$s: M\to E$, the function $\langle \phi,s\rangle: M\to \bR$ is smooth. Equivalently, given 
$C^\infty$ frame $\{ s_{\alpha 1},\ldots, s_{\alpha r}\}$ of $E|_{U_\alpha}$, we declare
that $\{ s_{\alpha 1}^*,\ldots, s_{\alpha r}^*\}$ is a $C^\infty$ frame of
$E^*|_{U_\alpha}$. For any $\phi\in \Gamma(U_\alpha, E^*|_{U_\alpha})$ we may write
$$
\phi(x) =\sum_{i=1}^r a_i(x) s_{\alpha i}^*(x),\quad x\in U_\alpha.
$$
$\phi$ is a smooth section, i.e., $\phi\in C^\infty(U_\alpha,E^*|_{U_\alpha})$, if and 
only if $a_1,\ldots, a_r$ are smooth functions on $U_\alpha$. 
\end{remark}





Let $F$ be another smooth vector bundle over $M$. We may apply operations on vector spaces to construct new smooth vector bundles. For example, we can construct $E \oplus F$ and $E \otimes F$ whose fibers are given by $E_p \oplus F_p$ and $E_p \otimes F_p$ respectively. As another example, we can take $\text{Hom}(E,F)$ whose fibers are given by $\text{Hom}(E,F)_p = \text{Hom}(E_p, F_p)$. Note that $\text{Hom}(E,F) \simeq E^* \otimes F$. 
We can also take the $k$-th exterior power $\Lambda^kE$, where $k\leq r$. 

In each above example, the smooth structure is given by the following. For each point $p \in M$, we take a neighborhood $U$ of $p$ such that there is a  $C^\infty$ frame $\{ e_1, \ldots, e_r\}$ for $E|_U$ and 
a $C^\infty$ frame $\{ f_1, \ldots, f_s\}$ for $F|_U$. 
\begin{itemize}
\item The dual frame $\{e_1^*, \ldots, e_r^*\}$  is a $C^\infty$ frame for $E^*|_U$. 
\item $\{ e_1, \ldots, e_r, f_1, \ldots, f_s\}$, we get a $C^\infty$ frame for $(E \oplus F)|_U$. 
\item $\{ e_i \otimes f_j: 1\leq i\leq r, 1\leq j\leq s\} $ is a $C^\infty$ frame for $(E \otimes F)|_U$. 
\item $\{ e_{i_1} \wedge \cdots \wedge e_{i_k} : 1 \le i_1 < \cdots < i_k \le r \} $ is
 a $C^\infty$ frame of $\Lambda^k E$. (Here $k\leq r$.) 
\end{itemize}


\section{Wednesday, October 7, 2015}


\begin{definition}
Let $M$ be a smooth manifold. The {\em cotangent space} at $p \in M$ is the space 
$T^*_pM := (T_pM)^*$, the dual vector space of the tangent space $T_p M$ to $M$ at $p$. 
A {\em cotangent vector} at $p \in M$ is an element of $T_p^*M$. The {\em cotangent bundle} of $M$ is 
$T^*M:= (TM)^*$, the dual of the tangent bundle $TM$ of $M$.
\end{definition}

\begin{definition}
Let $M$ be a smooth manifold. 
\begin{enumerate}
\item[(i)] A {\em smooth $(r,s)$-tensor} on $M$ is a smooth section of 
\[
T_s^rM := (TM)^{\otimes r} \otimes (T^*M)^{\otimes s}.
\]
\item[(ii)] A {\em smooth $s$-form} on $M$ is a smooth section of $\Lambda^sT^*M$. 
\end{enumerate}
\end{definition}

\begin{example}
A vector field is a (1,0)-tensor. An $s$-form is a particular type of $(0,s)$-tensor. A $1$-form is the same as a $(0,1)$-tensor. 
\end{example}

\begin{example}
Let $f : M \to \mathbb{R}$ be smooth. Then for any point $p \in M$, the differential $df_p$ is a linear map $df_p : T_pM \to \mathbb{R}$. It follows that $df_p \in T_p^*M$. 
Suppose that $(U,\phi)$ is a chart for $M$ and $\phi=(x_1,\ldots, x_n)$ are
local coordinates. Then
$$
\langle df, \frac{\partial}{\partial x_i}\rangle = \frac{\partial f}{\partial x_i}
$$
are smooth functions on $U$. This shows that $df$ is a smooth
section of $T^*M$, i.e.,  $df$ is a smooth 1-form on $M$.
The 1-form $df$ is called the {\em differential} of $f$.  
\end{example}

We now study tensors in local coordinates. Let $(U,\phi)$ be a chart for $M$ such that $\phi = (x_1, \ldots, x_n)$. Then we know that $\{\frac{\partial }{\partial x_1},\cdots,\frac{\partial}{\partial x_n}\}$ is a smooth frame for $TM|_U = TU$. The differentials $dx_i$ of the coordinate functions are smooth
sections of $T^*M|_U =T^*U$ and 
$$
dx_i(\frac{\partial}{\partial x_j}) =\delta_{ij}.
$$
So $\{ dx_1,\ldots, dx_n\}$ is a $C^\infty$ frame of $T^*U$ dual 
to the $C^\infty$ frame $\{\frac{\partial}{\partial x_1},\ldots, 
\frac{\partial}{\partial x_n}\}$
For any smooth function $f : U \to \mathbb{R}$, we may write 
\[
df = \sum_{i=1}^n \frac{\partial f}{\partial x_i} dx_i.
\] 
More generally, any smooth $(r,s)$-tensor can be written in terms of the local frames:
\[
\sum_{\substack{1 \le i_1, \cdots, i_r \le n \\ 1 \le j_1, \cdots,  j_s \le n}} a^{i_1 \cdots i_r}_{j_1 \cdots j_s} \frac{\partial }{\partial x_{i_1}} \otimes \cdots \otimes \frac{\partial }{\partial x_{i_r}} \otimes dx_{j_1} \otimes \cdots \otimes dx_{j_s}
\]
where $a^{i_1 \cdots i_r}_{j_1 \cdots j_s} \in C^\infty(U)$. 

\bigskip

\noindent
{\bf \large Pullback of $(0,s)$ tensors under a $C^\infty$ map}
\begin{definition}
Let $\phi : M \to N$ be a smooth map between smooth manifolds. Let $p$ be a point of $M$. Then $d\phi_p : T_pM \to T_{\phi(p)}N$ is a linear map. We get a dual linear map $d\phi_p^* : T_{\phi(p)}^*N \to T_p^*M$. Then for any $(0,s)$-tensor $T$ on $N$, we let $\phi^*T$ denote the $(0,s)$-tensor of $M$ described by 
\[
\phi^*T(p) = (d\phi_p^*)^{\otimes s}(T(\phi(p))). 
\]
\end{definition}

\begin{definition}
We let $\Omega^s(N)$ denote the space of smooth $s$-forms on $N$, that is, the space of smooth sections of $\Lambda^sT^*N$. The above definition implies that we may pull back $s$-forms. 
\end{definition}

\begin{lemma}
For any smooth function $f: N \to \mathbb{R}$, we have 
\[
\phi^*(df) = d(\phi^*f).
\]
\end{lemma}

\begin{proof}
For any $p \in M$, we compute 
\[
(\phi^*df)(p) = d\phi_p^*(df_{\phi(p)}) = df_{\phi(p)} \circ d\phi_p = d(f \circ \phi)_p = d(\phi^*f)(p). 
\]
\end{proof}

\begin{example}
Let $\phi : (0,\infty) \times \mathbb{R} \to \mathbb{R}^2$ be the map 
\[
\phi(r,\theta) = (r \cos \theta, r \sin \theta). 
\]
Note $\phi^*x = r\cos\theta$ and $\phi^*y = r\sin\theta$. Then 
\[
\phi^*dx = d(\phi^*x) = d(r\cos \theta) = \cos\theta dr - r \sin \theta d\theta
\]
and 
\[
\phi^*dy = d(\phi^*y) = d(r \sin\theta) = \sin\theta dr + r \cos\theta d\theta.
\]
We also compute that 
\[
\phi^*(dx \wedge dy) = r dr \wedge d\theta. 
\]
\end{example}

\noindent
{\bf \large Pullback and pushforward of tensors under
a $C^\infty$ diffeomorphism}

\begin{definition}
Let $\phi : M \to N$ be a smooth diffeomorphism. It follows that $d\phi_p : T_pM \to T_{\phi(p)}N$ is an invertible linear map with inverse $d(\phi^{-1})_{\phi(p)}$. Then we get a map 
$\phi^* : C^\infty(N,T^r_sN) \to C^\infty(M, T^r_sM)$ called the {\em pullback} described by 
\begin{align*}
\phi^*T(p) = [(d(\phi^{-1})_{\phi(p)})^{\otimes r} \otimes (d\phi_p^*)^{\otimes s}]T(\phi(p))
\end{align*}
We also get a map $\phi_* : C^\infty(M, T^r_sM) \to C^\infty(N, T^r_sN)$ called the {\em pushforward} described by $\phi_* = (\phi^{-1})^*$. 
\end{definition}

\begin{example}
If $X$ is a smooth vector field, then 
\[
\phi_*X(q) = d\phi_{\phi^{-1}(q)}X(\phi^{-1}(q))
\]
for any $q \in N$. 
\end{example}

\begin{lemma}
If $\phi : M_1 \to M_2$ and $\psi :M_2 \to M_3$ are smooth maps. 
\begin{enumerate}
\item[(i)] Then $(\psi \circ \phi)^* = \phi^* \circ \psi^*$. 
\item[(ii)] If $\phi,\psi$ are diffeomorphisms, then $(\psi \circ \phi)_* = \psi_* \circ \phi_*$. 
\end{enumerate}
\end{lemma}

\noindent
{\large \bf Lie derivatives on tensors}

Let $X$ be a smooth vector field on $M$. We have already defined $L_Xf = X(f)$ for $f : M \to \mathbb{R}$ a smooth function. We have also defined $L_X(Y) = [X,Y]$ for a smooth vector field $Y$ on $M$. Now we want to define $L_XT$ for any smooth tensor $T$. 

Recall from before that if $\phi_t : U \to M$ is the local flow of $X$ around $p \in M$, then 
\[
L_Xf(p) = \frac{d}{dt}\Big|_{t=0} (\phi_t^*f)(p)
\]
and 
\[
L_XY(p) = \frac{d}{dt}\Big|_{t=0}(\phi_t^*Y)(p).
\]
Note that $\phi_t^* = (\phi_t^{-1})_* =(\phi_{-t})_*$. 

\begin{definition}
Let $M$ be a smooth manifold and let $X$ be a smooth vector field. We can define the 
{\em Lie derivative} with respect to $X$ to be the map $L_X : C^\infty(M,T_s^rM) \to C^\infty(M,T_s^rM)$ by the rule 
\[
L_XT(p) := \frac{d}{dt}\Big|_{t=0}(\phi_t^*T)(p)
\]
where $\phi_t: U \to M$ is the local flow of $X$. 
\end{definition}

\begin{lemma}
The Lie derivative satisfies the following properties 
\begin{enumerate}
\item[(i)] For a smooth function $f$, we have $L_Xf = X(f)$.
\item[(ii)] For a smooth vector field $Y$, we have $L_XY = [X,Y]$. 
\item[(iii)] For a (0,1)-tensor $\alpha$ and $Y$ a vector field, we have 
\[
(L_X\alpha)(Y) = L_X(\alpha(Y)) - \alpha(L_XY) = X(\alpha(Y)) - \alpha([X,Y]).
\]
\item[(iv)] For tensors $S$ and $T$, we have 
\[
L_X(S \otimes T) = L_X(S) \otimes T + S \otimes L_X(T).
\]
In particular, if $f$ is a smooth function, then 
\[
L_X(fT) = X(f)T + fL_XT
\]
\end{enumerate}
\end{lemma}

\begin{proof}
To see (iii), we can check that 
\[
\phi_t^*(\alpha(Y)) = (\phi_t^*\alpha)(\phi_t^*(Y)).
\]
For (iv), we can check that 
\[
\phi_t^*(S \otimes T) = \phi_t^*S \otimes \phi_t^*T.
\]
\end{proof}

\begin{remark}
Alternatively, one can use properties (i) through (iv) to define the Lie derivative. 
\end{remark}

\begin{lemma}
$
L_X \circ L_Y - L_Y \circ L_X = L_{[X,Y]}. \\
$
This means that the map $L : C^\infty(M,TM) \to \mathfrak{gl}(C^\infty(M, T^r_sM))$ given by $X \mapsto L_X$  is a Lie algebra homomorphism. 
\end{lemma}
\begin{proof}
Assignment 5 (1).
\end{proof}




\noindent
{\large \bf Exterior derivative on forms}


\begin{definition}
Define $d : \Omega^s(M) \to \Omega^{s+1}(M)$ to be the unique $\mathbb{R}$-linear map satisfying 
\begin{enumerate}
\item[(i)] If $f$ is a smooth function on $M$, then $df$ is the differential of $f$. 
\item[(ii)] For any smooth function $f$ on $M$, we have $ddf = 0$. 
\item[(iii)] (Leibniz rule): If $\alpha$ is an $r$-form and $\beta$ is an $s$-form, then 
\[
d(\alpha \wedge \beta) = d\alpha \wedge \beta + (-1)^r \alpha \wedge d\beta. 
\]
\end{enumerate}
\end{definition}

In terms of local coordinates, we have the following. If $\alpha$ is an $s$-form and $(U,\phi)$ is a local coordinate chart, then we may write 
\[
\alpha = \sum_{1 \le j_1 < \cdots j_s \le n} a_{j_1 \cdots j_s} dx_{j_1} \wedge \cdots \wedge dx_{j_s}
\]
and we compute 
\[
d\alpha =  \sum_{1 \le j_1 < \cdots j_s \le n} da_{j_1 \cdots j_s} \wedge dx_{i_1} \wedge \cdots \wedge dx_{j_s}
\]

\begin{proposition}
Let $\omega$ be an $s$-form on $M$. Then we have the following. 
\begin{enumerate}
\item[(i)]  $dd\omega = 0$. 
\item[(ii)] If $\phi : M' \to M$ is a smooth map, then $d(\phi^*\omega) = \phi^*(d \omega)$, that is, $d$ commutes with pullbacks. 
\item[(iii)] If $X$ is a smooth vector field on $M$, then $d(L_X\omega) = L_X(d\omega)$, that is, $d$ commutes with Lie derivatives. 
\item[(iv)] For an $s$-form $\omega$ and vector fields $X_0, \ldots, X_s$, we compute 
\begin{align*}
d\omega(X_0, \ldots, X_s) &= \sum_{i=0}^s (-1)^iX_i(\omega(X_0, \ldots, \hat{X}_i, \ldots, X_s)) \\
&\;\;\;+ \sum_{0 \le i, j \le s}(-1)^{i+j} \omega([X_i, X_j], X_0, \ldots, \hat{X}_i, \ldots, \hat{X}_j, \ldots, X_s). 
\end{align*}
\end{enumerate}
\end{proposition}

\begin{proof}
The proofs of (i) and (ii) are straightforward. Taking $\phi = \phi_t$ in (ii), we get (iii). The proof of (iv) is Assignment 5 (3). 
\end{proof}



\noindent
{\bf \large Interior derivatives on forms}

\begin{definition}
Let $X$ be a smooth vector field on a smooth manifold $M$. Define $i_X : \Omega^s(M) \to \Omega^{s-1}(M)$ by the rules 
\begin{enumerate}
\item[(i)] $i_Xf = 0$ for a smooth function $f : M \to \mathbb{R}$ and 
\item[(ii)] For an $s$-form $\alpha$, we have $i_X\alpha(X_1, \ldots, X_{s-1}) = \alpha(X, X_1, \ldots, X_{s-1})$. 
\end{enumerate}
\end{definition}

\begin{lemma}
We have the following. 
\begin{enumerate}
\item[(i)] $i_X \circ i_X = 0$
\item[(ii)] $i_X(\alpha \wedge \beta) = i_X\alpha \wedge \beta + (-1)^{\deg(\alpha)} \alpha \wedge i_X\beta$.
 \item[(iii)] (Cartan's formula): We have $d \circ i_X + i_X \circ d = L_X$. 
\end{enumerate}
\end{lemma}

\begin{proof}
(i) and (ii) are straightfoward to check. (iii) is Assignment 5 (2a).
\end{proof}


\section{Monday, October 12, 2015} 

\noindent
{\bf \large Riemannian Metrics}

\begin{definition}
Let $M$ be a smooth manifold. A {\em Riemannian metric} $g$ on $M$ is a smooth $(0,2)$-tensor such that 
for any $p\in M$, $g(p) : T_pM \times T_pM \to \bR$ is a inner product on $T_pM$. 
We say such a pair $(M,g)$ is a {\em Riemannian manifold}.
\end{definition}


The tensor bundle $T^0_2M$ can be written as a direct sum of two  $C^\infty$ subbundles:
$$
T^0_2 M = (T^*M)^{\otimes 2} = S^2(T^*M)\oplus \Lambda^2(T^*M)
$$
where $S^2(T^*M)$ is the symmetric square of $T^*M$.

Let $n=\dim(M)$. For any $p\in M$, 
\begin{itemize}
\item $(T_p^*M)^{\otimes 2}$ is the space of bilinear forms on $T_pM$, which is $n^2$ dimensional;
\item $S^2T_p^*M$ is the space of symmetric bilinear forms on $T_pM$, which is $\frac{1}{2}n(n+1)$ dimensional; 
\item $\Lambda^2 T_p^*M$ 
is the space of skew-symmetric bilinear forms on $T_pM$, which is $\frac{1}{2}n(n-1)$ 
dimensional. 
\end{itemize}

Let $\Omega \subset C^\infty(M, S^2T^*M)$ denote the space of Riemannian metrics on $M$.  
Then we claim that $\Omega$ is a convex subset. This is because if $g_0, g_1 \in \Omega$, then 
$(1-t)g_0 + tg_1$ is a Riemannian metric for $t\in [0,1]$. In particular, we see that $\Omega$ is contractible. 

\bigskip

We now discuss  Riemannian metrics in local coordinates. Let $(U,\phi)$ be a chart for $M$ and write $\phi = (x_1, \ldots, x_n)$. Then $\{dx_1,\ldots, dx_n\}$ is a $C^\infty$ frame for $T^*M|_U=T^*U$. If we let 
\[
dx_i dx_j = \frac{1}{2}(dx_i \otimes dx_j + dx_j \otimes dx_i)
\]
then we see that $\{dx_idx_j: 1\leq i\leq j\leq n\}$ is a $C^\infty$ frame for $S^2T^*M|_U$. Then we know that on $U$, we may write 
\[
g = \sum_{i,j} g_{ij} dx_idx_j
\]
for some smooth functions $g_{ij}$, where $g_{ij} = g_{ji}$. 
For any $p$, the collection $(g_{ij}(p))$ forms a symmetric, positive definite, 
$n\times n$ matrix with entries in $\bR$. 

\begin{example}
Let $M = \bR^n$. Then we let $g_0(\frac{\partial}{\partial x_i}, \frac{\partial}{\partial x_j}) = \delta_{ij}$. 
This is called the Euclidean metric. In terms of global coordinates $(x_1,\ldots,x_n)$ on $\bR^n$,  
\[
g_0 = dx_1^2 + \cdots + dx_n^2.
\]
\end{example}

\begin{example}
On $\bR^2$, let $(x,y)$ be the cartesian coordinates, so that the Eulidean metric
$g_0$ can be written as $g_0= dx^2 + dy^2$. The polar coordinates $(r,\theta)$, which
are local coordinates around any point in $\bR^2-\{(0,0)\}$, are related to
$(x,y)$ by
$$
x = r\cos\theta,\quad  y = r \sin \theta
$$
In terms of the polar coordinates, the Euclidean metric is of the form  
\[
g_0 = E dr^2 + F(drd\theta + d\theta dr) + Gd\theta^2 = E dr^2 +2F drd\theta + Gd\theta^2,
\]
where 
$$
E = g_0(\frac{\partial}{\partial r}, \frac{\partial}{\partial r}),\quad
F = g_0(\frac{\partial}{\partial r}, \frac{\partial}{\partial \theta}),\quad
G = g_0(\frac{\partial}{\partial \theta}, \frac{\partial}{\partial \theta}).
$$
We have
\begin{eqnarray*}
\frac{\partial}{\partial r} &=& 
\frac{\partial x}{\partial r} \frac{\partial}{\partial x} + \frac{\partial y}{\partial r} \frac{\partial}{\partial y} 
= \cos\theta  \frac{\partial}{\partial x} + \sin\theta  \frac{\partial}{\partial y} 
= \frac{x \frac{\partial}{\partial x} + y \frac{\partial}{\partial y}}{\sqrt{x^2 + y^2}},\\
\frac{\partial}{\partial \theta} &=& 
-r\sin\theta  \frac{\partial}{\partial x} + r\cos\theta \frac{\partial}{\partial y}=
-y \frac{\partial}{\partial x} + x \frac{\partial}{\partial y}
\end{eqnarray*}
We compute that $E = 1$, $F = 0$ and $G = r^2$. It follows that 
\[
g_0 = dr^2 + r^2d\theta.
\]

$\{ \frac{\partial}{\partial x}, \frac{\partial}{\partial y}\}$ is an $C^\infty$ orthonormal frame
for $T\bR^2$. 

$\{ \frac{\partial}{\partial r}, \frac{1}{r}\frac{\partial}{\partial \theta}\}$ is
a $C^\infty$ orthonormal frame for $T\bR^2|_{\bR^2-\{(0,0)\}}$.  

\end{example}

\begin{example}
On $\bR^3$, the Euclean metric is $g_0 = dx^2 + dy^2 + dz^2$ in terms of
the cartesian coordinates $(x,y,z)$. The spherical coordinates $(\rho,\phi,\theta)$
are local coordinates around any point in $U:= (\bR^2-\{(0,0)\})\times \bR$, 
the complement of the $z$-axis $x=y=0$; they are related to
the cartesian coordinates by 
$$
x = \rho \sin \phi \cos \theta,\quad
y = \rho \sin \phi \sin \theta,\quad
z = \rho \cos \phi.
$$
We find that 
\begin{align*}
\frac{\partial}{\partial \rho} &= \frac{x \frac{\partial}{\partial x} + y \frac{\partial}{\partial y} + z \frac{\partial}{\partial z}}{\sqrt{x^2 + y^2 + z^2}} \\
\frac{\partial}{\partial \theta} &= -y \frac{\partial}{\partial x} + x \frac{\partial}{\partial y} \\
\frac{\partial}{\partial \phi} &= \frac{1}{\sqrt{x^2 + y^2}}\left(xz \frac{\partial}{\partial x} + yz \frac{\partial}{\partial y} - (x^2 + y^2) \frac{\partial}{\partial z}\right).
\end{align*}
$\rho=\sqrt{x^2+y^2+z^2}$ is a smooth function on $U$; indeed it is a smooth function on $\bR^3-\{(0,0,0)\}$.
Although $\phi$ and $\theta$ are well-defined only locally but not globally on $U$, the above
computations show that $\frac{\partial}{\partial \rho}$, $\frac{\partial}{\partial \theta}$, 
$\frac{\partial}{\partial \phi}$ are well-defined $C^\infty$ vector fields on $U$ 
and form a $C^\infty$ frame for $T\bR^3|_U$; $d\phi$ and $d\theta$ are well-defined, smooth 1-forms on $U$, and
$\{ d\rho, d\theta, d\phi\}$ is a $C^\infty$ frame for $T^*\bR^3|_U$.

We compute that  
\[
g_0 = d\rho^2 + \rho^2 d\phi^2 + \rho^2 \sin^2\phi d\theta^2.
\]

$\{\frac{\partial}{\partial x}, \frac{\partial}{\partial y}, \frac{\partial}{\partial z} \}$
is a $C^\infty$ orthonormal frame for $T\bR^3$.

$\{ \frac{\partial}{\partial \rho}, \frac{1}{\rho}\frac{\partial}{\partial \phi}, 
\frac{1}{\rho \sin\phi} \frac{\partial}{\partial \theta}\}$
is a $C^\infty$ orthonormal frame for $T\bR^3|_U$. 
\end{example}

\bigskip

Let $f : M \to N$ be a smooth map between smooth manifolds. If $g$ is
a Riemannian metric on $N$, then $g\in C^\infty(N, S^2(T^*N))$, so 
$f^*g\in C^\infty(M,S^2T^*M)$.  Given $p\in M$, 
$(f^*g)(p)$ is an inner product on $T_p M$ iff $df_p: T_pM\to T_{f(p)}M$
is injective iff $f$ is an immersion at $p$. Therefore, if $f$ is an immersion
then $f^*g$ is a Riemannian metric on $M$.



\begin{definition}
Let $f : M \to N$ be a smooth immersion and let $g$ be a Riemannian metric on $N$. Then $f^*g$ is a Riemannian metric on $M$ called the {\em pullback}. 
\end{definition}

\begin{example}
Let $i_r : S^2(r) \to \mathbb{R}^3$. Then $g_{can} := i_r^*g_0$ is known as the 
{\em canonical metric} or {\em round metric} on the sphere of radius $r$. 

It is convenient to use the coordinates 
$$
x = r \sin \phi \cos \theta,\quad 
y = r \sin \phi \sin \theta,\quad
z = r \cos \phi.
$$
Then we find that 
\[
g_{can} = i_r^*g_0 = r^2(d\phi^2 + \sin^2\phi d\theta^2)
\]
\end{example}

\begin{definition}
Let $f : (M,g_1) \to (N, g_2)$ be a smooth map between Riemannian manifolds. We say that $f$ is 
\begin{enumerate}
\item[(i)] an {\em isometric immersion} (resp. {\em embedding}) if $f$ is an 
immersion (resp. embedding) and $f^*g_2=  g_1$ (in other words, if the differential preserves the inner product). 
\item[(ii)] a {\em (local) isometry} if $f$ is a (local) diffeomorphism and $f^*g_2=  g_1$. 
\end{enumerate}
\end{definition}

Suppose that $i : (M_1, g_1) \hookrightarrow (M_2, g_2)$ is an isometric embedding.  Then $i(M_1)$ is a Riemannian submanifold of $(M_2, g_2)$. This means that it is a submanifold when equipped with the Riemannian metric given by pulling back the metric on $M_2$ under inclusion. 

\begin{example}
Let $i_r : S^n(r) \to \mathbb{R}^{n+1}$. Then $g_{can} = i_r^* g_0$ is the round metric on the $n$-sphere of radius $r > 0$. 
\end{example}

\begin{example}
Let $A \in GL(n,\mathbb{R})$. Then $A$ defines an invertible linear map $A : \mathbb{R}^n \to \mathbb{R}^n$. In particular, $A$ is a smooth diffeomorphism. Then we can pull back the Euclidean metric. We find that 
\begin{align*}
A^*g_0 &= \sum_{i} d\Big(\sum_{j} A_{ij} dx_j\Big) d\Big(\sum_{k} A_{ik} dx_k\Big) = 
\sum_{j,k} \Big(\sum_{i} A_{ij} A_{ik}\Big) dx_j dx_k \\
&= \sum_{j,k} (A^T A)_{jk} dx_j dx_k.
\end{align*}
We see that $A$ is an isometry if and only if $A^*g_0 = g_0$, which happens if and only if $A^TA = I$, which means that $A \in O(n)$. 
\end{example}

We will see later the following.
\begin{theorem}
A smooth map $\phi : (\mathbb{R}^n, g_0) \to (\mathbb{R}^n, g_0)$ is an isometry 
if and only if $\phi$ is a rigid motion, i.e. $\phi(x) = Ax + b$ 
for some $A \in O(n)$ and $b \in \bR^n$. 
\end{theorem}

\begin{example}
Let $A \in O(n+1)$. Then $A(S^n(r)) = S^n(r)$. It follows that the restriction $A : (S^n(r), g_{can}) \to (S^n(r), g_{can})$ is an isometry. 
We will see later that, these are all of the isometrics of the round sphere.
\end{example}

\begin{example}
Let $\phi : \bR \to S^1$ be the map $\phi(t) = (\cos t, \sin t)$. This is a smooth local diffeomorphism. On $\bR$, we have the metric $dt^2$ and on $\bR^2$, we have the metric $dx^2 + dy^2$, which induces the metric $g_\mathrm{can}$ on $S^1$. Then we find that 
\[
\phi^* g_\mathrm{can} = (i\circ \phi)^*(dx^2 + dy^2) 
=  (-\sin t dt)^2 + (\cos t dt)^2 = dt^2.
\]
It follows that $\phi: (\bR, dt^2)\to (S^1,g_{\mathrm{can}})$ is a local isometry. 
\end{example}

\begin{definition}
Let $(M_1, g_1)$ and $(M_2, g_2)$ be Riemannian manifolds and let $M_1 \times M_2$ denote the product manifold. For $i=1, 2$, let $\pi_i : M_1 \times M_2 \to M_i$. We define the 
{\em product metric} on $M_1 \times M_2$ to be 
\[
g_1 \times g_2 = \pi_1^*g_1 + \pi_2^*g_2. 
\]
In this way, the metric on $T_{(p_1, p_2)}(M_1 \times M_2)$ ensures the space decomposes as an orthogonal sum $T_{p_1}M_1 \oplus T_{p_2}M_2$. This means that 
\[
(g_1 \times g_2)(p_1, p_2)((u_1,v_1), (u_2, v_2)) = \langle u_1, u_2 \rangle_{p_1} + \langle v_1, v_2 \rangle_{p_2}.
\]
\end{definition}

\begin{example}
Let $T^n$ denote the torus $\underbrace{S^1 \times \cdots \times S^1}_{n\textup{ copies}}$.
The flat metric on $T$ is 
$g=\underbrace{g_{\mathrm{can}}\times \cdots\times g_{\mathrm{can}} }_{n \textup{ times}}$.
Let $\phi : \mathbb{R}^n \to T^n$ be the map 
\[
(t_1, \ldots, t_n) \mapsto ((\cos t_1, \sin t_1), \ldots, (\cos t_n, \sin t_n)).
\]
Then $\phi$ is a local isometry from $(\bR^n, g_0)$ to $(T^n,g)$. 
\end{example}

\begin{definition}
Let $M$ be a smooth manifold (note that we are assuming that $M$ is Hausdorff with a countable basis). A smooth {\em partition of unity} on $M$ is a collection of smooth functions $\{f_\gamma \in C^\infty(M) : \gamma \in \Gamma\}$ such that 
\begin{enumerate}
\item[(i)] (nonnegative) We have $f_\gamma \ge 0$ for each $\gamma$
\item[(ii)] (locally finite) The collection $\{\text{supp} f_\gamma : \gamma \in \Gamma\}$ is {\em locally finite} in the sense that for each $p \in M$, there is a neighborhood $W$ of $p$ such that only finitely many $\text{supp} f_\gamma$ intersect $W$. 
\item[(iii)] For each $p \in M$, we have 
\[
\sum_{\gamma \in \Gamma}f_\gamma(p) = 1.
\]
Note that the left hand side is a finite sum by (ii). 
\end{enumerate}
Moreover we say that a partition of unity $\{f_\gamma\}$ is {\em subordinate to an open cover} $\mathcal{A} = \{A_\alpha : \alpha  \in I \}$ if for each $\gamma \in \Gamma$, there is an $\alpha \in I$ such that $\text{supp} f_\gamma \subseteq A_\alpha$. 
\end{definition}

\begin{theorem}\label{partition-of-unity}
Let $M$ be a smooth manifold and let $\mathcal{A} = \{A_\alpha : \alpha \in I\}$ be an open cover of $M$. Then there is a partition of unity $\{f_\gamma : \gamma \in \Gamma\}$ subordinate to the open cover $\mathcal{A}$. 
\end{theorem}

\begin{proof}
See \cite[Chapter V Section 4]{Bo}. 
\end{proof}

The proofs of the following two propositions rely on Theorem  \ref{partition-of-unity}
and will be presented on the roundtable on October 16.

\begin{proposition}
Let $M$ be a smooth manifold. Then there is a Riemannian metric on $M$. 
\end{proposition}


\begin{proposition}
Let $M$ be a compact Hausdorff smooth $n$-manifold. Then $M$ can be smoothly embedded in $\bR^{2n+1}$. 
\end{proposition}

We have the following classical theorems.

\begin{theorem}[Weak Whitney Embedding]
Let $M$ be a smooth $n$-manifold (Hausdorff and countable basis). Then $M$ can be smoothly embedded in $\bR^{2n+1}$ as a closed submanifold. 
\end{theorem}

\begin{theorem}[Strong Whitney Emdedding]
Let $M$ be a smooth $n$-manifold (Hausdorff with countable basis). Then $M$ can be smoothly embedded in $\bR^{2n}$ as a closed submanifold. 
\end{theorem}



\begin{theorem}[Nash Embedding Theorem]
Any Riemannian $n$-manifold can be isometrically embedded in $\bR^{n(n+1)(3n+11)/2}$. 
Any compact Riemannian $n$-manifold can be isometrically embedded in 
$\bR^{n(3n+11)/2}$.
\end{theorem}



\section{Wednesday, October 14, 2015}


\noindent
{\large \bf Volume form}

\begin{definition}[Volume Form]
Let $M$ be a smooth $n$-manifold. A {\em volume form} on $M$ is a smooth $n$-form $\nu$ on $M$ such that $\nu(p) \ne 0$ for any $p \in M$.  
\end{definition}

\begin{lemma}
If $M$ is a smooth $n$-manifold, the following are equivalent. 
\begin{enumerate}
\item[(i)] There is a volume form on $M$.
\item[(ii)] $\Lambda^n T^*M$ is trivial. 
\item[(iii)] $M$ is orientable. 
\end{enumerate}
\end{lemma}

\begin{proof}
(i)$\Leftrightarrow$ (ii): Item (i) means that there is a global smooth frame for $\Lambda^nT^*M$. This happens if and only if $\Lambda^nT^*M$ is a trivial vector bundle of rank $1$ by  a previous lemma. 

\noindent
(i)$\Rightarrow$ (iii): 
Assume that (i) holds. Call the volume form $\nu$. Let $\{(U_\alpha, \phi_\alpha) : \alpha \in I\}$ be a smooth atlas for $M$ such that each $U_\alpha$ is connected. We define a smooth atlas $\{(U_\alpha, \phi_\alpha') : \alpha \in I\}$ as follows: On $U_\alpha$, we may write $\nu = f_\alpha dx_1^\alpha \wedge \cdots \wedge dx_n^\alpha$ where $n$ is the dimension of $M$ and $\phi_\alpha = (x_1^\alpha, \ldots, x_n^\alpha)$ are local coordinates on $U_\alpha$. We know that $f_\alpha \ne 0$, and $U_\alpha$ is connected. It follows that either $f_\alpha > 0$ or $f_\alpha < 0$ on $U_\alpha$. 
\begin{itemize}
\item If $f_\alpha > 0$, define $(U_\alpha',\phi_\alpha') = (U_\alpha, \phi_\alpha)$. 
\item If $f_\alpha < 0$, then let $r$ be the map $r(x_1, \ldots, x_n) = (-x_1, x_2, \ldots, x_n)$ and define $(U_\alpha, \phi_\alpha' = r \circ \phi_\alpha)$. 
\end{itemize}
Then we can check that $\{(U_\alpha, \phi_\alpha') : \alpha \in I\}$ defines an orientation on $M$. 

\noindent
(iii)$\Rightarrow$ (i):
Assume that (iii) holds. Suppose that $\{(U_\alpha, \phi_\alpha) : \alpha \in I\}$ is an orientation on $M$, that is, $\{(U_\alpha, \phi_\alpha) : \alpha \in I\}$ is a smooth atlas on $M$ such that $\det d(\phi_\beta \circ \phi_\alpha^{-1}) > 0$ on 
$\phi_\alpha(U_\alpha\cap U_\beta)$. Equip $M$ with a Riemannian metric $g$. On $U_\alpha$, write $\phi_\alpha = (x_1^\alpha, \ldots, x_n^\alpha)$. Then 
\[
g = \sum_{i,j=1}^n g_{ij}^\alpha dx_i^\alpha dx_j^\alpha
\]
where $g_{ij}^\alpha=\langle \frac{\partial}{\partial x_i^\alpha},
\frac{\partial}{\partial x_j^\alpha} \rangle\in C^\infty(U_\alpha)$,
and $(g_{ij}^\alpha(p))$ is a positive definite symmetric $n\times n$ matrix for every $p\in U_\alpha$.


 Define $\nu_\alpha\in \Omega^n(U_\alpha)$ to be $\nu_\alpha = \sqrt{\det(g_{ij}^\alpha)} dx_1^\alpha \wedge \cdots \wedge dx_n^\alpha$. For each $p \in U_\alpha$, we know that $g_{ij}^\alpha(p)$ is a symmetric positive definite matrix and so $\det(g_{ij}^\alpha) : U_\alpha \to (0,\infty)$. Then $\nu_\alpha$ is a smooth nowhere zero section of $(\Lambda^nT^*M)|_{U_\alpha}$. If $U_\alpha \cap U_\beta \ne \varnothing$, then 
\[
g_{kl}^\beta = \langle \frac{\partial}{\partial x_k^\beta}, \frac{\partial}{\partial x_l^\beta} \rangle = \langle \sum_{i} \frac{\partial x_i^\alpha}{\partial x_k^\beta} \frac{\partial}{\partial x_i^\alpha}, \sum_{j} \frac{\partial x_j^\alpha}{\partial x_l^\beta} \frac{\partial}{\partial x_j^\alpha} \rangle = \sum_{i,j} \frac{\partial x_i^\alpha}{\partial x_k^\beta} \frac{\partial x_j^\alpha}{\partial x_l^\beta} g_{ij}^\alpha.
\]
Write $A_{ij} = g_{ij}^\alpha$ and $B_{kl} = g_{kl}^\beta$ and $C_{ik} = \frac{\partial x_i^\alpha}{\partial x_k^\beta}$. Then  $B = C^tAC$. It follows that
$\det B = \det A (\det C)^2 \Rightarrow \det B = \det A \sqrt{\det C}$ 
(since $A$, $B$ are symmetric and positive definite, and $\det C>0$).  We also have
$$
dx_1^\alpha \wedge \cdots \wedge dx_n^\alpha = 
\det C dx_1^\beta\wedge \cdots \wedge dx_n^\beta.
$$
On $U_\alpha\cap U_\beta$, 
\begin{eqnarray*}
\nu_\alpha  &=& \sqrt{\det A}  dx_1^\alpha \wedge \cdots \wedge dx_n^\alpha
=\sqrt{\det A}\det C dx_1^\beta\wedge \cdots \wedge dx_n^\beta\\
&=& \det B dx_1^\beta\wedge \cdots \wedge dx_n^\beta =\nu_\beta.
\end{eqnarray*}
\end{proof}

\begin{remark}
Let $(M,g)$ be an oriented Riemannian manifold of dimension $n$. Then there is a unique volume form $\nu$ compatible with the orientation and the Riemannian metric, namely, the one we constructed. For any $p \in M$, choose an orthonormal basis $(e_1, \ldots, e_n)$ for $T_pM$ compatible with the orientation in the sense that if $\{(U_\alpha, \phi_\alpha)\}$ is an orientation and $\phi_\alpha = (x_1^\alpha, \ldots, x_n^\alpha)$, then $(dx_1^\alpha \wedge \cdots \wedge dx_n^\alpha)_p(e_1, \ldots, e_n) > 0$. Then we let $\nu(p) = e_1^* \wedge \cdots \wedge e_n^*$, where $(e_1^*,\ldots, e_n^*)$ is
the dual basis of $T^*_p M$. This is well-defined because if $(f_1, \ldots, f_n)$ is another orthonormal basis which is compatible with the orientation then 
\[
f_i = \sum_{j=1}^n a_{ij} e_j
\]
where $a_{ij} = A \in O(n)$ and $\det (A) > 0$ (which means that $A \in SO(n)$) and so 
\[
f_1^* \wedge \cdots \wedge f_n^* = e_1^* \wedge \cdots \wedge e_n^*. 
\]
\end{remark}

\begin{example}
For $(\bR^n, g_0 = dx_1^2 + \cdots + dx_n^2)$, we let $e_i = \frac{\partial}{\partial x_i}$ and $e_i^* = dx_i$ and so $\nu = dx_1 \wedge \cdots \wedge dx_n$. 
\end{example}

\begin{example}
Let $j : (S^n, g_{can}) \hookrightarrow (\mathbb{R}^{n+1}, g_0)$ be the round unit sphere isometrically embedded in $\mathbb{R}^{n+1}$. For any $x  = (x_1, \ldots, x_{n+1})\in S^n$, we know that 
\[
T_xS^n = \{v \in \mathbb{R}^{n+1} : x \cdot v = 0\}. 
\]
Then we find that,
\[
\text{vol}_{S^n, g_{can}} = \pm j^*(i_X(dx_1 \wedge \cdots \wedge dx_{n+1}))
\]
where $\pm$ depends on the orientation on $S^n$, and
$X = \sum_{j=1}^{n+1} x_j \frac{\partial}{\partial x_j}$.
\end{example}

\begin{example}
More generally, let $(N^{n+1}, g)$ be an oriented Riemannian manifold. Let $j : M^n \hookrightarrow N^{n+1}$ be a submanifold of codimension $1$ equipped with the Riemannian metric $j^*g$. If $M$ is also oriented, then we have volume forms $\nu_M \in \Omega^n(M)$ and $\nu_N \in \Omega^{n+1}(M)$ which are compatible with the orientations and metrics. Suppose that there is a vector field $X$ on $N$ such that for any $p \in M$, we have $|X(p)| = 1$ and $X(p) \perp T_pM$.  By replacing $X$ by $-X$ if necessary, we may 
further assume that $(X(p), e_1, \ldots, e_n)$ is an orthonormal basis for $T_pN$ which is compatible with the orientation on $N$ where $e_1, \ldots, e_n$ is an orthonormal basis for $T_p M$ compatible with the orientation on $M$. Then $j^*(i_X\nu_N) = \nu_M$. 
\end{example}

\noindent
{\bf \large Integration on an oriented manifold}

Let $(M,g)$ be a smooth $n$-manifold equipped with an orientation defined by a $C^\infty$ atlas 
$\{ (U_\alpha,\phi_\alpha): \alpha\in I\}$. Let $\phi_\alpha= (x_1^\alpha,\ldots, x_n^\alpha)$.
Given a smooth $n$-form $\omega$ and a compact subset $R$ of $\omega$, the integral
$$
\int_R \omega
$$
is characterized by the following properties. 

\begin{enumerate}
\item Suppose that $R$ is contained in $U_\alpha$ for some $\alpha\in I$, and let $(x_1^\alpha,\ldots, x_n^\alpha)$
be local coordinates on $U_\alpha$. On $U_\alpha$, any smooth $n$-form can be written
as $\omega = f_\alpha dx_1^\alpha\wedge \cdots dx_n^\alpha$ for some $f_\alpha\in C^\infty(U_\alpha)$. We define
$$
\int_R \omega = \int_{\phi_\alpha(R)} f_\alpha(x)dx_1^\alpha \cdots dx_n^\alpha. 
$$
\item If $R_1$ and $R_2$ are disjoint compact subsets of $M$ then
$$
\int_{R_1\cup R_2}  \omega = \int_{R_1}\omega  + \int_{R_2} \omega.
$$
\item If $\omega_1,\omega_2\in \Omega^n(M)$ and $c_1,c_2\in \bR$ then
$$
\int_R (c_1\omega_1+c_2\omega_2)= c_1 \int_R \omega_1 + c_2\int_R \omega_2. 
$$
\end{enumerate}

Let $\{ f_\gamma: \gamma\in \Lambda\}$ be a partition of unity subordinate to the open cover $\{ U_\alpha:\alpha\in I\}$. 
Given any $\omega\in \Omega^n(M)$, 
$$
\int_R \omega = \int_R \sum_{\gamma\in \Lambda} f_\gamma \omega
=\sum_{\gamma\in \Lambda} \int_R f_\gamma \omega = \sum_{\gamma\in \Lambda} \int_{R_\gamma} f_\gamma \omega.
$$
where $R_\gamma:= R\cap \mathrm{Supp}(f_\gamma)$ is a compact set contained in some $U_\alpha$, so 
we define $\int_{R_\gamma} f_\gamma\omega$ by (1). 


\begin{definition}
Let $(M,g)$ be an oriented Riemannian manifold and let $\nu_g$ be a volume form compatible with the orientation and Riemannian metric $g$.
Given a compact set $R$ in $M$, we define the {\em volume} of $R$  
\[
\text{volume}_g(R) = \int_R \nu_g. 
\]
\end{definition}


\begin{example}
Equip $S^2$ with the metric $g_{\textup{can}} = d\phi^2 + \sin^2\phi d\theta^2$. 
Let $U= S^2\setminus \{(0,0,1),(0,0,-1)\}$. An orthonormal frame for $TS^2|_U$  is 
\[
\frac{\partial}{\partial \phi}, \frac{1}{\sin\phi}\frac{\partial}{\partial \theta}
\]
and the dual coframe is $d\phi, \sin\phi d\theta$. (In general, if $e_1, \ldots, e_n$ is an orthonormal basis of $T_pM$ which is compatible with the orientation, then $g(p) = e_1^* \otimes e_1^* + \cdots + e_n^* \otimes e_n^*$ and $\nu(p) = e_1^* \wedge \cdots \wedge e_n^*$.) 
Then we see that the volume form is 
\[
\nu_{g_\textup{can}} = \sin\phi d\phi \wedge d\theta,
\]
and so 
\[
\text{volume}_{g_\textup{can}}(S^2) = \int_0^{2\pi} \int_0^{\pi} \sin\phi d\phi d\theta = 4\pi. 
\]
\end{example}

\noindent
{\bf \large Length}

\begin{definition}
Let $\gamma : (a,b) \to (M,g)$ be a smooth curve. Then the {\em length of $\gamma$} is 
\[
\text{length}(\gamma) = \int_a^b \lVert \gamma'(t) \rVert dt,
\textup{ where } \lVert \gamma'(t) \rVert =
 \sqrt{\langle \gamma'(t),\gamma'(t)\rangle_{\gamma(t)} }.
\]
\end{definition}

\begin{example}
We consider upper half plane $\mathbb{H}^2 = \{(x,y) : \bR^2 : y > 0\}$. We endow this with the metric 
\[
g = \frac{dx^2 + dy^2}{y^2}.
\]
Pick points $x_1 > x_0$ and $y_1 > y_0 > 0$ in $\bR$. Let $\gamma_1$ be the straight line from $(x_0, y_0)$ to $(x_1, y_0)$ 
and let $\gamma_2$ be the straight line from $(x_0, y_0)$ to $(x_0, y_1)$:
$$
\gamma_1(t) = (t,y_0),\quad t\in (x_0,x_1);\quad
\gamma_2(t)=  (x_0,t),\quad t\in (y_0, y_1).
$$
We compute that $\gamma_1'=\frac{\partial}{\partial x}$, 
\[
\langle \gamma'_1(t), \gamma'_1(t) \rangle_{\gamma(t)} = \frac{1}{y_0^2}.
\]
Hence we find that 
\[
\text{length}(\gamma_1) = \int_{x_0}^{x_1} |\gamma'_1| dt = \frac{x_1 - x_0}{y_0}. 
\]
On the other hand, we compute that  $\gamma_2'=\frac{\partial}{\partial y}$, 
\[
\langle \gamma_2'(t), \gamma_2'(t) \rangle_{\gamma(t)}  = \frac{1}{t^2}
\]
and hence 
\[
\text{length}(\gamma_2) = \int_{y_0}^{y_1} \frac{dt}{t} = \log(y_1/y_0). 
\]

For any $a > 0$ ,we can consider $F_a : \mathbb{H} \to \mathbb{H}$ given by $F_a(x,y) = (ax,ay)$ and then 
\[
F^*g = \frac{d(ax)^2 + d(ay)^2}{(ay)^2} = g. 
\]
It follows that $F_a$ is an isometry. 
\end{example}






\section{Monday, October 19, 2015} 

\noindent
{\bf \large Distance}

\begin{definition}
If $(M,g)$ is a connected Riemannian manifold and $p,q \in M$, 
then for any $p,q$ in $M$ there exists a piecewise smooth curve $\gamma:[0,1]\to M$ such that
$\gamma(0)=1$ and $\gamma(1)=q$. We define the {\em distance from $p$ to $q$} to be 
\[
d_g(p,q) = \inf\{
\text{length}(\gamma) : \gamma : [0,1] \to M \text{piecewise smooth}, \gamma(0) = p, \gamma(1) = q\}.
\]
\end{definition}

From the above definition, it is clear that for $p,q,r\in M$,
\begin{itemize}
\item $d_g(p,q) \in [0,\infty)$ and $\text{dist}_g(p,p)=0$;
\item $d_g(p,q)=\text{dist}_g(q,p)$;
\item $d_g(p,q) + \text{dist}_g(q,r) \geq \text{dist}_g(p,r)$. 
\end{itemize}
We will see later that if $M$ is Hausdorff then $d_g(p,q)=0\Rightarrow p=q$, so that
$(M,d_g)$ is a metric space (in the sense of topology). The topology defined
by $d_g$ agrees with the topology on $M$. 

\begin{lemma}\label{eqn:dist-iso}
The distance is preserved by isometry. That is, if $\phi : (M_1, g_1) \to (M_2, g_2)$ is an isometry, then 
\[
d_{g_1}(p,q) = d_{g_2}(\phi(p), \phi(q)). 
\]
\end{lemma}
\begin{proof} Note that $\gamma:I\to M_1$ is a piecewise smooth curve in $M_1$ if and only if $\phi\circ\gamma: I\to M_2$ is a piecewise 
smooth curve in $M_2$, and in this case, we have 
$\text{length}(\phi\circ\gamma)=\text{length}(\gamma)$. 
\end{proof} 


\begin{example}
For $(\bR^n, g_0)$, $d_{g_0}(\vec{x},\vec{y}) = |\vec{x} - \vec{y}|$. 
To see this, by Lemma \ref{eqn:dist-iso} and the fact that rigid motions are isometries, we may assume
$\vec{x}=(0,\ldots,0)$ and $\vec{y}=(d,0,\ldots,0)$, where $d\geq 0$. Details are left as an exercise.  
\end{example}

\bigskip

\noindent
{\bf \large Discrete group actions}

\begin{definition}
Let $G$ be a group and $M$ a set.  We say that $G$ {\em acts on $M$ on the left} (resp. {\em on the right})  if there is a map 
$\phi : G \times M \longrightarrow M$, $\phi(m,g)= \phi_g(m) = g\cdot m$ (resp. $m\cdot g$),
satsifying the following (i) and (ii) (resp. (ii)').  
\begin{enumerate}
\item[(i)] If  $e \in G$ is the identity of $G$ then $\phi_e: M\to M$ is the identity map.
\item[(ii)] (left action) For any $g_1,g_2\in G$, we have $\phi_{g_1g_2} = \phi_{g_1} \circ \phi_{g_2}$,\\ i.e. 
$(g_1g_2)\cdot m = g_1\cdot (g_2\cdot m)$ for all $m\in M$.
\item[{(ii)'}] (right action) For any $g_1,g_2\in G$, we have
$\phi_{g_1g_2}=\phi_{g_2}\circ \phi_{g_1}$, \\ i.e.,
$m\cdot (g_1g_2)=(m\cdot g_1)\cdot g_2$ for all $m\in M$. 
\end{enumerate}
\end{definition}

\begin{remark}
A left (resp. right) $G$-action on a set $M$ is the same thing as
a group homomorphism $G \to (\text{Perm}(M),\circ)$ given by  $g\mapsto \phi_g$
(resp. $g\mapsto \phi_{g^{-1}}$.) 
\end{remark}

\begin{definition}
Let $G$ be a group and $M$ a topological space. Then we say that {\em $G$ acts on $M$ on the left} (resp. {\em on the right}) if there is a map 
$\phi : G \times M \to M$ satisfying (i) and (ii) (resp. (ii)') above and also 
\begin{itemize}
\item[(iii)] The map $\phi_g : M \to M$ is continuous for each $g \in G$. 
\end{itemize}
\end{definition}

\begin{remark}
A left (resp. right) $G$-action on a topological space $M$
is the same thing as a group homomorphism $G \to (\text{Homeo}(M),\circ) $ given by
$g\mapsto \phi_g$ (resp. $g\mapsto \phi_{g^{-1}}$.) 
\end{remark}

\begin{definition}\label{proper-discon}
Let $G$ be a group and $M$ a topological space and suppose $G$ acts on $M$ on the left. The action of $G$ on $M$ is called 
{\em properly discontinuous} if for each point $p \in M$, there is a neighborhood $U$ of $p$ in $M$ such that for each $g \in G\setminus\{e\}$, we have $\phi_g(U) \cap U = \varnothing$. 
\end{definition}

\begin{remark}
Let $U$ be as in Definition \ref{proper-discon}.
If $g_1, g_2 \in G$ are distinct then $\phi_{g_1}(U) \cap \phi_{g_2}(U) = \varnothing$. 
In particular, a properly discontinuous action is {\em free} in the sense that if $g \in G$ and $p \in M$, then $g \cdot p = p$ implies that $g =e$. 
\end{remark}

\begin{proposition}
If a group $G$ acts on a topological space $M$ properly discontinuously, then the map $\pi : M \to M/G$ is a covering map,
where $M/G$ is equipped with the quotient topology. 
\end{proposition}

\begin{proof}
For a point $\bar{p} \in M/G$, there is a $p \in M$ such that $\pi(p) = \bar{p}$. There is an open neighborhood $U$ of $p$ in $M$ such that if $g$ is not the identity, then $g(U) \cap U = \varnothing$. Let $\bar{U}$ be $\pi(U)$. Then $\bar{p} \in \bar{U}$ and 
$\pi^{-1}(\bar{U})$ is the disjoint union $\sqcup_{g \in G}\phi_g(U)$, where each $\phi_g(U)$ is open. 
It follows that $\bar{U}$ is an open neighborhood of $\bar{p}$ in $M/G$. Moreover, the restriction 
$\pi|_{\phi_g(U)} : \phi_g(U) \to \bar{U}$ 
is a homeomorphism for any $g \in G$. 
\end{proof}

\begin{definition}
Let $G$ be a group and let $M$ be a smooth manifold. We say that {\em $G$ acts on $M$ on the left} (resp. {\em on the right}) 
if there is a map $\phi : G \times M \to M$ satisfying (i) and (ii) (resp. (ii)'), and also
\begin{enumerate}
 \item[(iii)'] The map $\phi_g : M \to M$ is a smooth for each $g \in G$.
\end{enumerate} 
\end{definition}

\begin{remark}
A left (resp. right) $G$-action on a smooth manifold $M$
is the same thing as a homomorphism $G \to \text{Diffeo}(M)$ given by 
$g\mapsto \phi_g$ (resp. $g\mapsto \phi_{g^{-1}}$). 
\end{remark}

\begin{proposition}
Suppose that a group $G$ acts on a smooth manifold $M$ properly discontinuously. Then 
\begin{enumerate}
\item[(i)] There is a unique smooth structure on $M/G$ such that $\pi : M \to M/G$ is a local smooth diffeomorphism. 
\item[(ii)] If $h$ is a Riemannian metric on $M$ and $\phi_g$ is an isometry of $(M,h)$ for each $g\in G$ (in this case we say that $G$ 
{\em acts isometrically on $(M,h)$}), then there is a unique Riemannian metric $\hat{h}$ on $M/G$ such that $\pi^*\hat{h} = h$. 
\end{enumerate}
\end{proposition}

\begin{example}
Let $G = \{\pm1\}$ and let $M = S^n$. Let $\phi_1 = \text{id}$ and let $\phi_{-1}$ be the antipodal map $A:S^n\to S^n$,
$A(x)=-x$.  Then $G$ acts properly discontinuously and isometrically on $(S^n, g_{can})$. It follows that there is a metric 
$\hat{g}$ on $P_n(\bR)$ such that $\pi^* \hat{g} = g_{can}$.  When $n=1$,  $(P_1(\bR),\hat{g})$ is isometric
to $S^1(\frac{1}{2})$ (circle of radius $\frac{1}{2}$). 
\end{example}

\begin{example}
Let $G=\bZ^n$ acts $\bR^n$ by 
\[
(m_1, \ldots, m_n) \cdot (x_1, \ldots, x_n) \mapsto (x_1 + m_1, \ldots, x_n + m_n),
\]
where $(m_1,\ldots,m_n)\in \bZ^n$ and $(x_1,\ldots,x_n)\in \bR^n$, i.e., 
$\phi_{(m_1,\ldots,m_n)}:\bR^n\to \bR^n$ is translation by the vector $(m_1,\ldots,m_n)$.
This action is properly discontinuous and preserves the Euclidean metric $g_0$, so it
descendents to a Riemannian metric $\hat{g}_0$, known as the flat metric, on
the quotient $T^n=\bR^n/\bZ^n$.  
There is an isometry $(\mathbb{R}^n/ \mathbb{Z}^n, \hat{g}_0) \to (S^1(\frac{1}{2\pi}))^n$. 
\end{example}


We now discuss orientation. 

\begin{definition}
Let $V$ be a real vector space of dimension $n$. An {\em orientation} on $V$ is an equivalence class of ordered bases, where two bases are equivalent if the change of coordinates matrix has positive determinant. 
\end{definition}

Let $(U_\alpha, \phi_\alpha)$ be a smooth atlas on a smooth manifold $M$ and say it defines an orientation, meaning that the 
transition functions have positive Jacobian. Choose local coordinates $\phi_\alpha = (x_1^\alpha, \ldots, x_n^\alpha)$ around 
$p \in M$. Then the basis $\{\frac{\partial}{\partial x_i}(q)\}$ defines an orientation on $T_qM$ for each $q \in U_\alpha$. 

\begin{definition}
Suppose that $f : M_1 \to M_2$ is a local diffeomorphism between oriented smooth manifolds. We say that $f$ is {\em orientation preserving}
(resp. {\em orientation reversing}) at 
$p \in M_1$ if given an ordered basis $(e_1, \ldots, e_n)$ of $T_pM_1$ compatible with the orientation on $M_1$, the ordered basis 
$(df_p(e_1), \ldots, df_p(e_n))$ of $T_{f(p)}M_2$ is compatible (resp. not compatible) with the orientation on $M_2$. 

We say $f$ is {\em orientation preserving} (resp. {\em orientation reversing}) if it is orientation preserving
(resp. orientation reversing) at all $p\in M_1$.
\end{definition}

\begin{remark}
If $M_1$ and $M_2$ are connected, then $f$ is orientation preserving (reversing) at some point $p \in M_1$ if and only if $f$ is orientation preserving (reversing) at all $p \in M_1$. 
\end{remark}

\begin{example}
The antipodal map $A : S^n \to S^n$ is orientation preserving if and only if $n$ is odd. (cf. Problem (4) of Assignment 6)
\end{example}

\begin{example}
The action of $\bZ^n$ on $\bR^n$ by translation is orientation preserving. 
\end{example}


\section{Wednesday, October 21, 2015}

\noindent
{\bf \large Lie groups}

\begin{definition}
A {\em Lie group} $G$ is a group together with the structure of a smooth manifold such that $\lambda : G \times G \to G$ given by $\lambda(x,y) = xy^{-1}$ is a smooth map. 
\end{definition}

\begin{remark}
From the definition:
\begin{itemize}
\item (inverse) The map $G \to G$ given by $x \mapsto x^{-1}$ is smooth.
\item (multiplication) The map $G \times G \to G$ given by $(x,y)\mapsto xy$ is smooth.
\item (left multiplication) For any $x\in G$, the map $L_x:G\to G$ given by 
$L_x(y)=xy$ (left multiplication by $x$) is a  smooth map.
\item (right multiplication) For any $x\in G$, the map $R_x:G\to G$ given by 
$R_x(y)=yx$ (right multiplication by $x$) is a  smooth map. 
\end{itemize}
Indeed, $G$ acts on $G$ on the right (resp. on the left) by right (resp. left) multiplication, so 
$L_x$ and $R_x$ are smooth diffeomorphisms for any $x\in G$.
\end{remark}


\begin{example}
$(\bR^n, +)$  is a Lie group.
\end{example}

\begin{example}
The set $GL(n,\bR)$ of invertible $n\times n$ matrices is a smooth manifold with a smooth group operation given by matrix multiplication. 
This manifold has two connected components, namely, $GL(n,\bR)_+=\{ A\in GL(n,\bR):\det A>0\}$ and 
$GL(n,\mathbb{R})_-=\{ A\in GL(n,\bR):\det A<0\}$. $GL(n,\bR)_+$ is a connected Lie group. 
The special linear group $SL(n,\bR)=\{A\in GL(n,\bR):\det A = 1\} $ is a Lie subgroup of $GL(n,\bR)$.
 \end{example}

\begin{example}
The orthogonal group $O(n) = \{A \in GL(n,\bR) : A^T A = I_n\}$
is a Lie subgroup of $GL(n,\bR)$. It has two connected components;
$SO(n)=O(n)\cap SL(n,\bR)$, the connected component of the identity,
is a Lie subgroup of $SL(n,\bR)$. 
\end{example}

\begin{definition}\label{invariant-tensor}
Let $G$ be a Lie group. A tensor $T$ on $G$ is {\em left} (resp. {\em right}) {\em invariant} if $L_x^*T = T$
(resp. $R_x^*T=T$)  for each $x \in G$. 
If a tensor $T$ on $G$ is both left-invariant and right-invariant, then $T$ is called {\em bi-invariant}. 
\end{definition}

\begin{remark}
Note that if $T$ is left (resp. right) invariant then $T$ is determined by 
$T(e)$, the value of $T$ at the identity $e\in G$.  In particular:
\begin{itemize}
\item A function on $G$ is left (resp. right) invariant 
if and only if it a constant function.
\item A vector field $X$ on $G$ is left (resp. right) invariant if and only if for each $x\in G$, we have
$X(x)=d(L_x)_e(X(e))$ (resp. $X(x)=d(R_x)_e(X(e))$).
\end{itemize}
\end{remark}



Let $\mathfrak{X}(G)^L$ (resp. $\mathfrak{X}(G)^R$) denote the space of left (resp. right) invariant vector fields. 
We have an $\bR$-linear isomorphisms $T_eG \stackrel{\simeq}{\longrightarrow} \mathfrak{X}(G)^L$ 
(resp. $T_eG \stackrel{\simeq}{\longrightarrow} \mathfrak{X}(G)^R$) described by 
$\xi \mapsto X_\xi^L$ (resp. $\xi\mapsto X_\xi^R$),
where $X_\xi^L$ (resp. $X_\xi^R$) is the unique left (resp. right) invariant vector field on 
$G$ such that $X_\xi^L(e)=\xi$ (resp. $X_\xi^R(e)=\xi$). More explicitly,
$X_\xi^L(x)=d(L_x)_e(\xi)$ and $X_\xi^R(x)=d(R_x)_e(\xi)$, $x\in G$. 


\begin{definition}
Let $F : M \to N$ be smooth and let $X$ be a smooth vector field on $M$ and $Y$ a smooth vector field on $N$. We say that $X$ and $Y$ are 
{\em $F$-related} if for each $p \in M$, we have $dF_p(X(p)) = Y(F(p))$. 
\end{definition}

\begin{remark}
If $F$ is a diffeomorphism then $X$ and $Y$ are $F$-related if and only if $Y = F_*X$. 
\end{remark}

\begin{remark}\label{F-related}
More generally, $X$ and $Y$ are $F$-related if and only if for each $f \in C^\infty(N)$, we have $X(F^*f) = F^*(Y(f))$. 
\end{remark}

\begin{proposition}
Let $F : M \to N$ be smooth, let $X_1, X_2$ be smooth vector fields on $M$ and let $Y_1, Y_2$ be smooth vector fields on $N$. Suppose that $X_i$ and $Y_i$ are $F$-related. Then $[X_1, X_2]$ and $[Y_1, Y_2]$ are $F$-related. 
\end{proposition}

\begin{proof}
Let $f$ be a smooth function on $N$. Then  
\begin{align*}
[X_1, X_2] (F^*f) &= X_1(X_2 F^*f) - X_2(X_1F^*f) \\
&= X_1(F^*(Y_2f)) - X_2(F^*(Y_1f)) \\
&= F^*(Y_1Y_2f) - F^*(Y_2Y_1f) \\
&= F^*([Y_1,Y_2]f),
\end{align*}
where the second and the third equalities follow from Remark \ref{F-related}.
By Remark \ref{F-related}, $[X_1, X_2]$ and $[Y_1, Y_2]$ are $F$-related. 
\end{proof}

\begin{corollary}
If $F$ is a smooth diffeomorphism and $X_1, X_2$ are vector fields on $M$, then 
\[
[F_*X_1, F_*X_2] = F_*[X_1, X_2].
\]
\end{corollary}

\begin{corollary}
The set of left invariant vector fields $\mathfrak{X}(G)^L$ is a Lie subalgebra of $\mathfrak{X}(G)$. 
So is the set $\mathfrak{X}(G)^R$. 
\end{corollary}



\begin{definition}
We define $[-,-] : T_eG \times T_eG \to T_eG$ by 
\[
(\xi, \eta) \mapsto [X_\xi^L, X_\eta^L](e).
\]
We define the {\em Lie algebra} $\mathfrak{g}$ of $G$ to be $T_eG$ with the above Lie bracket. 
Then we note that we have an isomorphism $\mathfrak{g} \simeq \mathfrak{X}(G)^L$ as Lie algebras. 
\end{definition}

\begin{remark}[Assignment 7 (1)]
If we let $i : G \to G$ denote the map $g \mapsto g^{-1}$, then $i^2 = \text{id}$ and $di_e(\xi) = -\xi$. We have 
\[
\xymatrix{
G \ar[r]^i \ar[d]_{L_a}& G\ar[d]^{R_{a^{-1}}} \\
G \ar[r]^i & G.
}
\]
It follows that $X_\xi^R = -i_*X_\xi^L$. Hence,
\[
[X_\xi^R, X_\eta^R] = [i_*X_\xi^L, i_*X_\eta^L]  = i_*[X_\xi^L, X_\eta^L] = i_*X_{[\xi,\eta]}^L = - X_{[\xi, \eta]}^R.
\]
\end{remark}

\begin{proposition}\label{TG}
The tangent bundle of a Lie group is trivial. 
\end{proposition}

\begin{proof}
Let $\xi_1, \ldots, \xi_n$ be a basis of $\mathfrak{g} = T_eG$. Then $X_{\xi_1}^L, \ldots, X_{\xi_n}^L$ forms a global $C^\infty$ frame of $TG$.
Let $\phi: G \times \mathfrak{g} \to TG$ be the map 
\[
(x,\xi) \mapsto (x, X_\xi^L(x)).
\]
Then $\phi^{-1}: TG\to G\times \mathfrak{g}$ is a global trivialization of $TG$.
\end{proof}

\begin{example}
Let $G = (\bR^n, +)$. For any $a_1,\ldots, a_n\in \bR$, the vector field 
$\sum_{i=1}^n a_i\frac{\partial }{\partial x_i}$ is bi-invariant. We have
$$
\mathfrak{X}(G)^L = \mathfrak{X}(G)^R = \{ \sum_{i=1}^n a_i \frac{\partial}{\partial x_i}: (a_1,\ldots, a_n)\in \bR^n\} \cong \bR^n.
$$
The Lie bracket on $T_0\bR^n$ is trivial. The map $\phi$ in the proof of Proposition \ref{TG} is given by
$$
\phi: \bR^n\times \bR^n\to T \bR^n,\quad (x,y) \mapsto (x, \sum_{i=1}^n y_i \frac{\partial}{\partial x_i})
$$
where $x=(x_1,\ldots,x_n)$, $y=(y_1,\ldots,y_n)$.
\end{example}

\begin{example}
Let $G = GL(n,\bR)$. Recall that $\mathfrak{g} = M_n(\mathbb{R})$. For $\xi \in M_n(\mathbb{R})$, then 
$d(L_A)_{I_n}(\xi) = A \xi$ and $d(R_A)_{I_n}(\xi) = \xi A$. Because of this, we see that 
\begin{eqnarray*}
X_\xi^L(A) &=&  A\xi =  \sum_{i,j} (\sum_k a_{ik} \xi_{kj}) \frac{\partial}{\partial a_{ij}}\\
X_\xi^R(A) &=&  \xi A =  \sum_{i,j}(\sum_k\xi_{ik} a_{kj}) \frac{\partial}{\partial a_{ij}}
\end{eqnarray*}
The map $\phi: GL(n,\bR) \times \mathfrak{g} \to TG = GL(n,\bR)\times M_n(\bR)$ is described by 
\[
(A, \xi) \mapsto (A, A\xi). 
\] 

Moreover, if $H$ is a Lie subgroup of $G = GL(n,\bR)$, $\phi$ restricts to
$H\times \mathfrak{h}\subset G\times \mathfrak{h}\to TH \subset TG$. For example,
$H=SL(n,\bR)$, $\mathfrak{h}=\mathfrak{sl}(n,\bR)$; $H=O(n)$ or $SO(n)$, 
$\mathfrak{h}=\mathfrak{so}(n)$. 
 \end{example}

\begin{remark}
This argument of trivializing a bundle will work also for the cotangent bundle of $G$ and more generally for any tensor bundle $T_s^rG$ of $G$. 
Indeed, if $E\to M$ is a trivial vector bundle then the dual bundle $E^*\to M$ is also trivial and more generally
$E^{\otimes r}\otimes (E^*)^{\otimes s}$ is a trivial vector bundle for any $r,s\in \bZ_{\geq 0}$.  
\end{remark}

\begin{lemma}\label{left-flow}
Let $\phi_\xi^L$ be the flow of $X_\xi^L$ and $\phi_\xi^R$ the flow of $X_\xi^R$. Then \begin{enumerate}
\item[(i)] For each $a \in G$, we have 
\[
L_a \circ \phi_\xi^L(t,x) = \phi_\xi^L(t, ax)
\]
\item[(ii)] For each $a \in G$, we have 
\[
R_a \circ \phi_\xi^R(t,x) = \phi_\xi^R(t, xa). 
\]
\end{enumerate}
\end{lemma}

\begin{remark}
This is saying that left (resp. right) multiplication by $a$ carries an integral curve 
of a left (resp. right) invariant vector field to another integral curve of this vector field.  
\end{remark}

\begin{proof}[Proof of Lemma \ref{left-flow}]
It suffices to show that 
\begin{enumerate}
\item[(a)] $(L_a \circ \phi_\xi^L)(0,x) = ax$  
\item[(b)] $\frac{d}{dt}(L_a \circ \phi_\xi^L)(t, x) = X_\xi^L((L_a \circ \phi_\xi^L)(t,x))$.
\end{enumerate}
To see (a), we note that 
\[
L_a \circ \phi_\xi^L(0,x) = a \cdot \phi_\xi^L(0,x) = ax.
\]
For (b), we note that 
\begin{align*}
\frac{d}{dt}(L_a \circ \phi_\xi^L)(t, x) &= d(L_a)_{\phi_\xi^{L}(t,x)}(\frac{d}{dt}\phi_\xi^L(t,x))  \\
&= d(L_a)_{\phi_\xi^{L}(t,x)} (X_\xi^L(\phi_\xi^L(t,x))) \\
&= X_\xi^L(L_a \circ \phi_\xi^L(t,x)).
\end{align*}
\end{proof}

\begin{proposition}
If $G$ is a Lie group and $\xi \in \mathfrak{g}$, then $\phi_\xi^L, \phi_\xi^R$ are defined on $\mathbb{R} \times G$. 
\end{proposition}

\begin{proof}
There is an $\epsilon > 0$ and an open neighborhood $V$ of $e$ in $G$ such that $\phi(t,x)$ is defined for $(t,x) \in (-\epsilon, \epsilon) \times V$. By the previous result, we see that $\phi_t(x)$ is defined for $(t,x) \in (-\epsilon, \epsilon) \times G$. Then we see that $\phi_{nt}(x) = \phi_t \circ \cdots \circ \phi_t(x)$ is defined for all $n \in \mathbb{N}$, $t\in (-\epsilon,\epsilon)$, $x\in G$,  and hence $\phi(t,x)$ is defined for all $(t,x)\in \bR\times G$.  
\end{proof}

\begin{example}
If $G=GL(n,\bR)$ or any Lie subgroup of $GL(n,\bR)$, then, we see that $X_\xi^L(A) = A\xi$ and also that 
\begin{eqnarray*}
&&  X_\xi^L(A)=A\xi, \quad \phi_\xi^L(t,A) = A \exp(t\xi),\\
&& X_\xi^R(A)=\xi A,\quad \phi_\xi^R(t,A) = \exp(t \xi)A. 
\end{eqnarray*}
Here $\exp(B)=\sum_{n=0}^\infty \frac{B^n}{n!}$, $B\in M_n(\bR)$. 
We want to use this observation to extend the definition of the exponential to any Lie group. 
\end{example}

\begin{definition}[Exponential map]
If $G$ is a Lie group. Define the {\em exponential map} $\exp : \mathfrak{g} \to G$ by the rule 
\[
\xi \mapsto \phi_\xi^L(1,e)
\]
where $e$ is the identity of $G$. 
\end{definition}

\begin{remark}
We note that $\phi_\xi^L(t,x) = \phi_{t\xi}^L(1,x) = \phi_{t\xi}^L(1,x\cdot e) = x \phi_{t\xi}^L(1,e)=  x \exp(t\xi)$. It follows that 
\[
\phi_\xi^L(t,x) = x \exp(t\xi).
\]
In other words 
\[
(\phi_\xi^L)_t = R_{\exp(t\xi)} : G \to G. 
\]
\end{remark}




\section{Wednesday, October 28, 2015}

As a special case of Definition \ref{invariant-tensor}:

\begin{definition}
Let $G$ be a Lie group and let $g$ be a Riemannian metric on $G$.   We say $g$ is {\em left-invariant} if 
$L_x^*g=g$ for all $x\in G$. Equivalently, $g$ is left-invariant if and only if for each $x\in G$, 
$L_x: (G,g)\to (G,g)$ is an isometry.
\end{definition}

\begin{remark} We have a one-to-one correspondence:
$$
\{ \textup{left-invariant metrics on }G\} \leftrightarrow \{ \textup{inner products on }T_eG\}.
$$
Indeed, $g$ is left-invariant if and only if for each $x \in G$ and for each $U,V \in T_xG$,  
\[
g(x)(U,V) = g(e)(d(L_{x^{-1}})_xU,d(L_{x^{-1}})_xV ).
\]
\end{remark}

\begin{example} $G=(\bR^n, +)$, $g_0=dx_1^2+\cdots dx_n^2$. 
For any $x\in \bR^n$, $L_x^*g=R_x^*g=g$. So $g$ is bi-invariant.
\end{example}

\begin{example}
Let 
$$
G = \{g : \bR \to \bR,  t\mapsto  yt + x :  x \in \bR, y \in (0,\infty)\},
$$
that is, the group of proper affine transformations of $\bR$. Define multiplication by composition: $g_1(t) = y_1t + x_1$ and $g_2(t) = y_2t + x_2$, then 
\[
(g_1 \circ g_2)(t) = g_1(y_2 t+ x_2)= y_1(y_2t+x_2) + x_1 = y_1y_2 t + (y_1 x_2+x_1).
\]


We may identify $G$ with the upper half plane: $G=\{(x,y)\in \bR^2: y>0\}$. With this identification, the multiplication 
is given by
\[
(x_1, y_1) \cdot (x_2, y_2) = (y_1x_2 + x_1, y_1y_2). 
\]
So the multiplication defines a smooth map $G\times G\to G$. The identity element is $e = (0,1)$. The inverse map is given by 
\[
(x_1, y_1)^{-1} = (-x_1y_1^{-1}, y_1^{-1}),
\]
which is smooth. So $G$ is indeed a Lie group. 

We note that 
\[
L_{(a,b)}(x,y) = b(x,y) + (a,0).
\]
And hence 
\[
d(L_{(a,b)})_{(x,y)}(v) = bv. 
\]


Let $g$ be the unique left-invariant metric on $G$ such that $g(0,1) =  dx^2 + dy^2$. We know that $g$ is of the form $g = Edx^2 + 2Fdxdy + Gdy^2$ for some smooth 
functions $E,F,G$, where $E(0,1)=G(0,1)=1$, and $F(0,1)=0$. We compute
$$
L^*_{(a,b)}dx = d (bx+a)=bdx,\quad
L^*_{(a,b)}dy = d(by)=bdy.
$$
So
$$
L_{(a,b)}^*g(x,y) = E(bx+a,by)b^2 dx^2 + 2F(bx+a,by)b^2 dxdy + G(bx+a,by)b^2 dy^2.
$$
$$
L_{(a,b)}^*g(0,1) = E(a,b)b^2 dx^2 + 2F(a,b)b^2 dxdy + G(a,b)b^2 dy^2.
$$
Since $g$ is left-invariant, $(L_{(a,b)}^*g)(0,1)= g(0,1)= dx^2 + dy^2$, so 
$$
E(a,b)=\frac{1}{b^2},\quad F(a,b)=0,\quad G(a,b)=\frac{1}{b^2}.
$$
We conclude that 
$$
g=\frac{dx^2+dy^2}{y^2}.
$$
We find that 
\[
g = \frac{dx^2 + dy^2}{y^2}. 
\]
We remark that there is a natural inclusion $G \hookrightarrow \text{Isom}(G,g)$ given by $x \mapsto L_x$. 

We can check that this metric is not right-invariant. Indeed 
\[
R_{(a,b)}(x,y) = (ay + x, by).
\]
So we find that 
\begin{align*}
R_{(a,b)}^*dx = dR_{(a,b)}^*x = dx + ady \\
R_{(a,b)}^*dy = dR_{(a,b)}^*y = bdy.
\end{align*}
And hence 
\[
R^*_{(a,b)} g = \frac{ (dx+ady)^2 + (bdy)^2}{(by)^2}= \frac{dx^2 + 2a dxdy + (a^2 + b^2)dy^2}{b^2y^2}.
\]
\end{example}

\medskip


John Milnor proved the following:
\begin{theorem}[{
\cite[Lemma 7.5]{Mi}}]
A connected Lie group admits a bi-invariant Riemannian metric if and only
if it is isomorphic to the direct product of a compact Lie group and an additive vector group.
\end{theorem}


\begin{definition}[Adjoint representation]
Let $G$ be a Lie group. Given an element $a\in G$, the map $R_{a^{-1}} \circ L_a : G \to G$ 
is a diffeomorphism sending $e$ to $e$, and hence we get a linear isomorphism 
\[
\Ad(a) := d(R_{a^{-1}} \circ L_a)_e : T_eG \to T_eG.
\]
This means that we get a group homomorphism
\begin{align*}
\Ad : G &\to GL(\fg) \\
a &\mapsto \Ad(a)
\end{align*}
where $GL(\fg)$ is the space of $\bR$-linear isomorphisms of $\fg$. This is a representation of $G$ called the 
{\em adjoint representation}. 
\end{definition}

\begin{example}
\begin{enumerate}
\item Let $G = (\bR^n, +)$. For any $a\in \bR^n$, $R_{a^{-1}} \circ L_a = \mathrm{id}$ is the identity map, and hence 
\begin{align*}
\Ad(a) = \text{id}_{\fg}
\end{align*}
for each $a \in G$. 
\item More generally, for any abelian Lie group, the adjoint representation is trivial.
 
\item Let $G = GL(n,\bR)$ or any subgroup of $GL(n,\bR)$. In this case 
\[
\Ad(A)(\xi) = A \xi A^{-1},\quad \textup{where }A\in GL(n,\bR),\ \xi\in \fgl(\bR).
\]
\end{enumerate}
\end{example}

\begin{proposition}[{\cite[page 41]{dC}}]
Let $\xi, \eta \in \fg$. Then 
\[
[\xi, \eta] = \frac{d}{dt}\Big|_{t=0}\Ad(\exp(t \xi))\eta.
\]
We set $\ad(\xi)\eta = [\xi, \eta]$. The map $\ad : \fg \to \fgl(\fg)$ is called {\em adjoint representation} of the Lie algebra. 
\end{proposition}

\begin{proof}
We note that 
\begin{align*}
\Ad(\exp(t\xi))\eta &= d(R_{-\exp(t\xi)})_{\exp(t\xi)}d(L_{\exp(t\xi)})_e\eta \\
&= d(R_{-\exp(t\xi)})_{\exp(t\xi)}(X_\eta^L(\exp(t\xi))) \\
&= \phi_t^*X_\eta^L(e)
\end{align*}
where $\phi_t = R_{\exp(t\xi)}$ is the flow of $X^L_\xi$. So 
\[
\frac{d}{dt}\Big|_{t=0} \Ad(\exp(t\xi))\eta = \frac{d}{dt}\Big|_{t=0} (\phi_t^*X_\eta^L)(e)
= [X_\xi^L, X_\eta^L](e) = [\xi, \eta].
\]
\end{proof}

\begin{example}
Let $G = GL(n,\mathbb{R})$ or a subgroup. Then for $\xi,\eta\in \fgl(n,\bR)$ 
\[
[\xi, \eta] = \frac{d}{dt}\Big|_{t=0}e^{t\xi}\eta e^{-t\xi} = \xi \eta - \eta \xi.
\]
\end{example}

\bigskip

\noindent
{\bf \large Continuous group actions}

\begin{definition}
Let $G$ be a group and $M$ a set. Suppose that $G$ acts on $M$ on the left. For any $p \in M$:
\begin{itemize} 
\item Let $G_p$ denote the {\em stabilizer of $p$}, that is, 
$G_p = \{g \in G : g \cdot p = p\}$
\item  Let $G \cdot p$ denote the {\em orbit of $p$}, that is, 
$G \cdot p = \{g \cdot p : g \in G\}$.
\end{itemize}
We say $G$ acts on $M$ {\em freely} if $G_p = \{e\}$ for each $p \in M$. 
We say that $G$ acts {\em transitively} if $M = G\cdot p$ for some $p \in M$ (which implies
$M=G\cdot p$ for all $p\in M$). 
\end{definition}

\begin{definition}[topological group]
We say that $G$ is a {\em topological group} if $G$ is a topological space together with a group structure such that the map $G \times G \to G$ given by $(x,y) \mapsto xy^{-1}$ is continuous. 
\end{definition}

\begin{definition}
Let $G$ be a group and $M$ a set, and suppose that $G$ acts on $M$ on the left. Let $\phi : G \times M \to M$ denote the action. 
\begin{enumerate}
\item[(i)] If $G$ is a topological group and $M$ is a topological space, we say the action is {\em continuous} if $\phi$ is continuous as a map from the product space. 
\item[(ii)] If $G$ is a Lie group and $M$ is smooth, then we say that the action is {\em smooth} if $\phi$ is {smooth} if $\phi$ is smooth as a map from the product manifold. 
\end{enumerate}
\end{definition}

\begin{lemma}
Let $G$ be a group, let $M$ be a topological space. Equip $G$ with the discrete topology. Then $\phi : G \times M \to M$ is continuous if and only if for each $g \in G$, the map $\phi_g : M \to M$ is continuous. 
\end{lemma}

\begin{proof}
($\Rightarrow$) If $\phi$ is continuous, then we note that $\phi_g = \phi \circ i_g$, where $i_g : M \to G\times M$ is the map $i_g(p) = (g,p)$, which is continuous, since $G$ is given the discrete topology. 

($\Leftarrow$) Suppose that each $\phi_g$ is continuous. Let $U$ be an open subset of $M$. Then we note that 
\[
\phi^{-1}(U) = \bigcup_{g \in G}(\{g\} \times \phi_g^{-1}(U)).
\]
Each of the sets in the union is open, and hence so is the union. 
\end{proof}



\begin{definition}
Let $G$ be a topological group and let $M$ be a Hausdorff topological manifold. 
Suppose $G$ acts on $M$ on the left continuously. We say that the action is {\em proper} if 
for any compact $K\subset M$, the set $G_K:=\{ g\in G: \phi_g(K)\cap K \neq \varnothing\}$
is relative compact in $G$, i.e. the closure of $G_K$ is compact. (This is automatic
if $G$ is compact.) 
\end{definition}



\begin{example}
Suppose that $\mathbb{C}^*$ acts on $\mathbb{C}$ by multiplication. Then this action is not proper. On the other hand if $\mathbb{C}^*$ acts on $\mathbb{C}^*$, then the action is proper. 
\end{example}

\begin{remark}
\begin{enumerate}
\item[(i)] Suppose that $G$ is a discrete group. The action is continuous and proper if and only if for each compact subset $K \subset M$, the set $G_K$ is finite. In particular, when $K = \{p\}$,
$G_K=G_p$, we see that the stabilizer $G_p$ of $p$ is finite.
\item[(ii)] Suppose that $G$ is discrete. Suppose that the action is continuous, proper, and free. We already know that for any $p \in M$ there is an open neighborhood $U$ of $p$ in $M$ such that $\overline{U}$ is compact. Because $G_{\overline{U}}$ is finite, we claim that $(G \cdot p) \cap U$ is finite. Because $M$ is Hausdorff, there is an open neighborhood $U'$ of $p$ such that 
$U' \cap \phi_g(U') = \varnothing$ for each $g \in G\setminus\{e\}$. This means that the action is ``properly discontinuous.''
\end{enumerate}
\end{remark}

\begin{example}
Let $S^1 = \{z \in \mathbb{C} : |z| = 1\}$, a Lie group. Also $S^{2n+1} = \{(z_0, \ldots, z_n) \in \mathbb{C}^{n+1} : |z_0|^2 + \cdots |z_n|^2 = 1\}$. Let $S^1$ act on $S^{2n+1}$ by the rule 
\[
\lambda \cdot(z_0, \ldots, z_n)  = (\lambda z_0, \ldots, \lambda z_n).
\]
This action is smooth. The action is also proper because $S^1$ is compact. Moreover the action is free. 
\end{example}

\begin{theorem}\label{thm:MG}
Let $G$ be a Lie group and let $M$ be a smooth manifold. If $G$ acts on $M$ smoothly, freely, and properly, then there is a unique smooth structure on $M/G$ such that $\pi : M \to M/G$ is a smooth submersion. 
\end{theorem}

\begin{example}
Let $\pi : S^{2n+1} \to P_n(\bC) = S^{2n+1}/S^1$ be the projection. We already constructed a $C^\infty$ atlas on $P_n(\bC)$. We can check that $\pi$ is a $C^\infty$ submersion with respect to 
this $C^\infty$ structure on $P_n(\bC)$. Theorem \ref{thm:MG} implies that this $C^\infty$  structure is unique with these properties. It follows that $P_n(\bC)$ is diffeomorphic to $S^{2n+1}/S^1$, 
where $S^{2n+1}/S^1$ is equipped with the unique $C^\infty$ structure given by 
Theorem \ref{thm:MG}.
\end{example}

\section{Wednesday, November 4, 2015}



\begin{definition}[Smooth fibration]
A map $\pi : E \to B$ is a {\em smooth fibration} with {\em total space} $E$, {\em base} $B$, and 
{\em fiber} $F$ if 
\begin{enumerate}
\item[(i)] $E,B,F$ are smooth manifolds. 
\item[(ii)] $\pi$ is a surjective smooth map. 
\item[(iii)] There is an open cover $\{U_\alpha:\alpha\in I\}$ of $B$ and smooth diffeomorphisms
\[
h_\alpha : \pi^{-1}(U_\alpha) \to U_\alpha \times F
\]
such that $\pi|_{\pi^{-1}(U_\alpha)} = \text{pr}_1 \circ h_\alpha$, where
$\mathrm{pr}_1:U_\alpha\times F\to U_\alpha$ is the projection to the first factor.
(It follows that $\pi$ is a submersion.)

\end{enumerate}
\end{definition}

\begin{example}
Take $E = B \times F$ with $\pi : E \to B$ being projection onto the first factor. This is called the {\em product fiber bundle} with base $B$ and fiber $F$.
\end{example}

\begin{definition}
We say that $\pi : E \to B$ is a {\em trivial fiber bundle} over $B$ with fiber $F$ if there is a smooth diffeomorphism $h : E \to B \times F$ such that $\pi = \mathrm{pr}_1 \circ h$. 
\end{definition}

\begin{example}
If $\pi : E \to B$ is a smooth vector bundle of rank $r$, then $\pi : E \to B$ is a smooth fibration with fiber $\bR^r$. But the converse is not true: the transition functions for a vector bundle need to satisfy some additional linearity requirement. 
\end{example}

\begin{example}
A covering space is a smooth fibration with discrete fiber. 
\end{example}

\begin{theorem}
Let $G$ be a Lie group and let $M$ be a smooth manifold. If $G$ acts on $M$ smoothly, freely, and properly, then there is a unique smooth structure on $M/G$ such that $\pi : M \to M/G$ is a smooth fibration with fiber $G$. 
\end{theorem}

\begin{example}
The map $\pi : S^{2n+1} \to P_n(\mathbb{C})$ is a smooth circle bundle, known as the {\em Hopf fibration}. 
\end{example}

\noindent
{\bf \large Riemannian submersions}

Let $f : (M,g) \to (N,h)$ be a smooth submersion between Riemannian manifolds. For a point $p \in M$, let $q = f(p) \in N$. Then we have an exact sequence of the form
\[
0 \to T_pf^{-1}(q) \to T_pM \stackrel{df_p}{\to} T_qN \to 0.
\]
Let $H_p$ be the orthogonal complement of $T_pf^{-1}(q)$ in $T_pM$ (using the metric $\langle - , - \rangle_p$). If we restrict $df_p$ to $H_p$, then we see that $df_p|_{H_p}$ gives a
linear isomorphism $H_p \cong T_qN$. 

\begin{definition}[Riemannian submersion]
We say that $f : (M,g) \to (N,h)$ is a {\em Riemannian submersion} if $df|_{H_p} : H_p \to T_{f(p)}N$ is an inner product space isomorphism. This means that for any $u,v \in H_p$, we have 
\[
\langle u, v \rangle_p = \langle df_p(u), df_p(v) \rangle_q
\]
\end{definition}

\begin{theorem}
If $(M,g)$ is a Riemannian manifold and $G$ is a Lie group acting smoothly, freely, and properly on $M$ and in addition the action is by isometries, then there is a unique Riemannian metric $\hat{g}$ on $M/G$ such that $\pi : (M,g) \to (M/G, \hat{g})$ is a Riemannian submersion. 
\end{theorem}

\begin{proof}
To determine this metric, we write 
\[
\hat{g}(q)(u,v) = g(p)((d\pi |_{H_p})^{-1}u, (d\pi |_{H_p})^{-1}v )
\]
where $p\in \pi^{-1}(q)$.  The right hand side is independent of choice of $p\in \pi^{-1}(q)= G\cdot p$
since $(d\phi_g)_p$ defines an isomoetry from $H_p$ to $H_{g\cdot p}$.
\end{proof}

\begin{example}
Use the round metric $g_{can}$ on $S^{2n+1}$ induced by the Euclidean metric on $\mathbb{R}^{2n+2}$. Then $S^1$ acts on $S^{2n+1}$ smoothly, freely, properly, and isometrically. So there is a unique Riemannian metric $\hat{g}_{can}$ on $P_n(\mathbb{C})$ such that $\pi : S^{2n+1} \to P_n(\mathbb{C})$ is a Riemannian submersion. When $n = 1$, the space $P_n(\mathbb{C})$ is diffeomorphic to $S^2$ and $(P_1(\mathbb{C}), \hat{g}_{can})$ is isometric to $(S^2, \frac{1}{4}g_{can})$. (See Example
\ref{hopf} below.) So $\pi : S^3(1) \to S^2(\frac{1}{2})$ is a Riemannian submersion. 
\end{example}

\begin{theorem}
Let $G$ be a Lie group and let $H$ be a closed Lie subgroup. Then there is a unique smooth structure on $G/H$ such that 
\begin{itemize}
\item $\pi : G \to G/H$ is a smooth submersion and 
\item the action $\phi : G \times G/H \to G/H$ is smooth. 
\end{itemize}
\end{theorem}

\begin{theorem}
If $G$ is a Lie group and $M$ is a smooth manifold, then the following are equivalent. 
\begin{enumerate}
\item[(i)] $G$ acts on $M$ transitively, smoothly, and $H$ is the stabilizer of some $p \in M$
\item[(ii)] $M$ is diffeomorphic to $G/H$. 
\end{enumerate}
\end{theorem}

\begin{example}
Let $\phi : SO(n+1) \times S^n \to S^n$ be the smooth map described by $(A,x) \mapsto Ax$. The action is smooth, transitive. The stabilizer of $(0,0,\ldots, 0,1)$ is 
\[
\left\{\begin{bmatrix}
A & 0 \\ 0 & 1
\end{bmatrix} : A \in SO(n)\right\} \simeq SO(n).
\]
So there is a map 
\begin{align*}
SO(n+1)/SO(n) &\to S^n \\
A \cdot SO(n) &\mapsto A\left[\begin{array}{c}0\\ \vdots \\ 0\\1\end{array}\right],
\end{align*}
which is a diffeomorphism. 

By Assignment 7 (3), there is a bi-invariant metric $g$ on $SO(n+1)$. There is a unique metric $\hat{g}$ on $SO(n+1)/SO(n)$ such that $\pi$ is a Riemannian submersion. 

Assignment 8 (2): $(SO(n+1)/SO(n), \hat{g})$ is isometric to
$(S^n,\lambda g_\mathrm{can})$ for some constant $\lambda >0$.
\end{example}


\begin{example}
Let $Gr(k,n) = \{V \subset \mathbb{R}^n : V \; \text{$k$-dimensional subspace of $\mathbb{R}^n$}\}$.  In particular we have $\mathbb{P}_n(\mathbb{R}) = \text{Gr}(1, n+1)$. Note that $O(n)$ acts transitively on $\text{Gr}(k,n)$ and the stabilizer of $\mathbb{R}^k \times \{0\}$ can be identified with $O(k) \times O(n-k)$. We may identify 
\[
\text{Gr}(k,n) = O(n)/(O(k) \times O(n-k))
\]
where the right hand side is a homogeneous space, which is a smooth manifold.
The bi-invariant metric on $O(n)$ induces a Riemannian metric on $Gr(k,n)$,
and $O(n)$ isometrically on $\text{Gr}(k,n)$.

For example, we may write 
$$
\text{Gr}(1, n+1) = \frac{O(n+1)}{O(1) \times O(n)} \\
=  \frac{1}{\{\pm1\}} \frac{O(n+1)}{O(n)}  \\
= \frac{1}{\{\pm1\}} \frac{SO(n+1)}{SO(n)} \\
= \frac{1}{\{\pm1\}} S^n.
$$
\end{example}

\begin{example}\label{hopf}
We have a diagram 
\[
\xymatrix{ S^3 \ar[r]^{\pi} \ar[rd]_{p}& S^2 \ar[d]^{j}\\
& P_1(\mathbb{C})
}
\]
where the diffeormophism $j^{-1}: P_1(\mathbb{C}) \to S^2$ is 
\[
[z_1, z_2] \mapsto \left(\frac{2z_1\bar{z_2}}{|z_1|^2 + |z_2|^2}, \frac{|z_2|^2 - |z_1|^2}{|z_1|^2 + |z_2|^2}\right)
\]
and 
$$
\pi : S^3=\{ (z_1,z_2)\in \bC^2: |z_1|^2+|z_2|^2=1\} \to S^2=\{
(w,z)\in \bC\times \bR: |w|^2+z^2=1\}
$$
is given by 
\[
(z_1, z_2) \mapsto (2z_1 \bar{z}_2, |z_1|^2 - |z_2|^2)
\]
Let $\hat{g}_\mathrm{can}$ be the  unique metric on $P_1(\mathbb{C})$ such that 
$p: (S^3,g_\mathrm{can})\to  (P_1(\mathbb{C}), \hat{g}_{can})$ is a Riemannian
submersion. We want to compute $\hat{g}=j^*\hat{g}_\mathrm{can}$.


 

Write 
\[
\begin{cases}
z_1 = \sin \lambda e^{i\theta_1} \\
z_2 = \cos\lambda e^{i\theta_2}
\end{cases}.
\]
These coordinates cover almost all of $S^3$ and because metrics are continuous, this is sufficient for our purposes. On $S^2$ we use spherical coordinates
\[
\begin{cases}
x = \sin \phi \cos\theta \\
y = \sin \phi \sin \theta \\
z = \cos \phi
\end{cases}.
\]
We already know that $g_{can}^{S^2(1)} = d\phi^2 + (\sin^2\phi )d\theta^2$. If we write $z_j = x_j + iy_j$, then we note that  
\[
\begin{cases}
x_1 = \sin\lambda \cos\theta_1 \\
y_1 = \sin\lambda \sin\theta_1 \\
x_2 = \cos\lambda \cos \theta_2 \\
y_2 = \cos\lambda \sin \theta_2
\end{cases}.
\]
We compute that 
\begin{align*}
g_{can}^{S^3(1)} = d\lambda^2 + \sin^2\lambda d\theta_1^2 + \cos^2\lambda d\theta_2^2.
\end{align*}
In these coordinates, we find that 
\[
(\sin\lambda e^{i\theta_1}, \cos\lambda e^{i\theta_2}) \mapsto (\sin(2\lambda)e^{i(\theta_1 - \theta_2)}, \cos^2\lambda - \sin^2\lambda).
\]
In other words, $\phi = 2 \lambda$ and $\theta = \theta_1 - \theta_2$. We find that 
$$
d\pi(\frac{\partial}{\partial \lambda}) = 2 \frac{\partial}{\partial \phi},\quad
d\pi(\frac{\partial}{\partial \theta_1}) = \frac{\partial}{\partial \theta},\quad
d\pi(\frac{\partial}{\partial \theta_2}) = - \frac{\partial}{\partial \theta}.
$$
We note that 
\[
\ker(d\pi) = \bR(\frac{\partial}{\partial \theta_1} + \frac{\partial}{\partial \theta_2}).
\]
We find that the horizontal subspace is  
\[
H= (\ker d\pi)^\perp = \bR \frac{\partial}{\partial \lambda} \oplus 
\bR(\cos^2\lambda \frac{\partial}{\partial \theta_1} - \sin^2\lambda \frac{\partial}{\partial \theta_2}).
\]
Let $\tilde{X}$ denote the horizontal lift of $X$. Then we find that 
$$
\widetilde{\frac{\partial}{\partial \phi}} = \frac{1}{2} \frac{\partial}{\partial \lambda},\quad
\widetilde{\frac{\partial}{\partial \theta}} = \cos^2\lambda \frac{\partial}{\partial \theta_1} - \sin^2\lambda \frac{\partial}{\partial \theta_2}.
$$
We know that 
\begin{eqnarray*}
\hat{g}(\frac{\partial}{\partial \phi}, \frac{\partial}{\partial \phi}) &=& g_\mathrm{can}^{S^3(1)} (\widetilde{\frac{\partial}{\partial \phi}}, \widetilde{\frac{\partial}{\partial \phi}}) 
= g^{S^3(1)}_\mathrm{can} (\frac{1}{2} \frac{\partial}{\partial \lambda}, \frac{1}{2} \frac{\partial}{\partial \lambda}) 
= \frac{1}{4},\\
\hat{g}(\frac{\partial}{\partial \phi}, \frac{\partial}{\partial \theta}) &=&
g_\mathrm{can}^{S^3(1)} (\widetilde{\frac{\partial}{\partial \phi}}, \widetilde{\frac{\partial}{\partial \theta}}) 
=0\\
\hat{g}(\frac{\partial}{\partial \theta}, \frac{\partial}{\partial \theta}) &=&
g_\mathrm{can}^{S^3(1)} (\widetilde{\frac{\partial}{\partial \theta}}, \widetilde{\frac{\partial}{\partial \theta}}) =\cos^4\lambda \sin^2\lambda +\sin^4\lambda \cos^2\lambda\\
&=&
\sin^2\lambda \cos^2\lambda = \frac{1}{4}\sin(2\lambda)^2 = \frac{1}{4} \sin^2 \phi.
\end{eqnarray*}
We see that 
\[
\hat{g} = \frac{1}{4}(d \phi^2 + \sin^2\phi d\theta^2) = \frac{1}{4}g_{can}^{S^2(1)}. 
\]
\end{example}




\section{Monday, November 9, 2015}

\noindent
{\bf \large Affine connections}

\begin{definition}[affine connection]
An {\em affine connection} $\nabla$ on a smooth manifold $M$ is a map 
$$
\nabla : \mathfrak{X}(M) \times \mathfrak{X}(M) \to \mathfrak{X}(M),\quad
(X,Y) \mapsto \nabla_XY
$$
such that for each $X,Y, Z \in \mathfrak{X}(M)$ and $f,g \in C^\infty(M)$, we have 
\begin{enumerate}
\item[(i)] $\nabla_{fX + gY}Z = f \nabla_XZ + g\nabla_YZ$. \\

\item[(ii)] $\nabla_X(Y + Z) = \nabla_XY + \nabla_XZ$
\item[(iii)] $\nabla_X(fY) = f\nabla_XY + X(f)Y$.\\

\end{enumerate}
\end{definition}

\begin{remark} 
\begin{itemize}
\item In the above definition:\\
i): for fixed $Y\in  \mathfrak{X}(M)$, the map $X\mapsto \nabla_X Y$  is $C^\infty(M)$-linear.\\
ii) and iii): for fixed $X\in \mathfrak{X}(M)$, the map
$\nabla_X: \fX(M)\to \fX(M)$ is $\bR$-linear, and satisfies the Leibniz rule.
\item The Lie derivative $L : \fX(M) \times \fX(M) \to \fX(M)$, $(X,Y)\to L_X Y=[X,Y]$, is
NOT an affine connection: it does not satisfy (i), although it satisfies (ii) and (iii). 
\end{itemize}
\end{remark}

\begin{remark}
If $\nabla_1$ and $\nabla_2$ are affine connections, then for $X \in \mathfrak{X}(M)$, the map 
\[
(\nabla_1)_X  -(\nabla_2)_X : \mathfrak{X}(M) \to \mathfrak{X}(M)
\]
is $C^\infty(M)$-linear and can be viewed as a section of $\text{End}(TM)$. That is, we may write 
\[
\nabla_1 - \nabla_2 \in C^\infty(M, T^*M \otimes T^*M \otimes TM)
\]
The space of affine connections is an affine space associated to the vector space $C^\infty(M, T_2^1M)$. 
\end{remark}

We now study connections in local coordinates. Let $(U,\phi)$ be a chart for $M$ and write $\phi = (x_1, \ldots, x_n)$. The list $\frac{\partial}{\partial x_1},
\ldots, \frac{\partial}{\partial x_n}$ form a smooth frame for $TM|_{U} = TU$. Then 
\[
\nabla_{\frac{\partial}{\partial x_i}}(\frac{\partial}{\partial x_j}) = \sum_{k} \Gamma_{ij}^k \frac{\partial}{\partial x_k}
\]
for some $\Gamma_{ij}^k \in C^\infty(U)$. 

If $X$ and $Y$ are smooth vector fields on $U$, we may write 
\[
X = \sum_{i} a_i \frac{\partial}{\partial x_i} \hspace{5mm} \text{and} \hspace{5mm} Y = \sum_{j} b_j \frac{\partial}{\partial x_j}
\]
where $a_i, b_j\in C^\infty(U)$.
We find that 
\[
\nabla_XY = \sum_{k=1}^n \left( \sum_{i=1}^n a_i \frac{\partial b_k}{\partial x_i} + \sum_{i,j=1}^n \Gamma_{ij}^k a_i b_j\right) \frac{\partial}{\partial x_k}.
\]


\begin{definition}[Vector field along a curve]
Let $M$ be a smooth manifold and $c : I \to M$ a smooth curve. A smooth vector field along $c$ is a smooth map $V :I \to TM$ such that $\pi \circ V = c$, that is, for each $t \in I$, we have $V(t) \in T_{c(t)}M$. 
\end{definition}

In local coordinates, if we restrict $c$ to $I'$ such that $c(I') \subset U$. Then  
\[
V(t) = \sum_{i=1}^n a_i(t) \frac{\partial}{\partial x_i}\Big|_{c(t)}
\]
for $a_i \in C^\infty(I')$. 

\begin{example}
The tangent vector field $\frac{dc}{dt}$ is a smooth vector field along $c$. 
\end{example}

\begin{proposition}\label{DV-inner-product}
Let $M$ be a smooth manifold with an affine connection $\nabla$. Then there is a unique correspondence taking a smooth curve $c:I\to M$ together with a smooth vector field $V:I\to TM$ along $c$ to a smooth vector field $\frac{DV}{dt} : I\to TM$  along $c$,  called the {\em covariant derivative of $V$ along $c$} such that 
\begin{enumerate}
\item[(i)] $\frac{D}{dt}(V + W) = \frac{DV}{dt} + \frac{DW}{dt}$ 
\item[(ii)] $\frac{D}{dt}(fV) = \frac{df}{dt}V + f \frac{DV}{dt}$ 
\item[(iii)] If $V = Y \circ c$ for some $Y \in \fX(M)$, then 
$$
\frac{DV}{dt}(t) = \nabla_{\frac{dc}{dt}(t)}Y.
$$ 
\end{enumerate}
\end{proposition}

In local coordinates, consider the case $c :I \to U$, where $(U,\phi)$ is a local coordinate chart.
Then $\phi\circ c: I\to \phi(U)\subset \bR^n$ is given by 
$\phi\circ c(t)= (x_1(t),\ldots, x_n(t))$, where $x_i\in C^\infty(I)$.
 On $U$, we may write 
\[
\nabla_{\frac{\partial}{\partial x_i}}\frac{\partial}{\partial x_j} = \sum_{k=1}^n \Gamma_{ij}^k \frac{\partial}{\partial x_k}.
\]
And we may write 
\[
V(t) = \sum_{i=1}^n a_i(t) \frac{\partial}{\partial x_i}\Big|_{c(t)}, \quad
\frac{dc}{dt}(t)=\sum_{i=1}^n \frac{dx_i}{dt}(t) \frac{\partial}{\partial x_i}\Big|_{c(t)}
\] 
Then 
\begin{align*}
\frac{DV}{dt} &= \frac{D}{dt} \left(\sum_{i=1}^n a_i(t) \frac{\partial}{\partial x_i}\Big|_{c(t)}\right) \\
&= \sum_{i=1}^n \frac{D}{dt}(a_i(t) \frac{\partial}{\partial x_i}\Big|_{c(t)}) \\
&= \sum_{i=1}^n \frac{d a_i}{\partial t}(t) \frac{\partial}{\partial x_i}\Big|_{(c(t))} + 
a_i \frac{D}{dt}(\frac{\partial}{\partial x_i}\Big|_{c(t)}) 
\end{align*}
where 
\[
\frac{D}{dt}(\frac{\partial}{\partial x_i}|_{c(t)})  = 
\nabla_{\frac{dc}{dt}(t)} \frac{\partial}{\partial x_i} =
\sum_{j=1}^n \frac{dx_j}{dt}(t)
\nabla_{\frac{\partial}{\partial x_i}|_{c(t)}}\frac{\partial}{\partial x_i}
=\sum_{j=1}^n \frac{dx_j}{dt}(t)\sum_{k=1}^n \Gamma_{ji}^k(c(t)) \frac{\partial}{\partial x_k}\Big|_{c(t)}
\]

Then we conclude that 
\[
\frac{DV}{dt} = \sum_{k=1}^n \left(\frac{da_k}{dt} + \sum_{i,j=1}^n \Gamma_{ij}^k \frac{dx_i}{dt}a_j\right)\frac{\partial}{\partial x_k}.
\]


\bigskip

\noindent
{\bf \large Parallel transport}


\begin{definition}
Let $M$ be a smooth manifold with an affine connection $\nabla$. A smooth vector field $V$ along
smooth curve $c: I\to M$ is {\em parallel} if $\frac{DV}{dt}(t)=0$ for all $t\in I$.
\end{definition}

\begin{proposition}
Let $M$ be a smooth manifold with an affine connection $\nabla$. Let $c : I \to M$ be a smooth curve and let $t_0 \in I$. For each tangent vector $V_0 \in T_{c(t_0)} M$ there is a unique parallel vector field
$V(t)$ along $c(t)$ with $V(t_0) = V_0$. 
The vector field $V(t)$ is called the {\em parallel transport} of $V_0$ along $c$. 
\end{proposition}

\begin{proof}
We may assume that $c(I) \subset U$ where $U$ is a coordinate chart. We may write 
\[
V_0 = \sum_{i} a_i \frac{\partial}{\partial x_i}|_{c(t_0)}
\]
for some $a_i\in \bR$. We want to solve $\frac{DV}{dt} = 0$ and $V(t_0) = V_0$. In terms of local coordinates, this means that, for $k=1,\ldots,n$,
\[
\begin{cases}
\displaystyle{\frac{da_k}{dt} + \sum_{i,j=1}^n \Gamma_{ij}^k \frac{dx_i}{dt}a_j = 0} \\
a_k(t_0) = a_k
\end{cases}
\]
If we write 
$$
\vec{a}(t) = \left[\begin{array}{c} a_1(t)\\ \vdots \\ a_n(t)\end{array}\right],
\quad \vec{a}=\left[\begin{array}{c} a_1\\ \vdots \\a_n\end{array}\right].
$$
and let $A(t)=(A_{kj}(t))$, where
$$
A_{kj}(t) = -\sum_{i=1}^n \Gamma_{ij}^k (x_1(t),\ldots, x_n(t)) 
\frac{d x_i}{d t}(t)
$$
Then these conditions are equivalent to 
\[
\begin{cases}
\frac{d}{dt}\vec{a}(t) = A(t)\vec{a}(t) \\
\vec{a}(t_0) = \vec{a}
\end{cases}.
\]
So the proposition follows from the existence and uniqueness of solutions to first order ODE's. 
\end{proof}

\begin{example}
On $\bR^n$ we can take the trivial connection $\nabla_{\frac{\partial}{\partial x_i}} \frac{\partial}{\partial x_j} = 0$. Then the parallel vector fields are just constant along curves.
\end{example}


\newpage

\noindent 
{\bf \Large Riemannian Connection}

\begin{definition}
An affine connection $\nabla$ on a smooth manifold $M$ is said to be {\em symmetric} if for any smooth vector fields $X,Y \in \mathfrak{X}(M)$, we have 
\[
\nabla_XY - \nabla_Y X = [X,Y].
\]
In terms of local coordinates, this places the requirement that $\Gamma_{ij}^k = \Gamma_{ji}^k$. 
\end{definition}



\begin{definition}
Let $(M,g)$ be a Riemannian manifold with affine connection $\nabla$. We say that $\nabla$ is 
{\em compatible with the metric $g$} if for each $X,Y,Z \in \fX(M)$, we have 
\[
X(g(Y,Z)) = g(\nabla_XY, Z) + g(Y, \nabla_XZ)
\]
\end{definition}

\begin{theorem}[Levi-Civita]
Given a Riemannian manifold $(M,g)$, there is a unique affine connection $\nabla$ on $M$ such that 
\begin{enumerate}
\item[(i)] $\nabla$ is symmetric and 
\item[(ii)] $\nabla$ is compatible with $g$. 
\end{enumerate}
This connection is known as the {\em Riemannian connection} or the {Levi-Civita connection}
on the Riemannian manifolds $(M,g)$. 
\end{theorem}

\begin{proof}
For uniqueness, suppose that $\nabla$ is an affine connection satisfying (i) and (ii). Then
for any $X,Y,Z\in \fX(M)$,
\begin{align*}
&\;\;\;\;\:X(g(Y,Z)) + Y(g(Z,X)) - Z(g(X,Y)) \\
&= g(\nabla_X Y + \nabla_YX, Z) + g([X,Z], Y) + g([Y,Z], X) \\
&= g([X,Y] + 2 \nabla_YX, Z) + g([X,Z], Y) + g([Y,Z], X).
\end{align*}
It follows that 
\begin{equation}\label{eqn:Levi-Civita}
\begin{aligned}
 g(\nabla_YX, Z) &= \frac{1}{2}\left(X(g(Y,Z)) + Y(g(Z,X))\right) -Z(g(X,Y)) \\
 &\;\; - g([X,Z], Y) - g([Y,Z], X) - g([X,Y],Z).
\end{aligned}
\end{equation}
Since $Z$ is arbitrary, Equation \eqref{eqn:Levi-Civita} uniquely determines $\nabla_YX$. 

For existence, one defines $\nabla_YX$ by \eqref{eqn:Levi-Civita} and shows that this is an affine connection satisfying (i) and (ii). 
\end{proof}

In terms of local coordinates: in \eqref{eqn:Levi-Civita}, let
$$
X=\frac{\partial}{\partial x_j},\quad Y=\frac{\partial}{\partial x_i},\quad
Z=\frac{\partial}{\partial x_k}.
$$
We obtain
\[
g(\nabla_{\frac{\partial}{\partial x_i}} \frac{\partial}{\partial x_j}, \frac{\partial}{\partial x_k}) = \frac{1}{2} \left( \frac{\partial}{\partial x_j} g_{ik} + \frac{\partial}{\partial x_i} g_{kj} - \frac{\partial}{\partial x_k} g_{ij}\right),
\]
where 
$\displaystyle{ \nabla_{\frac{\partial}{\partial x_i}} \frac{\partial}{\partial x_j}
=\sum_{l=1}^n \Gamma_{ij}^l 
\frac{\partial}{\partial x_l} }$, so 
\[
\sum_{l=1}^n \Gamma^l_{ij} g_{lk} = \frac{1}{2} \left( \frac{\partial}{\partial x_j} g_{ik} + \frac{\partial}{\partial x_i} g_{kj} - \frac{\partial}{\partial x_k} g_{ij}\right) 
\]
and hence 
\[
\Gamma_{ij}^l = \frac{1}{2} \sum_{k=1}^n g^{lk}( \frac{\partial}{\partial x_j} g_{ik} + \frac{\partial}{\partial x_i} g_{kj} - \frac{\partial}{\partial x_k} g_{ij})
\]
where $g^{lk}$ is the $l,k$ entry of the inverse of $g$. 







\section{Wednesday, November 11, 2015}

Recall that the Levi-Civita connection on a Riemannian manifold $(M,g)$ is
the unique affine connection which is symmetric and compatible with the Riemannian
metric $g$.

\begin{definition}
Let $\nabla$ be an affine connection on a smooth manifold $M$. The 
{\em torsion} of $\nabla$ is defined to be 
\begin{align*}
T_\nabla : \fX(M) \times \fX(M) &\to \fX(M) \\
(X,Y) &\mapsto  \nabla_XY - \nabla_YX - [X,Y].
\end{align*}
\end{definition}
It is straighforward to check that:
\begin {lemma}
\begin{enumerate}
\item[(i)] $T_\nabla$ is antisymmetric: $T_\nabla(X,Y)=-T_\nabla(Y,X)$.
\item[(ii)] $T_\nabla$ is $C^\infty(M)$-bilinear. 
\end{enumerate}
So $T_\nabla \in C^\infty(M, \Lambda^2 T^*M \otimes TM)$ is a (1,2)-tensor on $M$.
\end{lemma}
By definition, an affine connection $\nabla$ is symmetric if and only of $T_\nabla=0$. So 
the ``symmetric" condition is also known as the ``torsion free" condition.

\begin{proposition}\label{compatible-g}
Let $(M,g)$ be a Riemannian manifold, and let $\nabla$ be an affine connection on $M$ 
compatible with the Riemannian metric $g$. If $V,W$ are smooth vector fields along
a smooth curve $c:I\to M$ then
$$
\frac{d}{dt}\langle V, W\rangle =\langle\frac{DV}{dt}, W\rangle + \langle V,\frac{DW}{dt}\rangle,
$$
where $\langle\ , \ \rangle$ is the inner product defined by $g$, and 
$\frac{D}{dt}$ is the covariant derivative along $c$ determined by $\nabla$.
In particular, if $V, W$ are parallel vector fields along $c$ then 
$\langle V, W\rangle$ is a constant function on $I$.
\end{proposition}

We will see later that Proposition \ref{compatible-g} is a special case of a
more general result.

\begin{example}
Let $M = \mathbb{R}^n$ and let $g_0 = dx_1^2 + \cdots + dx_n^2$. Since all the $g_{ij}$'s are constant, we find that $\Gamma_{ij}^k = 0$. This means that $\nabla_\frac{\partial}{\partial x_i} \frac{\partial}{\partial x_j} = 0$. This implies that if $X = \sum_{i} a_i \frac{\partial}{\partial x_i}$ and $Y = \sum_{j} b_j \frac{\partial}{\partial x_j}$, then we see that 
\[
\nabla_XY = \sum_{i,j} \left(a _i  \frac{\partial b_j}{\partial x_i}\right) \frac{\partial}{\partial x_j}.
\]
Recall that 
\[
L_XY = \sum_{i,j} \left( a_i \frac{\partial b_j}{\partial x_i} - b_i \frac{\partial a_j}{\partial x_i}\right) \frac{\partial}{\partial x_j} = \nabla_XY - \nabla_YX.
\]
This shows that $\nabla$ is indeed torsion free. 
\end{example}

\begin{example}\label{two-sphere-nabla}
Let $S^2$ be equipped with the round metric. 
Use spherical coordinates 
\[
\begin{cases}
x = \sin \phi \cos \theta \\
y = \sin \phi \sin \theta \\
z = \cos \phi
\end{cases}.
\]
In these coordinates, we know that $g_{can} = d\phi^2 + \sin^2 \phi d \theta^2$. Write $(x_1,x_2) = (\phi, \theta)$. In terms of these coordinates, we have 
\[
g = \begin{bmatrix}
1 & 0 \\ 0 & \sin^2 \phi
\end{bmatrix}
\hspace{5mm} \text{and} \hspace{5mm}
g^{-1} = \begin{bmatrix}
1 & 0 \\ 0 & \frac{1}{\sin^2\phi}
\end{bmatrix}.
\]
Let $g_{ij,k}=\frac{\partial }{\partial x_k}g_{ij}$. We
compute the Christoffel symbols of the Levi-Civita connection to be 
\begin{align*}
\Gamma_{11}^1 &= 0 \\
\Gamma_{11}^2 &= 0 \\
\Gamma_{12}^1 &= \Gamma_{21}^1 = 0 \\
\Gamma_{12}^2 &= \Gamma_{21}^2 = \frac{1}{2} g^{22}(g_{22,1} + g_{12,2} - g_{12,2}) = \frac{1}{2} \frac{1}{\sin^2\phi} 2 \sin\phi \cos \phi = \cot \phi \\
\Gamma_{22}^1 &= \frac{1}{2} g^{11}(2g_{21,2}-g_{22,1})= - \sin\phi \cos\phi \\
\Gamma_{22}^2 &= 0.
\end{align*}
\begin{eqnarray*}
&& \nabla_{\frac{\partial}{\partial \phi}} \frac{\partial}{\partial \phi}
=\Gamma_{11}^1 \frac{\partial}{\partial \phi}+\Gamma_{11}^2\frac{\partial}{\partial \theta}=0 \\
&& \nabla_{\frac{\partial}{\partial \phi}} \frac{\partial}{\partial \theta}
=\nabla_{\frac{\partial}{\partial \theta}} \frac{\partial}{\partial \phi}
=\Gamma_{12}^1 \frac{\partial}{\partial \phi}+\Gamma_{12}^2\frac{\partial}{\partial \theta}
=\cot \phi \frac{\partial}{\partial \theta}\\
&&\nabla_{\frac{\partial}{\partial \theta}} \frac{\partial}{\partial \theta}
=\Gamma_{22}^1 \frac{\partial}{\partial \phi}+\Gamma_{22}^2\frac{\partial}{\partial \theta}=
-\sin\phi\cos\phi \frac{\partial}{\partial \phi}.
\end{eqnarray*}


\noindent
{\bf Parallel transport along a meridian $\theta=\theta_0$.}

The vector field $\frac{\partial}{\partial \phi}$ is parallel along  $\theta=\theta_0$
since $\nabla_{\frac{\partial}{\partial \phi} }\frac{\partial}{\partial \phi} =0$. 
From Proposition \ref{compatible-g}, the vector field 
$\frac{1}{\sin \phi}\frac{\partial}{\partial \theta}$ 
is also parallel along $\theta=\theta_0$ since it is perpendicular to $\frac{\partial}{\partial \phi}$
and of constant length $1$. We now verify this directly:
$$
\nabla_{\frac{\partial}{\partial \phi}} (\frac{1}{\sin \phi} \frac{\partial}{\partial\theta})
=\frac{-\cos\phi}{\sin^2\phi} \frac{\partial}{\partial \theta}
+\frac{1}{\sin\phi}\cdot \cot \phi\frac{\partial}{\partial \theta}=0.
$$
Any parallel vector field along a meridian $\theta=\theta_0$ is of the form
$$
a \frac{\partial}{\partial \phi} + b\cdot \frac{1}{\sin\phi}\frac{\partial}{\partial \theta}
$$
where $a,b\in \bR$ are constants.

\noindent
{\bf Parallel transport along a parallel $\phi=\phi_0$.}
Write $(x_1(\theta), x_2(\theta))= (\phi_0,\theta)$. A vector field 
$V(\theta)= a_1(\theta)\frac{\partial}{\partial\phi}+a_2(\theta)\frac{\partial}{\partial \theta}$
along $\phi=\phi_0$ is parallel if and only if
$$
\begin{cases}
\displaystyle{\frac{da_1}{d\theta} + \Gamma^1_{22}a_2 =0}&\\ 
\displaystyle{\frac{da_2}{d\theta}+ \Gamma_{21}^2 a_1 =0} &
\end{cases}
$$
where $\Gamma^1_{22}=-\sin\phi_0 \cos\phi_0$, $\Gamma_{21}^2 = \cot\phi_0$. 
The above two equations can be rewritten as
$$
\frac{d}{d\theta}\left[ \begin{array}{c} a_1(\theta)\\ \sin\phi_0 a_2(\theta)\end{array}\right]
=\left[ \begin{array}{cc} 0 & \cos\phi_0\\ -\cos\phi_0 &0 \end{array}\right]
 \left[ \begin{array}{c} a_1(\theta)\\ \sin\phi_0 a_2(\theta)\end{array}\right]
$$
The solution is 
\[
\begin{bmatrix}
a_1(\theta) \\ \sin\phi_0 a_2(\theta)
\end{bmatrix} = \begin{bmatrix}
\cos((\cos \phi_0)\theta) & \sin((\cos \phi_0)\theta) \\
- \sin((\cos \phi_0)\theta) & \cos((\cos \phi_0)\theta)
\end{bmatrix}\begin{bmatrix}
a_1(0) \\ \sin\phi_0a_2(0)
\end{bmatrix}
\]
Let $a_1(0)=1$ and $a_2(0)=0$, we see that the parallel transport of the unit vector $\frac{\partial}{\partial \phi}$ 
along $\phi=\phi_0$ is 
 \[
\cos((\cos\phi_0) \theta) \frac{\partial}{\partial \phi} 
- \frac{\sin((\cos \phi_0) \theta)}{\sin(\phi_0)}\frac{\partial}{\partial \theta}.
\]
Let $a_1(0)=0$ and  $a_2(0)=\frac{1}{\sin \phi_0}$, we see that  the
parallel transport of the unit vector $\frac{1}{\sin \phi_0} \frac{\partial}{\partial \theta}$ along $\phi = \phi_0$ is
\[
\sin((\cos \theta_0)\theta) \frac{\partial}{\partial \phi} + \frac{\cos((\cos \phi_0) \theta)}{\sin \phi_0} \frac{\partial}{\partial \theta}
\]


Another way to see it is to consider a cone $C$ tangent to $S^2$ along the circle $\phi=\phi_0$. 
Then for any $p$ on the circle $\phi=\phi_0$, $T_p C = T_p S^2$. 
By Assignment 8 (4), the parallel tranport along $\phi=\phi_0$ defined
by the Levi-Civita connection on $C$ and the Levi-Civita connection on $S^2$ are the same.
See page 79 of \cite{GHL} for details. 
\end{example}


\bigskip

\noindent
{\large \bf Geodesics}

\begin{definition}
Let $M$ be a Riemannian manifold and let $\gamma : I \to M$ be a smooth curve. Then we say that $\gamma$ is {\em geodesic} at $t_0 \in I$ if $\frac{D}{dt}(\frac{d \gamma}{dt})(t_0) = 0$, where we are using the Levi-Civita connection $\nabla$. We say that $\gamma$ is a {\em geodesic} if it is geodesic at each point of its domain.  
\end{definition}



By Proposition \ref{compatible-g}, if $\gamma$ is a geodesic, then $| \frac{d \gamma}{dt}|$ is constant. Assume that $| \frac{d \gamma}{dt}| = c > 0$. We may parametrize by arc length to get $| \frac{d \gamma}{dt}| = 1$. In terms of local coordinates $\phi \circ \gamma(t) = (x_1(t) , \ldots, x_n(t))$, we get the equation 
\[
\frac{d^2 x_k}{d t^2} + \sum_{i,j} \Gamma_{ij}^k \frac{d x_i}{dt} \frac{d x_j}{d t} = 0.
\]

\begin{example}[Euclidean space]
$M=\bR^n$ equipped with the Euclidean metric $g_0=dx_1^2+\cdots + dx_n^2$.
Then  $\Gamma_{ij}^k= 0$. geodesic $\gamma : I \to \mathbb{R}^2$ satisfies $\frac{d^2x_k}{dt^2} = 0$ and hence $x_k(t) = a_k + b_k t$ for $a_k,b_k \in \mathbb{R}$. It follows that $\gamma$ is affine linear in each coordinate. We conclude the following: for each $\vec{a} \in \bR^n$ and 
$\vec{b} \in T_{\vec{a}}\bR^n$, the line $\gamma(t) = \vec{a} + \vec{b}t$ is the 
unique geodesic such that $\gamma(0) = \vec{a}$ and $\gamma'(0) = \vec{b}$. 
\end{example}

\begin{example}[round sphere]
Geodesics in a round sphere are great circles. See Assignment 9 (2).
\end{example}



\section{Monday, November 16, 2015}

\begin{proposition}\label{geodesic-exist}
Let $(M,g)$ be a Riemannian manifold. Let $p$ be a point of $M$ and $v \in T_pM$. Then 
\begin{itemize}
\item (Existence) There is an open interval $I = (a,b)$, where $-\infty\leq a<0<b\leq+\infty$, and a geodesic 
$\gamma :I \to M$, such that $\gamma(0) = p$ and $\gamma'(0) = v$. 
\item (Uniqueness) If $\beta : I' \to M$ is another geodesic satisfying $\beta(0) = p$ and $\beta'(0) =v$ then $I' \subset I$ and 
$\beta = \gamma|_{I'}$. 
\end{itemize}
\end{proposition}

There is a reformulation using the notion of a geodesic field. 


\medskip

\noindent
{\bf \large Geodesic field and geodesic flow}

\begin{definition}\label{tgamma}
Given a smooth curve $\gamma:I\to M$, we define $\tilde{\gamma}:I\to TM$ by 
$\tilde{\gamma}(t)=(\gamma(t),\gamma'(t))$. Then $\tilde{\gamma}$ is a smooth curve in $TM$.
\end{definition} 

Any smooth curve $w:I\to TM$ is of the form $w(t)=(c(t),V(t))$, where $c:I\to M$ is 
a smooth curve in $M$ and $V(t)$ is a smooth vector field along $c(t)$; $w$ is
equal to $\tilde{\gamma}$ for some {\em geodesic} $\gamma:I\to M$ if and only if
\begin{equation}\label{eqn:cV}
c'(t)= V(t),\quad \frac{DV}{dt}(t)=0. 
\end{equation}

Suppose that $c(I)$ is contained in a coordinate neighborhood $U\subset M$. Then $w(I)$ is contained in $TU\subset TM$.
$\phi: U\to \phi(U)\subset \bR^n$ and $\tilde{\phi}: TU\to \phi(U)\times \bR^n \subset \bR^n\times \bR^n$, 
\begin{eqnarray*}
\phi\circ c(t) &=& (x_1(t),\ldots, x_n(t)),\\
V(t) &=& \sum_{i=1}^n y_i(t)\frac{\partial}{\partial x_i}|_{c(t)},\\
\tilde{\phi}\circ w(t) &=& (x_1(t),\ldots, x_n(t),y_1(t),\ldots, y_n(t)).
\end{eqnarray*}
Then \eqref{eqn:cV} is equivalent to the following system of $2n$ 1st order ODE's. 
\begin{equation} \label{eqn:cVxy}
\frac{dx_k}{dt}(t) = y_k(t),\quad \frac{dy_k}{dt} = - \sum_{i,j} \Gamma_{ij}^k(x) y_iy_j,\quad k=1,\ldots,n.
\end{equation}
These are equations for the integral curve of the following smooth vector field on $TU$: 
\[
G = \sum_{k} y_k \frac{\partial}{\partial x_k} - \sum_{i,j,k} \Gamma_{ij}^k(x_1,\ldots,x_n) y_iy_j \frac{\partial}{\partial y_k}.
\]
$G$ is independent of choice of coordinates. We obtain a smooth vector field $G$ on $TM$, known as the {\em geodesic field}.
Proposition \ref{geodesic-exist} follows from the existence and uniqueness of integral curves of $G\in \fX(TM)$.  

Given $(p,v)\in TM$, where $p\in M$ and $v\in T_pM$, let $\gamma:I\to M$ be the unique geodesic with
$\gamma(0)=p$ and $\gamma'(0)=v$ in Proposition \ref{geodesic-exist}, and define
$\tilde{\gamma}:I\to TM$ as in Definition \ref{tgamma}.
Then $\tilde{\gamma}(0)= (p,v)$ and $\tilde{\gamma}'(0)=G(p,v)\in T_{(p,v)}(TM)$.

Applying the existence/uniqueness theorem for flows of vector fields on $TM$, we find the following: 
for each $(p,v) \in TM$, where $p\in M$ and $v\in T_pM$, there is an open neighborhood $U$ of $(p,v)$ in $TM$, a positive number $\delta > 0$, 
and a smooth map 
\[
\phi : (-\delta, \delta) \times U \to TM
\]
such that 
\[
\begin{cases}
\frac{\partial \phi}{\partial t}(t,q,w) = G(\phi(t,q,w))\\
\phi(0,q,w) = (q,w)
\end{cases}
\]
Let $\gamma = \pi \circ \phi : (-\delta, \delta) \times U \to M$. Then for a fixed $(q,w) \in U \subset TM$, we find that 
\[
\gamma_{q,w}(t) := \gamma(t,q,w) = \pi(\phi(t,q,w))
\]
is a geodesic such that $\gamma_{q,w}(0) = q$ and $\frac{d \gamma_{q,w}}{dt}(0) = w$. 
For $t \in (-\delta, \delta)$, we get $\phi_t : U \to TM$, the flow of $G$, called the {\em geodesic flow}. 

\begin{example}
When $(M,g)=(\bR, dx^2)$, we can identify $T\bR$ with $\bR^2$ via the map 
$(x, y \frac{\partial}{\partial x}) \mapsto (x,y)$. Then we see that 
\[
G = y \frac{\partial}{\partial x}.
\]
The flow $\phi_t : T\bR \to T\bR$ is given by 
\[
\phi_t(x,y) = (x + ty, y)
\]
where $t\in \bR$.
\end{example}

\begin{example}
More generally, when  $(M,g)= (\bR^n, g_0)$, we find that 
\[
G = \sum_{i} y_i \frac{\partial}{\partial x_i}
\]
and $\phi_t : \bR^{2n} \to \bR^{2n}$ is given by 
\[
\phi_t(x,y) = (x + ty, y), 
\]
where $x,y\in \bR^n$.
\end{example}

\medskip

\noindent
{\bf \large Connections on vector bundles}



\begin{definition} \label{connection-E}
Let $M$ be a smooth manifold and let $\pi : E \to M$ be a smooth vector bundle of rank $r$. A \textbf{connection on $E$} is an
$\bR$-bilinear map $\nabla : \mathfrak{X}(M) \times C^{\infty}(M,E) \to C^{\infty}(M, E)$ written $(X,s) \mapsto \nabla_{X}s$ such that 
for any $X\in \fX(M)$, $s\in C^\infty(M,E)$, and $f\in C^\infty(M)$,
\begin{enumerate}
\item[(i)] $\nabla_{fX}s = f\nabla_X s$, i.e., $\nabla$ is $\mathbb{C}^{\infty}(M)$-linear in the first factor;
\item[(ii)] $\nabla_X(fs)=X(f) s + f\nabla_X s$, i.e., for fixed $X\in \fX(M)$, the map
$\nabla_X:C^\infty(M,E)\to C^\infty(M,E)$ sending $s$ to  $\nabla_X s$  satisfies the Leibniz rule.
\end{enumerate}
\end{definition} 

\begin{example}
An affine connection on $M$ is the same as a connection on $TM$. 
\end{example}

We introduce the following notation. We denote by $\Omega^p(M,E)$ the space of $E$-valued $p$-forms on $E$, that is, 
\[
\Omega^p(M,E) = C^\infty(M, \Lambda^p T^*M \otimes E).
\]
With this notation, Definition \ref{connection-E} can be reformulated as follows.
\begin{definition}
A connection on $E$ is an $\bR$-linear map $\nabla : \Omega^0(M,E) \to \Omega^1(M,E)$ 
written $s \mapsto \nabla s$ such that for each $f \in C^{\infty}(M)$ and each $s \in \Omega^0(M,E)$, we have 
\[
\nabla(fs) = df \otimes s + f \nabla s.
\]
\end{definition}

\begin{lemma}
If $\nabla_1$ and $\nabla_2$ are connections on $E$, then $\nabla_1 - \nabla_2 :\Omega^0(M, E) \to \Omega^{1}(M,E)$ is 
$C^{\infty}(M)$-linear. 
\end{lemma}

\begin{proof}
For $f : M \to \mathbb{R}$ a smooth function and $s : M \to E$ a smooth section, we have 
\begin{align*}
(\nabla_1 - \nabla_2)(fs) &= \nabla_1(fs) - \nabla_2(fs) \\
&= df \otimes s + f \nabla_1 s - df \otimes s - f \nabla_2s \\
&= f (\nabla_1 - \nabla_2)s.
\end{align*}
\end{proof}

It follows that $\phi := \nabla_1 - \nabla_2$ can be viewed as an element of $\Omega^1(M, \text{End} E)$. 
The space of connections on $E$ is an affine space whose associated vector space is $\Omega^1(M, \text{End}E)$. 

In general if $E,F$ are smooth vector bundles and $\phi : C^\infty(M, E) \to C^{\infty}(M,F)$ is a $C^\infty(M)$-linear map, then we can view $\phi$ as an element of $C^\infty(M, E^* \otimes F)$:
\[
\phi(s)(p) = \phi(p)s(p) \in F_p.
\]

Now we want to express our connection in terms of local coordinates. Let $(U,\phi)$ be a chart for $M$ and write $\phi = (x_1, \ldots, x_n)$. We get a smooth frame $\{\frac{\partial}{\partial x_i}\}$ for the tangent bundle $TM|_U$. We may suppose that we have a trivialization $h : E|_U \to U \times \mathbb{R}^r$. We get a smooth frame $e_1, \ldots, e_r$ for $E|_U$. On $U$, we have
\[
\nabla_{\frac{\partial}{\partial x_i}} e_j = \sum_{k=1}^r \Gamma_{ij}^k e_k
\]
for some $\Gamma_{ij}^k\in C^\infty(U)$. The element $\nabla e_j$ is an $E$-valued one-form on $U$ and we note that 
$$
\nabla e_j =  \sum_{i=1}^n \sum_{k=1}^r \Gamma_{ij}^k dx_i \otimes e_k = \sum_{k=1}^r \omega_j^k e_k
$$
where $\omega_j^k = \sum_{i=1}^n \Gamma_{ij}^k dx_i$ are smooth 1-forms on $U$. To define the connection one-forms 
$\omega_j^k \in \Omega^1(U)$ we only 
need a trivialization of $E|_U$ but not $TM|_U$ 
$$
\nabla e_j =\sum_{k=1}^j \omega_j^k e_k
$$
where $\omega_j^k\in \Omega^1(U)$. 

Let $\{U_\alpha : \alpha \in I\}$ be an open cover of $M$ such that $h_\alpha : \pi^{-1}(U_\alpha) \to U_\alpha \times \mathbb{R}^r$ are local trivializations. Let $\{e_{1,\alpha}, \ldots, e_{r,\alpha}\}$ be a $C^\infty$-frame of $E|_{U_\alpha}$, so that
$h_\alpha^{-1}$ is given by $h_\alpha^{-1}(x, (v_1,\ldots, v_r)) = (x,\sum_{i=1}^r v_i e_{i,\alpha}(x))$, where
$x\in U_\alpha$ and $(v_1,\ldots, v_r)\in \bR^r$. On $U_\alpha$, define $(\omega_\alpha)_j^k\in \Omega^1(U_\alpha)$ by 
\[
\nabla e_{j,\alpha} = \sum_{k=1}^r (\omega_\alpha)_j^k \otimes e_{k,\alpha}. 
\]

For a global smooth section $s \in C^\infty(M,E)$, we can expand $s$ on $U_\alpha$ as 
\[
s = \sum_{j=1}^r s_\alpha^j e_{j,\alpha}
\]
for some $s_\alpha^j$ in $C^\infty(U_\alpha)$. By Leibniz rule,
\[
\nabla s = \sum_{j=1}^r ds_\alpha^j e_{j,\alpha} + \sum_{j=1}^r s_\alpha^j \nabla e_{j,\alpha} = \sum_{j=1}^r ds_\alpha^j e_{j,\alpha} + \sum_{j,k=1}^r s_{\alpha}^j (\omega_\alpha)_j^k e_{k,\alpha}.
\]
On $U_\alpha$, define $(\nabla s)^j_\alpha \in \Omega^1(U_\alpha)$ by 
\[
\nabla s = \sum_{j=1}^r (\nabla s)_\alpha^j e_{j,\alpha}.
\]
We see that 
\[
(\nabla s)_\alpha^j = ds_\alpha^j + \sum_{k=1}^r (\omega_\alpha)_k^j s_\alpha^k,
\]
or equivalently,
\begin{equation}\label{eqn:nabla-d}
\begin{bmatrix}
(\nabla s)_\alpha^1 \\ \vdots \\ (\nabla s)_\alpha^r
\end{bmatrix} = \begin{bmatrix}
ds_\alpha^1 \\ \vdots \\ ds_\alpha^r
\end{bmatrix} +  \begin{bmatrix}
(\omega_\alpha)_1^1 & \cdots & (\omega_\alpha)_r^1 \\
\vdots & \ddots & \vdots  \\
(\omega_\alpha)_1^r & \cdots & (\omega_\alpha)_r^r
\end{bmatrix}
\begin{bmatrix}
s_\alpha^1 \\ \vdots \\ s_\alpha^r
\end{bmatrix}.
\end{equation}
We define
\begin{equation}\label{eqn:s-nablas}
s_\alpha := \begin{bmatrix} s_\alpha^1 \\ \vdots \\ s_\alpha^r \end{bmatrix}\in C^\infty(U_\alpha,\bR^r),\quad
(\nabla s)_\alpha := \begin{bmatrix} (\nabla s)_\alpha^1 \\ \vdots \\ (\nabla s)_\alpha^r \end{bmatrix}\in \Omega^1(U_\alpha,\bR^r),
\end{equation}
and define a matrix-valued 1-form
\begin{equation}\label{eqn:omega-matrix} 
\omega_\alpha:=  \begin{bmatrix}
(\omega_\alpha)_1^1 & \cdots & (\omega_\alpha)_r^1 \\
\vdots & \ddots & \vdots  \\
(\omega_\alpha)_1^r & \cdots & (\omega_\alpha)_r^r
\end{bmatrix} \in \Omega^1(U_\alpha,\fgl(r,\bR)).
\end{equation}
Then \eqref{eqn:nabla-d} can be written as
\[
(\nabla s)_\alpha = d s_\alpha + \omega_\alpha s_\alpha
\]
where $(\nabla s)_\alpha$ and $ds_\alpha$ are column vectors with components that are 1-forms,
$\omega_\alpha$ is a matrix with entries that are 1-forms, and $s_\alpha$ is a column vector
with components that are smooth functions.  



\section{Wednesday, November 18, 2015}

Let $\pi:E\to M$ be a smooth vector bundle of rank $r$ over a smooth manifold $M$.
Suppose that $\{ U_\alpha: \alpha\in I\}$ is an open cover of $M$ and 
$h_\alpha: \pi^{-1}(U_\alpha)\to U_\alpha\times \bR^r$ are local trivializations.
The local trivialization $h_\alpha$ gives a smooth frame $\{e_{i,\alpha}:i=1,\ldots,r\}$ for
$E|_{U_\alpha}$ such that $h_\alpha^{-1}(x,\vec{v})= (x, \sum_{i=1}^r v_i e_{i,\alpha}(x))$.
When $U_\alpha\cap U_\beta$ is nonempty, we also have transition functions
$$
h_{\alpha\beta} = h_\alpha\circ h_\beta^{-1}: (U_\alpha\cap U_\beta)\times \bR^r 
\to (U_\alpha\cap U_\beta)\times \bR^r,\quad (x,v)\mapsto (x,t_{\alpha\beta}(x)v)
$$
where $t_{\alpha\beta}$ is a smooth map from $U_\alpha\cap U_\beta$ to $GL(r,\bR)$. Then
$t_{\alpha\alpha}(x)= I_r$ for all $x\in U_\alpha$, where $I_r$ is the $r\times r$ identity matrix,
and $t_{\alpha\beta}(x)t_{\beta\gamma}(x)t_{\gamma\alpha}(x)=I_r$ for all $x\in U_\alpha\cap U_\beta\cap U_\gamma$. 

Conversely, given an open cover $\{U_\alpha:\alpha\in I\}$ of $M$ and 
smooth maps $t_{\alpha\beta}:U_\alpha\cap U_\beta \to GL(r,\bR)$ satisfying
$t_{\alpha\alpha}(x)=I_r$ for all $x\in U_\alpha$ and 
$t_{\alpha\beta}(x)t_{\beta\gamma}(x)t_{\gamma\alpha}(x)=I_r$ for all $x\in U_\alpha\cap U_\beta\cap U_\gamma$, 
we may construct a smooth rank $r$ vector bundle $E$ over $M$ by gluing the rank $r$
product vector bundles $\{ U_\alpha\times \bR^r\to U_\alpha:\alpha\in I\}$
along $(U_\alpha\cap U_\beta)\times \bR^r$ using $t_{\alpha\beta}$. 

Let $s \in C^\infty(M,E)$ be a global section, and let $s_\alpha\in C^\infty(U_\alpha,\bR^r)$ be defined
as the previous lecture. Then $h_\alpha(x)=(x,s_\alpha(x))$ for $x\in U_\alpha$. On $U_\alpha\cap U_\beta$,
$$
(x,s_\alpha(x))= h_\alpha(x)= h_\alpha\circ h_\beta^{-1}\circ h_\beta(x)= h_\alpha \circ h_\beta^{-1}(x,s_\beta(x))
=(x,t_{\alpha\beta}(x)s_\beta(x)).
$$
So we have
\begin{equation}\label{eqn:st}
s_\alpha = t_{\alpha\beta} s_\beta.
\end{equation}
In a similar fashion, let $(\nabla s)_\alpha\in \Omega^1(U_\alpha,\bR^r)$ be defined as in the previous lecture. The 
\begin{equation}\label{eqn:nablast}
(\nabla s)_\alpha = t_{\alpha\beta}(\nabla s)_\beta.
\end{equation}
The left hand side of \eqref{eqn:nablast} is
$$
ds_\alpha +\omega_\alpha s_\alpha = d(t_{\alpha\beta}s_\beta) + \omega_\alpha t_{\alpha\beta} s_\beta
= (dt_{\alpha\beta}) s_\beta + t_{\alpha\beta} (ds_\beta) + \omega_\alpha t_{\alpha\beta} s_\beta
$$
and the right hand side of \eqref{eqn:nablast} is
$$
t_{\alpha\beta}ds_\beta + t_{\alpha\beta} \omega_\beta s_\beta . 
$$
Therefore,
\begin{equation}\label{eqn:omegat}
\omega_\beta = t_{\alpha\beta}^{-1} dt_{\alpha\beta} + t_{\alpha\beta}^{-1} \omega_\alpha t_{\alpha\beta}
\end{equation}
on $U_\alpha\cap U_\beta$. A connection $\nabla: \Omega^0(M,E)\to \Omega^1(M,E)$ is equivalent to 
a collection $\{ \omega_\alpha \in \Omega^1(U_\alpha,\fgl(r,\bR))\}$ satisfying \eqref{eqn:omegat}
on $U_\alpha\cap U_\beta$. 


\medskip

\noindent
{\bf \large Pullback bundle}

\smallskip

Let $f : M \to N$ be a smooth map between smooth manifolds, and let $\pi :E \to N$ be a smooth vector bundle on $N$. Then we can define a 
bundle $\tilde{\pi} : f^*E \to M$ called the {\em pullback bundle} in the following manner. As a set 
\[
f^*E = \bigcup_{p \in M}E_{f(p)} = \{(p,q) \in M \times E : f(p) = \pi(p)\}.
\]
The smooth structure is determined in the following manner. If $s : N \to E$ is a smooth section of $E$, then $f^*s : M \to f^*E$ given by 
\[
f^*s(p) = s(f(p)) \in E_{f(p)} =: (f^*E)_p
\]
is a smooth section of $f^*E$. If $e_1, \ldots, e_r$ are a smooth frame for $E|_U$, where $U$ is an open set in $N$, then
$f^*e_1, \ldots, f^*e_r$ are a smooth frame of $f^*E|_{f^{-1}(U)}$.  A section $s : f^{-1}(U) \to f^*E|_{f^{-1}(U)}$ is smooth if and only if 
we can write 
\[
s = \sum_{j=1}^r a_j f^*e_j
\]
where the $a_j$ are smooth functions on $f^{-1}(U)$.  We have a pullback map 
$$
f^*:C^\infty(N,E)\to C^\infty(M,f^*E).
$$

Suppose that $\{U_\alpha:\alpha\in I\}$ is an open cover of $N$ with local trivializations 
$h_\alpha: \pi^{-1}(U_\alpha)\to U_\alpha\times \bR^r$, and define transition functions
$t_{\alpha\beta}:U_\alpha\cap U_\beta \to GL(r,\bR)$ as before. Then
$$
f^*t_{\alpha\beta}:= t_{\alpha\beta}\circ f :f^{-1}(U_\alpha\cap U_\beta) =f^{-1}(U_\alpha)\cap f^{-1}(U_\beta) \to GL(r,\bR)
$$ 
are the transition functions of $f^*E$. 


\begin{definition}[pullback connection]
Let $f : M \to N$ be a smooth map between smooth manifolds, and let $\pi : E \to N$ 
be a smooth vector bundle together with a connection $\nabla$. Then there is a unique connection 
$f^*\nabla$ on $f^*E$, called the {\em pullback connection}, such that 
\[
(f^*\nabla)(f^*s) = f^*(\nabla s)
\]
for a smooth section $s : N \to E$. 
\end{definition}

In other words, if $s : N \to E$ is a smooth section, $p$ is a point of $M$, and $X \in T_pM$, then 
\[
(f^*\nabla)_X(f^*s) = f^*(\nabla_{df_p(X)}s). 
\]

In terms of local trivializations, we know that if $e_1, \ldots, e_r$ are a smooth frame of $E|_U$, then $f^*e_1, \ldots, f^*e_r$ are 
a smooth frame for $f^*E|_{f^{-1}(U)}$. On $U$, we know that 
\[
\nabla e_j = \sum_{k=1}^r \omega_j^k \otimes e_k.
\]
Then 
\[
(f^*\nabla)(f^*e_j) = f^*(\nabla e_j) = \sum_{k=1}^r f^* \omega_j^k \otimes f^*e_k.
\]
Therefore, if $\{\omega_\alpha\in \Omega^1(U_\alpha,\fgl(r,\bR)):\alpha\in I\}$ are connection 1-forms
of the connection $\nabla$ on $E\to N$, then 
 $\{f^*\omega_\alpha\in \Omega^1(f^{-1}(U_\alpha),\fgl(r,\bR)):\alpha\in I\}$ are connection 1-forms
of the pullback connection $f^*\nabla$ on $f^*E\to M$.


We next consider the special case $E=TN$.
\begin{definition}
Let $F : M \to N$ be a smooth map between smooth manifolds. Define a
pushforward map 
$$ 
F_* : \fX(M)=C^\infty(M,TM) \to C^\infty(M, F^*TN)
$$
by 
\[
(F_*X)(p) = (dF_p)(X(p)) \in T_{F(p)}N = (F^*TN)_p,
\]
and define a pullback map
$$
F^* : \fX(N)=C^\infty(N,TN) \to C^\infty(M, F^*TN)
$$ 
by 
\[
(F^*Y)(p) = Y(F(p)) \in T_{F(p)}N = (F^*TN)_p
\]
\end{definition}

\begin{remark}
Let $X\in \fX(M)$ be a smooth vector field on $M$, and let
$Y\in \fX(N)$ be a smooth vector field on $N$. Then $X$ and $Y$ are $F$-related
in the sense of Definition \ref{F-related} if and only of
$$
F_* X = F^*Y \in C^\infty(M, F^*TN). 
$$
\end{remark}

\begin{definition}
An element in $C^\infty(M,F^*TN)$ is a smooth map $V : M \to F^*TN$ is such that the diagram 
\[
\xymatrix{ 
& TN \ar[d]^{\pi} \\
M \ar[ru]^{V} \ar[r]^{F}& N
}
\]
commutes. Following \cite{dC}, we call $V$ a {\em smooth vector field along $F :M \to N$}. 
\end{definition}

As special cases of the above definition:
\begin{itemize}
\item In \cite[Chapter 2]{dC}, we consider vector fields
along a parametrized curve $\gamma:I\to N$, where $I$ is an open interval in $\bR$
and $\gamma$ is a smooth map.
\item In \cite[Chapter 3]{dC}, we consider vector fields along
a parametrized surface $s:A\to N$, where $A$ is an open set in $\bR^2$ and $s$ is a smooth map.
\end{itemize}


\begin{proposition}\label{pullback-properties}
Suppose that we have a smooth map $F : M \to N$ from a smooth manifold  $M$ to a Riemannian manifold $(N,h)$,
so that we have a pushforward map $F_*:\fX(M)\to C^\infty(M,f^*TN)$.
Let $\nabla$ be an affine connection on $N$, and let $D:= F^*\nabla$ be the pull-back connection
on $F^*TN$. 
\begin{enumerate}
\item[(i)] If $\nabla$ is compatible with the Riemannian metric $h$ then 
\begin{equation}\label{eqn:pullback-inner-product}
X\langle V, W \rangle = \langle D_XV, W\rangle + \langle V, D_X W \rangle \  \forall X \in \mathfrak{X}(M)
\  \forall V,W \in C^\infty(M, F^*TN).
\end{equation}
Here the inner product $\langle \ , \ \rangle$ is defined by $h$.

\item[(ii)] If $\nabla$ is symmetric then 
\begin{equation}\label{eqn:pullback-symmetry}
D_X F_*Y - D_Y F_*X = F_*([X,Y]) \quad \forall X,Y\in \mathfrak{X}(M). 
\end{equation}
\end{enumerate}
In particular, if $\nabla$ is the Levi-Civita connection then the pullback connection $D$
satisfies \eqref{eqn:pullback-inner-product} and \eqref{eqn:pullback-symmetry}.  
\end{proposition}
\begin{proof} Assignment 10 (1).
\end{proof}


Let $N$ be a smooth manifold with an affine connection $\nabla$.

Let $\gamma : I \to N$ be a smooth curve in $N$,  and let $V$ be a smooth vector field along $\gamma$. 
The covariant derivative along $\gamma$ is given by   
\[
\frac{DV}{dt} = (\gamma^* \nabla)_{\frac{\partial}{\partial t}}V.
\]
The following proposition, which is the same as Proposition \ref{compatible-g}, 
is a special case of part (i) of Proposition \ref{pullback-properties}.
\begin{proposition} 
If $\nabla$ is compatible with a Riemannian metric $h$ on $N$ then the covariant derivative
along a parametrize curve $\gamma:I\to N$ satisfies
\[
\frac{d}{dt} \langle V, W \rangle = \langle \frac{DV}{dt}, W \rangle + \langle V, \frac{DW}{dt} \rangle
\]
for any vector fields $V,W$ along $\gamma$, where the inner product $\langle \ , \ \rangle$ is defined by $h$.
\end{proposition}

Let $s : A \to N$ be a parametrized surface in $N$,
where $A$ is an open set in $\bR^2$. Let $(u,v)$ be coordinates on $\bR^2$.
Then $\{ \frac{\partial}{\partial u}, \frac{\partial}{\partial v}\}$ is a smooth frame for $TA$. 
Let 
\[
\frac{\partial s}{\partial u} := s_* \frac{\partial}{\partial u},\
\frac{\partial s}{\partial v}:= s_*\frac{\partial}{\partial v} \in C^\infty(A,s^*TN).
\]
Let $W$ be a vector field along this parametrized surface, that is, $W \in C^\infty(A, s^*TN)$. Then we define 
\[
\frac{DW}{\partial u}  := (s^*\nabla)_{\frac{\partial}{\partial u}}W,\ 
\frac{DW}{\partial v} := (s^*\nabla)_{\frac{\partial}{\partial v}} W \in C^\infty(A,s^*TN). 
\]
\begin{proposition} If $\nabla$ is symmetric then the covariant derivative along the parametrized surface $s:A\to N$
satisfies 
\[
\frac{D}{\partial v} \frac{\partial s}{\partial u} = \frac{D}{\partial u} \frac{\partial s}{\partial v}.
\]
\end{proposition}
\begin{proof} Let $D:=s^*\nabla$ be the pullback connection on $s^*TN$. Then
$$
\frac{D}{\partial v} \frac{\partial s}{\partial u}- \frac{D}{\partial u} \frac{\partial s}{\partial v}
=D_{\frac{\partial}{\partial v}} \big( s_*\frac{\partial}{\partial u}\big) -
D_{\frac{\partial}{\partial u}} \big(s_*\frac{\partial}{\partial v}\big) = s_*([ \frac{\partial}{\partial v},\frac{\partial}{\partial u}])=0
$$
where the second equality follows from part (ii) of Proposition \ref{pullback-properties}.
\end{proof}




We now study the homogeneity of the geodesics. Let 
\[
\phi : (-\delta, \delta) \times U \to TM
\]
be the geodesic flow defined on some open subset $U \subset TM$. Let $\gamma=\pi\circ \phi: (-\delta, \delta) \times U \to M$.
Then $\phi(t,q,v)=(\gamma(t,q,v),\frac{\partial \gamma}{\partial t}(t,q,v))$. 

\begin{lemma}\label{geodesic-homo}
If the map $\gamma(t,q,v)$ is defined for $t \in (-\delta, \delta)$, then for each $a > 0$, the map $\gamma(t,q,av)$ 
is defined for $t \in (-\delta/a, \delta/a)$ and $\gamma(t,q,av) = \gamma(at, q,v)$. 
\end{lemma}
\begin{proof} Observe that, if $\beta:(-\delta,\delta)\to M$ is a geodesic 
with $\beta(0)=q\in M$ and $\beta'(0)= v\in T_qM$, then
$\tilde{\beta}:(-\delta/a,\delta/a)\to M$ defined by $\tilde{\beta}(t)=\beta(at)$ is 
a geodesic with $\tilde{\beta}(t)=q$ and $\tilde{\beta}'(0)=av$. 
\end{proof}

\begin{remark}
If $M$ is compact, the tangent bundle $TM$ is not compact, so the flow may not exist for all time $t$. However, we can consider the sphere bundle $S(TM)=\{(x,v)\in TM: |v|=1\}$, which is compact. The geodesic field $G$ on $TM$ is tangent to 
$S(TM)$, so it restricts to a vector field $\tilde{G}$ on $S(TM)$. By Lemma \ref{flow-compact}, the flow
of $\tilde{G}$ is defined on $\bR\times S(TM)$: $\tilde{\phi}: \bR\times S(TM)\to S(TM)$. 
By the above Lemma \ref{geodesic-homo}, the geodesic flow $\phi$ is defined on $\bR\times TM$. 
\end{remark}







\section{Monday, November 23, 2015}
Given $p \in M$, there is an open neighborhood $V$ of $p$ in $M$, an $\epsilon > 0$ and a $\delta > 0$ such that 
$\gamma(t,q,v)$ is defined for $- \delta < t < \delta$, $q\in V$, and $|v| < \epsilon$. By Lemma \ref{geodesic-homo}, 
$\gamma(t,q,v)$ is defined for $-2 < t < 2$, $q \in V$, and $|v| < \epsilon \delta/2$. So for any $p \in M$, 
there is an open neighborhood $V$ of $p$ in $M$ and an $\epsilon > 0$ such that $\gamma(t,q,v)$ is defined for $-2 < t < 2$, $q \in V$, and $|v| < \epsilon$. 

\begin{definition}[Exponential Map]
Let $U_{(V,\epsilon)} = \{(q, w) \in TM : q \in V, |w| < \epsilon\}$. Define 
$$
\exp : U_{(V,\epsilon)} \longrightarrow M,\quad \exp(q,w)=\gamma(1, q, w).
$$
Also define 
$$
\exp_p : B_\epsilon(0) \longrightarrow  M,\quad \exp_p(v) = \gamma(1,p,v),
$$
where $B_\epsilon(0)\subset T_pM$ is the open ball with center at the origin and with radius $\epsilon>0$. 
(Geometrically, this means that we find the unique geodesic passing through $p$ with velocity $v$ and we flow for unit amount of time.) 
\end{definition}

\begin{lemma}
The map $(d \exp_p)_0 : T_0(T_pM)= T_pM \to T_pM$ is the identity map.
\end{lemma}

\begin{proof}
$$
(d \exp_p)_0(v) = \frac{d}{dt}\Big|_{t=0} \exp_p(tv) 
= \frac{d}{dt}\Big|_{t=0} \gamma(1, p, tv) 
= \frac{d}{dt}\Big|_{t=0} \gamma(t, p, v) = v.
$$ 
\end{proof}

\begin{corollary}\label{exp-diffeo}
There is an open neighborhood $U$ of $0$ in $T_pM$ such that $\exp_p : U \to V := \exp_p(U)$ is a diffeomorphism. 
\end{corollary}

\begin{definition}
In Corollary \ref{exp-diffeo}, the open neighborhood $V$ is called a {\em normal neighborhood of $p$ in $M$}.  
If $\overline{B_\epsilon(0)} \subset U$, then $B_\epsilon(p) := \exp_p(B_\epsilon(0)) \subset M$ is called a {\em normal ball} (or {\em geodesic ball}) 
of radius $\epsilon > 0$ centered at $p$. The boundary $S_\epsilon(p)=\partial B_\epsilon(p)$ of this geodesic ball is called the {\em normal sphere}
(or {\em geodesic sphere}) of radius $\epsilon>0$ centered at $p$. 
\end{definition}

\begin{example}
The exponential $\exp_p : T_p\bR^n \to \bR^n$ is given by $\exp_p(v) = p + v$, which is a global diffeomorphism. 
\end{example}

\begin{example}
The map $\exp_p : T_pS^n \to S^n$ is given by 
\[
\exp_p(v) = \begin{cases}
p, & v=0,\\
\cos(|v|)p + \sin(|v|)\frac{v}{|v|}, & v\neq 0
\end{cases}
\]
This is a diffeomorphism of $B_\pi(0)$ onto $S^n \setminus \{-p\}$.  
\end{example}

\medskip

\noindent
{\bf \large Minimizing properties of geodesics}

\begin{lemma}[Gauss]
Let $p \in M$ and $v \in T_pM$ such that $\exp_p(v)$ is defined. Identify $T_pM$ with $T_v(T_pM)$. Then for $w \in T_pM$, we have 
\[
\langle (d \exp_p)_v(v), (d \exp_p)_v(w) \rangle= \langle v, w \rangle.
\]
\end{lemma}

\begin{proof} There exist $\delta, \epsilon >0$ small enough such that
$f(s,t):= \exp_p(t(v+sw))$ is defined for $t\in (-\delta,1+\delta)$ and $s\in (-\epsilon, \epsilon)$.
For any $s\in (-\epsilon,\epsilon)$, the curve $f_s:(-\delta, 1+\delta)\to M$ defined by $f_s(t):= f(s,t)=\exp_p(t(v+sw))$ is a geodesic
with $f_s(0)=p$ and $f_s'(0)=v+sw$. So we have
\begin{equation}\label{eqn:ftt}
\frac{D}{\partial t}\frac{\partial f}{\partial t}(s,t)= \frac{D}{dt}f_s'(t)= 0
\end{equation}
and $|\frac{\partial f}{\partial t}(s,t)|= |f_s'(t)|= |f_s'(0)|=|v+sw|\Rightarrow$
\begin{equation}\label{eqn:ft}
\langle \frac{\partial f}{\partial t},\frac{\partial f}{\partial t}\rangle (s,t)= |v+sw|^2 = |v|^2+ 2s\langle v, w\rangle + s^2|w|^2. 
\end{equation}
We also have
\begin{eqnarray*}
\frac{\partial f}{\partial s}(s,t)=(d\exp_p)_{t(v+sw)}(tw) & \Rightarrow&  \frac{\partial f}{\partial s}(0,t)=  (d\exp_p)_{tv}(tw);\\
\frac{\partial f}{\partial t}(s,t)=(d\exp_p)_{t(v+sw)}(v+sw) & \Rightarrow&  \frac{\partial f}{\partial t}(0,t)= (d\exp_p)_{tv}(v).
\end{eqnarray*}
So 
\begin{eqnarray*}
\langle \frac{\partial f}{\partial t}, \frac{\partial f}{\partial s}\rangle (0,1) &=&  \langle (d \exp_p)_v(v), (d \exp_p)_v(w) \rangle ,\\
\langle \frac{\partial f}{\partial t}, \frac{\partial f}{\partial s}\rangle (0,0) &=& 0.
\end{eqnarray*}
\begin{eqnarray*}
&& \langle (d \exp_p)_v(v), (d \exp_p)_v(w) \rangle\\
&& =\langle \frac{\partial f}{\partial t}, \frac{\partial f}{\partial s}\rangle (0,1)- \langle \frac{\partial f}{\partial t}, \frac{\partial f}{\partial s}\rangle (0,0)
= \int_0^1 \frac{\partial}{\partial t} \langle \frac{\partial f}{\partial t}, \frac{\partial f}{\partial s}\rangle(0,t) dt.
\end{eqnarray*}
\begin{eqnarray*}
&& \frac{\partial}{\partial t}\langle \frac{\partial f}{\partial t},\frac{\partial f}{\partial s}\rangle  = 
\langle \frac{D}{dt}\frac{\partial f}{\partial t}, \frac{\partial f}{\partial s}\rangle 
+ \langle \frac{\partial f}{\partial t},\frac{D}{\partial t}\frac{\partial f}{\partial s}\rangle \\
&& = \langle \frac{\partial f}{\partial t},\frac{D}{\partial t}\frac{\partial f}{\partial s}\rangle 
=  \langle \frac{\partial f}{\partial t},\frac{D}{\partial s}\frac{\partial f}{\partial t}\rangle 
= \frac{1}{2}\frac{\partial}{\partial s}\langle  \frac{\partial f}{\partial t},  \frac{\partial f}{\partial t}\rangle\\
&& = \frac{1}{2}\frac{\partial}{\partial s} (|v|^2 + 2s \langle v,w\rangle + s^2|w|^2) \\
&& = \langle v,w \rangle + s |w|^2.
\end{eqnarray*} 
The first equality follows from part (i) of Proposition \ref{pullback-properties}; 
the second equality follows from \eqref{eqn:ftt};
the third equality follows from part (ii) of Proposition \ref{pullback-properties};
the fourth equality follows from part (i) of Proposition \ref{pullback-properties};
the fifth equality follows from \eqref{eqn:ft}.
$$
\frac{\partial}{\partial t}\langle \frac{\partial f}{\partial t},\frac{\partial f}{\partial s}\rangle(0,t) 
=  \langle v,w\rangle \Rightarrow
\langle (d \exp_p)_v(v), (d \exp_p)_v(w) \rangle =\int_0^1 \langle v,w\rangle dt =\langle v, w\rangle.
$$
\end{proof}


\begin{proposition}
Let $(M,g)$ be a Riemannian manifold, $p \in M$, and $U$ a normal neighborhood of $p$. Let $B \subset U$ be a normal ball with center $p$, that is, $B = \exp_p(B_\delta(0))$ for some $\delta > 0$. Suppose that $\gamma : [0,1] \to B$ is a geodesic segment such that $\gamma(0) = p$ and $\gamma(1) = q$. Let $c : [0,1] \to M$ be a piecewise smooth curve such that
$c(0)=p$ and $c(1)=q$. Then $l(\gamma) \le l(c)$, with equality if and only if the curves $c$ and $\gamma$ have the same image.
\end{proposition}

\begin{proof}
We may assume that $c([0,1]) \subset B$, since $l(c)\geq l(c|_{[0,t_1]})$ where $c(t_1)\in \partial B$ and $c(t)\subset B$ for $0\leq t<t_1$.
We may also assume that $c(t) \ne p$ for $t > 0$, otherwise consider $c|_{[t_2, 1]}$ where $c(t_2) = p$ and $c(t) \ne p$ for $t_2 < t \le 1$. 

Define $b:[0,1]\to B_\delta(0)\subset T_p M$ by $b(t)=\exp_p^{-1}(c(t))$. Then $b:[0,1]\to T_pM$ is a piecewise smooth curve in $T_pM$,
and $c(t) = \exp_p(b(t))$. Since $c(t)\neq p$ for $t>0$, $b(t)\neq 0$ for $t>0$, so for $t\in (0,1]$ we may write
$$
b(t)= r(t)v(t)
$$
where $r(t)=|b(t)|>0$ and $v(t)=b(t)/|b(t)|$ are piecewise smooth. We have 
$$
\langle v(t), v(t)\rangle =1,\quad \langle v(t),v'(t)\rangle =0. 
$$
 
\[
\frac{dc}{dt}(t) = (d \exp_p)_{b(t)} (b'(t)) = r'(t) (d \exp_p)_{b(t)}(v(t)) + r(t) (d \exp_p)_{b(t)}(v'(t)).
\]
Therefore 
\begin{eqnarray*}
\Big|\frac{dc}{dt}(t)\Big|^2 &=&  r'(t)^2 |(d \exp_p)_{(b(t))}(v(t)) |^2 + r(t)^2 |(d \exp_p)_{b(t)}(v'(t))|^2\\
&& + 2r'(t)r(t) \langle (d \exp_p)_{(b(t))}(v(t)), (d \exp_p)_{(b(t))}(v'(t))\rangle 
\end{eqnarray*}
Note that $v(t)$ is a scalar multiple of $b(t)$, so by Gauss's lemma.
\begin{eqnarray*}
&& |(d \exp_p)_{(b(t))}(v(t)) |^2 = |v(t)|^2 =1,\\
&& \langle (d \exp_p)_{(b(t))}(v(t)), (d \exp_p)_{(b(t))}(v'(t))\rangle =\langle v(t),v'(t)\rangle =0.
\end{eqnarray*}
Therefore,
$$
\left|\frac{dc}{dt}(t)\right| =\sqrt{ r'(t)^2 + r(t)^2 |(d \exp_p)_{b(t)}(v'(t))|^2} \geq |r'(t)| \geq r'(t). 
$$
So the length of $c$ satisfies 
\[
l(c) = \int_0^1 \left|\frac{dc}{dt}(t)\right|dt  \ge \int_0^1r'(t) dt =  r(1) - r(0) = l(\gamma).
\]
Equality holds if and only if $v'(t) = 0$ and $\frac{dr}{dt} \ge0$. In this case, $v(t)=v$ is a constant unit vector, and  
\[
c(t) = \exp_p(r(t)v)
\]
which has the same image as $\gamma(t)=\exp_p(l(\gamma)t v)$. 
\end{proof}




\section{Wednesday, November 25, 2015}

\begin{theorem}
Let $(M,g)$ be a Riemannian manifold and let $p$ be a point of $M$. Then there is an open neighborhood $W$ of $p$ in $M$ and $\delta > 0$ 
such that for any $q \in W$, $\exp_q$ is a diffeomorphism from $B_\delta(0) \subset T_qM$ onto the geodesic ball $B_\delta(q)$,  
and $W \subset B_\delta(q)$. 

In particular, $W$ is a normal neighborhood of $q$ for any $q \in W$. We call $W$ a {\em totally geodesic neighborhood} of $p$ in $M$.
\end{theorem}


\begin{proof}
There is an open neighborhood $V$ of $p$ in $M$ and an $\epsilon > 0$ such that $\gamma(t,q,v)$ is defined for any $t \in (-2,2)$,
$q\in V$, and $|v|<\epsilon$. Then $\exp_q(v) = \gamma(1, q,v)$ is defined for $(q,v)\in U_{(V,\epsilon)} := \{(q,v) \in TM : q \in V, |v| < \epsilon\}$. 

Define $F : U_{(V,\epsilon)}  \to M \times M$ be 
\[
F(q,v) = (q,\exp_q(v)).
\]
We now compute
$$
dF_{(p,0)}: T_{(p,0)}TM = T_p M\times T_p M \longrightarrow T_{(p,p)}(M\times M) = T_p M\times T_p M.
$$
For any $q \in V$, we have $F(q,0) = (q,\exp_q(0)) = (q,q)$. This implies that 
\[
dF_{(p,0)}(u,0) = (u,u).
\] 
For any $v \in T_qM$, we have $F(p,v) = (p, \exp_pv)$. This implies that 
\[
dF_{(p,0)}(0,v) = (0, (d\exp_p)_0(v)) = (0,v).
\]
Therefore  
\[
dF_{(p,0)} = \begin{bmatrix}
I & 0 \\ I & I 
\end{bmatrix}
\]
where $I: T_pM\to T_pM$ is the identity map. In particular, $dF_{(p,0)}$ is a linear isomorphism. By the Inverse Function Theorem,
there exists an open neighborhood $V'$ of $p$ in $M$, $V'\subset V$, and $\delta \in (0,\epsilon)$, such that
$F|_{U_{(V',\delta)}}$ is a diffeomorphism onto its image $W':= F(U_{(V',\delta)})$, which is an open neighborhood
of $(p,p)$ in $M\times M$.  There is an open neighborhood $W$ of $p$ in $M$ such that 
$$
W \times W \subset W'=\bigcup_{q\in V'}\{q\} \times B_\delta(q). 
$$
Therefore $W\subset B_\delta(q)$ for all $q\in W$.  
\end{proof}

\begin{corollary}
For any $q_1, q_2 \in W$, there is a unique geodesic $\gamma$ joining $q_1$ and $q_2$. 
\end{corollary}

\begin{corollary}
Let $\gamma : [a,b] \to M$ be a piecewise smooth curve and write $\gamma(a) = p$ and $\gamma(b) = q$. Suppose that for any piecewise smooth curve $\beta : [c,d] \to M$ such that $\beta(c) = p$ and $\beta(d) = q$, the length of $\beta$ is at least the length of $\gamma$. Then $\gamma$ is a geodesic. 
\end{corollary}


\begin{definition}
Let $(M,g)$ be a Riemannian manifold. We say that an open subset $S\subset M$ is {\em strongly convex} if for each pair 
$q_1, q_2$ in the closure $\overline{S}$ of $S$, 
there is a unique minimizing geodesic $\gamma$ such that $\gamma(0) = q_1$, $\gamma(1) = q_2$, 
and $\gamma((0,1)) \subset S$.
\end{definition}

\begin{example}
Let $(M,g) = (\bR^n, g_0)$ be the Euclidean space.  Then strongly convex implies convex in the usual sense: $S\subset \bR^n$ is convex if for any $q_1, q_2\in S$, the
line segment $\overline{q_1q_2}$ connecting $q_1$ and $q_2$ is contained in $S$.
An open ball in $(\bR^n,g_0)$ is strongly convex, thus convex. 
The set $(0,1)^n$ is convex but not strongly convex.
\end{example}

\begin{proposition}
For each $p \in M$ there is a $\beta > 0$ such that $B_{\beta}(p)$ is strongly convex. 
\end{proposition}
\begin{proof}
See \cite[Chapter 3, Section 4]{dC}. 
\end{proof}

\begin{example}
Let $p$ be any point in the Euclidean space $(\bR^n,g_0)$. Then the geodesic ball $B_r(p)$ is strongly convex for $r>0$.

Let $p$ be a point in the round sphere $(S^n,g_{\mathrm{can}})$ of radius 1. Then the geodesic ball
$B_r(p)$ is strongly convex when $0<r<\pi/2$, but not strongly convex when $\pi/2\leq r< \pi$. 
\end{example}



\noindent
{\bf \large Curvature}

Let $(M,g)$ be a Riemannian manifold with $\nabla$ the Levi-Civita connection. Let $\mathfrak{X}(M)$ be the space of smooth vector fields on $M$. 

\begin{definition}
For $X,Y \in \mathfrak{X}(M)$, define an $\mathbb{R}$-linear map $R(X,Y) : \mathfrak{X}(M) \to \mathfrak{X}(M)$ by the rule 
\[
R(X,Y)Z = \nabla_Y \nabla_X Z - \nabla_X \nabla_Y Z + \nabla_{[X,Y]}Z = [\nabla_Y, \nabla_X]Z - \nabla_{[Y,X]}Z
\]
\end{definition}

\begin{proposition}
The map $R : \mathfrak{X}(M) \times \mathfrak{X}(M) \times  \mathfrak{X}(M) \to \mathfrak{X}(M)$ given by $(X,Y,Z) \mapsto R(X,Y)Z$ 
\begin{enumerate}
\item[(i)] is anti-symmetric in $X,Y$
\item[(ii)] is $C^\infty(M)$-linear in $X,Y,Z$. 
\end{enumerate}
Therefore $R$ can be viewed as an element of 
$$
\Omega^2(M, \End TM) := C^\infty(M, \Lambda^2 T^*M \otimes T^*M \otimes TM),
$$
that is, $R$ is an $\End(TM)$ valued $2$-form on $M$. In particular, $R$ is a $(1,3)$-tensor. 
\end{proposition}

\begin{proof} (i) is clear from the definition. Given (i), it remains to show that
for any $X,Y,Z\in \fX(M)$ and any $f\in C^\infty(M)$, 
\begin{enumerate}
\item[(a)] $R(fX,Y)Z = fR(X,Y)Z$, and 
\item[(b)] $R(X,Y)(fZ)=fR(X,Y)Z$ 
\end{enumerate}
\begin{eqnarray*}
R(fX,Y)Z & =& \nabla_Y\nabla_{fX}Z -\nabla_{fX}\nabla_Y Z +\nabla_{[fX,Y]} Z\\
&=& \nabla_Y (f\nabla_X Z)-f\nabla_X \nabla_Y Z + \nabla_{f[X,Y]-Y(f)X} Z\\
&=& Y(f) \nabla_X Z + f\nabla_Y\nabla_X Z -f\nabla_X \nabla_Y Z + f\nabla_{[X,Y]}Z -Y(f)\nabla_X Z\\
&=&  fR(X,Y)Z
\end{eqnarray*} 
\begin{eqnarray*}
R(X,Y)(fZ)&=& \nabla_Y\nabla_X(fZ) -\nabla_X\nabla_Y(fZ) +\nabla_{[X,Y]} (fZ)\\
&=& \nabla_Y (X(f) Z+ f\nabla_X Z) - \nabla_X (Y(f) Z+ f\nabla_Y Z) + ([X,Y]f)Z + f\nabla_{[X,Y]} Z\\
&=& \quad YX(f) Z + X(f)\nabla_Y Z + Y(f)\nabla_X Z + f\nabla_X\nabla_Y Z\\
&&    -XY(f) Z - Y(f)\nabla_X Z - X(f)\nabla_Y Z - f\nabla_Y\nabla_X Z \\
&&     + (XY(f)-YX(f))Z  + f\nabla_{[X,Y]} Z\\
&=&  fR(X,Y)Z
\end{eqnarray*}
\end{proof}

\begin{proposition}[Bianchi identity]
We have 
\[
R(X,Y)Z + R(Y,Z)X + R(Z,X)Y = 0.
\]
\end{proposition}

\begin{proof} See \cite{dC} page 91.
\end{proof}

\begin{definition}
For $X,Y,Z,T \in \mathfrak{X}(M)$, define 
\[
R(X,Y,Z,T) := \langle R(X,Y)Z, T \rangle. 
\]
Then $R(X,Y,Z,T)$ is $C^\infty(M)$-linear in each slot, so it is a $(0,4)$ tensor. 
\end{definition}

\begin{proposition}\label{Rsymmetry}
The $(0,4)$ tensor $R(X,Y,Z,T)$ satisfies the following properties.
\begin{enumerate}
\item[(a)] $R(X,Y,Z,T) + R(Y,Z,X,T) + R(Z,X,Y, T) = 0$.  (the Bianchi identity)
\item[(b)] $R\in C^\infty\big(M, \mathrm{Sym}^2(\Lambda^2 T^*M)\big)$, i.e.
\begin{enumerate}
\item[(b1)] $R(X,Y,Z,T) = -R(Y,X,Z,T)$
\item[(b2)] $R(X,Y,Z,T) = -R(X,Y,T,Z)$
\item[(b3)] $R(X,Y,Z,T) = R(Z,T,X,Y)$ 
\end{enumerate}
\end{enumerate}
\end{proposition}

\begin{proof} See \cite{dC} page 91-92.
\end{proof}




\section{Monday, November 30, 2015}

\noindent
{\bf \large The Riemannian curvature tensor in local coordinates}

Let $(U,\phi)$ be a $C^\infty$ chart in $M$. Let $(x_1,\ldots,x_n)$ be local coordinates
on $U$. Let $T$ be any $(r,s)$ tensor on $M$. Then on $U$,
$$
T=\sum_{\substack{i_1,\ldots, i_r\\ j_1,\ldots, j_s}} T^{i_1 \cdots i_r}_{j_1\cdots j_s} 
\frac{\partial}{\partial x_{i_1}} \otimes \cdots \otimes \frac{\partial}{\partial x_{i_r}}
\otimes dx_{j_1}\otimes \cdots \otimes dx_{j_s}
$$
where $T^{i_1\cdots i_r}_{j_1\cdots j_s} \in C^\infty(U)$.

As a $(1,3)$ tensor,
$$
R = \sum_{i,j,k,m} {R_{ijk}}^m  dx_i\otimes dx_j\otimes dx_k\otimes \frac{\partial}{\partial x_m},
$$
where ${R_{ijk}}^m \in C^\infty(U)$ is determined by
$$
R(\frac{\partial}{\partial x_i}, \frac{\partial}{\partial x_j}) \frac{\partial}{\partial x_k}
=\sum_l {R_{ijk}}^m \frac{\partial}{\partial x_m}.
$$

As a $(0,4)$ tensor,
$$
R =\sum_{i,j,k,l} R_{ijkl} dx_i\otimes dx_j\otimes dx_k\otimes dx_l,
$$
where 
$$
R_{ijkl}= R(\frac{\partial}{\partial x_i}, \frac{\partial}{\partial x_j},
\frac{\partial}{\partial x_k}, \frac{\partial}{\partial x_l}) 
= \langle R(\frac{\partial}{\partial x_i}, \frac{\partial}{\partial x_j})\frac{\partial}{\partial x_k},\frac{\partial}
{\partial x_l}\rangle = \sum_m {R_{ijk}}^m g_{ml} \in C^\infty(U). 
$$

By Proposition \ref{Rsymmetry}, 
$$
R_{ijkl}+ R_{jkil} + R_{kijl}=0,\quad
R_{ijkl}=-R_{jikl},\quad
R_{ijkl}= -R_{ijlk},\quad
R_{ijkl}=R_{klij}.
$$


We now express ${R_{ijk}}^m$ in terms of the Christoffel symbol $\Gamma_{ij}^k$. 
$$
R(\frac{\partial}{\partial x_i}, \frac{\partial}{\partial x_j})\frac{\partial}{\partial x_k}
=\nabla_{\frac{\partial}{\partial x_j}}\nabla_{\frac{\partial}{\partial x_i}}\frac{\partial}{\partial x_k}
-\nabla_{\frac{\partial}{\partial x_i}}\nabla_{\frac{\partial}{\partial x_j}}\frac{\partial}{\partial x_k}
+\nabla_{[\frac{\partial}{\partial x_i},\frac{\partial}{\partial x_j}] } \frac{\partial}{\partial x_k}
$$
where $[\frac{\partial}{\partial x_i},\frac{\partial}{\partial x_j} ]=0$, and
\begin{eqnarray*}
\nabla_{\frac{\partial}{\partial x_j}}\nabla_{\frac{\partial}{\partial x_i}}\frac{\partial}{\partial x_k} 
&=& \nabla_{\frac{\partial}{\partial x_j}}(\sum_l \Gamma_{ik}^l \frac{\partial}{\partial x_l}) \\
&=& \sum_l \frac{\partial \Gamma_{ik}^l}{\partial x_j} \frac{\partial}{\partial x_l}
+  \sum_l \Gamma_{ik}^l \nabla_{\frac{\partial}{\partial x_j}} \frac{\partial}{\partial x_l} \\
&=& \sum_m \frac{\partial \Gamma_{ik}^m}{\partial x_j} \frac{\partial}{\partial x_m}
+\sum_{l,m} \Gamma_{ik}^l \Gamma_{jl}^m \frac{\partial}{\partial x_m}
\end{eqnarray*}
So
$$
R(\frac{\partial}{\partial x_i}, \frac{\partial}{\partial x_j})\frac{\partial}{\partial x_k}
=\sum_m (\frac{\partial \Gamma_{ik}^m}{\partial x_j}- \frac{\partial \Gamma_{jk}^m}{\partial x_i}
+ \sum_l \Gamma_{ik}^l \Gamma_{jl}^m -\sum_l\Gamma_{jk}^l \Gamma_{il}^m ) \frac{\partial}{\partial x_m}.
$$

$$
\boxed{ {R_{ijk}}^m = \frac{\partial \Gamma_{ik}^m}{\partial x_j}- \frac{\partial \Gamma_{jk}^m}{\partial x_i}
+ \sum_l \Gamma_{ik}^l \Gamma_{jl}^m -\sum_l\Gamma_{jk}^l \Gamma_{il}^m }
$$
 
\noindent
{\bf \large Sectional Curvature}

If we fix a point $p$ in a Riemannian manifold $(M,g)$, then 
$V=T_p M$ is an inner product space. 

In general, an inner product on a vector space  $V\cong \bR^n$
induces an inner product on $\Lambda^2 V$ as follows: 
if $\{e_1,\ldots, e_n\}$ be an orthonormal basis of $V$
then $\{ e_i\wedge e_j: 1\leq i<j\leq n\}$ is an orthonormal basis of $\Lambda^2 V$.
Equivalently, if $x,y\in V$ then
$$
|x\wedge y|^2 =\langle x,x\rangle \langle y, y\rangle - \langle x,y\rangle^2. 
$$

\begin{definition}
Let $(M,g)$ be a Riemannian manifold with $p$ a point of $M$ and $\sigma$ a 2 dimensional subspace of $T_pM$. 
Define the {\em sectional curvature} of $\sigma$, denoted $K(\sigma, p)$, to be 
\[
K(\sigma, p) = \frac{R(p)(x,y,x,y)}{|x \wedge y|^2}
\]
where $\{ x,y \}$ is a basis of $\sigma$. 
\end{definition}

This is well-defined because if $\{ x',y'\}$ is another basis of $\sigma$ then 
$x' = ax + by$ and $y' = cx + dy$ for some 
$$
\left(\begin{array}{cc}a & b\\ c& d\end{array}\right) \in GL(2,\bR).
$$
The by (b1) and (b2) of Proposition \ref{Rsymmetry},
\[
R(p)(x',y',x',y') = (ad - bc)^2 R(p)(x,y,x,y).
\]
We also have
$$
x' \wedge y'  = (ad-bc) x \wedge y \Rightarrow |x'\wedge y'|^2=(ad-bc)^2 |x\wedge y|^2. 
$$ 

\begin{lemma}\label{r-K}
Let $V$ be an inner product space.
Suppose that $r,r' : V \times V \times V \times V \to \bR$ are $\bR$-linear in each factor 
and satisfy 
\begin{enumerate}
\item[(a)] $r(x,y,z,t)+ r(y,z,x,t)+ r(z,x,y,t)=0$.
\item[(b1)] $r(x,y,z,t)=-r(y,x,z,t)$.
\item[(b2)] $r(x,y,z,t)=-r(x,y,t,z)$.
\item[(b3)] $r(x,y,z,t)= r(z,t,x,y)$.
\end{enumerate}

Define $K, K' : \text{Gr}(2,V) \to \bR$ by 
\[
K(\sigma) = \frac{r(x,y,x,y)}{|x \wedge y|^2},\quad \quad K'(\sigma) = \frac{r'(x,y,x,y)}{|x \wedge y|^2}
\]
where $\{x,y\}$ is any basis of the 2-dimensional subspace $\sigma$ of $V$; this is well-defined by
(b1) and (b2). If $K = K'$, then $r = r'$.  
\end{lemma}

\begin{proof}
Let $\Delta = r - r' : V \times V \times V \times V \to \bR$. Then 
\begin{enumerate}
\item[(1)] $\Delta$ is $\mathbb{R}$-linear in each factor. 
\item[(2)] $\Delta$ satisfies (a), (b1), (b2), (b3).  
\item[(3)] $\Delta(x,y,x,y) = 0$ for any $x,y \in V$.  
\end{enumerate}
We want to show that $\Delta \equiv 0$. 

For each $x,y,z \in V$, by (3), we have 
\begin{align*}
0  &=\Delta(x + z, y, x + z, y) - \Delta(x,y, x,y) - \Delta(z,y,z,y) \\
&= \Delta(x,y,z,y) + \Delta(z,y,x,y) &\text{by linearity}\\
&= 2 \Delta(x,y,z,y) &\text{by (b3).}
\end{align*}

For any $x,y,z,t \in V$, we have 
\begin{align*}
0 &= \Delta(x, y + t, z, y+t) - \Delta(x,y,z,y) - \Delta(x,t,z,t) &\text{by last paragraph} \\
&= \Delta(x,y,z,t) + \Delta(x,t,z,y) &\text{linearity} \\
&= \Delta(x,y,z,t) + \Delta(z,y,x,t) &\text{(b3)} \\
&= \Delta(x,y,z,t) - \Delta(y,z,x,t) &\text{(b1)}. 
\end{align*}
Therefore,
$$
\Delta(x,y,z,t) = \Delta(y,z,x,t) = \Delta(z,x,y,t).
$$
By (a), 
$$
\Delta(x,y,z,t) + \Delta(y,z,x,t) + \Delta(z,x,y,t) =0.
$$
We conclude that
$$
\Delta(x,y,z,t)=0
$$
for all $x,y,z,t\in V$. This completes the proof.
\end{proof}

\begin{corollary}
The sectional curvature determines the Riemannian curvature tensor. 
\end{corollary}


\begin{definition}
We say that $(M,g)$ has {\em constant sectional curvature} $K_0$ if 
for each $p \in M$ and for any $\sigma \in \text{Gr}(2, T_pM)$, we have $K(\sigma) = K_0$. 
\end{definition}

\begin{lemma}
Define $r': V \times V \times V \times V \to \mathbb{R}$ by 
\[
r'(x,y,z,t)= \langle x,z \rangle \langle y,t \rangle - \langle x,t \rangle \langle y,z \rangle. 
\]
Then 
\begin{enumerate}
\item[(1)] $r'$ is $\bR$-linear in each factor
\item[(2)] $r'$ satisfies (a), (b1), (b2), (b3) in Lemma \ref{r-K}. 
\item[(3)] For any $x,y \in V$, we have $r'(x,y, x,y) = |x\wedge y|^2$. 
\end{enumerate}
\end{lemma}

\begin{corollary}\label{constant-K}
The Riemannian manifold $(M,g)$ has constant sectional curvature $K_0$ if and only if for each 
$X,Y,Z,T \in \fX(M)$, we have 
\[
R(X,Y,Z,T) = K_0\left( \langle X, Z \rangle \langle Y, T \rangle - \langle X, T \rangle \langle Y, Z \rangle \right)
\]
\end{corollary}

\begin{definition}
We say a Riemannian manifold $(M,g)$ is {\em flat} if its Riemannian curvature tensor is identically zero. 
\end{definition}

\begin{remark}
By Corollary \ref{constant-K}, $(M,g)$ is flat if and only if $M$ has constant sectional curvature equal to zero. 
\end{remark}

\begin{example}
Euclidean space $(\bR^n, g_0=dx_1^2+\cdots + dx_n^2)$ is flat, since the Christoffel symbols are zero and hence 
$R_{ijkl}$ are zero. Hence $(\mathbb{R}^n, g_0)$ has constant sectional curvature equal to zero. 
\end{example}

\begin{lemma}
Let $f : (M_1, g_1) \to (M_2, g_2)$ be a local isometry, that is, $f$ is a local diffeomorphism and $f^*g_2 = g_1$. 
Let $R_1$ be the curvature tensor of $(M_1, g_1)$ and let $R_2$ be the curvature tensor of $(M_2, g_2)$. Then $R_1 = f^*R_2$. 
\end{lemma}

\begin{proof}
In terms of local coordinates, we see that the local coordinates are equal and the $g_{ij}$ are equal, hence so are the curvature tensors. 
\end{proof}

\begin{example}[Flat $n$-torus]
There is a local isometry from $(\bR^n,g_0)$ to $(T^n= (S^1)^n, g:= (g_{\mathrm{can}})^n)$. Therefore
$(T^n,g)$ is flat. 
\end{example}

\begin{example}
\begin{itemize}
\item At a future time, we will see that $(S^n, g_{\mathrm{can}})$ has constant sectional curvature equal to $+1$.
As a consequence,  $(S^n, r^2g_{can})$  (the round sphere of radius $r>0$) has constant sectional curvature equal to 
$K = 1/r^2$. 
\item We will also see that $\mathcal{H}^n=\{(y_1,\ldots,y_n)\in \bR^n: y_n>0\}$ (upper half space) equipped with 
\[
g_n = \frac{dy_1^2 + \cdots + dy_n^2}{y_n^2}
\]
has constant sectional curvature $K = -1$. 
\end{itemize}
\end{example}


\bigskip
\noindent
{\bf \large Two-dimensional case}

Let $(M,g)$ be a 2-dimensional Riemannian manifold. Let
$(U,\phi)$ be a $C^\infty$ chart on $M$, and let
$(x_1,x_2)$ be local coordinates on $U$. Then on $U$ we have
$$
g=g_{11} dx_1^2 + g_{12} dx_1 dx_2 + g_{21} dx_2 dx_1 + g_{22} dx_2^2 
= g_{11} dx_1^2 + 2g_{12} dx_1 dx_2 + g_{22} dx_2^2.
$$ 
$$
R= \sum_{i,j,k,l=1}^2 R_{ijkl} dx_i \otimes dx_j\otimes dx_k \otimes dx_l = R_{1212} (dx_1\wedge dx_2)\otimes (dx_1\wedge dx_2). 
$$
The only 2-dimensional subspace of $T_pM$ is itself. So in this case the sectional
curvature $K$ is a smooth function on $M$: $K(p)=K(p,T_pM)$ for $p\in M$.
$$
\boxed{ K=\frac{R_{1212}}{g_{11}g_{22}-g_{12}^2} } 
$$


\begin{example}
$(M,g)= (S^2, g_{\can} = d\phi^2 + \sin^2\phi d\theta^2)$. By Example \ref{two-sphere-nabla},
$$
\nabla_{\frac{\partial}{\partial \phi}}\frac{\partial}{\partial \phi}=0,\quad
\nabla_{\frac{\partial}{\partial \phi}}\frac{\partial}{\partial \theta}= 
\nabla_{\frac{\partial}{\partial \theta}}\frac{\partial}{\partial \phi}=\cot\theta\frac{\partial}{\partial\theta},\quad
\nabla_{\frac{\partial}{\partial \theta}}\frac{\partial}{\partial \theta}= -\sin\phi\cos\phi \frac{\partial}{\partial\phi}.
$$ 
Let $(x_1,x_2)=(\phi,\theta)$. Then
$$
R_{1212}= \langle R(\frac{\partial}{\partial \phi},\frac{\partial}{\partial\theta} \frac{\partial}{\partial \phi},\frac{\partial}{\partial\theta}\rangle
$$
where
\begin{eqnarray*}
R(\frac{\partial}{\partial \phi},\frac{\partial}{\partial\theta})\frac{\partial}{\partial \phi}
&=&  \nabla_{\frac{\partial}{\partial \theta}} \nabla_{\frac{\partial}{\partial \phi}}\frac{\partial}{\partial \phi}-
\nabla_{\frac{\partial}{\partial \phi}} \nabla_{\frac{\partial}{\partial \theta}}\frac{\partial}{\partial \phi}
+\nabla_{ [ \frac{\partial}{\partial \phi}, \frac{\partial}{\partial \theta}] } \frac{\partial}{\partial \phi}\\
&=& 0 - \nabla_{\frac{\partial}{\partial \phi}}(\cot \phi\frac{\partial}{\partial \theta}) + 0 
= \csc^2\phi \frac{\partial}{\partial \theta} -\cot^2\phi \frac{\partial}{\partial \theta} =\frac{\partial}{\partial \theta}. 
\end{eqnarray*}
$$
R_{1212}=\langle \frac{\partial}{\partial \theta}, \frac{\partial}{\partial \theta}\rangle = \sin^2\phi
$$
$$
g_{11} g_{22} -g_{12}^2 =\sin^2\phi.
$$
So 
$$
K=\frac{R_{1212}}{g_{11}g_{22}-g_{12}^2}=1.
$$
\end{example}

\bigskip

\noindent
{\bf \large Ricci curvature}


\begin{definition}
For any $p\in M$,  define a symmetric bilinear form $Q_p$ on $T_p M$ by   
\begin{align*}
Q_p(x,y) &:= \text{Trace}( T_pM \ni v \mapsto R(x,v,y) \in T_pM) \\
&= \sum_{i=1}^n R(x, e_i, y, e_i)
\end{align*}
for an orthonormal basis $\{e_i\}$ of $T_pM$. We then define 
\[
\mathrm{Ric}_p = \frac{1}{n-1}Q_p
\]
which is a symmetric $(0,2)$-tensor on $(M,g)$. (Note that this is the same type of tensor as $g$.)
\end{definition}

Why do we use $\frac{1}{n-1}$? Suppose that $(M,g)$ has constant sectional curvature $K_0$. Then 
$$
Q_p(x,y) = \sum_{i=1}^n K_0 \left(\langle x,y \rangle \langle e_i, e_i \rangle - \langle x, e_i \rangle \langle y, e_i \rangle\right)  
= K_0(n \langle x,y \rangle - \langle x , y \rangle)
= (n-1)K_0 \langle x,y \rangle.
$$
So then $\mathrm{Ric}_p(x,y) = K_0 \langle x, y\rangle$. 


In terms of local coordinates, we let 
\[
R(\frac{\partial}{\partial x_i}, \frac{\partial}{\partial x_k})\frac{\partial}{\partial x_j} = \sum_{l} {R_{ikj}}^l \frac{\partial}{\partial x_l}. 
\]
We let 
$$
R_{ij} := Q(\frac{\partial}{\partial x_i},\frac{\partial}{\partial x_j}) 
= \text{Trace}(\frac{\partial}{\partial x_k} \mapsto R(\frac{\partial}{\partial x_i}, \frac{\partial}{\partial x_k})\frac{\partial}{\partial x_j}) 
= \sum_{k} {R_{ikj}}^k 
= \sum_{k,l} R_{ikjl}g^{kl}.
$$
Then $Q = \sum_{i,j}R_{ij} dx_i \otimes dx_j$, where $R_{ij} = R_{ji}$. So 
$$
\mathrm{Ric} = \frac{1}{n-1} \sum_{i,j} R_{ij} dx_i \otimes dx_j,\quad
\textup{where } R_{ij} = \sum_{k,l}R_{ikjl}g^{kl}.
$$


\section{Wednesday, December 2, 2015}

\noindent
{\bf \large Scalar curvature}


\begin{definition} Let $(M,g)$ be a  Riemannian manifold. The scalar curvature
$S$ of $(M,g)$ is a smooth function on $M$ defined as follows. 
For each point $p \in M$, define a linear map $K_p : T_pM \to T_pM$ by 
\[
\langle K_p(x), y \rangle = Q_p(x,y).
\]
Then $K_p$ is self-adjoint, meaning $\langle K_p(x),y\rangle = \langle x, K_p(y)\rangle$. We then define 
\begin{eqnarray*}
S(p) &:=& \frac{1}{n(n-1)} \text{Trace}(K_p) 
= \frac{1}{n(n-1)} \sum_{i=1}^n Q_p(e_i, e_i) \\
&=& \frac{1}{n(n-1)} \sum_{i,j} R(p)(e_i, e_j, e_i, e_j) 
= \frac{1}{n} \sum_{i=1}^n \text{Ric}_p(e_i, e_i). 
\end{eqnarray*}
where $\{ e_1,\ldots, e_n\}$ is any orthonormal basis of $T_pM$. 
\end{definition}
We see that if $(M,g)$ has constant sectional curvature $K_0$, we have $\text{Ric} = K_0 g$ and hence $S(p) = K_0$
for all $p\in M$. 


In terms of local coordinates, we have 
\[
n(n-1)S = R_i^i = R_{ij}g^{ij} = R_{ijkl} g^{ik}g^{jl}
\]

In the special case, when $n = 2$, we have 
\[
R = R_{1212} (dx_1 \wedge dx_2) \otimes (dx_1 \wedge dx_2)
\]
and 
\begin{eqnarray*}
S &=& \frac{1}{2} (R_{1212}g^{11}g^{22}  + R_{2112}g^{21}g^{12} + R_{1221}g^{12}g^{21}+ R_{2121} g^{22}g^{11}) \\
&=&  \frac{1}{2} R_{1212} (2 g^{11}g^{22} - 2 (g^{12})^2) 
= R_{1212}(g^{22}g^{11} - (g^{12})^2) = \frac{R_{1212}}{g_{11}g_{22} - g_{12}^2} = K.
\end{eqnarray*}

\bigskip

\noindent
{\bf \large Covariant derivatives for tensors} 

References: \cite[Chapter 4 Section 5]{dC}, \cite[2B.3]{GHL})

\begin{proposition}
Let $\nabla$ be an affine connection on a smooth manifold $M$. 
Let $X$ be a smooth vector field on $M$ and let $\nabla_X : \mathfrak{X}(M) \to \mathfrak{X}(M)$ denote the covariant derivative along $X$. Then $\nabla_X$ has a unique extension 
\[
\nabla_X : C^\infty(M, T_{s}^rM) \to C^\infty(M, T_{s}^rM) 
\]
such that 
\begin{enumerate}
\item[(i)] $\nabla_X(c(S)) = c(\nabla_X(S))$ for any tensor $S$ and any contraction $c$
\item[(ii)] $\nabla_X(S \otimes T) = (\nabla_X S) \otimes T + S \otimes \nabla_X T$ for any tensors $S,T$. 
\end{enumerate}
\end{proposition}

\begin{proof}
For $f\in C^\infty(M)$, we must define $\nabla_Xf = X(f)$ by the Leibniz rule and (ii). For a $(0,1)$-tensor $\alpha \in \Omega^1(M)$  and a vector field $Y$, we must have 
\begin{eqnarray*}
X(\alpha(Y)) &=& \nabla_X(\alpha(Y)) 
= \nabla_X(c(Y \otimes \alpha)) 
= c(\nabla_X( Y \otimes \alpha))\\
& = & c(\nabla_X Y \otimes \alpha + Y \otimes \nabla_X \alpha ) 
= \alpha(\nabla_X Y) + (\nabla_X\alpha)(Y).
\end{eqnarray*}
This implies that 
\[
(\nabla_X\alpha)(Y) = X(\alpha(Y)) - \alpha(\nabla_X Y). 
\]

By (ii) the covariant derivative $\nabla_X$ along $X$ on $(r,s)$ tensors is
uniquely determined by the covariant derivative on $(1,0)$ tensors (vector fields) and 
$(0,1)$ tensors (1-forms). In particular, if
$T$ is a $(0,s)$-tensor and $Y_1, \ldots, Y_s \in \fX(M)$ then 
\begin{align*}
\nabla_XT(Y_1, \ldots, Y_s) &=  X(T(Y_1, \ldots, Y_s)) - \sum_{i=1}^s T(Y_1, \ldots, Y_{i-1}, \nabla_X Y_i, Y_{i+1}, \ldots, Y_s). 
\end{align*}
\end{proof}

Recall that the Lie derivative behaved similarly. In particular, we had 
\[
L_XT(Y_1, \ldots, Y_s) = X(T(Y_1, \ldots, Y_s)) - \sum_{i=1}^s T(Y_1, \ldots, L_XY_i, \ldots, Y_s). 
\]
This definition does not depend on the connection. However, the definition of $\nabla_XT$ does. 

\begin{remark}
Geometrically, the Lie derivative $L_X$ is the derivative of the pullback of a tensor under a flow $\phi_t$ of a vector field $X$. Also, there is a geometric interpretation of $\nabla_X$. We take an integral curve $\gamma$ of $X$ and we look at $\frac{D}{dt}T(\gamma(t))|_{t=0}$. 

The map $X \mapsto \nabla_XT$ is $C^\infty(M)$-linear in $X$, but the map $X \mapsto L_XT$ is $\mathbb{R}$-linear but not $C^\infty(M)$-linear in $X$. 

We may view $\nabla$ as a map 
\[
\nabla : C^{\infty}(M, T_s^r M) \to C^\infty(M, T_{s+1}^rM)
\]
by the map $T \mapsto \nabla T$ where 
\[
\nabla T(X_1, \ldots, X_{s+1}) = (\nabla_{X_{s+1}}T)(X_1, \ldots, X_s).
\]
\end{remark}


On a coordinate neighborhood $U$, let $\Gamma_{ij}^k\in C^\infty(U)$ be defined by 
\[
\boxed{\nabla_{\frac{\partial}{\partial x_i}}\frac{\partial}{\partial x_j} = \Gamma_{ij}^k \frac{\partial}{\partial x_k} }
\]
(The right hand side is a sum over $k$. We will continue to use this summation convention.) 
\[
(\nabla_{\frac{\partial}{\partial x_i}}dx_j)(\frac{\partial}{\partial x_k}) =
\frac{\partial}{\partial x_i}\Big(dx_j(\frac{\partial}{\partial x_k})\Big) 
- dx^j\left(\Gamma_{ik}^l \frac{\partial}{\partial x_l} \right) = - \Gamma_{ik}^j.
\]
So we find that 
\[
\boxed{\nabla_{\frac{\partial}{\partial x_i} }dx^j = -\Gamma_{ik}^j dx^k}
\]
If $T$ is an $(r,s)$ tensor, then on $U$ we can write 
\[
T = T^{i_1 \ldots i_r}_{j_1 \cdots j_s} \frac{\partial}{\partial x_{i_1}} \otimes \cdots \otimes \
\frac{\partial}{\partial x_{i_r}} \otimes dx^{j_1} \otimes \cdots \otimes dx^{j_s}.
\]
On $U$ we may write 
\[
\nabla T  =(\nabla T)_{j_1 \cdots j_{s+1}}^{i_1 \cdots i_r} \frac{\partial}{\partial x_{i_1}} \otimes \cdots \otimes 
\frac{\partial}{\partial x_{i_r}} \otimes dx^{j_1} \otimes \cdots \otimes dx^{j_{s+1}}.
\]
Our goal is to find $(\nabla T)_{j_1 \cdots j_{s+1}}^{i_1 \cdots i_r}$. We introduce the notation 
\[
T_{j_1 \cdots j_s, j_{s+1}}^{i_1 \cdots i_r} = (\nabla T)_{j_1 \cdots j_{s+1}}^{i_1 \cdots i_r} = 
(\nabla_{\frac{\partial}{\partial x_{j_{s+1}}}}T)^{i_1 \cdots i_r}_{j_1 \cdots j_s}.
\]
By this notation, we find that  
\[
\nabla_{\frac{\partial}{\partial x_k} }T = T_{j_1 \cdots j_s, k}^{i_1 \cdots i_r}
\frac{\partial}{\partial x_{i_1}} \otimes \cdots \otimes \frac{\partial}{\partial x_{i_r}} \otimes dx^{j_1} \otimes \cdots \otimes dx^{j_s}.
\]
On the other hand, we can apply Leibniz rule, and the above boxed equations to find that (see Assignment 12 Problem 4): 
\[
\boxed{T_{j_1 \cdots j_s, k}^{i_1 \cdots i_r} = \frac{\partial}{\partial x_k}(T^{i_1 \ldots i_r}_{j_1 \cdots j_s}) + \sum_{\alpha = 1}^r \Gamma_{kl}^{i_\alpha} T_{j_1 \cdots j_s}^{i_1 \cdots i_{\alpha - 1} l i_{\alpha + 1} \cdots i_r} - \sum_{\beta = 1}^s \Gamma_{k i_\beta}^{l}T_{j_1 \cdots j_{\beta - 1} l i_{\beta + 1} \cdots i_s }^{i_1 \cdots i_r}}
\]

\begin{proposition}
Let $\nabla$ be an affine connection on a Riemannian manifold $(M,g)$. Then $\nabla$ is compatible with $g$ if and only if $\nabla g = 0$. 
\end{proposition}

\begin{proof}
If $\nabla g = 0$, then $\nabla g(X,Y,Z) = 0$ for all $X,Y, Z \in \mathfrak{X}(M)$. But this implies that 
$$
0 = \nabla g(X,Y,Z) = (\nabla_Z g)(X,Y) = Z(g(X,Y)) - g(\nabla_Z X,Y) - g(X, \nabla_Z Y),
$$
which implies that $\nabla$ is compatible with $g$. This argument is reversible. 
\end{proof}

\begin{proposition}
Let $\nabla$ be an affine connection. Then $\nabla$ is symmetric (that is, $\nabla_XY - \nabla_YX = [X,Y]$) if and only if for any $1$-form $\alpha$ on $M$ and any vector fields $X,Y \in \mathfrak{X}(M)$, we have 
\[
(d\alpha)(X,Y) = (\nabla \alpha)(Y,X) - (\nabla \alpha)(X,Y).
\]
\end{proposition}

\begin{proof}
We have 
\[
d\alpha(X,Y) = X(\alpha(Y)) - Y(\alpha(X)) - \alpha([X,Y])
\]
and 
\[
(\nabla \alpha)(Y,X) = (\nabla_X\alpha)(Y) = X(\alpha(Y)) - \alpha(\nabla_X Y).
\]
The claim now follows easily. 
\end{proof}

Let $\nabla$ be the Levi-Civita connection on $(M,g)$. For a smooth function $f$, we get a one-form $\nabla f \in \Omega^1(M)$, defined by 
\[
(\nabla f)(X) = \nabla_X f = X(f),  
\]
so 
\[
\boxed{\nabla f = df.}
\]
In particular, we find that 
\[
\boxed{df = f_{,i}dx^i \hspace{10mm} f_{,i} = \frac{\partial f}{\partial x_i}}
\]

\bigskip

\noindent
{\bf \large Gradient, Divergence, Hessian, and Laplacian}

\begin{definition}
For a smooth function $f \in C^\infty(M)$, we define a vector field $\text{grad}(f) \in \mathfrak{X}(M)$, called 
the {\em gradient of $f$},  by the rule 
\[
\langle \text{grad}(f), X \rangle = df(X).
\]
\end{definition}
Write $\text{grad}(f)=\text{grad}(f)^j\frac{\partial}{\partial x_j}$. Then
$$
f_{,j} = \frac{\partial f}{\partial x_j}= df(\frac{\partial}{\partial x_j})=  \langle \text{grad}(f),\frac{\partial}{\partial x_j}\rangle
= \text{grad}(f)^i g_{ij} 
$$
Therefore,  
\[
\boxed{\text{grad} f = {f_,}^i \frac{\partial}{\partial x_i} \quad \quad  {f_,}^i = f_{,j}g^{ij} = \frac{\partial f}{\partial x_j}g^{ij}.}
\]

\begin{definition}
For a vector field $Y$ on $M$, we define a smooth function $\text{div}Y$, called the {\em divergence of $Y$} by the rule 
\[
\text{div}Y = c(\nabla Y)
\]
where $c$ denotes contraction. 
\end{definition}

Write $Y=Y^i \frac{\partial}{\partial x_i}$. Then
$$
\nabla Y = {Y^i}_{,j} \frac{\partial}{\partial x_i}\otimes dx_j,\quad \quad {Y^i}_{,j} =\frac{\partial Y^i}{\partial x_j} + \Gamma^i_{jk}Y^k. 
$$
Therefore,
\[
\boxed{\text{div}Y = {Y^i}_{,i} = \frac{\partial Y^i}{\partial x_i} + \Gamma_{ik}^i Y^k}
\]
where  $Y = Y^i \frac{\partial}{\partial x_i}$. 

\begin{definition}
For a smooth function $f$, we define a $(0,2)$-tensor, called the {\em Hessian of $f$} by the rule 
\[
\text{Hess}f = \nabla \nabla f = \nabla df = \nabla(f_{,i}dx^i) = f_{,ij} dx^i \otimes dx^j.
\]
\end{definition}

We compute that 
\[
f_{,ij} = \frac{\partial f_{,i}}{\partial x_j} - \Gamma_{ji}^k f_{,k} = \frac{\partial^2f}{\partial x_j \partial x_i} - \Gamma_{ji}^k \frac{\partial f}{\partial x_k} = f_{,ji}.
\]
It follows that $\text{Hess}f$ is a symmetric $(0,2)$-tensor. 

We also compute that 
\begin{eqnarray*}
&& \text{Hess}(f)(X,Y) = (\nabla df)(X,Y) = (\nabla_Ydf)(X)\\
&&\quad = Y(df(X)) - df(\nabla_Y X) 
= Y(X(f)) - (\nabla_YX)(f).
\end{eqnarray*}

\begin{definition}
For a smooth function $f$, we define a smooth function $\Delta f$, called the {\em Laplacian of $f$}, by the rule 
\[
\Delta f = \text{div}(\text{grad} f)= \text{div}({f_,}^i \frac{\partial}{\partial x_i}) = {f_{,i}}^i = f_{,ij}g^{ij}. 
\]
\end{definition}

Locally the Laplacian is given by 
\[
\Delta f = g^{ij} \left(\frac{\partial^2f}{\partial x_i\partial x_j} - \Gamma_{ij}^k \frac{\partial f}{\partial x_k}\right).
\]

In \emph{normal} coordinates at $p \in M$, we know that $g_{ij}(p) = g^{ij}(p)=  \delta_{ij}$ and $\Gamma_{ij}^k(p) = 0$. So we can compute that 
\begin{align*}
(\text{grad}f)(p) &= \sum_{i=1}^n \frac{\partial f}{\partial x_i}(p) \frac{\partial}{\partial x_i}|_{p} \\
(\text{div} Y)(p) &= \sum_{i=1}^n \frac{\partial Y^i}{\partial x_i}(p) \\
(\text{Hess}f)(p) &= \sum_{i,j} \frac{\partial^2 f}{\partial x_i \partial x_j}(p) dx^i|_p \otimes dx^j|_p \\
(\Delta f)(p) &= \sum_{i=1}^n \frac{\partial^2 f}{\partial x_i^2}(p)
\end{align*}





\section{Monday, December 7, 2015} 


\noindent
{\bf \large Curvature of a connection on a vector bundle}

Let $E \to M$ be a smooth vector bundle. Recall that a connection $\nabla$ on $E$ is an $\bR$-linear map 
\begin{align*}
\nabla : \Omega^0(M,E) &\to \Omega^1(M,E) \\
s &\mapsto \nabla s
\end{align*}
such that for $f\in C^\infty(M)$ and $s\in \Omega^0(M,E)$,
$$
\nabla(fs)= df \otimes s + f\nabla s.
$$

Given a vector field $X\in \fX(M)$ and a section $s\in \Omega^0(M,E)$, write $\nabla_X s = \nabla s(X) \in \Omega^0(M,E)$.
For vector fields $X,Y \in \fX(M)$, define 
\[
R_\nabla(X,Y) : C^\infty(M,E) \to C^\infty(M,E)
\]
by the rule 
\[
R_\nabla(X,Y)s = \nabla_X \nabla_Y s  - \nabla_X \nabla_Y s - \nabla_{[X,Y]}s.
\]
Then 
\begin{enumerate}
\item[(i)] $R_\nabla(X,Y) = - R_{\nabla}(Y,X)$
\item[(ii)] $R_\nabla(X,Y)$ is $C^\infty(M)$-linear in $X,Y,$ and $s$. 
\end{enumerate}
We may therefore view $R_\nabla$ as an element of 
\[
 \Omega^2(M, \End E ) = C^\infty(M, \Lambda^2 T^*M \otimes \End E) 
\]
We call $R_\nabla$ the curvature of $\nabla$. 


For a smooth map $f : N \to M$, we get a pullback connection $f^* \nabla$  on the pullback bundle $f^*E \to N$. Then the curvature
$R_{f^*\nabla}$ of the pull back connection $f^*\nabla$ is the pull back of the curvature $R_\nabla$ of $\nabla$: 
\[
R_{f^*\nabla} = f^*R_\nabla \in \Omega^2(N, \text{End} f^*E)
\]


\bigskip

\noindent
{\bf \large Jacobi Fields}

Let $(M,g)$ be a Riemannian manifold. A Jacobi field $J(t)$ is a smooth vector field along a 
geodesic $\gamma : I \to M$ which arises in the following way. Consider a smooth map
\begin{align*}
f : (-\epsilon, \epsilon) \times [0,a] &\to M \\
(s,t) &\mapsto f_s(t) = f(s,t)
\end{align*}
(which we think of as a family of geodesics parametrized by $s \in (-\epsilon, \epsilon)$) such that for any $s \in (-\epsilon, \epsilon)$, the map $f_s : [0,a] \to M$ is a geodesic and such that $f_0 = \gamma$. We then set 
\[
J(t) = \frac{\partial f}{\partial s}(0,t). 
\]


\begin{lemma}
Let $A = (-\epsilon, \epsilon) \times [0,a] \subset \mathbb{R}^2$. Let $f : A \to M$ be any smooth map. 
Then $\frac{\partial}{\partial s}, \frac{\partial}{\partial t}$ are global vector fields on $A$. Recall that we have defined
$$
\frac{\partial f}{\partial s} := f_*(\frac{\partial}{\partial s}),\quad
\frac{\partial f}{\partial t} := f_*(\frac{\partial}{\partial s})\in C^\infty(A, f^*TM).
$$
Let $\nabla$ be the Levi-Civita connection on $(M,g)$ and let $D = f^*\nabla$ be the pullback connection on 
$f^*TM$. Then 
\begin{equation}\label{eqn:f-symmetry}
\frac{D}{\partial s}\frac{\partial f}{\partial t}- \frac{D}{\partial t}\frac{\partial f}{\partial s} =0 
\end{equation}
\begin{equation}\label{eqn:f-R}
\frac{D^2}{dt^2} \frac{\partial f}{\partial s} - \frac{D}{ds} \left( \frac{D}{dt} \frac{\partial f}{\partial t}\right) + R(\frac{\partial f}{\partial t}, \frac{\partial f}{\partial s}) \frac{\partial f}{\partial t} = 0 
\end{equation}
\end{lemma}
\begin{proof}
By the symmetric of the pullback connection, we have 
\begin{equation}\label{eqn:A-symmetry}
0 = f_*[\frac{\partial}{\partial s}, \frac{\partial}{\partial t}] = D_{\frac{\partial}{\partial s}}f_*\frac{\partial}{\partial t} - D_{\frac{\partial}{\partial t}} f_*\frac{\partial}{\partial s}.
\end{equation}
which can be rewritten as \eqref{eqn:f-symmetry}.

We also have 
\begin{equation}\label{eqn:A-R}
D_{\frac{\partial}{\partial t}} D_{\frac{\partial}{\partial s}} f_* \frac{\partial}{\partial t} 
- D_{\frac{\partial}{\partial s}} D_{\frac{\partial}{\partial t}} f_* \frac{\partial}{\partial t} 
+ D_{[\frac{\partial}{\partial s},\frac{\partial}{\partial t}]} f_* \frac{\partial}{\partial t} = f^*R(\frac{\partial}{\partial s},\frac{\partial}{\partial t}) (f_*\frac{\partial}{\partial t}).
\end{equation}
where $[\frac{\partial}{\partial s}, \frac{\partial}{\partial t}]=0$. By \eqref{eqn:A-symmetry} and \eqref{eqn:A-R},
$$
\frac{D^2}{dt^2} \frac{\partial f}{\partial s} - \frac{D}{ds} \left( \frac{D}{dt} \frac{\partial f}{\partial t}\right) = R(\frac{\partial f}{\partial s}, \frac{\partial f}{\partial t}) \frac{\partial f}{\partial t},
$$
which is equivalent to \eqref{eqn:f-R}. 
\end{proof}


We now note that: $f_s : [0,a] \to M$ is a geodesic for any $s\in (-\epsilon,\epsilon)$  if and only if 
\[
\frac{D}{dt}\frac{\partial f}{\partial t}(s,t) = 0 \hspace{5mm} \text{for any $s,t$}. 
\]
Therefore, for a family of geodesics $f_s$, \eqref{eqn:f-R} becomes
\[
\frac{D^2}{dt^2} \frac{\partial f}{\partial s} + R(\frac{\partial f}{\partial t}, \frac{\partial f}{\partial s}) \frac{\partial f}{\partial t} = 0
\]
In particular, for $s = 0$, if we set 
\[
\frac{\partial f}{\partial t}(0,t) = \gamma'(t) \hspace{5mm} \text{and} \hspace{5mm} \frac{\partial f}{\partial s}(0,t) = J(t),
\]
then we see that 
\begin{equation}\label{eqn:Jacobi}
\boxed{\frac{D^2J}{dt^2} + R(\gamma', J)\gamma' = 0.}
\end{equation}

\begin{definition}
A vector field $J(t)$ along a geodesics $\gamma : [0,a] \to M$ is called a {\em Jacobi field} if it satisfies 
the Jacobi equation \eqref{eqn:Jacobi}. 
\end{definition}

\begin{proposition}\label{jacobi}
Let $\gamma : [0,a] \to M$ be a geodesic, with $\gamma(0) = p$ and $\gamma'(0) = v \in T_pM$ (so that $\gamma(t)=\exp_p(tv)$. Then 
\begin{enumerate}
\item[(a)] For any $u,w \in T_pM$, there is a unique Jacobi field $J(t)$ along $\gamma(t)$ with $J(0) = u$ and $\frac{DJ}{dt}(0) = w$. 
\item[(b)] If $J(t)$ is a Jacobi field along $\gamma(t)$, then there is a smooth map $f : (-\epsilon, \epsilon) \times [0,a] \to M$ written $f(s,t) = f_s(t)$ such that 
\begin{enumerate}
\item[(i)] for each $s \in (-\epsilon, \epsilon)$, the map $f_s : [0,a] \to M$ is a geodesic,
\item[(ii)] $f_0 = \gamma$, and 
\item[(iii)] $\frac{\partial f}{\partial s}(0,t) = J(t)$.  
\end{enumerate}
\end{enumerate}
\end{proposition}

\begin{example}
In Proposition \ref{jacobi}, suppose that $(M,g) = (\mathbb{R}^n, g_0)$ is the Euclidean space,
then $\gamma(t) = p + tv$. The Jacobi equation
is reduced to $\frac{D^2J}{dt^2}=0$. The unique solution in part (a)  is given by
$J(t) = u + tw$, and the smooth map $f$ in part (b) can be given by
$f(s,t) = (p + su) + t (v+ sw)$. 
\end{example}

\begin{proof}[Proof of Proposition \ref{jacobi}] $ $\\
(a) Let $e_1, \ldots, e_n$ be an orthonormal basis of $T_pM$ and let $e_i(t)$ be parallel transport of $e_i$ along $\gamma(t)$, that is, $e_i(t)$ is the unique parallel vector field along $\gamma(t)$ such that $e_i(0) = e_i$. Then for any $t \in [0,a]$, 
we see that $\{e_i(t)\}$ is an orthonormal basis of $T_{\gamma(t)}M$. If $J(t)$ is a smooth vector field along $\gamma(t)$, then we may write 
\[
J(t) = \sum_{i=1}^n f_i(t)e_i(t)
\]
for some smooth $f_i : [0,a] \to \mathbb{R}$. We see that $J(t)$ is a Jacobi field along $\gamma(t)$ if and only if the Jacobi equation holds, 
which holds if and only if 
\[
\sum_{i=1}^n f_i''(t)e_i(t) + \sum_{j=1}^n f_j(t) R(\gamma'(t), e_j(t))\gamma'(t) = 0.
\]
Taking inner product of the above equation and $e_i$, we see that the above equation is equivalent to
$$
f_i''(t) +  \sum_{j=1}^n f_j(t) R(\gamma'(t), e_j(t),\gamma'(t),e_i(t)) =0,\quad i=1,\ldots,n.
$$

Define $a_{ij}(t)\in C^\infty([0,a])$ by 
\[
a_{ij}(t) = R(\gamma'(t), e_j(t), \gamma'(t), e_i(t)).
\]
Then $a_{ij}(t)= a_{ij}(t)$. We see that $J(t)$ is a Jacobi field along $\gamma(t)$ if and only if 
\[
f_i''(t) + \sum_{j=1}^n a_{ij}(t) f_j(t) = 0 \quad \textup{for }i=1,\ldots,n
\]
if and only if 
\[
\frac{d^2}{dt^2}\vec{f}(t) + A(t)\vec{f}(t) = 0
\]
where $\vec{f}(t)=\displaystyle{ \left[\begin{array}{c} f_1(t)\\ \vdots \\ f_n(t) \end{array}\right]}$, and 
$A(t)$ is the matrix $(a_{ij}(t))$. We also have  
\[
\begin{cases}
J(0) = u \\
\frac{D J}{dt}(0) = w
\end{cases}\Leftrightarrow
\begin{cases}
\vec{f}(0) = \vec{u}   \\
\frac{d\vec{f}}{dt}(0) =\vec{w}
\end{cases}
\]
where
$$
\vec{u}=  \left[\begin{array}{c} \langle u,e_1\rangle \\ \vdots \\ \langle u,e_n\rangle \end{array}\right],\quad
\vec{w}=  \left[\begin{array}{c} \langle w,e_1\rangle \\ \vdots \\ \langle w,e_n\rangle \end{array}\right]
$$
The uniqueness of ODE's implies there is a unique solution satisfying these conditions. 


\smallskip

\noindent
(b) (cf. \cite{dC} Chapter 5 Exercise 2) \\
(Idea of the proof: set $u:=J(0), w:=\frac{DJ}{dt}(0)\in T_pM$.  When $(M,g)=(\bR^n,g_0)$, we have
$f(s,t)= (p+su)+t(v+sw) = \exp_{p+su}(t(v+sw))$. This motivates the construction of $f(s,t)$ in the general case:
$f(s,t)=\exp_{\lambda(s)}(t(v(s)+sw(s)))$, where
$\lambda(s)=\exp_p(su)$ and $v(s), w(s)\in T_{\lambda(s)}M$ are the parallel tranports of $v,w\in T_pM$ along
the curve $\lambda(s)$.)



Let $J(t)$ be a Jacobi field along $\gamma(t)=\exp_p(tv)$. Let $u:=J(0), w:=\frac{DJ}{dt}(0)\in T_pM$.
Define $\lambda:(-\epsilon,\epsilon)\to M$ by $\lambda(s)=\exp_p(su)$. Then $\lambda(0)=0$ and $\lambda'(0)=u$.
Let $v(s)$ (resp. $w(s)$) be the unique parallel vector field along the curve $\lambda(s)$ such that
$v(0)=v$ (resp. $w(0)=w$). Define a smooth map $f:(-\epsilon,\epsilon)\times [0,a]\to M$ by
$$
f(s,t)=\exp_{\lambda(s)}\big(t(v(s)+sw(s))\big).
$$ 
Then 
\begin{enumerate}
\item[(i)] For any $s\in (-\epsilon, \epsilon)$, $f_s:[0,a]\to M$ defeind by $f_s(t)=f(s,t)$ is the unique geodesic
with $f_s(0)=\lambda(s)$ and $f_s'(0)= v(s)+sw(s)$.  
\item[(ii)] $f_0(t)=\exp_p(tv)=\gamma(t)$.
\item[(iii)] $\bar{J}(t):= \frac{\partial f}{\partial s}(0,t)$ is a Jacobi field along $\gamma(t)$.
\end{enumerate}
It remains to show that $\bar{J}(0)=u$ and $\frac{D\bar{J}}{dt}(0)=w$ ($\Rightarrow \bar{J}(t)=J(t)$).
$$
f(s,0)=\lambda(s)\Rightarrow \bar{J}(0)= \frac{\partial f}{\partial s}(0,0) = \lambda'(0)=u.
$$
$$
\frac{\partial f}{\partial t}(s,0) = f_s'(0)= v(s)+ s w(s) \Rightarrow \frac{D}{\partial s}\frac{\partial f}{\partial t}(s,0)=w(s).
$$
$$
\frac{D \bar{J}}{dt}(0,0)=  \frac{D}{\partial t}\frac{\partial f}{\partial s}(0,0)=
\frac{D}{\partial s}\frac{\partial f}{\partial t}(0,0)= w(0)=w. 
$$
\end{proof}


We now consider the special case $u=0$ in part (b) of the above proof. 
Say that $J(t)$ is a Jacobi field along $\gamma(t)=\exp_p(tv)$ such that $J(0) = 0$ and $\frac{DJ}{dt}(0) = w$. 
Applying the construction from part (b) of the proof, we see that $\lambda(s)=p$ (the constant map) and 
$f(s,t) = \exp_p(t(v + sw))$. We see that 
\[
\frac{\partial f}{\partial s}(s,t) = (d \exp_p)_{t(v + sw)}(tw)
\]
and hence 
\[
J(t) = (d \exp_p)_{tv}(tw).
\]

\begin{proposition}
Let $\gamma : [0,a] \to M$ be a geodesic with $\gamma(0) = p$ and $\gamma'(0) = v \in T_pM$ (so that
$\gamma(t)=\exp_p(tv)$).
Let  $J(t)$ be a Jacobi field along $\gamma(t)$ such that $J(0) = 0$ and $\frac{DJ}{dt}(0) = w$. Then 
\[
J(t) = (d \exp_p)_{tv}(tw)
\]
for $t \in [0,a]$. 
\end{proposition}

\begin{lemma}\label{jacobi-tangent}
Let $\gamma : [0,a] \to M$ be a geodesic and $J(t)$ a Jacobi field along $\gamma(t)$. Then
\[
\langle J(t), \gamma'(t) \rangle = \langle J(0), \gamma'(0) \rangle + t \langle J'(0), \gamma'(0) \rangle
\]
where $J'(0) = \frac{DJ}{dt}(0)$. 
\end{lemma}

\begin{proof} Define a smooth function $f:[0,a]\to \bR$ by 
$f(t) = \langle J(t), \gamma'(t) \rangle$. 
The lemma says $f(t)=f(0)+ f'(0)t$. It suffices to show that $f''(t)=0$.
 
Recall that because $\gamma$ is a geodesic, we have $\frac{D}{dt}\gamma'(t) = 0$. 
Let $J'=\frac{DJ}{dt}$ and $J''=\frac{D^2J}{dt^2}$. Then
\begin{eqnarray*}
f' &=& \langle J', \gamma'(t) \rangle\\
f'' &=& \langle J'', \gamma' \rangle =  - \langle R(\gamma', J)\gamma', \gamma' \rangle = R(\gamma', J, \gamma', \gamma') = 0,
\end{eqnarray*}
where we use the Jacobi equation $J''+R(\gamma',J)\gamma=0$. 
\end{proof}

\begin{remark}
Note that $\gamma'(t)$ and $t\gamma'(t)$ are Jacobi fields along $\gamma(t)$ (by the Jacobi equation). 
By Lemma \ref{jacobi-tangent}, for any Jacobi field $J(t)$ along $\gamma(t)$, we have 
\[
J(t) = \left(\langle J(0), \gamma'(0) \rangle + t\langle J'(0), \gamma'(0) \rangle \right) \frac{\gamma'(t)}{|\gamma'(0)|^2} + J^\perp(t)
\]
where $J^\perp(t)$ is also a Jacobi field along $\gamma(t)$ and 
\[
\langle J^\perp, \gamma' \rangle = 0.
\]
\end{remark}






\section{Wednesday, December 9, 2015}


\noindent 
{\bf \large Jacobi fields on a manifold with constant sectional curvature}

Let $(M,g)$ be a Riemannian manifold with constant sectional curvature $K$.
Let $\gamma:[0,a]\to M$ be a {\em normalized} geodesic (i.e. $|\gamma'|=1$).
Let $p=\gamma(0)\in M$ and $v=\gamma'(0)\in T_p M$. Let $J(t)$ be a Jacobi field
along $\gamma(t)$ such that 
$$
J(0)=0, \quad \frac{DJ}{dt}(0)=w,\quad \langle w,v\rangle =0.
$$
Then $\langle J(t),\gamma'(t) \rangle =0$ for all $t\in [0,a]$. For any smooth
vector field $V(t)$ along $\gamma(t)$,
$$
\langle R(\gamma',J)\gamma',V\rangle = K(\langle \gamma',\gamma'\rangle \langle J,V\rangle
-\langle \gamma',V\rangle \langle \gamma', J\rangle) =\langle KJ,V\rangle.
$$
Therefore $R(\gamma',J)\gamma' = KJ$. So $J$ satisfies
$$
\frac{D^2}{dt^2} + KJ=0.
$$
Let $J(t)=f(t) w(t)$, where $f$ is a smooth function on $[0,a]$ and
$w(t)$ is the unique parallel vector field along $\gamma(t)$ with $w(0)=w$. Then
$$
\frac{D^2 J}{dt^2} + KJ =0,\quad J(0)=0,\quad \frac{DJ}{dt}(0)=w,
$$
are equivalent to
$$
f'' + Kf=0,\quad f(0)=0,\quad f'(0)=0.
$$
$$
f(t)=\begin{cases}
\frac{ \sin(\sqrt{K} t)}{\sqrt{K}}, & K>0;\\
t, & K=0;\\
\frac{ \sinh(\sqrt{-K} t)}{\sqrt{-K}}, & K<0.
\end{cases}
$$

Therefore, the unique Jacobi field $J(t)$ along $\gamma(t)$ with $J(0)=0$, $\frac{DJ}{dt}(0)=w$,
where $\langle w,\gamma'(0) \rangle =0$, is given by 
$$
J(t)=\begin{cases}
\frac{ \sin(\sqrt{K} t)}{\sqrt{K}}w(t), & K>0,\\
t w(t), & K=0,\\
\frac{ \sinh(\sqrt{-K} t)}{\sqrt{-K}} w(t), & K<0,
\end{cases}
$$
where $w(t)$ is the unique parallel vector field along $\gamma(t)$ with $w(0)=w$.

Similarly,
the unique Jacobi field $J(t)$ along $\gamma(t)$ with $J(0)=u$, $\frac{DJ}{dt}(0)=0$,
where $\langle u,\gamma'(0) \rangle =0$, is given by 
$$
J(t)=\begin{cases}
\cos(\sqrt{K} t)u(t), & K>0,\\
u(t), & K=0,\\
\cosh(\sqrt{-K} t) u(t), & K<0,
\end{cases}
$$
where $u(t)$ is the unique parallel vector field along $\gamma(t)$ with $u(0)=u$.


\bigskip




\noindent
{\bf \large Taylor Expansion of $g_{ij}$ in local coordinates}

\begin{proposition}\label{J-norm}
Let $(M,g)$ be a Riemannian manifold and $p$ a point $M$. Let $\gamma : [0,a] \to M$ be a geodesic with $\gamma(0) = p$ and $\gamma'(0) = v$. (This means that $\gamma(t) = \exp_p(tv)$.) Let $J(t)$ be a Jacobi field along $\gamma(t)$ with $J(0)= 0$ and $\frac{DJ}{dt}(0) = w \in T_pM$. (This means that $J(t) = (d \exp_p)_{tv}(tw)$.) Then 
\begin{align*}
|J(t)|^2 &= \langle w, w \rangle t^2 - \frac{1}{3} R(v,w,v,w)t^4 - \frac{1}{6} (\nabla_v R)(v,w,v,w) t^5 \\
&\;\;\; + \left[\frac{2}{45} \langle R(v,w)v, R(v,w)v \rangle - \frac{1}{20} (\nabla_v \nabla_v R)(v,w,v,w)\right] t^6 + o(t^6).
\end{align*}
\end{proposition}

\begin{corollary}
If $v$ and $w$ are orthonormal, then 
\[
|J(t)|^2 = t^2 - \frac{1}{3} K(p,\sigma)t^4 + o(t^4)
\]
where $\sigma$ is the span of $v$ and $w$. As a result, we also have (when $t>0$)
\[
|J(t)| = t - \frac{1}{6}K(p, \sigma)t^3 + o(t^3).
\]
\end{corollary}

We now prove the proposition. 

\begin{proof}[Proof of Proposition \ref{J-norm}]
Let $f = \langle J, J \rangle$. Need to compute $f^{(k)}(0)$ for $0 \le k \le 6$. 

Note that 
\begin{align*}
f' &= 2 \langle J', J \rangle \\
f'' &= 2\langle J'', J \rangle + 2 \langle J', J' \rangle \\
f^{(3)} &= 2 \langle J^{(3)}, J  \rangle + 6\langle J'', J' \rangle \\
f^{(4)} &= 2 \langle J^{(4)}, J \rangle + 8 \langle J^{(3)}, J' \rangle + 6 \langle J'', J'' \rangle \\
f^{(5)} &= 2 \langle J^{(5)}, J \rangle + 10 \langle J^{(4)}, J' \rangle + 20 \langle J^{(3)}, J'' \rangle \\
f^{(6)} &=2 \langle J^{(6)}, J \rangle + 12 \langle J^{(5)}, J' \rangle + 30 \langle J^{(4)}, J'' \rangle + 20 \langle J^{(3)}, J^{(3)} \rangle. 
\end{align*}

We now know that $J(0) = 0$ and $J'(0) = w$. We need to compute $J^{(k)}(0)$ for $2 \le k \le 5$. But we have the Jacobi equation, so we know that 
\begin{align*}
& J''= - R(\gamma', J) \gamma' \Rightarrow J''(0) \\
& J^{(3)} = - R'(\gamma', J)\gamma' - R(\gamma', J')\gamma' \Rightarrow J^{(3)}(0) = -R(v,w)v \\
& J^{(4)} = -R''(\gamma', J)\gamma' - 2R'(\gamma', J')\gamma' - R(\gamma', J'')\gamma' 
\Rightarrow J^{(4)}(0) = -2(\nabla_v R)(v,w)v \\
J^{(5)} &= -R'''(\gamma', J)\gamma' - 3 R''(\gamma', J')\gamma' - 3 R'(\gamma', J'')\gamma' - R(\gamma', J^(3))\gamma' \\
&\Rightarrow J^{(5)}(0)= -3(\nabla_v\nabla_vR)(v,w)v + R(v,R(v,w)v)v
\end{align*}
We then plug these results into the above expressions for $f^{(k)}$ to find 
\begin{align*}
f(0) &= 0 \\
f'(0) &= 0 \\
f''(0) &= 2 \langle w, w \rangle \\
f^{(3)}(0) &= 0 \\
f^{(4)}(0) &= -8 \langle R(v,w)v, w \rangle \\
f^{(5)}(0) &= -20 \langle (\nabla_vR)(v,w)v, w \rangle \\
f^{(6)}(0) &= 12 \langle -3(\nabla_v\nabla_vR)(v,w)v + R(v, R(v,w)v)v, w \rangle + 20 \langle R(v,w)v, R(v,w)v \rangle \\
&= -36 \langle (\nabla_v\nabla_vR)(v,w)v, v \rangle + 32 \langle R(v,w)v, R(v,w)v \rangle.
\end{align*}
Using the usual Taylor expansion, we find the desired result. 
\end{proof}

Proposition \ref{J-norm} implies
\begin{align*}
 & \langle (d \exp_p)_{tv}(u) , (d\exp_p)_{tv}w \rangle \\
= &\langle u, w \rangle - \frac{1}{3} R(v,u,v,w) t^2 - \frac{1}{6} (\nabla_v R)(v,u,v,w) t^3 \\
& \quad + \left[ \frac{2}{45} \langle R(v,u)v, R(v,w)v \rangle - \frac{1}{20} (\nabla_v\nabla_vR)(v,u,v,w) \right]t^4 + O(t^5)
\end{align*}


Let $\{e_1,\ldots, e_n\}$ be an orthonormal basis of $T_pM$. Then
\begin{align*}
&\langle (d \exp_p)_{v}(e_i) , (d\exp_p)_{v}e_j \rangle \\
=& \langle e_i, e_j \rangle - \frac{1}{3} R(v,e_i,v,e_j)  - \frac{1}{6} (\nabla_v R)(v,e_i,v,e_j)  \\
&\quad + \left[ \frac{2}{45} \langle R(v,e_i)v, R(v,e_j)v \rangle - \frac{1}{20} (\nabla_v\nabla_vR)(v,e_i,v,e_j) \right] + O(|v|^5)
\end{align*}

Suppose that $B_\epsilon(p)$ is a geodesic ball with center $p$ and radius $\epsilon>0$. 
Then
$$
q=\exp_p(\sum_{k=1}^n x_k e_k) \in B_\epsilon(q).
$$
where $(x_1,\ldots, x_n)$ are the normal coordinates determined by $(e_1,\ldots, e_n)$.
Then
$$
\frac{\partial}{\partial x_i}\Big|_q = (d\exp_p)_{\sum_{k=1}^n x_k e_k}(e_i).
$$
So
$$
g_{ij}(x_1,\ldots, x_n )= \langle (d\exp_p)_{\sum_{k=1}^n x_k e_k}(e_i), (d\exp_p)_{\sum_{l=1}^n x_l e_l} (e_j)\rangle
$$

On $B_\epsilon(p)$, 
\[
\nabla R = \sum_{i,j,k,l,m} R_{ijkl,m} dx^i \otimes dx^j \otimes dx^k \otimes dx^l \otimes dx^m
\]
and 
\[
\nabla \nabla R =\sum_{i,j,k,l,m,r,s} R_{ijkl, rs} dx^i \otimes dx^j \otimes dx^k \otimes dx^l \otimes dx^r \otimes dx^s
\]
We obtain the following Taylor expansion of $g_{ij}$:
\[
\boxed{
\begin{aligned}
g_{ij}(x) &= \delta_{ij} - \frac{1}{3} \sum_{k,l} R_{ikjl}(p) x_k x_l
 - \frac{1}{6} \sum_{k,l,m} R_{ijkl,m}(p) x_kx_lx_m \\
&\;\;\; - \frac{1}{20} \sum_{k,l,r,s}R_{ikjl,rs}(p) x_kx_lx_r x_s + \frac{2}{45} 
\sum_{k,l,r,s,m} R_{iklm}(p)R_{jrsm}(p) x_kx_lx_rx_s + O(|x|^5)
\end{aligned} }
\]

\bigskip

\noindent
{\bf \large Taylor Expansion of $\sqrt{ \det(g_{ij})}$}

\smallskip

Let $g(x)= (g_{ij}(x))$. Then
$$
g(x)= I + g^{(2)}(x) + g^{(3)}(x)+g^{(4)}(x) + O(|x|^5)
$$
where $I$ is the $n\times n$ identity matrix.
$$
\sqrt{ \det(g(x)) }= \exp(\frac{1}{2} \mathrm{Tr}\log(g(x)))
$$
where
$$
\log(g(x))= g^{(2)}(x)+ g^{(3)}(x) + g^{(4)}(x)-\frac{1}{2} g^{(2)}(x)^2 + O(|x|^5).
$$
\begin{eqnarray*}
-\frac{1}{2} \left(g^{(2)}(x)^2\right)_{ij}
&=&-\frac{1}{18} \sum_{k,l,r,s,m} R_{ikml}(p) R_{jrms}(p) x_k x_l x_r x_s\\
&=& -\frac{1}{18} \sum_{k,l,r,s,m} R_{iklm}(p) R_{jrsm}(p) x_k x_l x_r x_s
\end{eqnarray*}
\begin{eqnarray*}
&& \mathrm{Tr}\log(g(x))= 
- \frac{1}{3} \sum_{k,l} R_{kl}(p) x_k x_l
 - \frac{1}{6} \sum_{k,l,m} R_{kl,m}(p) x_kx_lx_m \\
&& \quad \quad - \frac{1}{20} \sum_{k,l,r,s}R_{kl,rs}(p) x_kx_lx_r x_s - \frac{1}{90} 
\sum_{i,k,l,r,s,m} R_{iklm}(p)R_{irsm}(p) x_kx_lx_rx_s + O(|x|^5)
\end{eqnarray*}


\begin{eqnarray*}
&& \sqrt{ \det(g(x))} = 
1 - \frac{1}{6} \sum_{k,l} R_{kl}(p) x_k x_l
 - \frac{1}{12} \sum_{k,l,m} R_{kl,m}(p) x_kx_lx_m \\
&& \quad \quad \sum_{k,l,r,s} \Big( - \frac{1}{40} \sum_{k,l,r,s}R_{kl,rs}(p) - \frac{1}{180} 
\sum_{i,m} R_{iklm}(p)R_{irsm}(p) +\frac{1}{72} R_{kl}(p) R_{rs}(p)\Big) 
x_k x_l x_r x_s + O(|x|^5)
\end{eqnarray*}


\section{Monday, December 14, 2015} 




Let $S^2=\{ (x,y,z)\in \bR^3: x^2+y^2+z^2=1\}$ be the round sphere of radius 1, and let $p=(0,0,1)$ be the north pole. 
The exponential map  $\exp_p: T_p S^2\to S^2$ sends
a circle of radius $\rho>0$ centered at the origin to the circle 
$$
\{ (x,y,z)\in \bR^3: x^2+y^2=\sin^2\rho,\ z=\cos\rho\}.
$$
Let $(\rho,\theta)$ be the polar coordinates on $T_p S^2 =\bR^2$. Then 
$$
\exp_p^*(dx^2+dy^2+dz^2)= d\rho^2+ \sin^2\rho d\theta^2.
$$ 

More generally, given $K>0$, let $S^2(\frac{1}{\sqrt{K}}) =\{ (x,y,z)\in \bR^3: x^2+y^2+z^2=\frac{1}{K}\}$
be the round sphere of radius $\frac{1}{\sqrt{K}}$, which has constant sectional curvature $K>0$.
Let $p=(0,0,\frac{1}{\sqrt{K}})$ be the north pole.
The exponential map $\exp_p: T_P S^2(\frac{1}{\sqrt{K}})\to S^2(\frac{1}{\sqrt{K}})$
sends a circle of radius $\rho>0$ centered at the origin  to the circle  
$$
\{ (x,y,z)\in \bR^3: x^2+y^2=\frac{\sin^2(\sqrt{K}\rho)}{K},\ z=\frac{\cos(\sqrt{K}\rho)}{\sqrt{K}}\}.
$$ 
Let $(\rho,\theta)$ be the polar coordinates on $T_p S^2(\frac{1}{\sqrt{K}}) =\bR^2$. Then 
$$
\exp_p^*(dx^2+dy^2+dz^2)= d\rho^2+ \Big(\frac{\sin(\sqrt{K}\rho)}{\sqrt{K}}\Big)^2 d\theta^2. 
$$


Let $(M,g)$ be a Riemannnian manifold with constant sectional curvature $K$. 
Let $\gamma:[0,a]\to M$ be a normalized geodesic, and let $J(t)$ be a Jacobi field
along $\gamma(t)$ with $J(0)=0$, $\frac{DJ}{dt}(0)= w$, where $\langle w, \gamma'(0)\rangle$. 
Then 
$$
J(t)= f_K(t) w(t),
$$
where 
$$
f_K(t)=\begin{cases}
\frac{\sin(\sqrt{K} t)}{\sqrt{K}}, & K>0,\\
t, & K=0,\\
\frac{\sinh(\sqrt{-K} t)}{\sqrt{-K}}, & K<0,
\end{cases}
$$

Let $B_\delta(p)$ be the geodesic ball with center $p$ and radius $\delta>0$.
Define a $C^\infty$ map 
$$
F:(0,\delta)\times S^{n-1} \to B_\delta(p),\quad (\rho,v)\mapsto \exp_p(\rho v).
$$
Then
$$
dF_{(\rho,v)}: T_{(\rho,v)} \Big( (0,\delta)\times S^{n-1} \Big) = \bR \frac{\partial}{\partial \rho}\oplus T_v S^{n-1}
\to T_{\exp_p(\rho v)}M
$$
is given by 
\begin{eqnarray*}
dF_{(\rho,v)} (\frac{\partial}{\partial \rho}) &=& (d\exp_p)_{\rho v}(v)\\
dF_{(\rho,v)} (w) &=& (d\exp_p)_{\rho v}(\rho w)
\end{eqnarray*}
where $w\in T_v S^{n-1}=\{ w\in \bR^n: \langle v,w\rangle =0\}$. By Gauss's lemma,
\begin{eqnarray*}
\langle  (d\exp_p)_{\rho v}(v),  (d\exp_p)_{\rho v}(v)\rangle &=& \langle v, v\rangle =1,\\
 \langle (d\exp_p)_{\rho v}(v), (d\exp_p)_{\rho v}(\rho w)\rangle &=& \rho \langle v, w\rangle =0.
\end{eqnarray*}
We have 
$$
(d\exp_p)_{\rho v} (\rho w)= f_K(\rho) w(\rho v)
$$
where $w(\rho v)\in T_{\exp_p(\rho v)} M$ is the parallel transport of $w\in T_p M$ along
the geodesic $t \mapsto \exp_p(tv)$. So
$$
|(d\exp_p)_{\rho v} (\rho w)|^2 = f_K(\rho)^2 |w|^2. 
$$

Therefore,
$$
F^* g = d\rho^2 + f_K(\rho)^2 g_{\can}^{S^{n-1}}=\begin{cases}
\displaystyle{ d\rho^2+\Big(\frac{\sin(\sqrt{K}\rho)}{\sqrt{K}}\Big)^2 g_\can^{S^{n-1}} }, & K>0;\\
& \\
\displaystyle{ d\rho^2+\rho^2 g_\can^{S^{n-1}} }, & K=0;\\
& \\
\displaystyle{ d\rho^2+\Big(\frac{\sinh(\sqrt{-K}\rho)}{\sqrt{-K}}\Big)^2 g_\can^{S^{n-1}} },&  K<0.
\end{cases} 
$$

\noindent
{\bf \large Conjugate points} 

See \cite{dC} Chapter 5  Section 3. 


\bigskip


\noindent
{\bf \large Divergence and Laplacian Revisited}

Let $(M,g)$ be a Riemannian manifold.

Given a vector field $Y\in \fX(M)$, we may write $Y= Y^i\frac{\partial}{\partial x_i}$
in a coordinate neighborhood $U$ with local coordinates $(x_1,\ldots,x_n)$, where $Y^i\in C^\infty(U)$.
Then
$$
\mathrm{div} Y = {Y^i}_{,i}=\frac{\partial Y^i}{\partial x_i}+\Gamma^i_{ik} Y^k.
$$
\begin{lemma}\label{divergence-local}
$$
\mathrm{div} Y =\frac{1}{\sqrt{ \det(g) }}\sum_i \frac{\partial}{\partial x_i}(\sqrt{\det(g)} Y^i).
$$
\end{lemma}

\begin{proof}
\begin{eqnarray*}
\sum_i \Gamma_{ik}^i &=& \frac{1}{2}\sum_{i,j} g^{ij}(\frac{\partial}{\partial x_i} g_{kj}
+\frac{\partial}{\partial x_k} g_{ji}-\frac{\partial}{\partial x_j} g_{ik})= \frac{1}{2}\sum_{i,j} g^{ij} \frac{\partial}{\partial x_k} g_{ji}\\
&=& \frac{1}{2}\mathrm{Tr}(g^{-1}\frac{\partial}{\partial x_k} g)
=\frac{\partial}{\partial x_k}\log\sqrt{\det(g)} 
=\frac{1}{\sqrt{\det(g)}}\frac{\partial}{\partial x_k}(\sqrt{\det(g)}).
\end{eqnarray*}
\begin{eqnarray*}
\mathrm{div} Y &=& {Y^i}_{,i}= \sum_i \frac{\partial Y^i}{\partial x_i}+ \sum_k\Big(\frac{1}{\sqrt{\det(g)}}\frac{\partial}{\partial x_k}(\sqrt{\det(g)}) \Big) Y^k\\
&=& \frac{1}{\sqrt{ \det(g) }}\sum_i \frac{\partial}{\partial x_i}(\sqrt{\det(g)} Y^i).
\end{eqnarray*}
\end{proof}

\begin{corollary}
Let $(M,g)$ be an oriented Riemannian manifold, and let $\omega$ be the volume form determined by the Riemannian  metric $g$ and the orientation.
Then 
\begin{equation}\label{eqn:d-i-omega-div}
d(i_Y\omega) = \mathrm{div}(Y)\omega.
\end{equation}
\end{corollary}

\begin{proof} It suffice to verify this in each coordinate neighborhood $U$. Choose local coordinates
$(x_1,\ldots,x_n)$ compatible with the orientation. Then
\begin{eqnarray*}
\omega &=&\sqrt{\det(g)} dx_1 \wedge \cdots \wedge dx_n,\\
i_Y \omega &=& \sum_{i=1}^n (-1)^{i-1} Y^i\sqrt{\det(g)} dx_1 \wedge \cdots dx_{i-1}\wedge dx_{i+1}\wedge \cdots \wedge dx_n 
\end{eqnarray*}
\begin{equation}\label{eqn:d-i-omega}
d(i_Y \omega) = \sum_{i=1}^n \frac{\partial}{\partial x_i}(Y^i\sqrt{\det(g)}) dx_1\wedge \cdots \wedge d x_n
\end{equation}
\begin{equation}\label{eqn:div-omega}
(\mathrm{\mathrm{div}}Y)\omega = \mathrm{div}Y \sqrt{\det(g)} dx_1\ldots dx_n. 
\end{equation}
Equation \eqref{eqn:d-i-omega-div} follows from \eqref{eqn:d-i-omega}, \eqref{eqn:div-omega}, and
Lemma \ref{divergence-local}.
\end{proof}

\begin{corollary} In local coordinates, the Laplacian of a smooth function $f$ is given by 
$$
\Delta f =\frac{1}{\sqrt{ \det(g) } }\sum_{i,j} \frac{\partial}{\partial x_i}
\Big( \sqrt{\det(g)} g^{ij}\frac{\partial f}{\partial x_j}\Big) ,
$$
\end{corollary}


\begin{thebibliography}{ABC}

\bibitem[Bo]{Bo} William M. Boothby, {\em An Introduction to Differential Manifolds and Riemannian Geometry}, revised second edition.

\bibitem[dC]{dC}  Manfredo Perdigao do Carmo, {\em Riemannian Geometry}.

\bibitem[GHL]{GHL}Sylvestre Gallot, Dominique Hulin, and Jacques Lafontaine, 
{\em Riemannian Geometry}, third edition.

\bibitem[Mi]{Mi} 
John Milnor, ''Curvatures of left invariant metrics on Lie groups,"
Advances in Math. {\bf 21} (1976), no. 3, 293--329.


\end{thebibliography}


\end{document}