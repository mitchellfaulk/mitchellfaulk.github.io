\documentclass[12pt]{article}

\usepackage{amsmath, amssymb, xypic}
\usepackage[margin=1in]{geometry}

\newcommand{\norm}[1]{\left\lVert#1\right\rVert}
\newcommand{\floor}[1]{\left\lfloor#1\right\rfloor}

\begin{document}
\begin{center}
Assignment 5\\
Intro to Modern Analysis
\end{center}



\noindent \textbf{1.} Let $n$ be a given fixed positive integer. Find a function such that $f^{(n-1)}$ is continuous, but $f^{(n)}$ is not. 

\medskip

\noindent \textbf{2.} Let $f$ be continuous on $[a,b]$ and differentiable on $(a,b)$. Suppose that $f'(x) \ne 0$ for each $x$ in $(a,b)$. Show that $f$ is injective on $[a,b]$. 

\medskip



\noindent \textbf{3.} Suppose the derivative $f'$ is continuous on $(a,b)$ and $f'(x) \ne 0$ for each $x$ in $(a,b)$.  
\begin{enumerate}
\item[(a)] Show that $f$ admits an inverse $g$ defined on the image of $f$. 
\item[(b)] Prove that $g$ is differentiable and the derivative satisfies 
\[
g'(f(x)) = \frac{1}{f'(x)}
\]
for each $x \in (a,b)$. 
\end{enumerate}

\medskip

\noindent \textbf{4.} Let $f$ be continuous on $[a,b]$ and differentiable on $(a,b)$. Set $f(a) = y$ and suppose that $|f'(x)| \leqslant M$ for each $x \in (a,b)$. How large can $f(b)$ be? How small can $f(b)$ be? Prove that the values you find are actually achieved by demonstrating two functions which achieve them. 

\medskip

\noindent \textbf{5.} If $n$ is a positive integer and $0 \leqslant y \leqslant x$, show that 
\[
ny^{n-1}(x - y) \leqslant x^n - y^n \leqslant nx^{n-1}(x -y). 
\]

\medskip

\noindent \textbf{6.} Suppose $f$ is continuous on $[a,b]$ and $f(x) \geqslant 0$ for each $x \in [a,b]$. Show that if  
\[
\int_a^b f(x)\:dx = 0,
\]
then $f(x) = 0$ for each $x \in [a,b]$. 

\medskip

\noindent \textbf{7.} Define $f$ on $[a,b]$ by 
\[
f(x) = \begin{cases}
1 & x \; \text{rational} \\
0 & x \; \text{irrational}
\end{cases}.
\]
Show that $f$ is not Riemann integrable on $[a,b]$.

\medskip

\noindent \textbf{8.} Let $f$ be defined on $(0,1]$. Suppose that $f$ is Riemann integrable on $(c,1]$ for each $c \in (0,1)$. 
\begin{enumerate}
\item[(a)] If $f$ is Riemann integrable on $[0,1]$, show that 
\[
\int_0^1 f(x)\: dx = \lim_{c \to 0}\int_c^1 f(x)\:dx.
\]
\item[(b)] Construct a function $f$ for which the limit in (a) exists, even though the same limit fails to exist for $|f|$ in place of $f$. 
\end{enumerate} 

\medskip

\noindent \textbf{9.} Let $p$ and $q$ be positive real numbers satisfying 
\[
\frac{1}{p} + \frac{1}{q} = 1.
\]
\begin{enumerate}
\item[(a)] Show that for any nonnegative numbers $u,v \geqslant 0$, we have 
\[
uv \leqslant \frac{u^p}{p} + \frac{v^q}{q}.
\]
\item[(b)] Show that if $f$ and $g$ are both nonnegative (i.e. $f \geqslant 0$ and $g \geqslant 0$) and satisfy 
\[
\int_a^b f(x)^p\: dx = \int_a^b g(x)^q \: dx = 1,
\]
then 
\[
\int_a^b f(x)g(x) \: dx \leqslant 1.
\]
\item[(c)] Show that for any two functions (not necessarily nonnegative), we have 
\[
\left|\int_a^b f(x)g(x) \: dx\right| \leqslant \left(\int_a^b |f(x)|^p\: dx\right)^{1/p}\left(\int_a^b |g(x)|^q\: dx\right)^{1/q}
\]
provided both sides make sense. This is called H\"older's inequality. 
\end{enumerate}

\medskip

\noindent \textbf{10.} Prove directly from the definitions that if $f$ is Riemann integrable on $[a,b]$ and $c$ is any positive constant, then $cf$ is also integrable on $[a,b]$ and moreover
\[
\int_a^b cf(x) \: dx = c \int_a^b f(x)\: dx.
\] 




\end{document}