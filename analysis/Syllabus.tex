\documentclass[12pt]{article}

\usepackage{amsmath, amssymb, xypic}
\usepackage[margin=1in]{geometry}

\begin{document}
\begin{center}
MATH S4061: INTRO TO MODERN ANALYSIS I  \\
Summer 2019  \\
Section 001
\end{center}


\noindent \textbf{Instructor:} Mitchell Faulk \\
\textbf{Time:} MTWR 10:45am-12:10pm  \\
\textbf{Place:} TBA \\
\textbf{Office Hours:} TR: 1:00pm-2:00pm \\
\textbf{Teaching Assistant:} TBA \\
\textbf{Webpage:} http://math.columbia.edu/$\sim$faulk/AnalysisSummer2019

\section*{Prerequisites} 

Multivariable Calculus and linear algebra, or the equivalent


\section*{Description and goals}


We plan to cover the first seven chapters of Rudin's \emph{Principles of Mathematical Analysis (third edition)}. The main concepts covered include metric spaces, sequences, series, continuity, differentiation, integration, sequences of functions, and series of functions. 




\section*{Policies}

\subsection*{Grading}

The course will be graded as follows 

\vspace{3mm}


\noindent Homework: 40\%\\
Warm-ups: 10\% \\
Midterm: 20\% \\
Final: 30\%



\subsection*{Homework}

There will be six written assignments, each due at the beginning of class, typically on Wednesdays. 
\begin{itemize}
\item Late submissions will \textbf{not} be accepted. 
\item The lowest homework grade will be dropped. 
\item Please staple your work. (There is a stapler in the library of the math building.)
\item Collaboration with others is acceptable, but \textbf{you must write your own solutions by yourself}. Violators will be subject to strict penalties. 
\end{itemize} 

\noindent \textbf{On writing quality for homework}: Written assignments constitute the largest portion of your overall grade; I expect them to be completed with this in mind. This course is as much of a course in learning about math as it is in learning \emph{how to write} about math. 

I implore you to each submission as a \emph{final} draft. In particular, poor presentation (failure to write in complete sentences, excessive cross-outs, and general disorganization) will \emph{not} be tolerated and will be penalized accordingly. (Your English professor would not tolerate a cross-out in the final draft of a term paper; I shall have similar expectations in the presentation of these written assignments.)

It is important to remember that what appears on the blackboard is often much less than what I would expect from a well-written homework assignment because, as a lecturer, I often fail to write out details or explanations (saying them aloud instead) and use shorthand to save time. You should do neither of these things in your written work. Instead you should write out every relevant step in logic and comply with standard English grammar and syntax, which means, in particular, that your proofs should be written in complete sentences, with proper punctuation, including periods and capitalization. 

Also please avoid trying to ``save space'' on the page: leave adequate margins for comments, leave spaces between problems, and separate distinct portions of an argument into separate paragraphs. 

As a final note, I would encourage (perhaps even demand) that you work out your solutions on separate scrap paper first, and then, once you've assembled a complete solution, transcribe that solution in an organized fashion onto your final draft. 

\subsection*{Warm-ups}

There will be twenty ``warm-up'' problems, due at the beginning of class of each day of lecture, except the first and last days. (In particular, there will be \emph{no} warm-up questions due on days of exams.) 

The warm-up questions will be graded not on a point system by rather on a scale of three possible ``grades'' as follows
\begin{itemize}
\item Check ($\checkmark$): The warm-up question was attempted with adequate effort largely in compliance with the expectations of written quality suggested above for written assignments. 
\item Check-minus ($\checkmark^-$): The warm-up question was attempted, but there was a significant lack of effort or there were significant errors in reasoning or presentation. 
\item Zero ($0$): The warm-up question was not attempted. 
\end{itemize} 
As with homework assignments, there is a strict policy of not accepting late warm-up solutions.  Also, note that I will expect the same standards of presentation for warm-up solutions as with written assignments. Use them as an opportunity to practice writing math well. 


\subsection*{Exams}

There will be one midterm exam and one final exam. The anticipated dates are 

\vspace{3mm}

\noindent Midterm exam date: 06/13 \\
Final exam date: 07/03

\vspace{3mm}

\noindent If you have a conflict with either of these dates, you \textbf{must} contact me ahead of time to make arrangements. (At least a week in advance is ideal.) 

\medskip

\noindent \textbf{On writing quality for exams:} Because of the time restraints during an exam, my expectations for the quality of written work on exams are much more relaxed than they are for the quality of written assignments. In particular, in the exam setting, it is \emph{acceptable} to use shorthand and to be somewhat less careful in the organization and quality of your writing. Nevertheless, it is still my expectation that complete solutions will include all relevant logical steps, and solutions which make flagrant jumps in logic will be penalized accordingly. 


\subsection*{Textbook}

The textbook is Rudin's \emph{Principles of Mathematical Analysis (third edition)}. A version of the textbook can be found online. 

\subsection*{Students with disabilities}

In order to receive any disability-related accommodations, students must be registered with Disability Services (DS). Students that have, or think they may have, a disability are encouraged to contact DS for more information regarding policies and services available. 


\section*{Other advice}

This is a fast-paced course, so keeping up to date with material is important. The warm-ups and written assignments are designed to help you in this task. In addition, I strongly encourage you to work on homework assignments early, read the textbook before lecture, and ask questions during lecture if you have any. 


\section*{Syllabus and schedule}

\begin{tabular}{| l | l | l | l |}\hline
Date & Material & Textbook & Announcements \\ \hline\hline
05/27 & HOLIDAY &  HOLIDAY  & HOLIDAY \\
05/28 & Ordered sets &  &  \\
05/29 & Ordered sets &  &  \\
05/30 & Topology &  & HW1 due \\
05/31 & Topology &  & Quiz 1 \\ \hline \hline  
06/03 & Topology &  &  \\
06/04 & Sequences &  &   \\
06/05 & Sequences &  & HW2 due \\
06/06 & Series &  & Quiz 2 \\ \hline \hline 
06/10 & Series &  &   \\
06/11 & Continuity &  &  \\
06/12 & Continuity &  & HW3 due \\
06/13 & \textbf{Midterm exam} &  &  \\ \hline \hline 
06/17 & Differentiation &  &  \\ 
06/18 & Differentiation &  &  \\
06/19 & Integration &  & HW4 due \\
06/20 & Integration &  & Quiz 3 \\ \hline \hline 
06/24 & Integration &  &  \\
06/25 & Sequences of functions &  & \\
06/26 & Sequences of functions &  & HW5 due \\
06/27 & Series of functions &  & Quiz 4 \\ \hline \hline 
07/01 & Stone-Weierstrass &  &  \\
07/02 & Review & & \\
07/03 & \textbf{Final exam} & & HW6 due  \\
07/04 & HOLIDAY & HOLIDAY & HOLIDAY \\ \hline
\end{tabular}




\end{document}