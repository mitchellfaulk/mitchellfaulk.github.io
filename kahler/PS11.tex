\documentclass[12pt]{article}

\usepackage{amsmath, amssymb, amsthm, setspace, hyperref, xypic, mathrsfs,enumitem}
\usepackage[margin=1.25in]{geometry}
\newcommand{\norm}[1]{\left\lVert#1\right\rVert}
\newcommand{\ddb}{\partial\bar{\partial}}

\DeclareMathOperator{\Spec}{Spec}



\theoremstyle{definition}
\newtheorem{definition}{Definition}
\newtheorem{exercise}[definition]{Exercise}
\newtheorem{example}[definition]{Example}
\newtheorem{remark}[definition]{Remark}
\newtheorem{note}[definition]{Note}
\newtheorem{notation}[definition]{Notation}

\theoremstyle{theorem}
\newtheorem{proposition}[definition]{Proposition}
\newtheorem{theorem}[definition]{Theorem}
\newtheorem{corollary}[definition]{Corollary}
\newtheorem{lemma}[definition]{Lemma}
\newtheorem{problem}[definition]{Problem}
\newtheorem{conjecture}[definition]{Conjecture}
\newtheorem{task}[definition]{Task}

\begin{document}
\noindent Problem Set 11 \\
\noindent Hilbert Mumford Criterion \\
\noindent Summer 2021 \\

\medskip

\noindent For simplicity, let's work over the field $\mathbb{C}$. We also simplify our lives by considering the following situation. Let $G$ be a linear reductive group acting on a scheme $X$, where $X \subset \mathbb{C}^n$ is affine, $G$ acts linearly (through a faithful embedding into $GL_n$), and the action is linearized using a character $\chi$ of $G$.\footnote{This situation is very different from the one in MFK, where, in particular, $X$ is projective and the action is linearized through any invertible sheaf $L$ over $X$, not necessarily the trivial sheaf. Nevertheless, certain similarities persist.} The last condition means the following. We view $G$ as acting on the trivial line bundle $X \times \mathbb{C}$ by the rule 
\[
g \cdot (x,z) = (g \cdot x, \chi(g)z)
\]
where $\chi : G \to GL_1 = \mathbb{C}^*$ is a fixed character. Let $L_\chi$ denote the trivial line bundle together with this choice of linearization. Note that with these conventions, we have a natural isomorphism 
\[
L_{\chi^{\otimes m}} \simeq (L_\chi)^{\otimes m}. 
\] 

A section of the trivial bundle can be identified with a regular function $s : X \to \mathbb{C}$. I will call such a section \emph{$\chi$-equivariant} if 
\[
s(g \cdot x) = \chi(g)s(x),
\]
that is, if $s$ is an equivariant morphism with respect to the actions of $G$ on $X$ and $\mathbb{C}$. The space of $\chi$-equivariant sections will be denoted $H^0(X, L_\chi)$. It is a subspace of the space of regular functions on $X$. A point $x \in X$ is called \emph{semistable} if there is a positive integer $m$ and a $\chi^m$-equivariant section $s \in H^0(X, L_\chi^{\otimes m})$ such that $s(x) \ne 0$. The subset of semistable points will be denoted $X_{\text{ss}}(L_\chi)$. 



\begin{enumerate}[label=\textbf{\arabic*.}]
\item Let $X = \mathbb{C}^n$. Let $G = \mathbb{C}^*$ act on $X$ through the usual diagonal action. A character $\chi$ of $G$ takes the form $\chi(t) = t^d$ for some integer $d \in \mathbb{Z}$. 
\begin{enumerate}
\item Check that $H^0(X, L_\chi)$ is isomorphic to the vector space of homogeneous polynomials of degree $d$. 
\item Show the following 
\[
X_{\text{ss}}(L_\chi) = \begin{cases}
\mathbb{C}^n \setminus 0 & d > 0 \\
\mathbb{C}^n & d = 0 \\
\varnothing & d < 0
\end{cases}.
\]
\end{enumerate}
\item Let $X = \mathbb{C}^3$. Let $G = \{(t,t^{-1},u) \in (\mathbb{C}^*)^3 : t \in \mathbb{C}^*, u = \pm 1\} \simeq \mathbb{C}^* \times \mu_2$. Let $G$ act on $X$ in the obvious way. Let $\chi : G \to \mathbb{C}^*$ denote the character 
\[
\chi(t,t^{-1}, u) = tu. 
\]
\begin{enumerate}
\item The regular functions on $X$ can be identified with the polynomials in the variables $x,y,z$. Show that a monomial $x^ay^bz^c$ is $\chi$-equivariant if and only if 
\[
a = b + 1, c \equiv 1 \; \text{mod}\; 2.
\]
\item Show that $H^0(X, L_\chi) \simeq xz\mathbb{C}[xy,z^2]$. 
\item Show that $X_{\text{ss}}(L_\chi) = \mathbb{C}^* \times \mathbb{C}^2$.
\end{enumerate}
\item For two positive integers $r \leqslant n$, let $X = \text{Hom}(\mathbb{C}^n, \mathbb{C}^r) = M_{r \times n}$ be the set of $r$-by-$n$ matrices. Let $G = GL_r = \text{Aut}(\mathbb{C}^r)$ be the general linear group acting by post-composition:
\[
g \cdot A = gA.
\]
Let $\chi : GL_r \to \mathbb{C}^*$ be the determinant. 
\begin{enumerate}
\item Let $s : X \to \mathbb{C}$ be a regular function determined by a minor of maximal rank, that is, by an $r$-by-$r$ minor. Show that $s$ is $\chi$-equivariant. 
\item Show that $X_{\text{ss}}(L_\chi)$ contains the subset of full rank matrices. 
\end{enumerate}
\end{enumerate}

\medskip

A one-parameter subgroup of $G$ is a morphism $\lambda : \mathbb{C}^* \to G$ of algebraic groups. 




\end{document}